% \iffalse meta-comment
%
% File: selinput.dtx
% Version: 2007/09/09 v1.2
% Info: Semi-automatic input encoding detection
%
% Copyright (C) 2007 by
%    Heiko Oberdiek <heiko.oberdiek at googlemail.com>
%
% This work may be distributed and/or modified under the
% conditions of the LaTeX Project Public License, either
% version 1.3c of this license or (at your option) any later
% version. This version of this license is in
%    http://www.latex-project.org/lppl/lppl-1-3c.txt
% and the latest version of this license is in
%    http://www.latex-project.org/lppl.txt
% and version 1.3 or later is part of all distributions of
% LaTeX version 2005/12/01 or later.
%
% This work has the LPPL maintenance status "maintained".
%
% This Current Maintainer of this work is Heiko Oberdiek.
%
% This work consists of the main source file selinput.dtx
% and the derived files
%    selinput.sty, selinput.pdf, selinput.ins, selinput.drv,
%    selinput-test1.tex, selinput-test2.tex, selinput-test3.tex,
%    selinput-test4.tex, selinput-test5.tex.
%
% Distribution:
%    CTAN:macros/latex/contrib/oberdiek/selinput.dtx
%    CTAN:macros/latex/contrib/oberdiek/selinput.pdf
%
% Unpacking:
%    (a) If selinput.ins is present:
%           tex selinput.ins
%    (b) Without selinput.ins:
%           tex selinput.dtx
%    (c) If you insist on using LaTeX
%           latex \let\install=y% \iffalse meta-comment
%
% File: selinput.dtx
% Version: 2007/09/09 v1.2
% Info: Semi-automatic input encoding detection
%
% Copyright (C) 2007 by
%    Heiko Oberdiek <heiko.oberdiek at googlemail.com>
%
% This work may be distributed and/or modified under the
% conditions of the LaTeX Project Public License, either
% version 1.3c of this license or (at your option) any later
% version. This version of this license is in
%    http://www.latex-project.org/lppl/lppl-1-3c.txt
% and the latest version of this license is in
%    http://www.latex-project.org/lppl.txt
% and version 1.3 or later is part of all distributions of
% LaTeX version 2005/12/01 or later.
%
% This work has the LPPL maintenance status "maintained".
%
% This Current Maintainer of this work is Heiko Oberdiek.
%
% This work consists of the main source file selinput.dtx
% and the derived files
%    selinput.sty, selinput.pdf, selinput.ins, selinput.drv,
%    selinput-test1.tex, selinput-test2.tex, selinput-test3.tex,
%    selinput-test4.tex, selinput-test5.tex.
%
% Distribution:
%    CTAN:macros/latex/contrib/oberdiek/selinput.dtx
%    CTAN:macros/latex/contrib/oberdiek/selinput.pdf
%
% Unpacking:
%    (a) If selinput.ins is present:
%           tex selinput.ins
%    (b) Without selinput.ins:
%           tex selinput.dtx
%    (c) If you insist on using LaTeX
%           latex \let\install=y% \iffalse meta-comment
%
% File: selinput.dtx
% Version: 2007/09/09 v1.2
% Info: Semi-automatic input encoding detection
%
% Copyright (C) 2007 by
%    Heiko Oberdiek <heiko.oberdiek at googlemail.com>
%
% This work may be distributed and/or modified under the
% conditions of the LaTeX Project Public License, either
% version 1.3c of this license or (at your option) any later
% version. This version of this license is in
%    http://www.latex-project.org/lppl/lppl-1-3c.txt
% and the latest version of this license is in
%    http://www.latex-project.org/lppl.txt
% and version 1.3 or later is part of all distributions of
% LaTeX version 2005/12/01 or later.
%
% This work has the LPPL maintenance status "maintained".
%
% This Current Maintainer of this work is Heiko Oberdiek.
%
% This work consists of the main source file selinput.dtx
% and the derived files
%    selinput.sty, selinput.pdf, selinput.ins, selinput.drv,
%    selinput-test1.tex, selinput-test2.tex, selinput-test3.tex,
%    selinput-test4.tex, selinput-test5.tex.
%
% Distribution:
%    CTAN:macros/latex/contrib/oberdiek/selinput.dtx
%    CTAN:macros/latex/contrib/oberdiek/selinput.pdf
%
% Unpacking:
%    (a) If selinput.ins is present:
%           tex selinput.ins
%    (b) Without selinput.ins:
%           tex selinput.dtx
%    (c) If you insist on using LaTeX
%           latex \let\install=y% \iffalse meta-comment
%
% File: selinput.dtx
% Version: 2007/09/09 v1.2
% Info: Semi-automatic input encoding detection
%
% Copyright (C) 2007 by
%    Heiko Oberdiek <heiko.oberdiek at googlemail.com>
%
% This work may be distributed and/or modified under the
% conditions of the LaTeX Project Public License, either
% version 1.3c of this license or (at your option) any later
% version. This version of this license is in
%    http://www.latex-project.org/lppl/lppl-1-3c.txt
% and the latest version of this license is in
%    http://www.latex-project.org/lppl.txt
% and version 1.3 or later is part of all distributions of
% LaTeX version 2005/12/01 or later.
%
% This work has the LPPL maintenance status "maintained".
%
% This Current Maintainer of this work is Heiko Oberdiek.
%
% This work consists of the main source file selinput.dtx
% and the derived files
%    selinput.sty, selinput.pdf, selinput.ins, selinput.drv,
%    selinput-test1.tex, selinput-test2.tex, selinput-test3.tex,
%    selinput-test4.tex, selinput-test5.tex.
%
% Distribution:
%    CTAN:macros/latex/contrib/oberdiek/selinput.dtx
%    CTAN:macros/latex/contrib/oberdiek/selinput.pdf
%
% Unpacking:
%    (a) If selinput.ins is present:
%           tex selinput.ins
%    (b) Without selinput.ins:
%           tex selinput.dtx
%    (c) If you insist on using LaTeX
%           latex \let\install=y\input{selinput.dtx}
%        (quote the arguments according to the demands of your shell)
%
% Documentation:
%    (a) If selinput.drv is present:
%           latex selinput.drv
%    (b) Without selinput.drv:
%           latex selinput.dtx; ...
%    The class ltxdoc loads the configuration file ltxdoc.cfg
%    if available. Here you can specify further options, e.g.
%    use A4 as paper format:
%       \PassOptionsToClass{a4paper}{article}
%
%    Programm calls to get the documentation (example):
%       pdflatex selinput.dtx
%       makeindex -s gind.ist selinput.idx
%       pdflatex selinput.dtx
%       makeindex -s gind.ist selinput.idx
%       pdflatex selinput.dtx
%
% Installation:
%    TDS:tex/latex/oberdiek/selinput.sty
%    TDS:doc/latex/oberdiek/selinput.pdf
%    TDS:doc/latex/oberdiek/test/selinput-test1.tex
%    TDS:doc/latex/oberdiek/test/selinput-test2.tex
%    TDS:doc/latex/oberdiek/test/selinput-test3.tex
%    TDS:doc/latex/oberdiek/test/selinput-test4.tex
%    TDS:doc/latex/oberdiek/test/selinput-test5.tex
%    TDS:source/latex/oberdiek/selinput.dtx
%
%<*ignore>
\begingroup
  \catcode123=1 %
  \catcode125=2 %
  \def\x{LaTeX2e}%
\expandafter\endgroup
\ifcase 0\ifx\install y1\fi\expandafter
         \ifx\csname processbatchFile\endcsname\relax\else1\fi
         \ifx\fmtname\x\else 1\fi\relax
\else\csname fi\endcsname
%</ignore>
%<*install>
\input docstrip.tex
\Msg{************************************************************************}
\Msg{* Installation}
\Msg{* Package: selinput 2007/09/09 v1.2 Semi-automatic input encoding detection (HO)}
\Msg{************************************************************************}

\keepsilent
\askforoverwritefalse

\let\MetaPrefix\relax
\preamble

This is a generated file.

Project: selinput
Version: 2007/09/09 v1.2

Copyright (C) 2007 by
   Heiko Oberdiek <heiko.oberdiek at googlemail.com>

This work may be distributed and/or modified under the
conditions of the LaTeX Project Public License, either
version 1.3c of this license or (at your option) any later
version. This version of this license is in
   http://www.latex-project.org/lppl/lppl-1-3c.txt
and the latest version of this license is in
   http://www.latex-project.org/lppl.txt
and version 1.3 or later is part of all distributions of
LaTeX version 2005/12/01 or later.

This work has the LPPL maintenance status "maintained".

This Current Maintainer of this work is Heiko Oberdiek.

This work consists of the main source file selinput.dtx
and the derived files
   selinput.sty, selinput.pdf, selinput.ins, selinput.drv,
   selinput-test1.tex, selinput-test2.tex, selinput-test3.tex,
   selinput-test4.tex, selinput-test5.tex.

\endpreamble
\let\MetaPrefix\DoubleperCent

\generate{%
  \file{selinput.ins}{\from{selinput.dtx}{install}}%
  \file{selinput.drv}{\from{selinput.dtx}{driver}}%
  \usedir{tex/latex/oberdiek}%
  \file{selinput.sty}{\from{selinput.dtx}{package}}%
  \usedir{doc/latex/oberdiek/test}%
  \file{selinput-test1.tex}{\from{selinput.dtx}{test,test1}}%
  \file{selinput-test2.tex}{\from{selinput.dtx}{test,test2}}%
  \file{selinput-test3.tex}{\from{selinput.dtx}{test,test3}}%
  \file{selinput-test4.tex}{\from{selinput.dtx}{test,test4}}%
  \file{selinput-test5.tex}{\from{selinput.dtx}{test,test5}}%
  \nopreamble
  \nopostamble
  \usedir{source/latex/oberdiek/catalogue}%
  \file{selinput.xml}{\from{selinput.dtx}{catalogue}}%
}

\catcode32=13\relax% active space
\let =\space%
\Msg{************************************************************************}
\Msg{*}
\Msg{* To finish the installation you have to move the following}
\Msg{* file into a directory searched by TeX:}
\Msg{*}
\Msg{*     selinput.sty}
\Msg{*}
\Msg{* To produce the documentation run the file `selinput.drv'}
\Msg{* through LaTeX.}
\Msg{*}
\Msg{* Happy TeXing!}
\Msg{*}
\Msg{************************************************************************}

\endbatchfile
%</install>
%<*ignore>
\fi
%</ignore>
%<*driver>
\NeedsTeXFormat{LaTeX2e}
\ProvidesFile{selinput.drv}%
  [2007/09/09 v1.2 Semi-automatic input encoding detection (HO)]%
\documentclass{ltxdoc}
\usepackage[T1]{fontenc}
\usepackage{textcomp}
\usepackage{lmodern}
\usepackage{holtxdoc}[2011/11/22]
\usepackage{color}
\begin{document}
  \DocInput{selinput.dtx}%
\end{document}
%</driver>
% \fi
%
% \CheckSum{389}
%
% \CharacterTable
%  {Upper-case    \A\B\C\D\E\F\G\H\I\J\K\L\M\N\O\P\Q\R\S\T\U\V\W\X\Y\Z
%   Lower-case    \a\b\c\d\e\f\g\h\i\j\k\l\m\n\o\p\q\r\s\t\u\v\w\x\y\z
%   Digits        \0\1\2\3\4\5\6\7\8\9
%   Exclamation   \!     Double quote  \"     Hash (number) \#
%   Dollar        \$     Percent       \%     Ampersand     \&
%   Acute accent  \'     Left paren    \(     Right paren   \)
%   Asterisk      \*     Plus          \+     Comma         \,
%   Minus         \-     Point         \.     Solidus       \/
%   Colon         \:     Semicolon     \;     Less than     \<
%   Equals        \=     Greater than  \>     Question mark \?
%   Commercial at \@     Left bracket  \[     Backslash     \\
%   Right bracket \]     Circumflex    \^     Underscore    \_
%   Grave accent  \`     Left brace    \{     Vertical bar  \|
%   Right brace   \}     Tilde         \~}
%
% \GetFileInfo{selinput.drv}
%
% \title{The \xpackage{selinput} package}
% \date{2007/09/09 v1.2}
% \author{Heiko Oberdiek\\\xemail{heiko.oberdiek at googlemail.com}}
%
% \maketitle
%
% \begin{abstract}
% This package selects the input encoding by specifying between
% input characters and their glyph names.
% \end{abstract}
%
% \tableofcontents
%
% \newcommand*{\EM}{\textcolor{blue}}
% \newcommand*{\ExampleText}{^^A
%   Umlauts:\ \EM{\"A\"O\"U\"a\"o\"u\ss}^^A
% }
%
% \section{Documentation}
%
% \subsection{Introduction}
%
% \LaTeX\ supports the direct use of 8-bit characters by means
% of package \xpackage{inputenc}. However you must know
% and specify the encoding, e.g.:
% \begin{quote}
%   \ttfamily
%   |\documentclass{article}|\\
%   |\usepackage[|\EM{latin1}|]{inputenc}|\\
%   |% or \usepackage[|\EM{utf8}|]{inputenc}|\\
%   |% or \usepackage[|\EM{??}|]{inputenc}|\\
%   |\begin{document}|\\
%   |  |\ExampleText\\
%   |\end{document}|
% \end{quote}
%
% If the document is transferred in an environment that
% uses a different encoding, then there are programs that
% convert the input characters. Examples for conversion
% of file \xfile{test.tex}
% from encoding latin1 (ISO-8859-1) to UTF-8:
% \begin{quote}
%   \ttfamily
%   |recode ISO-8859-1..UTF-8 test.tex|\\
%   |recode latin1..utf8 test.tex|\\
%   |iconv --from-code ISO-8859-1|\\
%   \hphantom{iconv}| --to-code UTF-8|\\
%   \hphantom{iconv}| --output testnew.tex|\\
%   \hphantom{iconv}| test.tex|\\
%   |iconv -f latin1 -t utf8 -o testnew.tex test.tex|
% \end{quote}
% However, the encoding name for package \xpackage{inputenc}
% must be changed:
% \begin{quote}
%    |\usepackage[latin1]{inputenc}| $\rightarrow$
%    |\usepackage[utf8]{inputenc}|\kern-4pt\relax
% \end{quote}
% Of course, unless you are using some clever editor
% that knows package \xpackage{inputenc}, recodes
% the file and adjusts the option at the same time.
% But most editors can perhaps recode the file, but
% they let the option untouched.
%
% Therefore package \xpackage{selinput} chooses another way for
% specifying the input encoding. The encoding name is not needed
% at all. Some 8-bit characters are identified by their glyph
% name and the package chooses an appropriate encoding, example:
% \begin{quote}
%   \ttfamily
%   |\documentclass{article}|\\
%   |\usepackage{selinput}|\\
%   |\SelectInputMappings{|\\
%   |  adieresis={|\EM{\"a}|}|,\\
%   |  germandbls={|\EM{\ss}|}|,\\
%   |  Euro={|\EM{\texteuro}|}|,\\
%   |}|\\
%   |\begin{document}|\\
%   |  |\ExampleText\\
%   |\end{document}|
% \end{quote}
%
% \subsection{User interface}
%
% \begin{declcs}{SelectInputEncodingList} \M{encoding list}
% \end{declcs}
% \cs{SelectInputEncodingList} expects a comma separated list of
% encoding names. Example:
% \begin{quote}
%   |\SelectInputEncodingList{utf8,ansinew,mac-roman}|
% \end{quote}
% The encodings of package \xpackage{inputenx} are used as default.
%
% \begin{declcs}{SelectInputMappings} \M{mapping pairs}
% \end{declcs}
% A mapping pair consists of a glyph name and its input
% character:
% \begin{quote}
%   |\SelectInputMappings{|\\
%   |  adieresis={|\EM{\"a}|}|,\\
%   |  germandbls={|\EM{\ss}|}|,\\
%   |  Euro={|\EM{\texteuro}|}|,\\
%   |}|
% \end{quote}
% The supported glyph names can be found in file \xfile{ix-name.def}
% of project \xpackage{inputenx} \cite{inputenx}. The names are
% basically taken from Adobe's glyphlists \cite{adobe:glyphlist,adobe:aglfn}.
% As many pairs are needed as necessary to identify the encoding.
% Example with insufficient pairs:
% \begin{quote}
%   \ttfamily
%   |\SelectInputEncodingSet{latin1,latin9}|\\
%   |\SelectInputMappings{|\\
%   |  adieresis={|\EM{\"a}|}|,\\
%   |  germandbls={|\EM{\ss}|}|,\\
%   |}|\\
%   \ExampleText| and Euro: |\EM{\textcurrency} (wrong)
% \end{quote}
% The first encoding \xoption{latin1} passes the constraints given
% by the mapping pairs. However the Euro symbol is not part of
% the encoding. Thus a mapping pair with the Euro symbol
% solves the problem. In fact the symbol alone already succeeds in selecting
% between \xoption{latin1} and \xoption{latin9}:
% \begin{quote}
%   \ttfamily
%   |\SelectInputEncodingSet{latin1,latin9}|\\
%   |\SelectInputMappings{|\\
%   |  Euro={|\EM{\texteuro}|},|\\
%   |}|\\
%   \ExampleText| and Euro: |\EM{\texteuro}
% \end{quote}
%
% \subsection{Options}
%
% \begin{description}
% \item[\xoption{warning}:]
%   The selected encoding is written
%   by \cs{PackageInfo} into the \xfile{.log} file only.
%   Option \xoption{warning} changes it to \cs{PackageWarning}.
%   Then the selected encoding is shown on the terminal as well.
% \item[\xoption{ucs}:]
%   The encoding file \xfile{utf8x} of package \cs{ucs} requires
%   that the package itself is loaded before.
%   If the package is not loaded, then the option \xoption{ucs}
%   will load package \xpackage{ucs} if the detected encoding is
%   UTF-8 (limited to the preamble, packages cannot be loaded later).
% \item[\xoption{utf8=\dots}:]
%   The option allows to specify other encoding files
%   for UTF-8 than \LaTeX's \xfile{utf8.def}. For example,
%   |utf8=utf-8| will load \xfile{utf-8.def} instead.
% \end{description}
%
% \subsection{Encodings}
%
% Package \xpackage{stringenc} \cite{stringenc}
% is used for testing the encoding. Thus the encoding
% name must be known by this package. Then the found
% encoding is loaded by \cs{inputencoding} by package
% \xpackage{inputenc} or \cs{InputEncoding} if package
% \xpackage{inputenx} is loaded.
%
% The supported encodings are present in the encoding list,
% thus usually the encoding names do not matter.
% If the list is set by \cs{SelectInputEncodingList},
% then you can use the names that work for package
% \xpackage{inputenc} and are known by package \xpackage{stringenc},
% for example: \xoption{latin1}, \xoption{x-iso-8859-1}. Encoding
% file names of package \xpackage{inputenx} are prefixed with \xfile{x-}.
% The prefix can be dropped, if package \xpackage{inputenx} is loaded.
%
% \StopEventually{
% }
%
% \section{Implementation}
%
%    \begin{macrocode}
%<*package>
\NeedsTeXFormat{LaTeX2e}
\ProvidesPackage{selinput}
  [2007/09/09 v1.2 Semi-automatic input encoding detection (HO)]%
%    \end{macrocode}
%
%    \begin{macrocode}
\RequirePackage{inputenc}
\RequirePackage{kvsetkeys}[2006/10/19]
\RequirePackage{stringenc}[2007/06/16]
\RequirePackage{kvoptions}
%    \end{macrocode}
%    \begin{macro}{\SelectInputEncodingList}
%    \begin{macrocode}
\newcommand*{\SelectInputEncodingList}{%
  \let\SIE@EncodingList\@empty
  \kvsetkeys{SelInputEnc}%
}
%    \end{macrocode}
%    \end{macro}
%    \begin{macro}{\SelectInputMappings}
%    \begin{macrocode}
\newcommand*{\SelectInputMappings}[1]{%
  \SIE@LoadNameDefs
  \let\SIE@StringUnicode\@empty
  \let\SIE@StringDest\@empty
  \kvsetkeys{SelInputMap}{#1}%
  \ifx\\SIE@StringUnicode\SIE@StringDest\\%
    \PackageError{selinput}{%
      No mappings specified%
    }\@ehc
  \else
    \EdefUnescapeHex\SIE@StringUnicode\SIE@StringUnicode
    \let\SIE@Encoding\@empty
    \@for\SIE@EncodingTest:=\SIE@EncodingList\do{%
      \ifx\SIE@Encoding\@empty
        \StringEncodingConvertTest\SIE@temp\SIE@StringUnicode
                                  {utf16be}\SIE@EncodingTest{%
          \ifx\SIE@temp\SIE@StringDest
            \let\SIE@Encoding\SIE@EncodingTest
          \fi
        }{}%
      \fi
    }%
    \ifx\SIE@Encoding\@empty
      \StringEncodingConvertTest\SIE@temp\SIE@StringDest
                                {ascii}{utf16be}{%
        \def\SIE@Encoding{ascii}%
        \SIE@Info{selinput}{%
          Matching encoding not found, but input characters%
          \MessageBreak
          are 7-bit (possibly editor replacements).%
          \MessageBreak
          Hence using ascii encoding%
        }%
      }{}%
    \fi
    \ifx\SIE@Encoding\@empty
      \PackageError{selinput}{%
        Cannot find a matching encoding%
      }\@ehd
    \else
      \ifx\SIE@Encoding\SIE@EncodingUTFviii
        \SIE@LoadUnicodePackage
        \ifx\SIE@UseUTFviii\@empty
        \else
          \let\SIE@Encoding\SIE@UseUTFviii
        \fi
      \fi
      \begingroup\expandafter\expandafter\expandafter\endgroup
      \expandafter\ifx\csname InputEncoding\endcsname\relax
        \inputencoding\SIE@Encoding
      \else
        \InputEncoding\SIE@Encoding
      \fi
      \SIE@Info{selinput}{Encoding `\SIE@Encoding' selected}%
    \fi
  \fi
}
%    \end{macrocode}
%    \end{macro}
%    \begin{macro}{\SIE@LoadNameDefs}
%    \begin{macrocode}
\def\SIE@LoadNameDefs{%
  \begingroup
    \endlinechar=\m@ne
    \catcode92=0 % backslash
    \catcode123=1 % left curly brace/beginning of group
    \catcode125=2 % right curly brace/end of group
    \catcode37=14 % percent/comment character
    \@makeother\[%
    \@makeother\]%
    \@makeother\.%
    \@makeother\(%
    \@makeother\)%
    \@makeother\/%
    \@makeother\-%
    \let\InputenxName\SelectInputDefineMapping
    \InputIfFileExists{ix-name.def}{}{%
      \PackageError{selinput}{%
        Missing `ix-name.def' (part of package `inputenx')%
      }\@ehd
    }%
    \global\let\SIE@LoadNameDefs\relax
  \endgroup
}
%    \end{macrocode}
%    \end{macro}
%    \begin{macro}{\SelectInputDefineMapping}
%    \begin{macrocode}
\newcommand*{\SelectInputDefineMapping}[1]{%
  \expandafter\gdef\csname SIE@@#1\endcsname
}
%    \end{macrocode}
%    \end{macro}
%    \begin{macrocode}
\kv@set@family@handler{SelInputMap}{%
  \@onelevel@sanitize\kv@key
  \ifx\kv@value\relax
    \PackageError{selinput}{%
      Missing input character for `\kv@key'%
    }\@ehc
  \else
    \@onelevel@sanitize\kv@value
    \ifx\kv@value\@empty
      \PackageError{selinput}{%
        Input character got lost?\MessageBreak
        Missing input character for `\kv@key'%
      }\@ehc
    \else
      \@ifundefined{SIE@@\kv@key}{%
        \PackageWarning{selinput}{%
          Missing definition for `\kv@key'%
        }%
      }{%
        \edef\SIE@StringDest{%
          \SIE@StringDest
          \kv@value
        }%
        \edef\SIE@StringUnicode{%
          \SIE@StringUnicode
          \csname SIE@@\kv@key\endcsname
        }%
      }%
    \fi
  \fi
}
%    \end{macrocode}
%    \begin{macrocode}
\kv@set@family@handler{SelInputEnc}{%
  \@onelevel@sanitize\kv@key
  \ifx\kv@value\relax
    \ifx\SIE@EncodingList\@empty
      \let\SIE@EncodingList\kv@key
    \else
      \edef\SIE@EncodingList{\SIE@EncodingList,\kv@key}%
    \fi
  \else
    \@onelevel@sanitize\kv@value
    \PackageError{selinput}{%
      Illegal key value pair (\kv@key=\kv@value)\MessagBreak
      in encoding list%
    }\@ehc
  \fi
}
%    \end{macrocode}
%
%    \begin{macro}{\SIE@LoadUnicodePackage}
%    \begin{macrocode}
\def\SIE@LoadUnicodePackage{%
  \@ifpackageloaded\SIE@UnicodePackage{}{%
    \RequirePackage\SIE@UnicodePackage\relax
  }%
  \SIE@PatchUCS
  \global\let\SIE@LoadUnicodePackage\relax
}
\let\SIE@show\show
\def\SIE@PatchUCS{%
  \AtBeginDocument{%
    \expandafter\ifx\csname ver@ucsencs.def\endcsname\relax
    \else
      \let\show\SIE@show
    \fi
  }%
}
\SIE@PatchUCS
%    \end{macrocode}
%    \end{macro}
%    \begin{macrocode}
\AtBeginDocument{%
  \let\SIE@LoadUnicodePackage\relax
}
%    \end{macrocode}
%    \begin{macro}{\SIE@EncodingUTFviii}
%    \begin{macrocode}
\def\SIE@EncodingUTFviii{utf8}
\@onelevel@sanitize\SIE@EncodingUTFviii
%    \end{macrocode}
%    \end{macro}
%    \begin{macro}{\SIE@EncodingUTFviiix}
%    \begin{macrocode}
\def\SIE@EncodingUTFviiix{utf8x}
\@onelevel@sanitize\SIE@EncodingUTFviiix
%    \end{macrocode}
%    \end{macro}
%
%    \begin{macrocode}
\let\SIE@UnicodePackage\@empty
\let\SIE@UseUTFviii\@empty
\let\SIE@Info\PackageInfo
%    \end{macrocode}
%    \begin{macrocode}
\SetupKeyvalOptions{%
  family=SelInput,%
  prefix=SelInput@%
}
\define@key{SelInput}{utf8}{%
  \def\SIE@UseUTFviii{#1}%
  \@onelevel@sanitize\SIE@UseUTFviii
}
\DeclareBoolOption{ucs}
\DeclareVoidOption{warning}{%
  \let\SIE@Info\PackageWarning
}
\ProcessKeyvalOptions{SelInput}
\ifSelInput@ucs
  \def\SIE@UnicodePackage{ucs}%
  \ifx\SIE@UseUTFviii\@empty
    \let\SIE@UseUTFviii\SIE@EncodingUTFviiix
  \fi
\else
  \ifx\SIE@UseUTFviii\@empty
    \@ifpackageloaded{ucs}{%
      \let\SIE@UseUTFviii\SIE@EncodingUTFviiix
    }{%
      \let\SIE@UseUTFviii\SIE@EncodingUTFviii
    }%
  \fi
\fi
%    \end{macrocode}
%
%    \begin{macro}{\SIE@EncodingList}
%    \begin{macrocode}
\edef\SIE@EncodingList{%
  utf8,%
  x-iso-8859-1,%
  x-iso-8859-15,%
  x-cp1252,% ansinew
  x-mac-roman,%
  x-iso-8859-2,%
  x-iso-8859-3,%
  x-iso-8859-4,%
  x-iso-8859-5,%
  x-iso-8859-6,%
  x-iso-8859-7,%
  x-iso-8859-8,%
  x-iso-8859-9,%
  x-iso-8859-10,%
  x-iso-8859-11,%
  x-iso-8859-13,%
  x-iso-8859-14,%
  x-iso-8859-15,%
  x-mac-centeuro,%
  x-mac-cyrillic,%
  x-koi8-r,%
  x-cp1250,%
  x-cp1251,%
  x-cp1257,%
  x-cp437,%
  x-cp850,%
  x-cp852,%
  x-cp855,%
  x-cp858,%
  x-cp865,%
  x-cp866,%
  x-nextstep,%
  x-dec-mcs%
}%
\@onelevel@sanitize\SIE@EncodingList
%    \end{macrocode}
%    \end{macro}
%
%    \begin{macrocode}
%</package>
%    \end{macrocode}
%
% \section{Test}
%
%    \begin{macrocode}
%<*test>
\NeedsTeXFormat{LaTeX2e}
\documentclass{minimal}
\usepackage{textcomp}
\usepackage{qstest}
%    \end{macrocode}
%    \begin{macrocode}
%<*test1|test2|test3>
\makeatletter
\let\BeginDocumentText\@empty
\def\TestEncoding#1#2{%
  \SelectInputMappings{#2}%
  \Expect*{\SIE@Encoding}{#1}%
  \Expect*{\inputencodingname}{#1}%
  \g@addto@macro\BeginDocumentText{%
    \SelectInputMappings{#2}%
    \Expect*{\SIE@Encoding}{#1}%
    \textbf{\SIE@Encoding:} %
    \kvsetkeys{test}{#2}\par
  }%
}
\def\TestKey#1#2{%
  \define@key{test}{#1}{%
    \sbox0{##1}%
    \sbox2{#2}%
    \Expect*{wd:\the\wd0, ht:\the\ht0, dp:\the\dp0}%
           *{wd:\the\wd2, ht:\the\ht2, dp:\the\dp2}%
    [#1=##1] % hash-ok
  }%
}
\RequirePackage{keyval}
\TestKey{adieresis}{\"a}
\TestKey{germandbls}{\ss}
\TestKey{Euro}{\texteuro}
\makeatother
\usepackage[
  warning,%
%<test2>  utf8=utf-8,
%<test3>  ucs,
]{selinput}
%<test1|test3>\inputencoding{ascii}
%<test2>\inputencoding{utf-8}
%<test3>\usepackage{ucs}
\begin{qstest}{preamble}{}
  \TestEncoding{x-iso-8859-15}{%
    adieresis=^^e4,%
    germandbls=^^df,%
    Euro=^^a4,%
  }%
  \TestEncoding{x-cp1252}{%
    adieresis=^^e4,%
    germandbls=^^df,%
    Euro=^^80,%
  }%
%<test1>  \TestEncoding{utf8}{%
%<test2>  \TestEncoding{utf-8}{%
%<test3>  \TestEncoding{utf8x}{%
    adieresis=^^c3^^a4,%
    germandbls=^^c3^^9f,%
%<!test2>    Euro=^^e2^^82^^ac,
  }%
\end{qstest}
%<test3>\let\ifUnicodeOptiongraphics\iffalse
\begin{document}
\begin{qstest}{document}{}
%<test3>\makeatletter
  \BeginDocumentText
\end{qstest}
%</test1|test2|test3>
%    \end{macrocode}
%
%    \begin{macrocode}
%<*test4>
\usepackage[warning,ucs]{selinput}
\SelectInputMappings{%
    adieresis=^^c3^^a4,%
    germandbls=^^c3^^9f,%
    Euro=^^e2^^82^^ac,%
}
\begin{qstest}{encoding}{}
  \Expect*{\inputencodingname}{utf8x}%
\end{qstest}
\begin{document}
  adieresis=^^c3^^a4, %
  germandbls=^^c3^^9f, %
  Euro=^^e2^^82^^ac%
%</test4>
%    \end{macrocode}
%
%    \begin{macrocode}
%<*test5>
\usepackage[warning,ucs]{selinput}
\SelectInputMappings{%
    adieresis={\"a},%
    germandbls={{\ss}},%
    Euro=\texteuro{},%
}
\begin{qstest}{encoding}{}
  \Expect*{\inputencodingname}{ascii}%
\end{qstest}
\begin{document}
  adieresis={\"a}, %
  germandbls={{\ss}}, %
  Euro=\texteuro{}%
%</test5>
%    \end{macrocode}
%
%    \begin{macrocode}
\end{document}
%</test>
%    \end{macrocode}
%
% \section{Installation}
%
% \subsection{Download}
%
% \paragraph{Package.} This package is available on
% CTAN\footnote{\url{ftp://ftp.ctan.org/tex-archive/}}:
% \begin{description}
% \item[\CTAN{macros/latex/contrib/oberdiek/selinput.dtx}] The source file.
% \item[\CTAN{macros/latex/contrib/oberdiek/selinput.pdf}] Documentation.
% \end{description}
%
%
% \paragraph{Bundle.} All the packages of the bundle `oberdiek'
% are also available in a TDS compliant ZIP archive. There
% the packages are already unpacked and the documentation files
% are generated. The files and directories obey the TDS standard.
% \begin{description}
% \item[\CTAN{install/macros/latex/contrib/oberdiek.tds.zip}]
% \end{description}
% \emph{TDS} refers to the standard ``A Directory Structure
% for \TeX\ Files'' (\CTAN{tds/tds.pdf}). Directories
% with \xfile{texmf} in their name are usually organized this way.
%
% \subsection{Bundle installation}
%
% \paragraph{Unpacking.} Unpack the \xfile{oberdiek.tds.zip} in the
% TDS tree (also known as \xfile{texmf} tree) of your choice.
% Example (linux):
% \begin{quote}
%   |unzip oberdiek.tds.zip -d ~/texmf|
% \end{quote}
%
% \paragraph{Script installation.}
% Check the directory \xfile{TDS:scripts/oberdiek/} for
% scripts that need further installation steps.
% Package \xpackage{attachfile2} comes with the Perl script
% \xfile{pdfatfi.pl} that should be installed in such a way
% that it can be called as \texttt{pdfatfi}.
% Example (linux):
% \begin{quote}
%   |chmod +x scripts/oberdiek/pdfatfi.pl|\\
%   |cp scripts/oberdiek/pdfatfi.pl /usr/local/bin/|
% \end{quote}
%
% \subsection{Package installation}
%
% \paragraph{Unpacking.} The \xfile{.dtx} file is a self-extracting
% \docstrip\ archive. The files are extracted by running the
% \xfile{.dtx} through \plainTeX:
% \begin{quote}
%   \verb|tex selinput.dtx|
% \end{quote}
%
% \paragraph{TDS.} Now the different files must be moved into
% the different directories in your installation TDS tree
% (also known as \xfile{texmf} tree):
% \begin{quote}
% \def\t{^^A
% \begin{tabular}{@{}>{\ttfamily}l@{ $\rightarrow$ }>{\ttfamily}l@{}}
%   selinput.sty & tex/latex/oberdiek/selinput.sty\\
%   selinput.pdf & doc/latex/oberdiek/selinput.pdf\\
%   test/selinput-test1.tex & doc/latex/oberdiek/test/selinput-test1.tex\\
%   test/selinput-test2.tex & doc/latex/oberdiek/test/selinput-test2.tex\\
%   test/selinput-test3.tex & doc/latex/oberdiek/test/selinput-test3.tex\\
%   test/selinput-test4.tex & doc/latex/oberdiek/test/selinput-test4.tex\\
%   test/selinput-test5.tex & doc/latex/oberdiek/test/selinput-test5.tex\\
%   selinput.dtx & source/latex/oberdiek/selinput.dtx\\
% \end{tabular}^^A
% }^^A
% \sbox0{\t}^^A
% \ifdim\wd0>\linewidth
%   \begingroup
%     \advance\linewidth by\leftmargin
%     \advance\linewidth by\rightmargin
%   \edef\x{\endgroup
%     \def\noexpand\lw{\the\linewidth}^^A
%   }\x
%   \def\lwbox{^^A
%     \leavevmode
%     \hbox to \linewidth{^^A
%       \kern-\leftmargin\relax
%       \hss
%       \usebox0
%       \hss
%       \kern-\rightmargin\relax
%     }^^A
%   }^^A
%   \ifdim\wd0>\lw
%     \sbox0{\small\t}^^A
%     \ifdim\wd0>\linewidth
%       \ifdim\wd0>\lw
%         \sbox0{\footnotesize\t}^^A
%         \ifdim\wd0>\linewidth
%           \ifdim\wd0>\lw
%             \sbox0{\scriptsize\t}^^A
%             \ifdim\wd0>\linewidth
%               \ifdim\wd0>\lw
%                 \sbox0{\tiny\t}^^A
%                 \ifdim\wd0>\linewidth
%                   \lwbox
%                 \else
%                   \usebox0
%                 \fi
%               \else
%                 \lwbox
%               \fi
%             \else
%               \usebox0
%             \fi
%           \else
%             \lwbox
%           \fi
%         \else
%           \usebox0
%         \fi
%       \else
%         \lwbox
%       \fi
%     \else
%       \usebox0
%     \fi
%   \else
%     \lwbox
%   \fi
% \else
%   \usebox0
% \fi
% \end{quote}
% If you have a \xfile{docstrip.cfg} that configures and enables \docstrip's
% TDS installing feature, then some files can already be in the right
% place, see the documentation of \docstrip.
%
% \subsection{Refresh file name databases}
%
% If your \TeX~distribution
% (\teTeX, \mikTeX, \dots) relies on file name databases, you must refresh
% these. For example, \teTeX\ users run \verb|texhash| or
% \verb|mktexlsr|.
%
% \subsection{Some details for the interested}
%
% \paragraph{Attached source.}
%
% The PDF documentation on CTAN also includes the
% \xfile{.dtx} source file. It can be extracted by
% AcrobatReader 6 or higher. Another option is \textsf{pdftk},
% e.g. unpack the file into the current directory:
% \begin{quote}
%   \verb|pdftk selinput.pdf unpack_files output .|
% \end{quote}
%
% \paragraph{Unpacking with \LaTeX.}
% The \xfile{.dtx} chooses its action depending on the format:
% \begin{description}
% \item[\plainTeX:] Run \docstrip\ and extract the files.
% \item[\LaTeX:] Generate the documentation.
% \end{description}
% If you insist on using \LaTeX\ for \docstrip\ (really,
% \docstrip\ does not need \LaTeX), then inform the autodetect routine
% about your intention:
% \begin{quote}
%   \verb|latex \let\install=y\input{selinput.dtx}|
% \end{quote}
% Do not forget to quote the argument according to the demands
% of your shell.
%
% \paragraph{Generating the documentation.}
% You can use both the \xfile{.dtx} or the \xfile{.drv} to generate
% the documentation. The process can be configured by the
% configuration file \xfile{ltxdoc.cfg}. For instance, put this
% line into this file, if you want to have A4 as paper format:
% \begin{quote}
%   \verb|\PassOptionsToClass{a4paper}{article}|
% \end{quote}
% An example follows how to generate the
% documentation with pdf\LaTeX:
% \begin{quote}
%\begin{verbatim}
%pdflatex selinput.dtx
%makeindex -s gind.ist selinput.idx
%pdflatex selinput.dtx
%makeindex -s gind.ist selinput.idx
%pdflatex selinput.dtx
%\end{verbatim}
% \end{quote}
%
% \section{Catalogue}
%
% The following XML file can be used as source for the
% \href{http://mirror.ctan.org/help/Catalogue/catalogue.html}{\TeX\ Catalogue}.
% The elements \texttt{caption} and \texttt{description} are imported
% from the original XML file from the Catalogue.
% The name of the XML file in the Catalogue is \xfile{selinput.xml}.
%    \begin{macrocode}
%<*catalogue>
<?xml version='1.0' encoding='us-ascii'?>
<!DOCTYPE entry SYSTEM 'catalogue.dtd'>
<entry datestamp='$Date$' modifier='$Author$' id='selinput'>
  <name>selinput</name>
  <caption>Semi-automatic detection of input encoding.</caption>
  <authorref id='auth:oberdiek'/>
  <copyright owner='Heiko Oberdiek' year='2007'/>
  <license type='lppl1.3'/>
  <version number='1.2'/>
  <description>
    This package selects the input encoding by specifying pairs
    of input characters and their glyph names.
    <p/>
    The package is part of the <xref refid='oberdiek'>oberdiek</xref>
    bundle.
  </description>
  <documentation details='Package documentation'
      href='ctan:/macros/latex/contrib/oberdiek/selinput.pdf'/>
  <ctan file='true' path='/macros/latex/contrib/oberdiek/selinput.dtx'/>
  <miktex location='oberdiek'/>
  <texlive location='oberdiek'/>
  <install path='/macros/latex/contrib/oberdiek/oberdiek.tds.zip'/>
</entry>
%</catalogue>
%    \end{macrocode}
%
% \begin{thebibliography}{9}
% \bibitem{inputenx}
%   Heiko Oberdiek: \textit{The \xpackage{inputenx} package};
%   2007-04-11 v1.1;
%   \CTAN{macros/latex/contrib/oberdiek/inputenx.pdf}.
%
% \bibitem{adobe:glyphlist}
%   Adobe: \textit{Adobe Glyph List};
%   2002-09-20 v2.0;
%   \url{http://partners.adobe.com/public/developer/en/opentype/glyphlist.txt}.
%
% \bibitem{adobe:aglfn}
%   Adobe: \textit{Adobe Glyph List For New Fonts};
%   2005-11-18 v1.5;
%   \url{http://partners.adobe.com/public/developer/en/opentype/aglfn13.txt}.
%
% \bibitem{stringenc}
%   Heiko Oberdiek: \textit{The \xpackage{stringenc} package};
%   2007-06-16 v1.1;
%   \CTAN{macros/latex/contrib/oberdiek/stringenc.pdf}.
%
% \end{thebibliography}
%
% \begin{History}
%   \begin{Version}{2007/06/16 v1.0}
%   \item
%     First version.
%   \end{Version}
%   \begin{Version}{2007/06/20 v1.1}
%   \item
%     Requested date for package \xpackage{stringenc} fixed.
%   \end{Version}
%   \begin{Version}{2007/09/09 v1.2}
%   \item
%     Line end fixed.
%   \end{Version}
% \end{History}
%
% \PrintIndex
%
% \Finale
\endinput

%        (quote the arguments according to the demands of your shell)
%
% Documentation:
%    (a) If selinput.drv is present:
%           latex selinput.drv
%    (b) Without selinput.drv:
%           latex selinput.dtx; ...
%    The class ltxdoc loads the configuration file ltxdoc.cfg
%    if available. Here you can specify further options, e.g.
%    use A4 as paper format:
%       \PassOptionsToClass{a4paper}{article}
%
%    Programm calls to get the documentation (example):
%       pdflatex selinput.dtx
%       makeindex -s gind.ist selinput.idx
%       pdflatex selinput.dtx
%       makeindex -s gind.ist selinput.idx
%       pdflatex selinput.dtx
%
% Installation:
%    TDS:tex/latex/oberdiek/selinput.sty
%    TDS:doc/latex/oberdiek/selinput.pdf
%    TDS:doc/latex/oberdiek/test/selinput-test1.tex
%    TDS:doc/latex/oberdiek/test/selinput-test2.tex
%    TDS:doc/latex/oberdiek/test/selinput-test3.tex
%    TDS:doc/latex/oberdiek/test/selinput-test4.tex
%    TDS:doc/latex/oberdiek/test/selinput-test5.tex
%    TDS:source/latex/oberdiek/selinput.dtx
%
%<*ignore>
\begingroup
  \catcode123=1 %
  \catcode125=2 %
  \def\x{LaTeX2e}%
\expandafter\endgroup
\ifcase 0\ifx\install y1\fi\expandafter
         \ifx\csname processbatchFile\endcsname\relax\else1\fi
         \ifx\fmtname\x\else 1\fi\relax
\else\csname fi\endcsname
%</ignore>
%<*install>
\input docstrip.tex
\Msg{************************************************************************}
\Msg{* Installation}
\Msg{* Package: selinput 2007/09/09 v1.2 Semi-automatic input encoding detection (HO)}
\Msg{************************************************************************}

\keepsilent
\askforoverwritefalse

\let\MetaPrefix\relax
\preamble

This is a generated file.

Project: selinput
Version: 2007/09/09 v1.2

Copyright (C) 2007 by
   Heiko Oberdiek <heiko.oberdiek at googlemail.com>

This work may be distributed and/or modified under the
conditions of the LaTeX Project Public License, either
version 1.3c of this license or (at your option) any later
version. This version of this license is in
   http://www.latex-project.org/lppl/lppl-1-3c.txt
and the latest version of this license is in
   http://www.latex-project.org/lppl.txt
and version 1.3 or later is part of all distributions of
LaTeX version 2005/12/01 or later.

This work has the LPPL maintenance status "maintained".

This Current Maintainer of this work is Heiko Oberdiek.

This work consists of the main source file selinput.dtx
and the derived files
   selinput.sty, selinput.pdf, selinput.ins, selinput.drv,
   selinput-test1.tex, selinput-test2.tex, selinput-test3.tex,
   selinput-test4.tex, selinput-test5.tex.

\endpreamble
\let\MetaPrefix\DoubleperCent

\generate{%
  \file{selinput.ins}{\from{selinput.dtx}{install}}%
  \file{selinput.drv}{\from{selinput.dtx}{driver}}%
  \usedir{tex/latex/oberdiek}%
  \file{selinput.sty}{\from{selinput.dtx}{package}}%
  \usedir{doc/latex/oberdiek/test}%
  \file{selinput-test1.tex}{\from{selinput.dtx}{test,test1}}%
  \file{selinput-test2.tex}{\from{selinput.dtx}{test,test2}}%
  \file{selinput-test3.tex}{\from{selinput.dtx}{test,test3}}%
  \file{selinput-test4.tex}{\from{selinput.dtx}{test,test4}}%
  \file{selinput-test5.tex}{\from{selinput.dtx}{test,test5}}%
  \nopreamble
  \nopostamble
  \usedir{source/latex/oberdiek/catalogue}%
  \file{selinput.xml}{\from{selinput.dtx}{catalogue}}%
}

\catcode32=13\relax% active space
\let =\space%
\Msg{************************************************************************}
\Msg{*}
\Msg{* To finish the installation you have to move the following}
\Msg{* file into a directory searched by TeX:}
\Msg{*}
\Msg{*     selinput.sty}
\Msg{*}
\Msg{* To produce the documentation run the file `selinput.drv'}
\Msg{* through LaTeX.}
\Msg{*}
\Msg{* Happy TeXing!}
\Msg{*}
\Msg{************************************************************************}

\endbatchfile
%</install>
%<*ignore>
\fi
%</ignore>
%<*driver>
\NeedsTeXFormat{LaTeX2e}
\ProvidesFile{selinput.drv}%
  [2007/09/09 v1.2 Semi-automatic input encoding detection (HO)]%
\documentclass{ltxdoc}
\usepackage[T1]{fontenc}
\usepackage{textcomp}
\usepackage{lmodern}
\usepackage{holtxdoc}[2011/11/22]
\usepackage{color}
\begin{document}
  \DocInput{selinput.dtx}%
\end{document}
%</driver>
% \fi
%
% \CheckSum{389}
%
% \CharacterTable
%  {Upper-case    \A\B\C\D\E\F\G\H\I\J\K\L\M\N\O\P\Q\R\S\T\U\V\W\X\Y\Z
%   Lower-case    \a\b\c\d\e\f\g\h\i\j\k\l\m\n\o\p\q\r\s\t\u\v\w\x\y\z
%   Digits        \0\1\2\3\4\5\6\7\8\9
%   Exclamation   \!     Double quote  \"     Hash (number) \#
%   Dollar        \$     Percent       \%     Ampersand     \&
%   Acute accent  \'     Left paren    \(     Right paren   \)
%   Asterisk      \*     Plus          \+     Comma         \,
%   Minus         \-     Point         \.     Solidus       \/
%   Colon         \:     Semicolon     \;     Less than     \<
%   Equals        \=     Greater than  \>     Question mark \?
%   Commercial at \@     Left bracket  \[     Backslash     \\
%   Right bracket \]     Circumflex    \^     Underscore    \_
%   Grave accent  \`     Left brace    \{     Vertical bar  \|
%   Right brace   \}     Tilde         \~}
%
% \GetFileInfo{selinput.drv}
%
% \title{The \xpackage{selinput} package}
% \date{2007/09/09 v1.2}
% \author{Heiko Oberdiek\\\xemail{heiko.oberdiek at googlemail.com}}
%
% \maketitle
%
% \begin{abstract}
% This package selects the input encoding by specifying between
% input characters and their glyph names.
% \end{abstract}
%
% \tableofcontents
%
% \newcommand*{\EM}{\textcolor{blue}}
% \newcommand*{\ExampleText}{^^A
%   Umlauts:\ \EM{\"A\"O\"U\"a\"o\"u\ss}^^A
% }
%
% \section{Documentation}
%
% \subsection{Introduction}
%
% \LaTeX\ supports the direct use of 8-bit characters by means
% of package \xpackage{inputenc}. However you must know
% and specify the encoding, e.g.:
% \begin{quote}
%   \ttfamily
%   |\documentclass{article}|\\
%   |\usepackage[|\EM{latin1}|]{inputenc}|\\
%   |% or \usepackage[|\EM{utf8}|]{inputenc}|\\
%   |% or \usepackage[|\EM{??}|]{inputenc}|\\
%   |\begin{document}|\\
%   |  |\ExampleText\\
%   |\end{document}|
% \end{quote}
%
% If the document is transferred in an environment that
% uses a different encoding, then there are programs that
% convert the input characters. Examples for conversion
% of file \xfile{test.tex}
% from encoding latin1 (ISO-8859-1) to UTF-8:
% \begin{quote}
%   \ttfamily
%   |recode ISO-8859-1..UTF-8 test.tex|\\
%   |recode latin1..utf8 test.tex|\\
%   |iconv --from-code ISO-8859-1|\\
%   \hphantom{iconv}| --to-code UTF-8|\\
%   \hphantom{iconv}| --output testnew.tex|\\
%   \hphantom{iconv}| test.tex|\\
%   |iconv -f latin1 -t utf8 -o testnew.tex test.tex|
% \end{quote}
% However, the encoding name for package \xpackage{inputenc}
% must be changed:
% \begin{quote}
%    |\usepackage[latin1]{inputenc}| $\rightarrow$
%    |\usepackage[utf8]{inputenc}|\kern-4pt\relax
% \end{quote}
% Of course, unless you are using some clever editor
% that knows package \xpackage{inputenc}, recodes
% the file and adjusts the option at the same time.
% But most editors can perhaps recode the file, but
% they let the option untouched.
%
% Therefore package \xpackage{selinput} chooses another way for
% specifying the input encoding. The encoding name is not needed
% at all. Some 8-bit characters are identified by their glyph
% name and the package chooses an appropriate encoding, example:
% \begin{quote}
%   \ttfamily
%   |\documentclass{article}|\\
%   |\usepackage{selinput}|\\
%   |\SelectInputMappings{|\\
%   |  adieresis={|\EM{\"a}|}|,\\
%   |  germandbls={|\EM{\ss}|}|,\\
%   |  Euro={|\EM{\texteuro}|}|,\\
%   |}|\\
%   |\begin{document}|\\
%   |  |\ExampleText\\
%   |\end{document}|
% \end{quote}
%
% \subsection{User interface}
%
% \begin{declcs}{SelectInputEncodingList} \M{encoding list}
% \end{declcs}
% \cs{SelectInputEncodingList} expects a comma separated list of
% encoding names. Example:
% \begin{quote}
%   |\SelectInputEncodingList{utf8,ansinew,mac-roman}|
% \end{quote}
% The encodings of package \xpackage{inputenx} are used as default.
%
% \begin{declcs}{SelectInputMappings} \M{mapping pairs}
% \end{declcs}
% A mapping pair consists of a glyph name and its input
% character:
% \begin{quote}
%   |\SelectInputMappings{|\\
%   |  adieresis={|\EM{\"a}|}|,\\
%   |  germandbls={|\EM{\ss}|}|,\\
%   |  Euro={|\EM{\texteuro}|}|,\\
%   |}|
% \end{quote}
% The supported glyph names can be found in file \xfile{ix-name.def}
% of project \xpackage{inputenx} \cite{inputenx}. The names are
% basically taken from Adobe's glyphlists \cite{adobe:glyphlist,adobe:aglfn}.
% As many pairs are needed as necessary to identify the encoding.
% Example with insufficient pairs:
% \begin{quote}
%   \ttfamily
%   |\SelectInputEncodingSet{latin1,latin9}|\\
%   |\SelectInputMappings{|\\
%   |  adieresis={|\EM{\"a}|}|,\\
%   |  germandbls={|\EM{\ss}|}|,\\
%   |}|\\
%   \ExampleText| and Euro: |\EM{\textcurrency} (wrong)
% \end{quote}
% The first encoding \xoption{latin1} passes the constraints given
% by the mapping pairs. However the Euro symbol is not part of
% the encoding. Thus a mapping pair with the Euro symbol
% solves the problem. In fact the symbol alone already succeeds in selecting
% between \xoption{latin1} and \xoption{latin9}:
% \begin{quote}
%   \ttfamily
%   |\SelectInputEncodingSet{latin1,latin9}|\\
%   |\SelectInputMappings{|\\
%   |  Euro={|\EM{\texteuro}|},|\\
%   |}|\\
%   \ExampleText| and Euro: |\EM{\texteuro}
% \end{quote}
%
% \subsection{Options}
%
% \begin{description}
% \item[\xoption{warning}:]
%   The selected encoding is written
%   by \cs{PackageInfo} into the \xfile{.log} file only.
%   Option \xoption{warning} changes it to \cs{PackageWarning}.
%   Then the selected encoding is shown on the terminal as well.
% \item[\xoption{ucs}:]
%   The encoding file \xfile{utf8x} of package \cs{ucs} requires
%   that the package itself is loaded before.
%   If the package is not loaded, then the option \xoption{ucs}
%   will load package \xpackage{ucs} if the detected encoding is
%   UTF-8 (limited to the preamble, packages cannot be loaded later).
% \item[\xoption{utf8=\dots}:]
%   The option allows to specify other encoding files
%   for UTF-8 than \LaTeX's \xfile{utf8.def}. For example,
%   |utf8=utf-8| will load \xfile{utf-8.def} instead.
% \end{description}
%
% \subsection{Encodings}
%
% Package \xpackage{stringenc} \cite{stringenc}
% is used for testing the encoding. Thus the encoding
% name must be known by this package. Then the found
% encoding is loaded by \cs{inputencoding} by package
% \xpackage{inputenc} or \cs{InputEncoding} if package
% \xpackage{inputenx} is loaded.
%
% The supported encodings are present in the encoding list,
% thus usually the encoding names do not matter.
% If the list is set by \cs{SelectInputEncodingList},
% then you can use the names that work for package
% \xpackage{inputenc} and are known by package \xpackage{stringenc},
% for example: \xoption{latin1}, \xoption{x-iso-8859-1}. Encoding
% file names of package \xpackage{inputenx} are prefixed with \xfile{x-}.
% The prefix can be dropped, if package \xpackage{inputenx} is loaded.
%
% \StopEventually{
% }
%
% \section{Implementation}
%
%    \begin{macrocode}
%<*package>
\NeedsTeXFormat{LaTeX2e}
\ProvidesPackage{selinput}
  [2007/09/09 v1.2 Semi-automatic input encoding detection (HO)]%
%    \end{macrocode}
%
%    \begin{macrocode}
\RequirePackage{inputenc}
\RequirePackage{kvsetkeys}[2006/10/19]
\RequirePackage{stringenc}[2007/06/16]
\RequirePackage{kvoptions}
%    \end{macrocode}
%    \begin{macro}{\SelectInputEncodingList}
%    \begin{macrocode}
\newcommand*{\SelectInputEncodingList}{%
  \let\SIE@EncodingList\@empty
  \kvsetkeys{SelInputEnc}%
}
%    \end{macrocode}
%    \end{macro}
%    \begin{macro}{\SelectInputMappings}
%    \begin{macrocode}
\newcommand*{\SelectInputMappings}[1]{%
  \SIE@LoadNameDefs
  \let\SIE@StringUnicode\@empty
  \let\SIE@StringDest\@empty
  \kvsetkeys{SelInputMap}{#1}%
  \ifx\\SIE@StringUnicode\SIE@StringDest\\%
    \PackageError{selinput}{%
      No mappings specified%
    }\@ehc
  \else
    \EdefUnescapeHex\SIE@StringUnicode\SIE@StringUnicode
    \let\SIE@Encoding\@empty
    \@for\SIE@EncodingTest:=\SIE@EncodingList\do{%
      \ifx\SIE@Encoding\@empty
        \StringEncodingConvertTest\SIE@temp\SIE@StringUnicode
                                  {utf16be}\SIE@EncodingTest{%
          \ifx\SIE@temp\SIE@StringDest
            \let\SIE@Encoding\SIE@EncodingTest
          \fi
        }{}%
      \fi
    }%
    \ifx\SIE@Encoding\@empty
      \StringEncodingConvertTest\SIE@temp\SIE@StringDest
                                {ascii}{utf16be}{%
        \def\SIE@Encoding{ascii}%
        \SIE@Info{selinput}{%
          Matching encoding not found, but input characters%
          \MessageBreak
          are 7-bit (possibly editor replacements).%
          \MessageBreak
          Hence using ascii encoding%
        }%
      }{}%
    \fi
    \ifx\SIE@Encoding\@empty
      \PackageError{selinput}{%
        Cannot find a matching encoding%
      }\@ehd
    \else
      \ifx\SIE@Encoding\SIE@EncodingUTFviii
        \SIE@LoadUnicodePackage
        \ifx\SIE@UseUTFviii\@empty
        \else
          \let\SIE@Encoding\SIE@UseUTFviii
        \fi
      \fi
      \begingroup\expandafter\expandafter\expandafter\endgroup
      \expandafter\ifx\csname InputEncoding\endcsname\relax
        \inputencoding\SIE@Encoding
      \else
        \InputEncoding\SIE@Encoding
      \fi
      \SIE@Info{selinput}{Encoding `\SIE@Encoding' selected}%
    \fi
  \fi
}
%    \end{macrocode}
%    \end{macro}
%    \begin{macro}{\SIE@LoadNameDefs}
%    \begin{macrocode}
\def\SIE@LoadNameDefs{%
  \begingroup
    \endlinechar=\m@ne
    \catcode92=0 % backslash
    \catcode123=1 % left curly brace/beginning of group
    \catcode125=2 % right curly brace/end of group
    \catcode37=14 % percent/comment character
    \@makeother\[%
    \@makeother\]%
    \@makeother\.%
    \@makeother\(%
    \@makeother\)%
    \@makeother\/%
    \@makeother\-%
    \let\InputenxName\SelectInputDefineMapping
    \InputIfFileExists{ix-name.def}{}{%
      \PackageError{selinput}{%
        Missing `ix-name.def' (part of package `inputenx')%
      }\@ehd
    }%
    \global\let\SIE@LoadNameDefs\relax
  \endgroup
}
%    \end{macrocode}
%    \end{macro}
%    \begin{macro}{\SelectInputDefineMapping}
%    \begin{macrocode}
\newcommand*{\SelectInputDefineMapping}[1]{%
  \expandafter\gdef\csname SIE@@#1\endcsname
}
%    \end{macrocode}
%    \end{macro}
%    \begin{macrocode}
\kv@set@family@handler{SelInputMap}{%
  \@onelevel@sanitize\kv@key
  \ifx\kv@value\relax
    \PackageError{selinput}{%
      Missing input character for `\kv@key'%
    }\@ehc
  \else
    \@onelevel@sanitize\kv@value
    \ifx\kv@value\@empty
      \PackageError{selinput}{%
        Input character got lost?\MessageBreak
        Missing input character for `\kv@key'%
      }\@ehc
    \else
      \@ifundefined{SIE@@\kv@key}{%
        \PackageWarning{selinput}{%
          Missing definition for `\kv@key'%
        }%
      }{%
        \edef\SIE@StringDest{%
          \SIE@StringDest
          \kv@value
        }%
        \edef\SIE@StringUnicode{%
          \SIE@StringUnicode
          \csname SIE@@\kv@key\endcsname
        }%
      }%
    \fi
  \fi
}
%    \end{macrocode}
%    \begin{macrocode}
\kv@set@family@handler{SelInputEnc}{%
  \@onelevel@sanitize\kv@key
  \ifx\kv@value\relax
    \ifx\SIE@EncodingList\@empty
      \let\SIE@EncodingList\kv@key
    \else
      \edef\SIE@EncodingList{\SIE@EncodingList,\kv@key}%
    \fi
  \else
    \@onelevel@sanitize\kv@value
    \PackageError{selinput}{%
      Illegal key value pair (\kv@key=\kv@value)\MessagBreak
      in encoding list%
    }\@ehc
  \fi
}
%    \end{macrocode}
%
%    \begin{macro}{\SIE@LoadUnicodePackage}
%    \begin{macrocode}
\def\SIE@LoadUnicodePackage{%
  \@ifpackageloaded\SIE@UnicodePackage{}{%
    \RequirePackage\SIE@UnicodePackage\relax
  }%
  \SIE@PatchUCS
  \global\let\SIE@LoadUnicodePackage\relax
}
\let\SIE@show\show
\def\SIE@PatchUCS{%
  \AtBeginDocument{%
    \expandafter\ifx\csname ver@ucsencs.def\endcsname\relax
    \else
      \let\show\SIE@show
    \fi
  }%
}
\SIE@PatchUCS
%    \end{macrocode}
%    \end{macro}
%    \begin{macrocode}
\AtBeginDocument{%
  \let\SIE@LoadUnicodePackage\relax
}
%    \end{macrocode}
%    \begin{macro}{\SIE@EncodingUTFviii}
%    \begin{macrocode}
\def\SIE@EncodingUTFviii{utf8}
\@onelevel@sanitize\SIE@EncodingUTFviii
%    \end{macrocode}
%    \end{macro}
%    \begin{macro}{\SIE@EncodingUTFviiix}
%    \begin{macrocode}
\def\SIE@EncodingUTFviiix{utf8x}
\@onelevel@sanitize\SIE@EncodingUTFviiix
%    \end{macrocode}
%    \end{macro}
%
%    \begin{macrocode}
\let\SIE@UnicodePackage\@empty
\let\SIE@UseUTFviii\@empty
\let\SIE@Info\PackageInfo
%    \end{macrocode}
%    \begin{macrocode}
\SetupKeyvalOptions{%
  family=SelInput,%
  prefix=SelInput@%
}
\define@key{SelInput}{utf8}{%
  \def\SIE@UseUTFviii{#1}%
  \@onelevel@sanitize\SIE@UseUTFviii
}
\DeclareBoolOption{ucs}
\DeclareVoidOption{warning}{%
  \let\SIE@Info\PackageWarning
}
\ProcessKeyvalOptions{SelInput}
\ifSelInput@ucs
  \def\SIE@UnicodePackage{ucs}%
  \ifx\SIE@UseUTFviii\@empty
    \let\SIE@UseUTFviii\SIE@EncodingUTFviiix
  \fi
\else
  \ifx\SIE@UseUTFviii\@empty
    \@ifpackageloaded{ucs}{%
      \let\SIE@UseUTFviii\SIE@EncodingUTFviiix
    }{%
      \let\SIE@UseUTFviii\SIE@EncodingUTFviii
    }%
  \fi
\fi
%    \end{macrocode}
%
%    \begin{macro}{\SIE@EncodingList}
%    \begin{macrocode}
\edef\SIE@EncodingList{%
  utf8,%
  x-iso-8859-1,%
  x-iso-8859-15,%
  x-cp1252,% ansinew
  x-mac-roman,%
  x-iso-8859-2,%
  x-iso-8859-3,%
  x-iso-8859-4,%
  x-iso-8859-5,%
  x-iso-8859-6,%
  x-iso-8859-7,%
  x-iso-8859-8,%
  x-iso-8859-9,%
  x-iso-8859-10,%
  x-iso-8859-11,%
  x-iso-8859-13,%
  x-iso-8859-14,%
  x-iso-8859-15,%
  x-mac-centeuro,%
  x-mac-cyrillic,%
  x-koi8-r,%
  x-cp1250,%
  x-cp1251,%
  x-cp1257,%
  x-cp437,%
  x-cp850,%
  x-cp852,%
  x-cp855,%
  x-cp858,%
  x-cp865,%
  x-cp866,%
  x-nextstep,%
  x-dec-mcs%
}%
\@onelevel@sanitize\SIE@EncodingList
%    \end{macrocode}
%    \end{macro}
%
%    \begin{macrocode}
%</package>
%    \end{macrocode}
%
% \section{Test}
%
%    \begin{macrocode}
%<*test>
\NeedsTeXFormat{LaTeX2e}
\documentclass{minimal}
\usepackage{textcomp}
\usepackage{qstest}
%    \end{macrocode}
%    \begin{macrocode}
%<*test1|test2|test3>
\makeatletter
\let\BeginDocumentText\@empty
\def\TestEncoding#1#2{%
  \SelectInputMappings{#2}%
  \Expect*{\SIE@Encoding}{#1}%
  \Expect*{\inputencodingname}{#1}%
  \g@addto@macro\BeginDocumentText{%
    \SelectInputMappings{#2}%
    \Expect*{\SIE@Encoding}{#1}%
    \textbf{\SIE@Encoding:} %
    \kvsetkeys{test}{#2}\par
  }%
}
\def\TestKey#1#2{%
  \define@key{test}{#1}{%
    \sbox0{##1}%
    \sbox2{#2}%
    \Expect*{wd:\the\wd0, ht:\the\ht0, dp:\the\dp0}%
           *{wd:\the\wd2, ht:\the\ht2, dp:\the\dp2}%
    [#1=##1] % hash-ok
  }%
}
\RequirePackage{keyval}
\TestKey{adieresis}{\"a}
\TestKey{germandbls}{\ss}
\TestKey{Euro}{\texteuro}
\makeatother
\usepackage[
  warning,%
%<test2>  utf8=utf-8,
%<test3>  ucs,
]{selinput}
%<test1|test3>\inputencoding{ascii}
%<test2>\inputencoding{utf-8}
%<test3>\usepackage{ucs}
\begin{qstest}{preamble}{}
  \TestEncoding{x-iso-8859-15}{%
    adieresis=^^e4,%
    germandbls=^^df,%
    Euro=^^a4,%
  }%
  \TestEncoding{x-cp1252}{%
    adieresis=^^e4,%
    germandbls=^^df,%
    Euro=^^80,%
  }%
%<test1>  \TestEncoding{utf8}{%
%<test2>  \TestEncoding{utf-8}{%
%<test3>  \TestEncoding{utf8x}{%
    adieresis=^^c3^^a4,%
    germandbls=^^c3^^9f,%
%<!test2>    Euro=^^e2^^82^^ac,
  }%
\end{qstest}
%<test3>\let\ifUnicodeOptiongraphics\iffalse
\begin{document}
\begin{qstest}{document}{}
%<test3>\makeatletter
  \BeginDocumentText
\end{qstest}
%</test1|test2|test3>
%    \end{macrocode}
%
%    \begin{macrocode}
%<*test4>
\usepackage[warning,ucs]{selinput}
\SelectInputMappings{%
    adieresis=^^c3^^a4,%
    germandbls=^^c3^^9f,%
    Euro=^^e2^^82^^ac,%
}
\begin{qstest}{encoding}{}
  \Expect*{\inputencodingname}{utf8x}%
\end{qstest}
\begin{document}
  adieresis=^^c3^^a4, %
  germandbls=^^c3^^9f, %
  Euro=^^e2^^82^^ac%
%</test4>
%    \end{macrocode}
%
%    \begin{macrocode}
%<*test5>
\usepackage[warning,ucs]{selinput}
\SelectInputMappings{%
    adieresis={\"a},%
    germandbls={{\ss}},%
    Euro=\texteuro{},%
}
\begin{qstest}{encoding}{}
  \Expect*{\inputencodingname}{ascii}%
\end{qstest}
\begin{document}
  adieresis={\"a}, %
  germandbls={{\ss}}, %
  Euro=\texteuro{}%
%</test5>
%    \end{macrocode}
%
%    \begin{macrocode}
\end{document}
%</test>
%    \end{macrocode}
%
% \section{Installation}
%
% \subsection{Download}
%
% \paragraph{Package.} This package is available on
% CTAN\footnote{\url{ftp://ftp.ctan.org/tex-archive/}}:
% \begin{description}
% \item[\CTAN{macros/latex/contrib/oberdiek/selinput.dtx}] The source file.
% \item[\CTAN{macros/latex/contrib/oberdiek/selinput.pdf}] Documentation.
% \end{description}
%
%
% \paragraph{Bundle.} All the packages of the bundle `oberdiek'
% are also available in a TDS compliant ZIP archive. There
% the packages are already unpacked and the documentation files
% are generated. The files and directories obey the TDS standard.
% \begin{description}
% \item[\CTAN{install/macros/latex/contrib/oberdiek.tds.zip}]
% \end{description}
% \emph{TDS} refers to the standard ``A Directory Structure
% for \TeX\ Files'' (\CTAN{tds/tds.pdf}). Directories
% with \xfile{texmf} in their name are usually organized this way.
%
% \subsection{Bundle installation}
%
% \paragraph{Unpacking.} Unpack the \xfile{oberdiek.tds.zip} in the
% TDS tree (also known as \xfile{texmf} tree) of your choice.
% Example (linux):
% \begin{quote}
%   |unzip oberdiek.tds.zip -d ~/texmf|
% \end{quote}
%
% \paragraph{Script installation.}
% Check the directory \xfile{TDS:scripts/oberdiek/} for
% scripts that need further installation steps.
% Package \xpackage{attachfile2} comes with the Perl script
% \xfile{pdfatfi.pl} that should be installed in such a way
% that it can be called as \texttt{pdfatfi}.
% Example (linux):
% \begin{quote}
%   |chmod +x scripts/oberdiek/pdfatfi.pl|\\
%   |cp scripts/oberdiek/pdfatfi.pl /usr/local/bin/|
% \end{quote}
%
% \subsection{Package installation}
%
% \paragraph{Unpacking.} The \xfile{.dtx} file is a self-extracting
% \docstrip\ archive. The files are extracted by running the
% \xfile{.dtx} through \plainTeX:
% \begin{quote}
%   \verb|tex selinput.dtx|
% \end{quote}
%
% \paragraph{TDS.} Now the different files must be moved into
% the different directories in your installation TDS tree
% (also known as \xfile{texmf} tree):
% \begin{quote}
% \def\t{^^A
% \begin{tabular}{@{}>{\ttfamily}l@{ $\rightarrow$ }>{\ttfamily}l@{}}
%   selinput.sty & tex/latex/oberdiek/selinput.sty\\
%   selinput.pdf & doc/latex/oberdiek/selinput.pdf\\
%   test/selinput-test1.tex & doc/latex/oberdiek/test/selinput-test1.tex\\
%   test/selinput-test2.tex & doc/latex/oberdiek/test/selinput-test2.tex\\
%   test/selinput-test3.tex & doc/latex/oberdiek/test/selinput-test3.tex\\
%   test/selinput-test4.tex & doc/latex/oberdiek/test/selinput-test4.tex\\
%   test/selinput-test5.tex & doc/latex/oberdiek/test/selinput-test5.tex\\
%   selinput.dtx & source/latex/oberdiek/selinput.dtx\\
% \end{tabular}^^A
% }^^A
% \sbox0{\t}^^A
% \ifdim\wd0>\linewidth
%   \begingroup
%     \advance\linewidth by\leftmargin
%     \advance\linewidth by\rightmargin
%   \edef\x{\endgroup
%     \def\noexpand\lw{\the\linewidth}^^A
%   }\x
%   \def\lwbox{^^A
%     \leavevmode
%     \hbox to \linewidth{^^A
%       \kern-\leftmargin\relax
%       \hss
%       \usebox0
%       \hss
%       \kern-\rightmargin\relax
%     }^^A
%   }^^A
%   \ifdim\wd0>\lw
%     \sbox0{\small\t}^^A
%     \ifdim\wd0>\linewidth
%       \ifdim\wd0>\lw
%         \sbox0{\footnotesize\t}^^A
%         \ifdim\wd0>\linewidth
%           \ifdim\wd0>\lw
%             \sbox0{\scriptsize\t}^^A
%             \ifdim\wd0>\linewidth
%               \ifdim\wd0>\lw
%                 \sbox0{\tiny\t}^^A
%                 \ifdim\wd0>\linewidth
%                   \lwbox
%                 \else
%                   \usebox0
%                 \fi
%               \else
%                 \lwbox
%               \fi
%             \else
%               \usebox0
%             \fi
%           \else
%             \lwbox
%           \fi
%         \else
%           \usebox0
%         \fi
%       \else
%         \lwbox
%       \fi
%     \else
%       \usebox0
%     \fi
%   \else
%     \lwbox
%   \fi
% \else
%   \usebox0
% \fi
% \end{quote}
% If you have a \xfile{docstrip.cfg} that configures and enables \docstrip's
% TDS installing feature, then some files can already be in the right
% place, see the documentation of \docstrip.
%
% \subsection{Refresh file name databases}
%
% If your \TeX~distribution
% (\teTeX, \mikTeX, \dots) relies on file name databases, you must refresh
% these. For example, \teTeX\ users run \verb|texhash| or
% \verb|mktexlsr|.
%
% \subsection{Some details for the interested}
%
% \paragraph{Attached source.}
%
% The PDF documentation on CTAN also includes the
% \xfile{.dtx} source file. It can be extracted by
% AcrobatReader 6 or higher. Another option is \textsf{pdftk},
% e.g. unpack the file into the current directory:
% \begin{quote}
%   \verb|pdftk selinput.pdf unpack_files output .|
% \end{quote}
%
% \paragraph{Unpacking with \LaTeX.}
% The \xfile{.dtx} chooses its action depending on the format:
% \begin{description}
% \item[\plainTeX:] Run \docstrip\ and extract the files.
% \item[\LaTeX:] Generate the documentation.
% \end{description}
% If you insist on using \LaTeX\ for \docstrip\ (really,
% \docstrip\ does not need \LaTeX), then inform the autodetect routine
% about your intention:
% \begin{quote}
%   \verb|latex \let\install=y% \iffalse meta-comment
%
% File: selinput.dtx
% Version: 2007/09/09 v1.2
% Info: Semi-automatic input encoding detection
%
% Copyright (C) 2007 by
%    Heiko Oberdiek <heiko.oberdiek at googlemail.com>
%
% This work may be distributed and/or modified under the
% conditions of the LaTeX Project Public License, either
% version 1.3c of this license or (at your option) any later
% version. This version of this license is in
%    http://www.latex-project.org/lppl/lppl-1-3c.txt
% and the latest version of this license is in
%    http://www.latex-project.org/lppl.txt
% and version 1.3 or later is part of all distributions of
% LaTeX version 2005/12/01 or later.
%
% This work has the LPPL maintenance status "maintained".
%
% This Current Maintainer of this work is Heiko Oberdiek.
%
% This work consists of the main source file selinput.dtx
% and the derived files
%    selinput.sty, selinput.pdf, selinput.ins, selinput.drv,
%    selinput-test1.tex, selinput-test2.tex, selinput-test3.tex,
%    selinput-test4.tex, selinput-test5.tex.
%
% Distribution:
%    CTAN:macros/latex/contrib/oberdiek/selinput.dtx
%    CTAN:macros/latex/contrib/oberdiek/selinput.pdf
%
% Unpacking:
%    (a) If selinput.ins is present:
%           tex selinput.ins
%    (b) Without selinput.ins:
%           tex selinput.dtx
%    (c) If you insist on using LaTeX
%           latex \let\install=y\input{selinput.dtx}
%        (quote the arguments according to the demands of your shell)
%
% Documentation:
%    (a) If selinput.drv is present:
%           latex selinput.drv
%    (b) Without selinput.drv:
%           latex selinput.dtx; ...
%    The class ltxdoc loads the configuration file ltxdoc.cfg
%    if available. Here you can specify further options, e.g.
%    use A4 as paper format:
%       \PassOptionsToClass{a4paper}{article}
%
%    Programm calls to get the documentation (example):
%       pdflatex selinput.dtx
%       makeindex -s gind.ist selinput.idx
%       pdflatex selinput.dtx
%       makeindex -s gind.ist selinput.idx
%       pdflatex selinput.dtx
%
% Installation:
%    TDS:tex/latex/oberdiek/selinput.sty
%    TDS:doc/latex/oberdiek/selinput.pdf
%    TDS:doc/latex/oberdiek/test/selinput-test1.tex
%    TDS:doc/latex/oberdiek/test/selinput-test2.tex
%    TDS:doc/latex/oberdiek/test/selinput-test3.tex
%    TDS:doc/latex/oberdiek/test/selinput-test4.tex
%    TDS:doc/latex/oberdiek/test/selinput-test5.tex
%    TDS:source/latex/oberdiek/selinput.dtx
%
%<*ignore>
\begingroup
  \catcode123=1 %
  \catcode125=2 %
  \def\x{LaTeX2e}%
\expandafter\endgroup
\ifcase 0\ifx\install y1\fi\expandafter
         \ifx\csname processbatchFile\endcsname\relax\else1\fi
         \ifx\fmtname\x\else 1\fi\relax
\else\csname fi\endcsname
%</ignore>
%<*install>
\input docstrip.tex
\Msg{************************************************************************}
\Msg{* Installation}
\Msg{* Package: selinput 2007/09/09 v1.2 Semi-automatic input encoding detection (HO)}
\Msg{************************************************************************}

\keepsilent
\askforoverwritefalse

\let\MetaPrefix\relax
\preamble

This is a generated file.

Project: selinput
Version: 2007/09/09 v1.2

Copyright (C) 2007 by
   Heiko Oberdiek <heiko.oberdiek at googlemail.com>

This work may be distributed and/or modified under the
conditions of the LaTeX Project Public License, either
version 1.3c of this license or (at your option) any later
version. This version of this license is in
   http://www.latex-project.org/lppl/lppl-1-3c.txt
and the latest version of this license is in
   http://www.latex-project.org/lppl.txt
and version 1.3 or later is part of all distributions of
LaTeX version 2005/12/01 or later.

This work has the LPPL maintenance status "maintained".

This Current Maintainer of this work is Heiko Oberdiek.

This work consists of the main source file selinput.dtx
and the derived files
   selinput.sty, selinput.pdf, selinput.ins, selinput.drv,
   selinput-test1.tex, selinput-test2.tex, selinput-test3.tex,
   selinput-test4.tex, selinput-test5.tex.

\endpreamble
\let\MetaPrefix\DoubleperCent

\generate{%
  \file{selinput.ins}{\from{selinput.dtx}{install}}%
  \file{selinput.drv}{\from{selinput.dtx}{driver}}%
  \usedir{tex/latex/oberdiek}%
  \file{selinput.sty}{\from{selinput.dtx}{package}}%
  \usedir{doc/latex/oberdiek/test}%
  \file{selinput-test1.tex}{\from{selinput.dtx}{test,test1}}%
  \file{selinput-test2.tex}{\from{selinput.dtx}{test,test2}}%
  \file{selinput-test3.tex}{\from{selinput.dtx}{test,test3}}%
  \file{selinput-test4.tex}{\from{selinput.dtx}{test,test4}}%
  \file{selinput-test5.tex}{\from{selinput.dtx}{test,test5}}%
  \nopreamble
  \nopostamble
  \usedir{source/latex/oberdiek/catalogue}%
  \file{selinput.xml}{\from{selinput.dtx}{catalogue}}%
}

\catcode32=13\relax% active space
\let =\space%
\Msg{************************************************************************}
\Msg{*}
\Msg{* To finish the installation you have to move the following}
\Msg{* file into a directory searched by TeX:}
\Msg{*}
\Msg{*     selinput.sty}
\Msg{*}
\Msg{* To produce the documentation run the file `selinput.drv'}
\Msg{* through LaTeX.}
\Msg{*}
\Msg{* Happy TeXing!}
\Msg{*}
\Msg{************************************************************************}

\endbatchfile
%</install>
%<*ignore>
\fi
%</ignore>
%<*driver>
\NeedsTeXFormat{LaTeX2e}
\ProvidesFile{selinput.drv}%
  [2007/09/09 v1.2 Semi-automatic input encoding detection (HO)]%
\documentclass{ltxdoc}
\usepackage[T1]{fontenc}
\usepackage{textcomp}
\usepackage{lmodern}
\usepackage{holtxdoc}[2011/11/22]
\usepackage{color}
\begin{document}
  \DocInput{selinput.dtx}%
\end{document}
%</driver>
% \fi
%
% \CheckSum{389}
%
% \CharacterTable
%  {Upper-case    \A\B\C\D\E\F\G\H\I\J\K\L\M\N\O\P\Q\R\S\T\U\V\W\X\Y\Z
%   Lower-case    \a\b\c\d\e\f\g\h\i\j\k\l\m\n\o\p\q\r\s\t\u\v\w\x\y\z
%   Digits        \0\1\2\3\4\5\6\7\8\9
%   Exclamation   \!     Double quote  \"     Hash (number) \#
%   Dollar        \$     Percent       \%     Ampersand     \&
%   Acute accent  \'     Left paren    \(     Right paren   \)
%   Asterisk      \*     Plus          \+     Comma         \,
%   Minus         \-     Point         \.     Solidus       \/
%   Colon         \:     Semicolon     \;     Less than     \<
%   Equals        \=     Greater than  \>     Question mark \?
%   Commercial at \@     Left bracket  \[     Backslash     \\
%   Right bracket \]     Circumflex    \^     Underscore    \_
%   Grave accent  \`     Left brace    \{     Vertical bar  \|
%   Right brace   \}     Tilde         \~}
%
% \GetFileInfo{selinput.drv}
%
% \title{The \xpackage{selinput} package}
% \date{2007/09/09 v1.2}
% \author{Heiko Oberdiek\\\xemail{heiko.oberdiek at googlemail.com}}
%
% \maketitle
%
% \begin{abstract}
% This package selects the input encoding by specifying between
% input characters and their glyph names.
% \end{abstract}
%
% \tableofcontents
%
% \newcommand*{\EM}{\textcolor{blue}}
% \newcommand*{\ExampleText}{^^A
%   Umlauts:\ \EM{\"A\"O\"U\"a\"o\"u\ss}^^A
% }
%
% \section{Documentation}
%
% \subsection{Introduction}
%
% \LaTeX\ supports the direct use of 8-bit characters by means
% of package \xpackage{inputenc}. However you must know
% and specify the encoding, e.g.:
% \begin{quote}
%   \ttfamily
%   |\documentclass{article}|\\
%   |\usepackage[|\EM{latin1}|]{inputenc}|\\
%   |% or \usepackage[|\EM{utf8}|]{inputenc}|\\
%   |% or \usepackage[|\EM{??}|]{inputenc}|\\
%   |\begin{document}|\\
%   |  |\ExampleText\\
%   |\end{document}|
% \end{quote}
%
% If the document is transferred in an environment that
% uses a different encoding, then there are programs that
% convert the input characters. Examples for conversion
% of file \xfile{test.tex}
% from encoding latin1 (ISO-8859-1) to UTF-8:
% \begin{quote}
%   \ttfamily
%   |recode ISO-8859-1..UTF-8 test.tex|\\
%   |recode latin1..utf8 test.tex|\\
%   |iconv --from-code ISO-8859-1|\\
%   \hphantom{iconv}| --to-code UTF-8|\\
%   \hphantom{iconv}| --output testnew.tex|\\
%   \hphantom{iconv}| test.tex|\\
%   |iconv -f latin1 -t utf8 -o testnew.tex test.tex|
% \end{quote}
% However, the encoding name for package \xpackage{inputenc}
% must be changed:
% \begin{quote}
%    |\usepackage[latin1]{inputenc}| $\rightarrow$
%    |\usepackage[utf8]{inputenc}|\kern-4pt\relax
% \end{quote}
% Of course, unless you are using some clever editor
% that knows package \xpackage{inputenc}, recodes
% the file and adjusts the option at the same time.
% But most editors can perhaps recode the file, but
% they let the option untouched.
%
% Therefore package \xpackage{selinput} chooses another way for
% specifying the input encoding. The encoding name is not needed
% at all. Some 8-bit characters are identified by their glyph
% name and the package chooses an appropriate encoding, example:
% \begin{quote}
%   \ttfamily
%   |\documentclass{article}|\\
%   |\usepackage{selinput}|\\
%   |\SelectInputMappings{|\\
%   |  adieresis={|\EM{\"a}|}|,\\
%   |  germandbls={|\EM{\ss}|}|,\\
%   |  Euro={|\EM{\texteuro}|}|,\\
%   |}|\\
%   |\begin{document}|\\
%   |  |\ExampleText\\
%   |\end{document}|
% \end{quote}
%
% \subsection{User interface}
%
% \begin{declcs}{SelectInputEncodingList} \M{encoding list}
% \end{declcs}
% \cs{SelectInputEncodingList} expects a comma separated list of
% encoding names. Example:
% \begin{quote}
%   |\SelectInputEncodingList{utf8,ansinew,mac-roman}|
% \end{quote}
% The encodings of package \xpackage{inputenx} are used as default.
%
% \begin{declcs}{SelectInputMappings} \M{mapping pairs}
% \end{declcs}
% A mapping pair consists of a glyph name and its input
% character:
% \begin{quote}
%   |\SelectInputMappings{|\\
%   |  adieresis={|\EM{\"a}|}|,\\
%   |  germandbls={|\EM{\ss}|}|,\\
%   |  Euro={|\EM{\texteuro}|}|,\\
%   |}|
% \end{quote}
% The supported glyph names can be found in file \xfile{ix-name.def}
% of project \xpackage{inputenx} \cite{inputenx}. The names are
% basically taken from Adobe's glyphlists \cite{adobe:glyphlist,adobe:aglfn}.
% As many pairs are needed as necessary to identify the encoding.
% Example with insufficient pairs:
% \begin{quote}
%   \ttfamily
%   |\SelectInputEncodingSet{latin1,latin9}|\\
%   |\SelectInputMappings{|\\
%   |  adieresis={|\EM{\"a}|}|,\\
%   |  germandbls={|\EM{\ss}|}|,\\
%   |}|\\
%   \ExampleText| and Euro: |\EM{\textcurrency} (wrong)
% \end{quote}
% The first encoding \xoption{latin1} passes the constraints given
% by the mapping pairs. However the Euro symbol is not part of
% the encoding. Thus a mapping pair with the Euro symbol
% solves the problem. In fact the symbol alone already succeeds in selecting
% between \xoption{latin1} and \xoption{latin9}:
% \begin{quote}
%   \ttfamily
%   |\SelectInputEncodingSet{latin1,latin9}|\\
%   |\SelectInputMappings{|\\
%   |  Euro={|\EM{\texteuro}|},|\\
%   |}|\\
%   \ExampleText| and Euro: |\EM{\texteuro}
% \end{quote}
%
% \subsection{Options}
%
% \begin{description}
% \item[\xoption{warning}:]
%   The selected encoding is written
%   by \cs{PackageInfo} into the \xfile{.log} file only.
%   Option \xoption{warning} changes it to \cs{PackageWarning}.
%   Then the selected encoding is shown on the terminal as well.
% \item[\xoption{ucs}:]
%   The encoding file \xfile{utf8x} of package \cs{ucs} requires
%   that the package itself is loaded before.
%   If the package is not loaded, then the option \xoption{ucs}
%   will load package \xpackage{ucs} if the detected encoding is
%   UTF-8 (limited to the preamble, packages cannot be loaded later).
% \item[\xoption{utf8=\dots}:]
%   The option allows to specify other encoding files
%   for UTF-8 than \LaTeX's \xfile{utf8.def}. For example,
%   |utf8=utf-8| will load \xfile{utf-8.def} instead.
% \end{description}
%
% \subsection{Encodings}
%
% Package \xpackage{stringenc} \cite{stringenc}
% is used for testing the encoding. Thus the encoding
% name must be known by this package. Then the found
% encoding is loaded by \cs{inputencoding} by package
% \xpackage{inputenc} or \cs{InputEncoding} if package
% \xpackage{inputenx} is loaded.
%
% The supported encodings are present in the encoding list,
% thus usually the encoding names do not matter.
% If the list is set by \cs{SelectInputEncodingList},
% then you can use the names that work for package
% \xpackage{inputenc} and are known by package \xpackage{stringenc},
% for example: \xoption{latin1}, \xoption{x-iso-8859-1}. Encoding
% file names of package \xpackage{inputenx} are prefixed with \xfile{x-}.
% The prefix can be dropped, if package \xpackage{inputenx} is loaded.
%
% \StopEventually{
% }
%
% \section{Implementation}
%
%    \begin{macrocode}
%<*package>
\NeedsTeXFormat{LaTeX2e}
\ProvidesPackage{selinput}
  [2007/09/09 v1.2 Semi-automatic input encoding detection (HO)]%
%    \end{macrocode}
%
%    \begin{macrocode}
\RequirePackage{inputenc}
\RequirePackage{kvsetkeys}[2006/10/19]
\RequirePackage{stringenc}[2007/06/16]
\RequirePackage{kvoptions}
%    \end{macrocode}
%    \begin{macro}{\SelectInputEncodingList}
%    \begin{macrocode}
\newcommand*{\SelectInputEncodingList}{%
  \let\SIE@EncodingList\@empty
  \kvsetkeys{SelInputEnc}%
}
%    \end{macrocode}
%    \end{macro}
%    \begin{macro}{\SelectInputMappings}
%    \begin{macrocode}
\newcommand*{\SelectInputMappings}[1]{%
  \SIE@LoadNameDefs
  \let\SIE@StringUnicode\@empty
  \let\SIE@StringDest\@empty
  \kvsetkeys{SelInputMap}{#1}%
  \ifx\\SIE@StringUnicode\SIE@StringDest\\%
    \PackageError{selinput}{%
      No mappings specified%
    }\@ehc
  \else
    \EdefUnescapeHex\SIE@StringUnicode\SIE@StringUnicode
    \let\SIE@Encoding\@empty
    \@for\SIE@EncodingTest:=\SIE@EncodingList\do{%
      \ifx\SIE@Encoding\@empty
        \StringEncodingConvertTest\SIE@temp\SIE@StringUnicode
                                  {utf16be}\SIE@EncodingTest{%
          \ifx\SIE@temp\SIE@StringDest
            \let\SIE@Encoding\SIE@EncodingTest
          \fi
        }{}%
      \fi
    }%
    \ifx\SIE@Encoding\@empty
      \StringEncodingConvertTest\SIE@temp\SIE@StringDest
                                {ascii}{utf16be}{%
        \def\SIE@Encoding{ascii}%
        \SIE@Info{selinput}{%
          Matching encoding not found, but input characters%
          \MessageBreak
          are 7-bit (possibly editor replacements).%
          \MessageBreak
          Hence using ascii encoding%
        }%
      }{}%
    \fi
    \ifx\SIE@Encoding\@empty
      \PackageError{selinput}{%
        Cannot find a matching encoding%
      }\@ehd
    \else
      \ifx\SIE@Encoding\SIE@EncodingUTFviii
        \SIE@LoadUnicodePackage
        \ifx\SIE@UseUTFviii\@empty
        \else
          \let\SIE@Encoding\SIE@UseUTFviii
        \fi
      \fi
      \begingroup\expandafter\expandafter\expandafter\endgroup
      \expandafter\ifx\csname InputEncoding\endcsname\relax
        \inputencoding\SIE@Encoding
      \else
        \InputEncoding\SIE@Encoding
      \fi
      \SIE@Info{selinput}{Encoding `\SIE@Encoding' selected}%
    \fi
  \fi
}
%    \end{macrocode}
%    \end{macro}
%    \begin{macro}{\SIE@LoadNameDefs}
%    \begin{macrocode}
\def\SIE@LoadNameDefs{%
  \begingroup
    \endlinechar=\m@ne
    \catcode92=0 % backslash
    \catcode123=1 % left curly brace/beginning of group
    \catcode125=2 % right curly brace/end of group
    \catcode37=14 % percent/comment character
    \@makeother\[%
    \@makeother\]%
    \@makeother\.%
    \@makeother\(%
    \@makeother\)%
    \@makeother\/%
    \@makeother\-%
    \let\InputenxName\SelectInputDefineMapping
    \InputIfFileExists{ix-name.def}{}{%
      \PackageError{selinput}{%
        Missing `ix-name.def' (part of package `inputenx')%
      }\@ehd
    }%
    \global\let\SIE@LoadNameDefs\relax
  \endgroup
}
%    \end{macrocode}
%    \end{macro}
%    \begin{macro}{\SelectInputDefineMapping}
%    \begin{macrocode}
\newcommand*{\SelectInputDefineMapping}[1]{%
  \expandafter\gdef\csname SIE@@#1\endcsname
}
%    \end{macrocode}
%    \end{macro}
%    \begin{macrocode}
\kv@set@family@handler{SelInputMap}{%
  \@onelevel@sanitize\kv@key
  \ifx\kv@value\relax
    \PackageError{selinput}{%
      Missing input character for `\kv@key'%
    }\@ehc
  \else
    \@onelevel@sanitize\kv@value
    \ifx\kv@value\@empty
      \PackageError{selinput}{%
        Input character got lost?\MessageBreak
        Missing input character for `\kv@key'%
      }\@ehc
    \else
      \@ifundefined{SIE@@\kv@key}{%
        \PackageWarning{selinput}{%
          Missing definition for `\kv@key'%
        }%
      }{%
        \edef\SIE@StringDest{%
          \SIE@StringDest
          \kv@value
        }%
        \edef\SIE@StringUnicode{%
          \SIE@StringUnicode
          \csname SIE@@\kv@key\endcsname
        }%
      }%
    \fi
  \fi
}
%    \end{macrocode}
%    \begin{macrocode}
\kv@set@family@handler{SelInputEnc}{%
  \@onelevel@sanitize\kv@key
  \ifx\kv@value\relax
    \ifx\SIE@EncodingList\@empty
      \let\SIE@EncodingList\kv@key
    \else
      \edef\SIE@EncodingList{\SIE@EncodingList,\kv@key}%
    \fi
  \else
    \@onelevel@sanitize\kv@value
    \PackageError{selinput}{%
      Illegal key value pair (\kv@key=\kv@value)\MessagBreak
      in encoding list%
    }\@ehc
  \fi
}
%    \end{macrocode}
%
%    \begin{macro}{\SIE@LoadUnicodePackage}
%    \begin{macrocode}
\def\SIE@LoadUnicodePackage{%
  \@ifpackageloaded\SIE@UnicodePackage{}{%
    \RequirePackage\SIE@UnicodePackage\relax
  }%
  \SIE@PatchUCS
  \global\let\SIE@LoadUnicodePackage\relax
}
\let\SIE@show\show
\def\SIE@PatchUCS{%
  \AtBeginDocument{%
    \expandafter\ifx\csname ver@ucsencs.def\endcsname\relax
    \else
      \let\show\SIE@show
    \fi
  }%
}
\SIE@PatchUCS
%    \end{macrocode}
%    \end{macro}
%    \begin{macrocode}
\AtBeginDocument{%
  \let\SIE@LoadUnicodePackage\relax
}
%    \end{macrocode}
%    \begin{macro}{\SIE@EncodingUTFviii}
%    \begin{macrocode}
\def\SIE@EncodingUTFviii{utf8}
\@onelevel@sanitize\SIE@EncodingUTFviii
%    \end{macrocode}
%    \end{macro}
%    \begin{macro}{\SIE@EncodingUTFviiix}
%    \begin{macrocode}
\def\SIE@EncodingUTFviiix{utf8x}
\@onelevel@sanitize\SIE@EncodingUTFviiix
%    \end{macrocode}
%    \end{macro}
%
%    \begin{macrocode}
\let\SIE@UnicodePackage\@empty
\let\SIE@UseUTFviii\@empty
\let\SIE@Info\PackageInfo
%    \end{macrocode}
%    \begin{macrocode}
\SetupKeyvalOptions{%
  family=SelInput,%
  prefix=SelInput@%
}
\define@key{SelInput}{utf8}{%
  \def\SIE@UseUTFviii{#1}%
  \@onelevel@sanitize\SIE@UseUTFviii
}
\DeclareBoolOption{ucs}
\DeclareVoidOption{warning}{%
  \let\SIE@Info\PackageWarning
}
\ProcessKeyvalOptions{SelInput}
\ifSelInput@ucs
  \def\SIE@UnicodePackage{ucs}%
  \ifx\SIE@UseUTFviii\@empty
    \let\SIE@UseUTFviii\SIE@EncodingUTFviiix
  \fi
\else
  \ifx\SIE@UseUTFviii\@empty
    \@ifpackageloaded{ucs}{%
      \let\SIE@UseUTFviii\SIE@EncodingUTFviiix
    }{%
      \let\SIE@UseUTFviii\SIE@EncodingUTFviii
    }%
  \fi
\fi
%    \end{macrocode}
%
%    \begin{macro}{\SIE@EncodingList}
%    \begin{macrocode}
\edef\SIE@EncodingList{%
  utf8,%
  x-iso-8859-1,%
  x-iso-8859-15,%
  x-cp1252,% ansinew
  x-mac-roman,%
  x-iso-8859-2,%
  x-iso-8859-3,%
  x-iso-8859-4,%
  x-iso-8859-5,%
  x-iso-8859-6,%
  x-iso-8859-7,%
  x-iso-8859-8,%
  x-iso-8859-9,%
  x-iso-8859-10,%
  x-iso-8859-11,%
  x-iso-8859-13,%
  x-iso-8859-14,%
  x-iso-8859-15,%
  x-mac-centeuro,%
  x-mac-cyrillic,%
  x-koi8-r,%
  x-cp1250,%
  x-cp1251,%
  x-cp1257,%
  x-cp437,%
  x-cp850,%
  x-cp852,%
  x-cp855,%
  x-cp858,%
  x-cp865,%
  x-cp866,%
  x-nextstep,%
  x-dec-mcs%
}%
\@onelevel@sanitize\SIE@EncodingList
%    \end{macrocode}
%    \end{macro}
%
%    \begin{macrocode}
%</package>
%    \end{macrocode}
%
% \section{Test}
%
%    \begin{macrocode}
%<*test>
\NeedsTeXFormat{LaTeX2e}
\documentclass{minimal}
\usepackage{textcomp}
\usepackage{qstest}
%    \end{macrocode}
%    \begin{macrocode}
%<*test1|test2|test3>
\makeatletter
\let\BeginDocumentText\@empty
\def\TestEncoding#1#2{%
  \SelectInputMappings{#2}%
  \Expect*{\SIE@Encoding}{#1}%
  \Expect*{\inputencodingname}{#1}%
  \g@addto@macro\BeginDocumentText{%
    \SelectInputMappings{#2}%
    \Expect*{\SIE@Encoding}{#1}%
    \textbf{\SIE@Encoding:} %
    \kvsetkeys{test}{#2}\par
  }%
}
\def\TestKey#1#2{%
  \define@key{test}{#1}{%
    \sbox0{##1}%
    \sbox2{#2}%
    \Expect*{wd:\the\wd0, ht:\the\ht0, dp:\the\dp0}%
           *{wd:\the\wd2, ht:\the\ht2, dp:\the\dp2}%
    [#1=##1] % hash-ok
  }%
}
\RequirePackage{keyval}
\TestKey{adieresis}{\"a}
\TestKey{germandbls}{\ss}
\TestKey{Euro}{\texteuro}
\makeatother
\usepackage[
  warning,%
%<test2>  utf8=utf-8,
%<test3>  ucs,
]{selinput}
%<test1|test3>\inputencoding{ascii}
%<test2>\inputencoding{utf-8}
%<test3>\usepackage{ucs}
\begin{qstest}{preamble}{}
  \TestEncoding{x-iso-8859-15}{%
    adieresis=^^e4,%
    germandbls=^^df,%
    Euro=^^a4,%
  }%
  \TestEncoding{x-cp1252}{%
    adieresis=^^e4,%
    germandbls=^^df,%
    Euro=^^80,%
  }%
%<test1>  \TestEncoding{utf8}{%
%<test2>  \TestEncoding{utf-8}{%
%<test3>  \TestEncoding{utf8x}{%
    adieresis=^^c3^^a4,%
    germandbls=^^c3^^9f,%
%<!test2>    Euro=^^e2^^82^^ac,
  }%
\end{qstest}
%<test3>\let\ifUnicodeOptiongraphics\iffalse
\begin{document}
\begin{qstest}{document}{}
%<test3>\makeatletter
  \BeginDocumentText
\end{qstest}
%</test1|test2|test3>
%    \end{macrocode}
%
%    \begin{macrocode}
%<*test4>
\usepackage[warning,ucs]{selinput}
\SelectInputMappings{%
    adieresis=^^c3^^a4,%
    germandbls=^^c3^^9f,%
    Euro=^^e2^^82^^ac,%
}
\begin{qstest}{encoding}{}
  \Expect*{\inputencodingname}{utf8x}%
\end{qstest}
\begin{document}
  adieresis=^^c3^^a4, %
  germandbls=^^c3^^9f, %
  Euro=^^e2^^82^^ac%
%</test4>
%    \end{macrocode}
%
%    \begin{macrocode}
%<*test5>
\usepackage[warning,ucs]{selinput}
\SelectInputMappings{%
    adieresis={\"a},%
    germandbls={{\ss}},%
    Euro=\texteuro{},%
}
\begin{qstest}{encoding}{}
  \Expect*{\inputencodingname}{ascii}%
\end{qstest}
\begin{document}
  adieresis={\"a}, %
  germandbls={{\ss}}, %
  Euro=\texteuro{}%
%</test5>
%    \end{macrocode}
%
%    \begin{macrocode}
\end{document}
%</test>
%    \end{macrocode}
%
% \section{Installation}
%
% \subsection{Download}
%
% \paragraph{Package.} This package is available on
% CTAN\footnote{\url{ftp://ftp.ctan.org/tex-archive/}}:
% \begin{description}
% \item[\CTAN{macros/latex/contrib/oberdiek/selinput.dtx}] The source file.
% \item[\CTAN{macros/latex/contrib/oberdiek/selinput.pdf}] Documentation.
% \end{description}
%
%
% \paragraph{Bundle.} All the packages of the bundle `oberdiek'
% are also available in a TDS compliant ZIP archive. There
% the packages are already unpacked and the documentation files
% are generated. The files and directories obey the TDS standard.
% \begin{description}
% \item[\CTAN{install/macros/latex/contrib/oberdiek.tds.zip}]
% \end{description}
% \emph{TDS} refers to the standard ``A Directory Structure
% for \TeX\ Files'' (\CTAN{tds/tds.pdf}). Directories
% with \xfile{texmf} in their name are usually organized this way.
%
% \subsection{Bundle installation}
%
% \paragraph{Unpacking.} Unpack the \xfile{oberdiek.tds.zip} in the
% TDS tree (also known as \xfile{texmf} tree) of your choice.
% Example (linux):
% \begin{quote}
%   |unzip oberdiek.tds.zip -d ~/texmf|
% \end{quote}
%
% \paragraph{Script installation.}
% Check the directory \xfile{TDS:scripts/oberdiek/} for
% scripts that need further installation steps.
% Package \xpackage{attachfile2} comes with the Perl script
% \xfile{pdfatfi.pl} that should be installed in such a way
% that it can be called as \texttt{pdfatfi}.
% Example (linux):
% \begin{quote}
%   |chmod +x scripts/oberdiek/pdfatfi.pl|\\
%   |cp scripts/oberdiek/pdfatfi.pl /usr/local/bin/|
% \end{quote}
%
% \subsection{Package installation}
%
% \paragraph{Unpacking.} The \xfile{.dtx} file is a self-extracting
% \docstrip\ archive. The files are extracted by running the
% \xfile{.dtx} through \plainTeX:
% \begin{quote}
%   \verb|tex selinput.dtx|
% \end{quote}
%
% \paragraph{TDS.} Now the different files must be moved into
% the different directories in your installation TDS tree
% (also known as \xfile{texmf} tree):
% \begin{quote}
% \def\t{^^A
% \begin{tabular}{@{}>{\ttfamily}l@{ $\rightarrow$ }>{\ttfamily}l@{}}
%   selinput.sty & tex/latex/oberdiek/selinput.sty\\
%   selinput.pdf & doc/latex/oberdiek/selinput.pdf\\
%   test/selinput-test1.tex & doc/latex/oberdiek/test/selinput-test1.tex\\
%   test/selinput-test2.tex & doc/latex/oberdiek/test/selinput-test2.tex\\
%   test/selinput-test3.tex & doc/latex/oberdiek/test/selinput-test3.tex\\
%   test/selinput-test4.tex & doc/latex/oberdiek/test/selinput-test4.tex\\
%   test/selinput-test5.tex & doc/latex/oberdiek/test/selinput-test5.tex\\
%   selinput.dtx & source/latex/oberdiek/selinput.dtx\\
% \end{tabular}^^A
% }^^A
% \sbox0{\t}^^A
% \ifdim\wd0>\linewidth
%   \begingroup
%     \advance\linewidth by\leftmargin
%     \advance\linewidth by\rightmargin
%   \edef\x{\endgroup
%     \def\noexpand\lw{\the\linewidth}^^A
%   }\x
%   \def\lwbox{^^A
%     \leavevmode
%     \hbox to \linewidth{^^A
%       \kern-\leftmargin\relax
%       \hss
%       \usebox0
%       \hss
%       \kern-\rightmargin\relax
%     }^^A
%   }^^A
%   \ifdim\wd0>\lw
%     \sbox0{\small\t}^^A
%     \ifdim\wd0>\linewidth
%       \ifdim\wd0>\lw
%         \sbox0{\footnotesize\t}^^A
%         \ifdim\wd0>\linewidth
%           \ifdim\wd0>\lw
%             \sbox0{\scriptsize\t}^^A
%             \ifdim\wd0>\linewidth
%               \ifdim\wd0>\lw
%                 \sbox0{\tiny\t}^^A
%                 \ifdim\wd0>\linewidth
%                   \lwbox
%                 \else
%                   \usebox0
%                 \fi
%               \else
%                 \lwbox
%               \fi
%             \else
%               \usebox0
%             \fi
%           \else
%             \lwbox
%           \fi
%         \else
%           \usebox0
%         \fi
%       \else
%         \lwbox
%       \fi
%     \else
%       \usebox0
%     \fi
%   \else
%     \lwbox
%   \fi
% \else
%   \usebox0
% \fi
% \end{quote}
% If you have a \xfile{docstrip.cfg} that configures and enables \docstrip's
% TDS installing feature, then some files can already be in the right
% place, see the documentation of \docstrip.
%
% \subsection{Refresh file name databases}
%
% If your \TeX~distribution
% (\teTeX, \mikTeX, \dots) relies on file name databases, you must refresh
% these. For example, \teTeX\ users run \verb|texhash| or
% \verb|mktexlsr|.
%
% \subsection{Some details for the interested}
%
% \paragraph{Attached source.}
%
% The PDF documentation on CTAN also includes the
% \xfile{.dtx} source file. It can be extracted by
% AcrobatReader 6 or higher. Another option is \textsf{pdftk},
% e.g. unpack the file into the current directory:
% \begin{quote}
%   \verb|pdftk selinput.pdf unpack_files output .|
% \end{quote}
%
% \paragraph{Unpacking with \LaTeX.}
% The \xfile{.dtx} chooses its action depending on the format:
% \begin{description}
% \item[\plainTeX:] Run \docstrip\ and extract the files.
% \item[\LaTeX:] Generate the documentation.
% \end{description}
% If you insist on using \LaTeX\ for \docstrip\ (really,
% \docstrip\ does not need \LaTeX), then inform the autodetect routine
% about your intention:
% \begin{quote}
%   \verb|latex \let\install=y\input{selinput.dtx}|
% \end{quote}
% Do not forget to quote the argument according to the demands
% of your shell.
%
% \paragraph{Generating the documentation.}
% You can use both the \xfile{.dtx} or the \xfile{.drv} to generate
% the documentation. The process can be configured by the
% configuration file \xfile{ltxdoc.cfg}. For instance, put this
% line into this file, if you want to have A4 as paper format:
% \begin{quote}
%   \verb|\PassOptionsToClass{a4paper}{article}|
% \end{quote}
% An example follows how to generate the
% documentation with pdf\LaTeX:
% \begin{quote}
%\begin{verbatim}
%pdflatex selinput.dtx
%makeindex -s gind.ist selinput.idx
%pdflatex selinput.dtx
%makeindex -s gind.ist selinput.idx
%pdflatex selinput.dtx
%\end{verbatim}
% \end{quote}
%
% \section{Catalogue}
%
% The following XML file can be used as source for the
% \href{http://mirror.ctan.org/help/Catalogue/catalogue.html}{\TeX\ Catalogue}.
% The elements \texttt{caption} and \texttt{description} are imported
% from the original XML file from the Catalogue.
% The name of the XML file in the Catalogue is \xfile{selinput.xml}.
%    \begin{macrocode}
%<*catalogue>
<?xml version='1.0' encoding='us-ascii'?>
<!DOCTYPE entry SYSTEM 'catalogue.dtd'>
<entry datestamp='$Date$' modifier='$Author$' id='selinput'>
  <name>selinput</name>
  <caption>Semi-automatic detection of input encoding.</caption>
  <authorref id='auth:oberdiek'/>
  <copyright owner='Heiko Oberdiek' year='2007'/>
  <license type='lppl1.3'/>
  <version number='1.2'/>
  <description>
    This package selects the input encoding by specifying pairs
    of input characters and their glyph names.
    <p/>
    The package is part of the <xref refid='oberdiek'>oberdiek</xref>
    bundle.
  </description>
  <documentation details='Package documentation'
      href='ctan:/macros/latex/contrib/oberdiek/selinput.pdf'/>
  <ctan file='true' path='/macros/latex/contrib/oberdiek/selinput.dtx'/>
  <miktex location='oberdiek'/>
  <texlive location='oberdiek'/>
  <install path='/macros/latex/contrib/oberdiek/oberdiek.tds.zip'/>
</entry>
%</catalogue>
%    \end{macrocode}
%
% \begin{thebibliography}{9}
% \bibitem{inputenx}
%   Heiko Oberdiek: \textit{The \xpackage{inputenx} package};
%   2007-04-11 v1.1;
%   \CTAN{macros/latex/contrib/oberdiek/inputenx.pdf}.
%
% \bibitem{adobe:glyphlist}
%   Adobe: \textit{Adobe Glyph List};
%   2002-09-20 v2.0;
%   \url{http://partners.adobe.com/public/developer/en/opentype/glyphlist.txt}.
%
% \bibitem{adobe:aglfn}
%   Adobe: \textit{Adobe Glyph List For New Fonts};
%   2005-11-18 v1.5;
%   \url{http://partners.adobe.com/public/developer/en/opentype/aglfn13.txt}.
%
% \bibitem{stringenc}
%   Heiko Oberdiek: \textit{The \xpackage{stringenc} package};
%   2007-06-16 v1.1;
%   \CTAN{macros/latex/contrib/oberdiek/stringenc.pdf}.
%
% \end{thebibliography}
%
% \begin{History}
%   \begin{Version}{2007/06/16 v1.0}
%   \item
%     First version.
%   \end{Version}
%   \begin{Version}{2007/06/20 v1.1}
%   \item
%     Requested date for package \xpackage{stringenc} fixed.
%   \end{Version}
%   \begin{Version}{2007/09/09 v1.2}
%   \item
%     Line end fixed.
%   \end{Version}
% \end{History}
%
% \PrintIndex
%
% \Finale
\endinput
|
% \end{quote}
% Do not forget to quote the argument according to the demands
% of your shell.
%
% \paragraph{Generating the documentation.}
% You can use both the \xfile{.dtx} or the \xfile{.drv} to generate
% the documentation. The process can be configured by the
% configuration file \xfile{ltxdoc.cfg}. For instance, put this
% line into this file, if you want to have A4 as paper format:
% \begin{quote}
%   \verb|\PassOptionsToClass{a4paper}{article}|
% \end{quote}
% An example follows how to generate the
% documentation with pdf\LaTeX:
% \begin{quote}
%\begin{verbatim}
%pdflatex selinput.dtx
%makeindex -s gind.ist selinput.idx
%pdflatex selinput.dtx
%makeindex -s gind.ist selinput.idx
%pdflatex selinput.dtx
%\end{verbatim}
% \end{quote}
%
% \section{Catalogue}
%
% The following XML file can be used as source for the
% \href{http://mirror.ctan.org/help/Catalogue/catalogue.html}{\TeX\ Catalogue}.
% The elements \texttt{caption} and \texttt{description} are imported
% from the original XML file from the Catalogue.
% The name of the XML file in the Catalogue is \xfile{selinput.xml}.
%    \begin{macrocode}
%<*catalogue>
<?xml version='1.0' encoding='us-ascii'?>
<!DOCTYPE entry SYSTEM 'catalogue.dtd'>
<entry datestamp='$Date$' modifier='$Author$' id='selinput'>
  <name>selinput</name>
  <caption>Semi-automatic detection of input encoding.</caption>
  <authorref id='auth:oberdiek'/>
  <copyright owner='Heiko Oberdiek' year='2007'/>
  <license type='lppl1.3'/>
  <version number='1.2'/>
  <description>
    This package selects the input encoding by specifying pairs
    of input characters and their glyph names.
    <p/>
    The package is part of the <xref refid='oberdiek'>oberdiek</xref>
    bundle.
  </description>
  <documentation details='Package documentation'
      href='ctan:/macros/latex/contrib/oberdiek/selinput.pdf'/>
  <ctan file='true' path='/macros/latex/contrib/oberdiek/selinput.dtx'/>
  <miktex location='oberdiek'/>
  <texlive location='oberdiek'/>
  <install path='/macros/latex/contrib/oberdiek/oberdiek.tds.zip'/>
</entry>
%</catalogue>
%    \end{macrocode}
%
% \begin{thebibliography}{9}
% \bibitem{inputenx}
%   Heiko Oberdiek: \textit{The \xpackage{inputenx} package};
%   2007-04-11 v1.1;
%   \CTAN{macros/latex/contrib/oberdiek/inputenx.pdf}.
%
% \bibitem{adobe:glyphlist}
%   Adobe: \textit{Adobe Glyph List};
%   2002-09-20 v2.0;
%   \url{http://partners.adobe.com/public/developer/en/opentype/glyphlist.txt}.
%
% \bibitem{adobe:aglfn}
%   Adobe: \textit{Adobe Glyph List For New Fonts};
%   2005-11-18 v1.5;
%   \url{http://partners.adobe.com/public/developer/en/opentype/aglfn13.txt}.
%
% \bibitem{stringenc}
%   Heiko Oberdiek: \textit{The \xpackage{stringenc} package};
%   2007-06-16 v1.1;
%   \CTAN{macros/latex/contrib/oberdiek/stringenc.pdf}.
%
% \end{thebibliography}
%
% \begin{History}
%   \begin{Version}{2007/06/16 v1.0}
%   \item
%     First version.
%   \end{Version}
%   \begin{Version}{2007/06/20 v1.1}
%   \item
%     Requested date for package \xpackage{stringenc} fixed.
%   \end{Version}
%   \begin{Version}{2007/09/09 v1.2}
%   \item
%     Line end fixed.
%   \end{Version}
% \end{History}
%
% \PrintIndex
%
% \Finale
\endinput

%        (quote the arguments according to the demands of your shell)
%
% Documentation:
%    (a) If selinput.drv is present:
%           latex selinput.drv
%    (b) Without selinput.drv:
%           latex selinput.dtx; ...
%    The class ltxdoc loads the configuration file ltxdoc.cfg
%    if available. Here you can specify further options, e.g.
%    use A4 as paper format:
%       \PassOptionsToClass{a4paper}{article}
%
%    Programm calls to get the documentation (example):
%       pdflatex selinput.dtx
%       makeindex -s gind.ist selinput.idx
%       pdflatex selinput.dtx
%       makeindex -s gind.ist selinput.idx
%       pdflatex selinput.dtx
%
% Installation:
%    TDS:tex/latex/oberdiek/selinput.sty
%    TDS:doc/latex/oberdiek/selinput.pdf
%    TDS:doc/latex/oberdiek/test/selinput-test1.tex
%    TDS:doc/latex/oberdiek/test/selinput-test2.tex
%    TDS:doc/latex/oberdiek/test/selinput-test3.tex
%    TDS:doc/latex/oberdiek/test/selinput-test4.tex
%    TDS:doc/latex/oberdiek/test/selinput-test5.tex
%    TDS:source/latex/oberdiek/selinput.dtx
%
%<*ignore>
\begingroup
  \catcode123=1 %
  \catcode125=2 %
  \def\x{LaTeX2e}%
\expandafter\endgroup
\ifcase 0\ifx\install y1\fi\expandafter
         \ifx\csname processbatchFile\endcsname\relax\else1\fi
         \ifx\fmtname\x\else 1\fi\relax
\else\csname fi\endcsname
%</ignore>
%<*install>
\input docstrip.tex
\Msg{************************************************************************}
\Msg{* Installation}
\Msg{* Package: selinput 2007/09/09 v1.2 Semi-automatic input encoding detection (HO)}
\Msg{************************************************************************}

\keepsilent
\askforoverwritefalse

\let\MetaPrefix\relax
\preamble

This is a generated file.

Project: selinput
Version: 2007/09/09 v1.2

Copyright (C) 2007 by
   Heiko Oberdiek <heiko.oberdiek at googlemail.com>

This work may be distributed and/or modified under the
conditions of the LaTeX Project Public License, either
version 1.3c of this license or (at your option) any later
version. This version of this license is in
   http://www.latex-project.org/lppl/lppl-1-3c.txt
and the latest version of this license is in
   http://www.latex-project.org/lppl.txt
and version 1.3 or later is part of all distributions of
LaTeX version 2005/12/01 or later.

This work has the LPPL maintenance status "maintained".

This Current Maintainer of this work is Heiko Oberdiek.

This work consists of the main source file selinput.dtx
and the derived files
   selinput.sty, selinput.pdf, selinput.ins, selinput.drv,
   selinput-test1.tex, selinput-test2.tex, selinput-test3.tex,
   selinput-test4.tex, selinput-test5.tex.

\endpreamble
\let\MetaPrefix\DoubleperCent

\generate{%
  \file{selinput.ins}{\from{selinput.dtx}{install}}%
  \file{selinput.drv}{\from{selinput.dtx}{driver}}%
  \usedir{tex/latex/oberdiek}%
  \file{selinput.sty}{\from{selinput.dtx}{package}}%
  \usedir{doc/latex/oberdiek/test}%
  \file{selinput-test1.tex}{\from{selinput.dtx}{test,test1}}%
  \file{selinput-test2.tex}{\from{selinput.dtx}{test,test2}}%
  \file{selinput-test3.tex}{\from{selinput.dtx}{test,test3}}%
  \file{selinput-test4.tex}{\from{selinput.dtx}{test,test4}}%
  \file{selinput-test5.tex}{\from{selinput.dtx}{test,test5}}%
  \nopreamble
  \nopostamble
  \usedir{source/latex/oberdiek/catalogue}%
  \file{selinput.xml}{\from{selinput.dtx}{catalogue}}%
}

\catcode32=13\relax% active space
\let =\space%
\Msg{************************************************************************}
\Msg{*}
\Msg{* To finish the installation you have to move the following}
\Msg{* file into a directory searched by TeX:}
\Msg{*}
\Msg{*     selinput.sty}
\Msg{*}
\Msg{* To produce the documentation run the file `selinput.drv'}
\Msg{* through LaTeX.}
\Msg{*}
\Msg{* Happy TeXing!}
\Msg{*}
\Msg{************************************************************************}

\endbatchfile
%</install>
%<*ignore>
\fi
%</ignore>
%<*driver>
\NeedsTeXFormat{LaTeX2e}
\ProvidesFile{selinput.drv}%
  [2007/09/09 v1.2 Semi-automatic input encoding detection (HO)]%
\documentclass{ltxdoc}
\usepackage[T1]{fontenc}
\usepackage{textcomp}
\usepackage{lmodern}
\usepackage{holtxdoc}[2011/11/22]
\usepackage{color}
\begin{document}
  \DocInput{selinput.dtx}%
\end{document}
%</driver>
% \fi
%
% \CheckSum{389}
%
% \CharacterTable
%  {Upper-case    \A\B\C\D\E\F\G\H\I\J\K\L\M\N\O\P\Q\R\S\T\U\V\W\X\Y\Z
%   Lower-case    \a\b\c\d\e\f\g\h\i\j\k\l\m\n\o\p\q\r\s\t\u\v\w\x\y\z
%   Digits        \0\1\2\3\4\5\6\7\8\9
%   Exclamation   \!     Double quote  \"     Hash (number) \#
%   Dollar        \$     Percent       \%     Ampersand     \&
%   Acute accent  \'     Left paren    \(     Right paren   \)
%   Asterisk      \*     Plus          \+     Comma         \,
%   Minus         \-     Point         \.     Solidus       \/
%   Colon         \:     Semicolon     \;     Less than     \<
%   Equals        \=     Greater than  \>     Question mark \?
%   Commercial at \@     Left bracket  \[     Backslash     \\
%   Right bracket \]     Circumflex    \^     Underscore    \_
%   Grave accent  \`     Left brace    \{     Vertical bar  \|
%   Right brace   \}     Tilde         \~}
%
% \GetFileInfo{selinput.drv}
%
% \title{The \xpackage{selinput} package}
% \date{2007/09/09 v1.2}
% \author{Heiko Oberdiek\\\xemail{heiko.oberdiek at googlemail.com}}
%
% \maketitle
%
% \begin{abstract}
% This package selects the input encoding by specifying between
% input characters and their glyph names.
% \end{abstract}
%
% \tableofcontents
%
% \newcommand*{\EM}{\textcolor{blue}}
% \newcommand*{\ExampleText}{^^A
%   Umlauts:\ \EM{\"A\"O\"U\"a\"o\"u\ss}^^A
% }
%
% \section{Documentation}
%
% \subsection{Introduction}
%
% \LaTeX\ supports the direct use of 8-bit characters by means
% of package \xpackage{inputenc}. However you must know
% and specify the encoding, e.g.:
% \begin{quote}
%   \ttfamily
%   |\documentclass{article}|\\
%   |\usepackage[|\EM{latin1}|]{inputenc}|\\
%   |% or \usepackage[|\EM{utf8}|]{inputenc}|\\
%   |% or \usepackage[|\EM{??}|]{inputenc}|\\
%   |\begin{document}|\\
%   |  |\ExampleText\\
%   |\end{document}|
% \end{quote}
%
% If the document is transferred in an environment that
% uses a different encoding, then there are programs that
% convert the input characters. Examples for conversion
% of file \xfile{test.tex}
% from encoding latin1 (ISO-8859-1) to UTF-8:
% \begin{quote}
%   \ttfamily
%   |recode ISO-8859-1..UTF-8 test.tex|\\
%   |recode latin1..utf8 test.tex|\\
%   |iconv --from-code ISO-8859-1|\\
%   \hphantom{iconv}| --to-code UTF-8|\\
%   \hphantom{iconv}| --output testnew.tex|\\
%   \hphantom{iconv}| test.tex|\\
%   |iconv -f latin1 -t utf8 -o testnew.tex test.tex|
% \end{quote}
% However, the encoding name for package \xpackage{inputenc}
% must be changed:
% \begin{quote}
%    |\usepackage[latin1]{inputenc}| $\rightarrow$
%    |\usepackage[utf8]{inputenc}|\kern-4pt\relax
% \end{quote}
% Of course, unless you are using some clever editor
% that knows package \xpackage{inputenc}, recodes
% the file and adjusts the option at the same time.
% But most editors can perhaps recode the file, but
% they let the option untouched.
%
% Therefore package \xpackage{selinput} chooses another way for
% specifying the input encoding. The encoding name is not needed
% at all. Some 8-bit characters are identified by their glyph
% name and the package chooses an appropriate encoding, example:
% \begin{quote}
%   \ttfamily
%   |\documentclass{article}|\\
%   |\usepackage{selinput}|\\
%   |\SelectInputMappings{|\\
%   |  adieresis={|\EM{\"a}|}|,\\
%   |  germandbls={|\EM{\ss}|}|,\\
%   |  Euro={|\EM{\texteuro}|}|,\\
%   |}|\\
%   |\begin{document}|\\
%   |  |\ExampleText\\
%   |\end{document}|
% \end{quote}
%
% \subsection{User interface}
%
% \begin{declcs}{SelectInputEncodingList} \M{encoding list}
% \end{declcs}
% \cs{SelectInputEncodingList} expects a comma separated list of
% encoding names. Example:
% \begin{quote}
%   |\SelectInputEncodingList{utf8,ansinew,mac-roman}|
% \end{quote}
% The encodings of package \xpackage{inputenx} are used as default.
%
% \begin{declcs}{SelectInputMappings} \M{mapping pairs}
% \end{declcs}
% A mapping pair consists of a glyph name and its input
% character:
% \begin{quote}
%   |\SelectInputMappings{|\\
%   |  adieresis={|\EM{\"a}|}|,\\
%   |  germandbls={|\EM{\ss}|}|,\\
%   |  Euro={|\EM{\texteuro}|}|,\\
%   |}|
% \end{quote}
% The supported glyph names can be found in file \xfile{ix-name.def}
% of project \xpackage{inputenx} \cite{inputenx}. The names are
% basically taken from Adobe's glyphlists \cite{adobe:glyphlist,adobe:aglfn}.
% As many pairs are needed as necessary to identify the encoding.
% Example with insufficient pairs:
% \begin{quote}
%   \ttfamily
%   |\SelectInputEncodingSet{latin1,latin9}|\\
%   |\SelectInputMappings{|\\
%   |  adieresis={|\EM{\"a}|}|,\\
%   |  germandbls={|\EM{\ss}|}|,\\
%   |}|\\
%   \ExampleText| and Euro: |\EM{\textcurrency} (wrong)
% \end{quote}
% The first encoding \xoption{latin1} passes the constraints given
% by the mapping pairs. However the Euro symbol is not part of
% the encoding. Thus a mapping pair with the Euro symbol
% solves the problem. In fact the symbol alone already succeeds in selecting
% between \xoption{latin1} and \xoption{latin9}:
% \begin{quote}
%   \ttfamily
%   |\SelectInputEncodingSet{latin1,latin9}|\\
%   |\SelectInputMappings{|\\
%   |  Euro={|\EM{\texteuro}|},|\\
%   |}|\\
%   \ExampleText| and Euro: |\EM{\texteuro}
% \end{quote}
%
% \subsection{Options}
%
% \begin{description}
% \item[\xoption{warning}:]
%   The selected encoding is written
%   by \cs{PackageInfo} into the \xfile{.log} file only.
%   Option \xoption{warning} changes it to \cs{PackageWarning}.
%   Then the selected encoding is shown on the terminal as well.
% \item[\xoption{ucs}:]
%   The encoding file \xfile{utf8x} of package \cs{ucs} requires
%   that the package itself is loaded before.
%   If the package is not loaded, then the option \xoption{ucs}
%   will load package \xpackage{ucs} if the detected encoding is
%   UTF-8 (limited to the preamble, packages cannot be loaded later).
% \item[\xoption{utf8=\dots}:]
%   The option allows to specify other encoding files
%   for UTF-8 than \LaTeX's \xfile{utf8.def}. For example,
%   |utf8=utf-8| will load \xfile{utf-8.def} instead.
% \end{description}
%
% \subsection{Encodings}
%
% Package \xpackage{stringenc} \cite{stringenc}
% is used for testing the encoding. Thus the encoding
% name must be known by this package. Then the found
% encoding is loaded by \cs{inputencoding} by package
% \xpackage{inputenc} or \cs{InputEncoding} if package
% \xpackage{inputenx} is loaded.
%
% The supported encodings are present in the encoding list,
% thus usually the encoding names do not matter.
% If the list is set by \cs{SelectInputEncodingList},
% then you can use the names that work for package
% \xpackage{inputenc} and are known by package \xpackage{stringenc},
% for example: \xoption{latin1}, \xoption{x-iso-8859-1}. Encoding
% file names of package \xpackage{inputenx} are prefixed with \xfile{x-}.
% The prefix can be dropped, if package \xpackage{inputenx} is loaded.
%
% \StopEventually{
% }
%
% \section{Implementation}
%
%    \begin{macrocode}
%<*package>
\NeedsTeXFormat{LaTeX2e}
\ProvidesPackage{selinput}
  [2007/09/09 v1.2 Semi-automatic input encoding detection (HO)]%
%    \end{macrocode}
%
%    \begin{macrocode}
\RequirePackage{inputenc}
\RequirePackage{kvsetkeys}[2006/10/19]
\RequirePackage{stringenc}[2007/06/16]
\RequirePackage{kvoptions}
%    \end{macrocode}
%    \begin{macro}{\SelectInputEncodingList}
%    \begin{macrocode}
\newcommand*{\SelectInputEncodingList}{%
  \let\SIE@EncodingList\@empty
  \kvsetkeys{SelInputEnc}%
}
%    \end{macrocode}
%    \end{macro}
%    \begin{macro}{\SelectInputMappings}
%    \begin{macrocode}
\newcommand*{\SelectInputMappings}[1]{%
  \SIE@LoadNameDefs
  \let\SIE@StringUnicode\@empty
  \let\SIE@StringDest\@empty
  \kvsetkeys{SelInputMap}{#1}%
  \ifx\\SIE@StringUnicode\SIE@StringDest\\%
    \PackageError{selinput}{%
      No mappings specified%
    }\@ehc
  \else
    \EdefUnescapeHex\SIE@StringUnicode\SIE@StringUnicode
    \let\SIE@Encoding\@empty
    \@for\SIE@EncodingTest:=\SIE@EncodingList\do{%
      \ifx\SIE@Encoding\@empty
        \StringEncodingConvertTest\SIE@temp\SIE@StringUnicode
                                  {utf16be}\SIE@EncodingTest{%
          \ifx\SIE@temp\SIE@StringDest
            \let\SIE@Encoding\SIE@EncodingTest
          \fi
        }{}%
      \fi
    }%
    \ifx\SIE@Encoding\@empty
      \StringEncodingConvertTest\SIE@temp\SIE@StringDest
                                {ascii}{utf16be}{%
        \def\SIE@Encoding{ascii}%
        \SIE@Info{selinput}{%
          Matching encoding not found, but input characters%
          \MessageBreak
          are 7-bit (possibly editor replacements).%
          \MessageBreak
          Hence using ascii encoding%
        }%
      }{}%
    \fi
    \ifx\SIE@Encoding\@empty
      \PackageError{selinput}{%
        Cannot find a matching encoding%
      }\@ehd
    \else
      \ifx\SIE@Encoding\SIE@EncodingUTFviii
        \SIE@LoadUnicodePackage
        \ifx\SIE@UseUTFviii\@empty
        \else
          \let\SIE@Encoding\SIE@UseUTFviii
        \fi
      \fi
      \begingroup\expandafter\expandafter\expandafter\endgroup
      \expandafter\ifx\csname InputEncoding\endcsname\relax
        \inputencoding\SIE@Encoding
      \else
        \InputEncoding\SIE@Encoding
      \fi
      \SIE@Info{selinput}{Encoding `\SIE@Encoding' selected}%
    \fi
  \fi
}
%    \end{macrocode}
%    \end{macro}
%    \begin{macro}{\SIE@LoadNameDefs}
%    \begin{macrocode}
\def\SIE@LoadNameDefs{%
  \begingroup
    \endlinechar=\m@ne
    \catcode92=0 % backslash
    \catcode123=1 % left curly brace/beginning of group
    \catcode125=2 % right curly brace/end of group
    \catcode37=14 % percent/comment character
    \@makeother\[%
    \@makeother\]%
    \@makeother\.%
    \@makeother\(%
    \@makeother\)%
    \@makeother\/%
    \@makeother\-%
    \let\InputenxName\SelectInputDefineMapping
    \InputIfFileExists{ix-name.def}{}{%
      \PackageError{selinput}{%
        Missing `ix-name.def' (part of package `inputenx')%
      }\@ehd
    }%
    \global\let\SIE@LoadNameDefs\relax
  \endgroup
}
%    \end{macrocode}
%    \end{macro}
%    \begin{macro}{\SelectInputDefineMapping}
%    \begin{macrocode}
\newcommand*{\SelectInputDefineMapping}[1]{%
  \expandafter\gdef\csname SIE@@#1\endcsname
}
%    \end{macrocode}
%    \end{macro}
%    \begin{macrocode}
\kv@set@family@handler{SelInputMap}{%
  \@onelevel@sanitize\kv@key
  \ifx\kv@value\relax
    \PackageError{selinput}{%
      Missing input character for `\kv@key'%
    }\@ehc
  \else
    \@onelevel@sanitize\kv@value
    \ifx\kv@value\@empty
      \PackageError{selinput}{%
        Input character got lost?\MessageBreak
        Missing input character for `\kv@key'%
      }\@ehc
    \else
      \@ifundefined{SIE@@\kv@key}{%
        \PackageWarning{selinput}{%
          Missing definition for `\kv@key'%
        }%
      }{%
        \edef\SIE@StringDest{%
          \SIE@StringDest
          \kv@value
        }%
        \edef\SIE@StringUnicode{%
          \SIE@StringUnicode
          \csname SIE@@\kv@key\endcsname
        }%
      }%
    \fi
  \fi
}
%    \end{macrocode}
%    \begin{macrocode}
\kv@set@family@handler{SelInputEnc}{%
  \@onelevel@sanitize\kv@key
  \ifx\kv@value\relax
    \ifx\SIE@EncodingList\@empty
      \let\SIE@EncodingList\kv@key
    \else
      \edef\SIE@EncodingList{\SIE@EncodingList,\kv@key}%
    \fi
  \else
    \@onelevel@sanitize\kv@value
    \PackageError{selinput}{%
      Illegal key value pair (\kv@key=\kv@value)\MessagBreak
      in encoding list%
    }\@ehc
  \fi
}
%    \end{macrocode}
%
%    \begin{macro}{\SIE@LoadUnicodePackage}
%    \begin{macrocode}
\def\SIE@LoadUnicodePackage{%
  \@ifpackageloaded\SIE@UnicodePackage{}{%
    \RequirePackage\SIE@UnicodePackage\relax
  }%
  \SIE@PatchUCS
  \global\let\SIE@LoadUnicodePackage\relax
}
\let\SIE@show\show
\def\SIE@PatchUCS{%
  \AtBeginDocument{%
    \expandafter\ifx\csname ver@ucsencs.def\endcsname\relax
    \else
      \let\show\SIE@show
    \fi
  }%
}
\SIE@PatchUCS
%    \end{macrocode}
%    \end{macro}
%    \begin{macrocode}
\AtBeginDocument{%
  \let\SIE@LoadUnicodePackage\relax
}
%    \end{macrocode}
%    \begin{macro}{\SIE@EncodingUTFviii}
%    \begin{macrocode}
\def\SIE@EncodingUTFviii{utf8}
\@onelevel@sanitize\SIE@EncodingUTFviii
%    \end{macrocode}
%    \end{macro}
%    \begin{macro}{\SIE@EncodingUTFviiix}
%    \begin{macrocode}
\def\SIE@EncodingUTFviiix{utf8x}
\@onelevel@sanitize\SIE@EncodingUTFviiix
%    \end{macrocode}
%    \end{macro}
%
%    \begin{macrocode}
\let\SIE@UnicodePackage\@empty
\let\SIE@UseUTFviii\@empty
\let\SIE@Info\PackageInfo
%    \end{macrocode}
%    \begin{macrocode}
\SetupKeyvalOptions{%
  family=SelInput,%
  prefix=SelInput@%
}
\define@key{SelInput}{utf8}{%
  \def\SIE@UseUTFviii{#1}%
  \@onelevel@sanitize\SIE@UseUTFviii
}
\DeclareBoolOption{ucs}
\DeclareVoidOption{warning}{%
  \let\SIE@Info\PackageWarning
}
\ProcessKeyvalOptions{SelInput}
\ifSelInput@ucs
  \def\SIE@UnicodePackage{ucs}%
  \ifx\SIE@UseUTFviii\@empty
    \let\SIE@UseUTFviii\SIE@EncodingUTFviiix
  \fi
\else
  \ifx\SIE@UseUTFviii\@empty
    \@ifpackageloaded{ucs}{%
      \let\SIE@UseUTFviii\SIE@EncodingUTFviiix
    }{%
      \let\SIE@UseUTFviii\SIE@EncodingUTFviii
    }%
  \fi
\fi
%    \end{macrocode}
%
%    \begin{macro}{\SIE@EncodingList}
%    \begin{macrocode}
\edef\SIE@EncodingList{%
  utf8,%
  x-iso-8859-1,%
  x-iso-8859-15,%
  x-cp1252,% ansinew
  x-mac-roman,%
  x-iso-8859-2,%
  x-iso-8859-3,%
  x-iso-8859-4,%
  x-iso-8859-5,%
  x-iso-8859-6,%
  x-iso-8859-7,%
  x-iso-8859-8,%
  x-iso-8859-9,%
  x-iso-8859-10,%
  x-iso-8859-11,%
  x-iso-8859-13,%
  x-iso-8859-14,%
  x-iso-8859-15,%
  x-mac-centeuro,%
  x-mac-cyrillic,%
  x-koi8-r,%
  x-cp1250,%
  x-cp1251,%
  x-cp1257,%
  x-cp437,%
  x-cp850,%
  x-cp852,%
  x-cp855,%
  x-cp858,%
  x-cp865,%
  x-cp866,%
  x-nextstep,%
  x-dec-mcs%
}%
\@onelevel@sanitize\SIE@EncodingList
%    \end{macrocode}
%    \end{macro}
%
%    \begin{macrocode}
%</package>
%    \end{macrocode}
%
% \section{Test}
%
%    \begin{macrocode}
%<*test>
\NeedsTeXFormat{LaTeX2e}
\documentclass{minimal}
\usepackage{textcomp}
\usepackage{qstest}
%    \end{macrocode}
%    \begin{macrocode}
%<*test1|test2|test3>
\makeatletter
\let\BeginDocumentText\@empty
\def\TestEncoding#1#2{%
  \SelectInputMappings{#2}%
  \Expect*{\SIE@Encoding}{#1}%
  \Expect*{\inputencodingname}{#1}%
  \g@addto@macro\BeginDocumentText{%
    \SelectInputMappings{#2}%
    \Expect*{\SIE@Encoding}{#1}%
    \textbf{\SIE@Encoding:} %
    \kvsetkeys{test}{#2}\par
  }%
}
\def\TestKey#1#2{%
  \define@key{test}{#1}{%
    \sbox0{##1}%
    \sbox2{#2}%
    \Expect*{wd:\the\wd0, ht:\the\ht0, dp:\the\dp0}%
           *{wd:\the\wd2, ht:\the\ht2, dp:\the\dp2}%
    [#1=##1] % hash-ok
  }%
}
\RequirePackage{keyval}
\TestKey{adieresis}{\"a}
\TestKey{germandbls}{\ss}
\TestKey{Euro}{\texteuro}
\makeatother
\usepackage[
  warning,%
%<test2>  utf8=utf-8,
%<test3>  ucs,
]{selinput}
%<test1|test3>\inputencoding{ascii}
%<test2>\inputencoding{utf-8}
%<test3>\usepackage{ucs}
\begin{qstest}{preamble}{}
  \TestEncoding{x-iso-8859-15}{%
    adieresis=^^e4,%
    germandbls=^^df,%
    Euro=^^a4,%
  }%
  \TestEncoding{x-cp1252}{%
    adieresis=^^e4,%
    germandbls=^^df,%
    Euro=^^80,%
  }%
%<test1>  \TestEncoding{utf8}{%
%<test2>  \TestEncoding{utf-8}{%
%<test3>  \TestEncoding{utf8x}{%
    adieresis=^^c3^^a4,%
    germandbls=^^c3^^9f,%
%<!test2>    Euro=^^e2^^82^^ac,
  }%
\end{qstest}
%<test3>\let\ifUnicodeOptiongraphics\iffalse
\begin{document}
\begin{qstest}{document}{}
%<test3>\makeatletter
  \BeginDocumentText
\end{qstest}
%</test1|test2|test3>
%    \end{macrocode}
%
%    \begin{macrocode}
%<*test4>
\usepackage[warning,ucs]{selinput}
\SelectInputMappings{%
    adieresis=^^c3^^a4,%
    germandbls=^^c3^^9f,%
    Euro=^^e2^^82^^ac,%
}
\begin{qstest}{encoding}{}
  \Expect*{\inputencodingname}{utf8x}%
\end{qstest}
\begin{document}
  adieresis=^^c3^^a4, %
  germandbls=^^c3^^9f, %
  Euro=^^e2^^82^^ac%
%</test4>
%    \end{macrocode}
%
%    \begin{macrocode}
%<*test5>
\usepackage[warning,ucs]{selinput}
\SelectInputMappings{%
    adieresis={\"a},%
    germandbls={{\ss}},%
    Euro=\texteuro{},%
}
\begin{qstest}{encoding}{}
  \Expect*{\inputencodingname}{ascii}%
\end{qstest}
\begin{document}
  adieresis={\"a}, %
  germandbls={{\ss}}, %
  Euro=\texteuro{}%
%</test5>
%    \end{macrocode}
%
%    \begin{macrocode}
\end{document}
%</test>
%    \end{macrocode}
%
% \section{Installation}
%
% \subsection{Download}
%
% \paragraph{Package.} This package is available on
% CTAN\footnote{\url{ftp://ftp.ctan.org/tex-archive/}}:
% \begin{description}
% \item[\CTAN{macros/latex/contrib/oberdiek/selinput.dtx}] The source file.
% \item[\CTAN{macros/latex/contrib/oberdiek/selinput.pdf}] Documentation.
% \end{description}
%
%
% \paragraph{Bundle.} All the packages of the bundle `oberdiek'
% are also available in a TDS compliant ZIP archive. There
% the packages are already unpacked and the documentation files
% are generated. The files and directories obey the TDS standard.
% \begin{description}
% \item[\CTAN{install/macros/latex/contrib/oberdiek.tds.zip}]
% \end{description}
% \emph{TDS} refers to the standard ``A Directory Structure
% for \TeX\ Files'' (\CTAN{tds/tds.pdf}). Directories
% with \xfile{texmf} in their name are usually organized this way.
%
% \subsection{Bundle installation}
%
% \paragraph{Unpacking.} Unpack the \xfile{oberdiek.tds.zip} in the
% TDS tree (also known as \xfile{texmf} tree) of your choice.
% Example (linux):
% \begin{quote}
%   |unzip oberdiek.tds.zip -d ~/texmf|
% \end{quote}
%
% \paragraph{Script installation.}
% Check the directory \xfile{TDS:scripts/oberdiek/} for
% scripts that need further installation steps.
% Package \xpackage{attachfile2} comes with the Perl script
% \xfile{pdfatfi.pl} that should be installed in such a way
% that it can be called as \texttt{pdfatfi}.
% Example (linux):
% \begin{quote}
%   |chmod +x scripts/oberdiek/pdfatfi.pl|\\
%   |cp scripts/oberdiek/pdfatfi.pl /usr/local/bin/|
% \end{quote}
%
% \subsection{Package installation}
%
% \paragraph{Unpacking.} The \xfile{.dtx} file is a self-extracting
% \docstrip\ archive. The files are extracted by running the
% \xfile{.dtx} through \plainTeX:
% \begin{quote}
%   \verb|tex selinput.dtx|
% \end{quote}
%
% \paragraph{TDS.} Now the different files must be moved into
% the different directories in your installation TDS tree
% (also known as \xfile{texmf} tree):
% \begin{quote}
% \def\t{^^A
% \begin{tabular}{@{}>{\ttfamily}l@{ $\rightarrow$ }>{\ttfamily}l@{}}
%   selinput.sty & tex/latex/oberdiek/selinput.sty\\
%   selinput.pdf & doc/latex/oberdiek/selinput.pdf\\
%   test/selinput-test1.tex & doc/latex/oberdiek/test/selinput-test1.tex\\
%   test/selinput-test2.tex & doc/latex/oberdiek/test/selinput-test2.tex\\
%   test/selinput-test3.tex & doc/latex/oberdiek/test/selinput-test3.tex\\
%   test/selinput-test4.tex & doc/latex/oberdiek/test/selinput-test4.tex\\
%   test/selinput-test5.tex & doc/latex/oberdiek/test/selinput-test5.tex\\
%   selinput.dtx & source/latex/oberdiek/selinput.dtx\\
% \end{tabular}^^A
% }^^A
% \sbox0{\t}^^A
% \ifdim\wd0>\linewidth
%   \begingroup
%     \advance\linewidth by\leftmargin
%     \advance\linewidth by\rightmargin
%   \edef\x{\endgroup
%     \def\noexpand\lw{\the\linewidth}^^A
%   }\x
%   \def\lwbox{^^A
%     \leavevmode
%     \hbox to \linewidth{^^A
%       \kern-\leftmargin\relax
%       \hss
%       \usebox0
%       \hss
%       \kern-\rightmargin\relax
%     }^^A
%   }^^A
%   \ifdim\wd0>\lw
%     \sbox0{\small\t}^^A
%     \ifdim\wd0>\linewidth
%       \ifdim\wd0>\lw
%         \sbox0{\footnotesize\t}^^A
%         \ifdim\wd0>\linewidth
%           \ifdim\wd0>\lw
%             \sbox0{\scriptsize\t}^^A
%             \ifdim\wd0>\linewidth
%               \ifdim\wd0>\lw
%                 \sbox0{\tiny\t}^^A
%                 \ifdim\wd0>\linewidth
%                   \lwbox
%                 \else
%                   \usebox0
%                 \fi
%               \else
%                 \lwbox
%               \fi
%             \else
%               \usebox0
%             \fi
%           \else
%             \lwbox
%           \fi
%         \else
%           \usebox0
%         \fi
%       \else
%         \lwbox
%       \fi
%     \else
%       \usebox0
%     \fi
%   \else
%     \lwbox
%   \fi
% \else
%   \usebox0
% \fi
% \end{quote}
% If you have a \xfile{docstrip.cfg} that configures and enables \docstrip's
% TDS installing feature, then some files can already be in the right
% place, see the documentation of \docstrip.
%
% \subsection{Refresh file name databases}
%
% If your \TeX~distribution
% (\teTeX, \mikTeX, \dots) relies on file name databases, you must refresh
% these. For example, \teTeX\ users run \verb|texhash| or
% \verb|mktexlsr|.
%
% \subsection{Some details for the interested}
%
% \paragraph{Attached source.}
%
% The PDF documentation on CTAN also includes the
% \xfile{.dtx} source file. It can be extracted by
% AcrobatReader 6 or higher. Another option is \textsf{pdftk},
% e.g. unpack the file into the current directory:
% \begin{quote}
%   \verb|pdftk selinput.pdf unpack_files output .|
% \end{quote}
%
% \paragraph{Unpacking with \LaTeX.}
% The \xfile{.dtx} chooses its action depending on the format:
% \begin{description}
% \item[\plainTeX:] Run \docstrip\ and extract the files.
% \item[\LaTeX:] Generate the documentation.
% \end{description}
% If you insist on using \LaTeX\ for \docstrip\ (really,
% \docstrip\ does not need \LaTeX), then inform the autodetect routine
% about your intention:
% \begin{quote}
%   \verb|latex \let\install=y% \iffalse meta-comment
%
% File: selinput.dtx
% Version: 2007/09/09 v1.2
% Info: Semi-automatic input encoding detection
%
% Copyright (C) 2007 by
%    Heiko Oberdiek <heiko.oberdiek at googlemail.com>
%
% This work may be distributed and/or modified under the
% conditions of the LaTeX Project Public License, either
% version 1.3c of this license or (at your option) any later
% version. This version of this license is in
%    http://www.latex-project.org/lppl/lppl-1-3c.txt
% and the latest version of this license is in
%    http://www.latex-project.org/lppl.txt
% and version 1.3 or later is part of all distributions of
% LaTeX version 2005/12/01 or later.
%
% This work has the LPPL maintenance status "maintained".
%
% This Current Maintainer of this work is Heiko Oberdiek.
%
% This work consists of the main source file selinput.dtx
% and the derived files
%    selinput.sty, selinput.pdf, selinput.ins, selinput.drv,
%    selinput-test1.tex, selinput-test2.tex, selinput-test3.tex,
%    selinput-test4.tex, selinput-test5.tex.
%
% Distribution:
%    CTAN:macros/latex/contrib/oberdiek/selinput.dtx
%    CTAN:macros/latex/contrib/oberdiek/selinput.pdf
%
% Unpacking:
%    (a) If selinput.ins is present:
%           tex selinput.ins
%    (b) Without selinput.ins:
%           tex selinput.dtx
%    (c) If you insist on using LaTeX
%           latex \let\install=y% \iffalse meta-comment
%
% File: selinput.dtx
% Version: 2007/09/09 v1.2
% Info: Semi-automatic input encoding detection
%
% Copyright (C) 2007 by
%    Heiko Oberdiek <heiko.oberdiek at googlemail.com>
%
% This work may be distributed and/or modified under the
% conditions of the LaTeX Project Public License, either
% version 1.3c of this license or (at your option) any later
% version. This version of this license is in
%    http://www.latex-project.org/lppl/lppl-1-3c.txt
% and the latest version of this license is in
%    http://www.latex-project.org/lppl.txt
% and version 1.3 or later is part of all distributions of
% LaTeX version 2005/12/01 or later.
%
% This work has the LPPL maintenance status "maintained".
%
% This Current Maintainer of this work is Heiko Oberdiek.
%
% This work consists of the main source file selinput.dtx
% and the derived files
%    selinput.sty, selinput.pdf, selinput.ins, selinput.drv,
%    selinput-test1.tex, selinput-test2.tex, selinput-test3.tex,
%    selinput-test4.tex, selinput-test5.tex.
%
% Distribution:
%    CTAN:macros/latex/contrib/oberdiek/selinput.dtx
%    CTAN:macros/latex/contrib/oberdiek/selinput.pdf
%
% Unpacking:
%    (a) If selinput.ins is present:
%           tex selinput.ins
%    (b) Without selinput.ins:
%           tex selinput.dtx
%    (c) If you insist on using LaTeX
%           latex \let\install=y\input{selinput.dtx}
%        (quote the arguments according to the demands of your shell)
%
% Documentation:
%    (a) If selinput.drv is present:
%           latex selinput.drv
%    (b) Without selinput.drv:
%           latex selinput.dtx; ...
%    The class ltxdoc loads the configuration file ltxdoc.cfg
%    if available. Here you can specify further options, e.g.
%    use A4 as paper format:
%       \PassOptionsToClass{a4paper}{article}
%
%    Programm calls to get the documentation (example):
%       pdflatex selinput.dtx
%       makeindex -s gind.ist selinput.idx
%       pdflatex selinput.dtx
%       makeindex -s gind.ist selinput.idx
%       pdflatex selinput.dtx
%
% Installation:
%    TDS:tex/latex/oberdiek/selinput.sty
%    TDS:doc/latex/oberdiek/selinput.pdf
%    TDS:doc/latex/oberdiek/test/selinput-test1.tex
%    TDS:doc/latex/oberdiek/test/selinput-test2.tex
%    TDS:doc/latex/oberdiek/test/selinput-test3.tex
%    TDS:doc/latex/oberdiek/test/selinput-test4.tex
%    TDS:doc/latex/oberdiek/test/selinput-test5.tex
%    TDS:source/latex/oberdiek/selinput.dtx
%
%<*ignore>
\begingroup
  \catcode123=1 %
  \catcode125=2 %
  \def\x{LaTeX2e}%
\expandafter\endgroup
\ifcase 0\ifx\install y1\fi\expandafter
         \ifx\csname processbatchFile\endcsname\relax\else1\fi
         \ifx\fmtname\x\else 1\fi\relax
\else\csname fi\endcsname
%</ignore>
%<*install>
\input docstrip.tex
\Msg{************************************************************************}
\Msg{* Installation}
\Msg{* Package: selinput 2007/09/09 v1.2 Semi-automatic input encoding detection (HO)}
\Msg{************************************************************************}

\keepsilent
\askforoverwritefalse

\let\MetaPrefix\relax
\preamble

This is a generated file.

Project: selinput
Version: 2007/09/09 v1.2

Copyright (C) 2007 by
   Heiko Oberdiek <heiko.oberdiek at googlemail.com>

This work may be distributed and/or modified under the
conditions of the LaTeX Project Public License, either
version 1.3c of this license or (at your option) any later
version. This version of this license is in
   http://www.latex-project.org/lppl/lppl-1-3c.txt
and the latest version of this license is in
   http://www.latex-project.org/lppl.txt
and version 1.3 or later is part of all distributions of
LaTeX version 2005/12/01 or later.

This work has the LPPL maintenance status "maintained".

This Current Maintainer of this work is Heiko Oberdiek.

This work consists of the main source file selinput.dtx
and the derived files
   selinput.sty, selinput.pdf, selinput.ins, selinput.drv,
   selinput-test1.tex, selinput-test2.tex, selinput-test3.tex,
   selinput-test4.tex, selinput-test5.tex.

\endpreamble
\let\MetaPrefix\DoubleperCent

\generate{%
  \file{selinput.ins}{\from{selinput.dtx}{install}}%
  \file{selinput.drv}{\from{selinput.dtx}{driver}}%
  \usedir{tex/latex/oberdiek}%
  \file{selinput.sty}{\from{selinput.dtx}{package}}%
  \usedir{doc/latex/oberdiek/test}%
  \file{selinput-test1.tex}{\from{selinput.dtx}{test,test1}}%
  \file{selinput-test2.tex}{\from{selinput.dtx}{test,test2}}%
  \file{selinput-test3.tex}{\from{selinput.dtx}{test,test3}}%
  \file{selinput-test4.tex}{\from{selinput.dtx}{test,test4}}%
  \file{selinput-test5.tex}{\from{selinput.dtx}{test,test5}}%
  \nopreamble
  \nopostamble
  \usedir{source/latex/oberdiek/catalogue}%
  \file{selinput.xml}{\from{selinput.dtx}{catalogue}}%
}

\catcode32=13\relax% active space
\let =\space%
\Msg{************************************************************************}
\Msg{*}
\Msg{* To finish the installation you have to move the following}
\Msg{* file into a directory searched by TeX:}
\Msg{*}
\Msg{*     selinput.sty}
\Msg{*}
\Msg{* To produce the documentation run the file `selinput.drv'}
\Msg{* through LaTeX.}
\Msg{*}
\Msg{* Happy TeXing!}
\Msg{*}
\Msg{************************************************************************}

\endbatchfile
%</install>
%<*ignore>
\fi
%</ignore>
%<*driver>
\NeedsTeXFormat{LaTeX2e}
\ProvidesFile{selinput.drv}%
  [2007/09/09 v1.2 Semi-automatic input encoding detection (HO)]%
\documentclass{ltxdoc}
\usepackage[T1]{fontenc}
\usepackage{textcomp}
\usepackage{lmodern}
\usepackage{holtxdoc}[2011/11/22]
\usepackage{color}
\begin{document}
  \DocInput{selinput.dtx}%
\end{document}
%</driver>
% \fi
%
% \CheckSum{389}
%
% \CharacterTable
%  {Upper-case    \A\B\C\D\E\F\G\H\I\J\K\L\M\N\O\P\Q\R\S\T\U\V\W\X\Y\Z
%   Lower-case    \a\b\c\d\e\f\g\h\i\j\k\l\m\n\o\p\q\r\s\t\u\v\w\x\y\z
%   Digits        \0\1\2\3\4\5\6\7\8\9
%   Exclamation   \!     Double quote  \"     Hash (number) \#
%   Dollar        \$     Percent       \%     Ampersand     \&
%   Acute accent  \'     Left paren    \(     Right paren   \)
%   Asterisk      \*     Plus          \+     Comma         \,
%   Minus         \-     Point         \.     Solidus       \/
%   Colon         \:     Semicolon     \;     Less than     \<
%   Equals        \=     Greater than  \>     Question mark \?
%   Commercial at \@     Left bracket  \[     Backslash     \\
%   Right bracket \]     Circumflex    \^     Underscore    \_
%   Grave accent  \`     Left brace    \{     Vertical bar  \|
%   Right brace   \}     Tilde         \~}
%
% \GetFileInfo{selinput.drv}
%
% \title{The \xpackage{selinput} package}
% \date{2007/09/09 v1.2}
% \author{Heiko Oberdiek\\\xemail{heiko.oberdiek at googlemail.com}}
%
% \maketitle
%
% \begin{abstract}
% This package selects the input encoding by specifying between
% input characters and their glyph names.
% \end{abstract}
%
% \tableofcontents
%
% \newcommand*{\EM}{\textcolor{blue}}
% \newcommand*{\ExampleText}{^^A
%   Umlauts:\ \EM{\"A\"O\"U\"a\"o\"u\ss}^^A
% }
%
% \section{Documentation}
%
% \subsection{Introduction}
%
% \LaTeX\ supports the direct use of 8-bit characters by means
% of package \xpackage{inputenc}. However you must know
% and specify the encoding, e.g.:
% \begin{quote}
%   \ttfamily
%   |\documentclass{article}|\\
%   |\usepackage[|\EM{latin1}|]{inputenc}|\\
%   |% or \usepackage[|\EM{utf8}|]{inputenc}|\\
%   |% or \usepackage[|\EM{??}|]{inputenc}|\\
%   |\begin{document}|\\
%   |  |\ExampleText\\
%   |\end{document}|
% \end{quote}
%
% If the document is transferred in an environment that
% uses a different encoding, then there are programs that
% convert the input characters. Examples for conversion
% of file \xfile{test.tex}
% from encoding latin1 (ISO-8859-1) to UTF-8:
% \begin{quote}
%   \ttfamily
%   |recode ISO-8859-1..UTF-8 test.tex|\\
%   |recode latin1..utf8 test.tex|\\
%   |iconv --from-code ISO-8859-1|\\
%   \hphantom{iconv}| --to-code UTF-8|\\
%   \hphantom{iconv}| --output testnew.tex|\\
%   \hphantom{iconv}| test.tex|\\
%   |iconv -f latin1 -t utf8 -o testnew.tex test.tex|
% \end{quote}
% However, the encoding name for package \xpackage{inputenc}
% must be changed:
% \begin{quote}
%    |\usepackage[latin1]{inputenc}| $\rightarrow$
%    |\usepackage[utf8]{inputenc}|\kern-4pt\relax
% \end{quote}
% Of course, unless you are using some clever editor
% that knows package \xpackage{inputenc}, recodes
% the file and adjusts the option at the same time.
% But most editors can perhaps recode the file, but
% they let the option untouched.
%
% Therefore package \xpackage{selinput} chooses another way for
% specifying the input encoding. The encoding name is not needed
% at all. Some 8-bit characters are identified by their glyph
% name and the package chooses an appropriate encoding, example:
% \begin{quote}
%   \ttfamily
%   |\documentclass{article}|\\
%   |\usepackage{selinput}|\\
%   |\SelectInputMappings{|\\
%   |  adieresis={|\EM{\"a}|}|,\\
%   |  germandbls={|\EM{\ss}|}|,\\
%   |  Euro={|\EM{\texteuro}|}|,\\
%   |}|\\
%   |\begin{document}|\\
%   |  |\ExampleText\\
%   |\end{document}|
% \end{quote}
%
% \subsection{User interface}
%
% \begin{declcs}{SelectInputEncodingList} \M{encoding list}
% \end{declcs}
% \cs{SelectInputEncodingList} expects a comma separated list of
% encoding names. Example:
% \begin{quote}
%   |\SelectInputEncodingList{utf8,ansinew,mac-roman}|
% \end{quote}
% The encodings of package \xpackage{inputenx} are used as default.
%
% \begin{declcs}{SelectInputMappings} \M{mapping pairs}
% \end{declcs}
% A mapping pair consists of a glyph name and its input
% character:
% \begin{quote}
%   |\SelectInputMappings{|\\
%   |  adieresis={|\EM{\"a}|}|,\\
%   |  germandbls={|\EM{\ss}|}|,\\
%   |  Euro={|\EM{\texteuro}|}|,\\
%   |}|
% \end{quote}
% The supported glyph names can be found in file \xfile{ix-name.def}
% of project \xpackage{inputenx} \cite{inputenx}. The names are
% basically taken from Adobe's glyphlists \cite{adobe:glyphlist,adobe:aglfn}.
% As many pairs are needed as necessary to identify the encoding.
% Example with insufficient pairs:
% \begin{quote}
%   \ttfamily
%   |\SelectInputEncodingSet{latin1,latin9}|\\
%   |\SelectInputMappings{|\\
%   |  adieresis={|\EM{\"a}|}|,\\
%   |  germandbls={|\EM{\ss}|}|,\\
%   |}|\\
%   \ExampleText| and Euro: |\EM{\textcurrency} (wrong)
% \end{quote}
% The first encoding \xoption{latin1} passes the constraints given
% by the mapping pairs. However the Euro symbol is not part of
% the encoding. Thus a mapping pair with the Euro symbol
% solves the problem. In fact the symbol alone already succeeds in selecting
% between \xoption{latin1} and \xoption{latin9}:
% \begin{quote}
%   \ttfamily
%   |\SelectInputEncodingSet{latin1,latin9}|\\
%   |\SelectInputMappings{|\\
%   |  Euro={|\EM{\texteuro}|},|\\
%   |}|\\
%   \ExampleText| and Euro: |\EM{\texteuro}
% \end{quote}
%
% \subsection{Options}
%
% \begin{description}
% \item[\xoption{warning}:]
%   The selected encoding is written
%   by \cs{PackageInfo} into the \xfile{.log} file only.
%   Option \xoption{warning} changes it to \cs{PackageWarning}.
%   Then the selected encoding is shown on the terminal as well.
% \item[\xoption{ucs}:]
%   The encoding file \xfile{utf8x} of package \cs{ucs} requires
%   that the package itself is loaded before.
%   If the package is not loaded, then the option \xoption{ucs}
%   will load package \xpackage{ucs} if the detected encoding is
%   UTF-8 (limited to the preamble, packages cannot be loaded later).
% \item[\xoption{utf8=\dots}:]
%   The option allows to specify other encoding files
%   for UTF-8 than \LaTeX's \xfile{utf8.def}. For example,
%   |utf8=utf-8| will load \xfile{utf-8.def} instead.
% \end{description}
%
% \subsection{Encodings}
%
% Package \xpackage{stringenc} \cite{stringenc}
% is used for testing the encoding. Thus the encoding
% name must be known by this package. Then the found
% encoding is loaded by \cs{inputencoding} by package
% \xpackage{inputenc} or \cs{InputEncoding} if package
% \xpackage{inputenx} is loaded.
%
% The supported encodings are present in the encoding list,
% thus usually the encoding names do not matter.
% If the list is set by \cs{SelectInputEncodingList},
% then you can use the names that work for package
% \xpackage{inputenc} and are known by package \xpackage{stringenc},
% for example: \xoption{latin1}, \xoption{x-iso-8859-1}. Encoding
% file names of package \xpackage{inputenx} are prefixed with \xfile{x-}.
% The prefix can be dropped, if package \xpackage{inputenx} is loaded.
%
% \StopEventually{
% }
%
% \section{Implementation}
%
%    \begin{macrocode}
%<*package>
\NeedsTeXFormat{LaTeX2e}
\ProvidesPackage{selinput}
  [2007/09/09 v1.2 Semi-automatic input encoding detection (HO)]%
%    \end{macrocode}
%
%    \begin{macrocode}
\RequirePackage{inputenc}
\RequirePackage{kvsetkeys}[2006/10/19]
\RequirePackage{stringenc}[2007/06/16]
\RequirePackage{kvoptions}
%    \end{macrocode}
%    \begin{macro}{\SelectInputEncodingList}
%    \begin{macrocode}
\newcommand*{\SelectInputEncodingList}{%
  \let\SIE@EncodingList\@empty
  \kvsetkeys{SelInputEnc}%
}
%    \end{macrocode}
%    \end{macro}
%    \begin{macro}{\SelectInputMappings}
%    \begin{macrocode}
\newcommand*{\SelectInputMappings}[1]{%
  \SIE@LoadNameDefs
  \let\SIE@StringUnicode\@empty
  \let\SIE@StringDest\@empty
  \kvsetkeys{SelInputMap}{#1}%
  \ifx\\SIE@StringUnicode\SIE@StringDest\\%
    \PackageError{selinput}{%
      No mappings specified%
    }\@ehc
  \else
    \EdefUnescapeHex\SIE@StringUnicode\SIE@StringUnicode
    \let\SIE@Encoding\@empty
    \@for\SIE@EncodingTest:=\SIE@EncodingList\do{%
      \ifx\SIE@Encoding\@empty
        \StringEncodingConvertTest\SIE@temp\SIE@StringUnicode
                                  {utf16be}\SIE@EncodingTest{%
          \ifx\SIE@temp\SIE@StringDest
            \let\SIE@Encoding\SIE@EncodingTest
          \fi
        }{}%
      \fi
    }%
    \ifx\SIE@Encoding\@empty
      \StringEncodingConvertTest\SIE@temp\SIE@StringDest
                                {ascii}{utf16be}{%
        \def\SIE@Encoding{ascii}%
        \SIE@Info{selinput}{%
          Matching encoding not found, but input characters%
          \MessageBreak
          are 7-bit (possibly editor replacements).%
          \MessageBreak
          Hence using ascii encoding%
        }%
      }{}%
    \fi
    \ifx\SIE@Encoding\@empty
      \PackageError{selinput}{%
        Cannot find a matching encoding%
      }\@ehd
    \else
      \ifx\SIE@Encoding\SIE@EncodingUTFviii
        \SIE@LoadUnicodePackage
        \ifx\SIE@UseUTFviii\@empty
        \else
          \let\SIE@Encoding\SIE@UseUTFviii
        \fi
      \fi
      \begingroup\expandafter\expandafter\expandafter\endgroup
      \expandafter\ifx\csname InputEncoding\endcsname\relax
        \inputencoding\SIE@Encoding
      \else
        \InputEncoding\SIE@Encoding
      \fi
      \SIE@Info{selinput}{Encoding `\SIE@Encoding' selected}%
    \fi
  \fi
}
%    \end{macrocode}
%    \end{macro}
%    \begin{macro}{\SIE@LoadNameDefs}
%    \begin{macrocode}
\def\SIE@LoadNameDefs{%
  \begingroup
    \endlinechar=\m@ne
    \catcode92=0 % backslash
    \catcode123=1 % left curly brace/beginning of group
    \catcode125=2 % right curly brace/end of group
    \catcode37=14 % percent/comment character
    \@makeother\[%
    \@makeother\]%
    \@makeother\.%
    \@makeother\(%
    \@makeother\)%
    \@makeother\/%
    \@makeother\-%
    \let\InputenxName\SelectInputDefineMapping
    \InputIfFileExists{ix-name.def}{}{%
      \PackageError{selinput}{%
        Missing `ix-name.def' (part of package `inputenx')%
      }\@ehd
    }%
    \global\let\SIE@LoadNameDefs\relax
  \endgroup
}
%    \end{macrocode}
%    \end{macro}
%    \begin{macro}{\SelectInputDefineMapping}
%    \begin{macrocode}
\newcommand*{\SelectInputDefineMapping}[1]{%
  \expandafter\gdef\csname SIE@@#1\endcsname
}
%    \end{macrocode}
%    \end{macro}
%    \begin{macrocode}
\kv@set@family@handler{SelInputMap}{%
  \@onelevel@sanitize\kv@key
  \ifx\kv@value\relax
    \PackageError{selinput}{%
      Missing input character for `\kv@key'%
    }\@ehc
  \else
    \@onelevel@sanitize\kv@value
    \ifx\kv@value\@empty
      \PackageError{selinput}{%
        Input character got lost?\MessageBreak
        Missing input character for `\kv@key'%
      }\@ehc
    \else
      \@ifundefined{SIE@@\kv@key}{%
        \PackageWarning{selinput}{%
          Missing definition for `\kv@key'%
        }%
      }{%
        \edef\SIE@StringDest{%
          \SIE@StringDest
          \kv@value
        }%
        \edef\SIE@StringUnicode{%
          \SIE@StringUnicode
          \csname SIE@@\kv@key\endcsname
        }%
      }%
    \fi
  \fi
}
%    \end{macrocode}
%    \begin{macrocode}
\kv@set@family@handler{SelInputEnc}{%
  \@onelevel@sanitize\kv@key
  \ifx\kv@value\relax
    \ifx\SIE@EncodingList\@empty
      \let\SIE@EncodingList\kv@key
    \else
      \edef\SIE@EncodingList{\SIE@EncodingList,\kv@key}%
    \fi
  \else
    \@onelevel@sanitize\kv@value
    \PackageError{selinput}{%
      Illegal key value pair (\kv@key=\kv@value)\MessagBreak
      in encoding list%
    }\@ehc
  \fi
}
%    \end{macrocode}
%
%    \begin{macro}{\SIE@LoadUnicodePackage}
%    \begin{macrocode}
\def\SIE@LoadUnicodePackage{%
  \@ifpackageloaded\SIE@UnicodePackage{}{%
    \RequirePackage\SIE@UnicodePackage\relax
  }%
  \SIE@PatchUCS
  \global\let\SIE@LoadUnicodePackage\relax
}
\let\SIE@show\show
\def\SIE@PatchUCS{%
  \AtBeginDocument{%
    \expandafter\ifx\csname ver@ucsencs.def\endcsname\relax
    \else
      \let\show\SIE@show
    \fi
  }%
}
\SIE@PatchUCS
%    \end{macrocode}
%    \end{macro}
%    \begin{macrocode}
\AtBeginDocument{%
  \let\SIE@LoadUnicodePackage\relax
}
%    \end{macrocode}
%    \begin{macro}{\SIE@EncodingUTFviii}
%    \begin{macrocode}
\def\SIE@EncodingUTFviii{utf8}
\@onelevel@sanitize\SIE@EncodingUTFviii
%    \end{macrocode}
%    \end{macro}
%    \begin{macro}{\SIE@EncodingUTFviiix}
%    \begin{macrocode}
\def\SIE@EncodingUTFviiix{utf8x}
\@onelevel@sanitize\SIE@EncodingUTFviiix
%    \end{macrocode}
%    \end{macro}
%
%    \begin{macrocode}
\let\SIE@UnicodePackage\@empty
\let\SIE@UseUTFviii\@empty
\let\SIE@Info\PackageInfo
%    \end{macrocode}
%    \begin{macrocode}
\SetupKeyvalOptions{%
  family=SelInput,%
  prefix=SelInput@%
}
\define@key{SelInput}{utf8}{%
  \def\SIE@UseUTFviii{#1}%
  \@onelevel@sanitize\SIE@UseUTFviii
}
\DeclareBoolOption{ucs}
\DeclareVoidOption{warning}{%
  \let\SIE@Info\PackageWarning
}
\ProcessKeyvalOptions{SelInput}
\ifSelInput@ucs
  \def\SIE@UnicodePackage{ucs}%
  \ifx\SIE@UseUTFviii\@empty
    \let\SIE@UseUTFviii\SIE@EncodingUTFviiix
  \fi
\else
  \ifx\SIE@UseUTFviii\@empty
    \@ifpackageloaded{ucs}{%
      \let\SIE@UseUTFviii\SIE@EncodingUTFviiix
    }{%
      \let\SIE@UseUTFviii\SIE@EncodingUTFviii
    }%
  \fi
\fi
%    \end{macrocode}
%
%    \begin{macro}{\SIE@EncodingList}
%    \begin{macrocode}
\edef\SIE@EncodingList{%
  utf8,%
  x-iso-8859-1,%
  x-iso-8859-15,%
  x-cp1252,% ansinew
  x-mac-roman,%
  x-iso-8859-2,%
  x-iso-8859-3,%
  x-iso-8859-4,%
  x-iso-8859-5,%
  x-iso-8859-6,%
  x-iso-8859-7,%
  x-iso-8859-8,%
  x-iso-8859-9,%
  x-iso-8859-10,%
  x-iso-8859-11,%
  x-iso-8859-13,%
  x-iso-8859-14,%
  x-iso-8859-15,%
  x-mac-centeuro,%
  x-mac-cyrillic,%
  x-koi8-r,%
  x-cp1250,%
  x-cp1251,%
  x-cp1257,%
  x-cp437,%
  x-cp850,%
  x-cp852,%
  x-cp855,%
  x-cp858,%
  x-cp865,%
  x-cp866,%
  x-nextstep,%
  x-dec-mcs%
}%
\@onelevel@sanitize\SIE@EncodingList
%    \end{macrocode}
%    \end{macro}
%
%    \begin{macrocode}
%</package>
%    \end{macrocode}
%
% \section{Test}
%
%    \begin{macrocode}
%<*test>
\NeedsTeXFormat{LaTeX2e}
\documentclass{minimal}
\usepackage{textcomp}
\usepackage{qstest}
%    \end{macrocode}
%    \begin{macrocode}
%<*test1|test2|test3>
\makeatletter
\let\BeginDocumentText\@empty
\def\TestEncoding#1#2{%
  \SelectInputMappings{#2}%
  \Expect*{\SIE@Encoding}{#1}%
  \Expect*{\inputencodingname}{#1}%
  \g@addto@macro\BeginDocumentText{%
    \SelectInputMappings{#2}%
    \Expect*{\SIE@Encoding}{#1}%
    \textbf{\SIE@Encoding:} %
    \kvsetkeys{test}{#2}\par
  }%
}
\def\TestKey#1#2{%
  \define@key{test}{#1}{%
    \sbox0{##1}%
    \sbox2{#2}%
    \Expect*{wd:\the\wd0, ht:\the\ht0, dp:\the\dp0}%
           *{wd:\the\wd2, ht:\the\ht2, dp:\the\dp2}%
    [#1=##1] % hash-ok
  }%
}
\RequirePackage{keyval}
\TestKey{adieresis}{\"a}
\TestKey{germandbls}{\ss}
\TestKey{Euro}{\texteuro}
\makeatother
\usepackage[
  warning,%
%<test2>  utf8=utf-8,
%<test3>  ucs,
]{selinput}
%<test1|test3>\inputencoding{ascii}
%<test2>\inputencoding{utf-8}
%<test3>\usepackage{ucs}
\begin{qstest}{preamble}{}
  \TestEncoding{x-iso-8859-15}{%
    adieresis=^^e4,%
    germandbls=^^df,%
    Euro=^^a4,%
  }%
  \TestEncoding{x-cp1252}{%
    adieresis=^^e4,%
    germandbls=^^df,%
    Euro=^^80,%
  }%
%<test1>  \TestEncoding{utf8}{%
%<test2>  \TestEncoding{utf-8}{%
%<test3>  \TestEncoding{utf8x}{%
    adieresis=^^c3^^a4,%
    germandbls=^^c3^^9f,%
%<!test2>    Euro=^^e2^^82^^ac,
  }%
\end{qstest}
%<test3>\let\ifUnicodeOptiongraphics\iffalse
\begin{document}
\begin{qstest}{document}{}
%<test3>\makeatletter
  \BeginDocumentText
\end{qstest}
%</test1|test2|test3>
%    \end{macrocode}
%
%    \begin{macrocode}
%<*test4>
\usepackage[warning,ucs]{selinput}
\SelectInputMappings{%
    adieresis=^^c3^^a4,%
    germandbls=^^c3^^9f,%
    Euro=^^e2^^82^^ac,%
}
\begin{qstest}{encoding}{}
  \Expect*{\inputencodingname}{utf8x}%
\end{qstest}
\begin{document}
  adieresis=^^c3^^a4, %
  germandbls=^^c3^^9f, %
  Euro=^^e2^^82^^ac%
%</test4>
%    \end{macrocode}
%
%    \begin{macrocode}
%<*test5>
\usepackage[warning,ucs]{selinput}
\SelectInputMappings{%
    adieresis={\"a},%
    germandbls={{\ss}},%
    Euro=\texteuro{},%
}
\begin{qstest}{encoding}{}
  \Expect*{\inputencodingname}{ascii}%
\end{qstest}
\begin{document}
  adieresis={\"a}, %
  germandbls={{\ss}}, %
  Euro=\texteuro{}%
%</test5>
%    \end{macrocode}
%
%    \begin{macrocode}
\end{document}
%</test>
%    \end{macrocode}
%
% \section{Installation}
%
% \subsection{Download}
%
% \paragraph{Package.} This package is available on
% CTAN\footnote{\url{ftp://ftp.ctan.org/tex-archive/}}:
% \begin{description}
% \item[\CTAN{macros/latex/contrib/oberdiek/selinput.dtx}] The source file.
% \item[\CTAN{macros/latex/contrib/oberdiek/selinput.pdf}] Documentation.
% \end{description}
%
%
% \paragraph{Bundle.} All the packages of the bundle `oberdiek'
% are also available in a TDS compliant ZIP archive. There
% the packages are already unpacked and the documentation files
% are generated. The files and directories obey the TDS standard.
% \begin{description}
% \item[\CTAN{install/macros/latex/contrib/oberdiek.tds.zip}]
% \end{description}
% \emph{TDS} refers to the standard ``A Directory Structure
% for \TeX\ Files'' (\CTAN{tds/tds.pdf}). Directories
% with \xfile{texmf} in their name are usually organized this way.
%
% \subsection{Bundle installation}
%
% \paragraph{Unpacking.} Unpack the \xfile{oberdiek.tds.zip} in the
% TDS tree (also known as \xfile{texmf} tree) of your choice.
% Example (linux):
% \begin{quote}
%   |unzip oberdiek.tds.zip -d ~/texmf|
% \end{quote}
%
% \paragraph{Script installation.}
% Check the directory \xfile{TDS:scripts/oberdiek/} for
% scripts that need further installation steps.
% Package \xpackage{attachfile2} comes with the Perl script
% \xfile{pdfatfi.pl} that should be installed in such a way
% that it can be called as \texttt{pdfatfi}.
% Example (linux):
% \begin{quote}
%   |chmod +x scripts/oberdiek/pdfatfi.pl|\\
%   |cp scripts/oberdiek/pdfatfi.pl /usr/local/bin/|
% \end{quote}
%
% \subsection{Package installation}
%
% \paragraph{Unpacking.} The \xfile{.dtx} file is a self-extracting
% \docstrip\ archive. The files are extracted by running the
% \xfile{.dtx} through \plainTeX:
% \begin{quote}
%   \verb|tex selinput.dtx|
% \end{quote}
%
% \paragraph{TDS.} Now the different files must be moved into
% the different directories in your installation TDS tree
% (also known as \xfile{texmf} tree):
% \begin{quote}
% \def\t{^^A
% \begin{tabular}{@{}>{\ttfamily}l@{ $\rightarrow$ }>{\ttfamily}l@{}}
%   selinput.sty & tex/latex/oberdiek/selinput.sty\\
%   selinput.pdf & doc/latex/oberdiek/selinput.pdf\\
%   test/selinput-test1.tex & doc/latex/oberdiek/test/selinput-test1.tex\\
%   test/selinput-test2.tex & doc/latex/oberdiek/test/selinput-test2.tex\\
%   test/selinput-test3.tex & doc/latex/oberdiek/test/selinput-test3.tex\\
%   test/selinput-test4.tex & doc/latex/oberdiek/test/selinput-test4.tex\\
%   test/selinput-test5.tex & doc/latex/oberdiek/test/selinput-test5.tex\\
%   selinput.dtx & source/latex/oberdiek/selinput.dtx\\
% \end{tabular}^^A
% }^^A
% \sbox0{\t}^^A
% \ifdim\wd0>\linewidth
%   \begingroup
%     \advance\linewidth by\leftmargin
%     \advance\linewidth by\rightmargin
%   \edef\x{\endgroup
%     \def\noexpand\lw{\the\linewidth}^^A
%   }\x
%   \def\lwbox{^^A
%     \leavevmode
%     \hbox to \linewidth{^^A
%       \kern-\leftmargin\relax
%       \hss
%       \usebox0
%       \hss
%       \kern-\rightmargin\relax
%     }^^A
%   }^^A
%   \ifdim\wd0>\lw
%     \sbox0{\small\t}^^A
%     \ifdim\wd0>\linewidth
%       \ifdim\wd0>\lw
%         \sbox0{\footnotesize\t}^^A
%         \ifdim\wd0>\linewidth
%           \ifdim\wd0>\lw
%             \sbox0{\scriptsize\t}^^A
%             \ifdim\wd0>\linewidth
%               \ifdim\wd0>\lw
%                 \sbox0{\tiny\t}^^A
%                 \ifdim\wd0>\linewidth
%                   \lwbox
%                 \else
%                   \usebox0
%                 \fi
%               \else
%                 \lwbox
%               \fi
%             \else
%               \usebox0
%             \fi
%           \else
%             \lwbox
%           \fi
%         \else
%           \usebox0
%         \fi
%       \else
%         \lwbox
%       \fi
%     \else
%       \usebox0
%     \fi
%   \else
%     \lwbox
%   \fi
% \else
%   \usebox0
% \fi
% \end{quote}
% If you have a \xfile{docstrip.cfg} that configures and enables \docstrip's
% TDS installing feature, then some files can already be in the right
% place, see the documentation of \docstrip.
%
% \subsection{Refresh file name databases}
%
% If your \TeX~distribution
% (\teTeX, \mikTeX, \dots) relies on file name databases, you must refresh
% these. For example, \teTeX\ users run \verb|texhash| or
% \verb|mktexlsr|.
%
% \subsection{Some details for the interested}
%
% \paragraph{Attached source.}
%
% The PDF documentation on CTAN also includes the
% \xfile{.dtx} source file. It can be extracted by
% AcrobatReader 6 or higher. Another option is \textsf{pdftk},
% e.g. unpack the file into the current directory:
% \begin{quote}
%   \verb|pdftk selinput.pdf unpack_files output .|
% \end{quote}
%
% \paragraph{Unpacking with \LaTeX.}
% The \xfile{.dtx} chooses its action depending on the format:
% \begin{description}
% \item[\plainTeX:] Run \docstrip\ and extract the files.
% \item[\LaTeX:] Generate the documentation.
% \end{description}
% If you insist on using \LaTeX\ for \docstrip\ (really,
% \docstrip\ does not need \LaTeX), then inform the autodetect routine
% about your intention:
% \begin{quote}
%   \verb|latex \let\install=y\input{selinput.dtx}|
% \end{quote}
% Do not forget to quote the argument according to the demands
% of your shell.
%
% \paragraph{Generating the documentation.}
% You can use both the \xfile{.dtx} or the \xfile{.drv} to generate
% the documentation. The process can be configured by the
% configuration file \xfile{ltxdoc.cfg}. For instance, put this
% line into this file, if you want to have A4 as paper format:
% \begin{quote}
%   \verb|\PassOptionsToClass{a4paper}{article}|
% \end{quote}
% An example follows how to generate the
% documentation with pdf\LaTeX:
% \begin{quote}
%\begin{verbatim}
%pdflatex selinput.dtx
%makeindex -s gind.ist selinput.idx
%pdflatex selinput.dtx
%makeindex -s gind.ist selinput.idx
%pdflatex selinput.dtx
%\end{verbatim}
% \end{quote}
%
% \section{Catalogue}
%
% The following XML file can be used as source for the
% \href{http://mirror.ctan.org/help/Catalogue/catalogue.html}{\TeX\ Catalogue}.
% The elements \texttt{caption} and \texttt{description} are imported
% from the original XML file from the Catalogue.
% The name of the XML file in the Catalogue is \xfile{selinput.xml}.
%    \begin{macrocode}
%<*catalogue>
<?xml version='1.0' encoding='us-ascii'?>
<!DOCTYPE entry SYSTEM 'catalogue.dtd'>
<entry datestamp='$Date$' modifier='$Author$' id='selinput'>
  <name>selinput</name>
  <caption>Semi-automatic detection of input encoding.</caption>
  <authorref id='auth:oberdiek'/>
  <copyright owner='Heiko Oberdiek' year='2007'/>
  <license type='lppl1.3'/>
  <version number='1.2'/>
  <description>
    This package selects the input encoding by specifying pairs
    of input characters and their glyph names.
    <p/>
    The package is part of the <xref refid='oberdiek'>oberdiek</xref>
    bundle.
  </description>
  <documentation details='Package documentation'
      href='ctan:/macros/latex/contrib/oberdiek/selinput.pdf'/>
  <ctan file='true' path='/macros/latex/contrib/oberdiek/selinput.dtx'/>
  <miktex location='oberdiek'/>
  <texlive location='oberdiek'/>
  <install path='/macros/latex/contrib/oberdiek/oberdiek.tds.zip'/>
</entry>
%</catalogue>
%    \end{macrocode}
%
% \begin{thebibliography}{9}
% \bibitem{inputenx}
%   Heiko Oberdiek: \textit{The \xpackage{inputenx} package};
%   2007-04-11 v1.1;
%   \CTAN{macros/latex/contrib/oberdiek/inputenx.pdf}.
%
% \bibitem{adobe:glyphlist}
%   Adobe: \textit{Adobe Glyph List};
%   2002-09-20 v2.0;
%   \url{http://partners.adobe.com/public/developer/en/opentype/glyphlist.txt}.
%
% \bibitem{adobe:aglfn}
%   Adobe: \textit{Adobe Glyph List For New Fonts};
%   2005-11-18 v1.5;
%   \url{http://partners.adobe.com/public/developer/en/opentype/aglfn13.txt}.
%
% \bibitem{stringenc}
%   Heiko Oberdiek: \textit{The \xpackage{stringenc} package};
%   2007-06-16 v1.1;
%   \CTAN{macros/latex/contrib/oberdiek/stringenc.pdf}.
%
% \end{thebibliography}
%
% \begin{History}
%   \begin{Version}{2007/06/16 v1.0}
%   \item
%     First version.
%   \end{Version}
%   \begin{Version}{2007/06/20 v1.1}
%   \item
%     Requested date for package \xpackage{stringenc} fixed.
%   \end{Version}
%   \begin{Version}{2007/09/09 v1.2}
%   \item
%     Line end fixed.
%   \end{Version}
% \end{History}
%
% \PrintIndex
%
% \Finale
\endinput

%        (quote the arguments according to the demands of your shell)
%
% Documentation:
%    (a) If selinput.drv is present:
%           latex selinput.drv
%    (b) Without selinput.drv:
%           latex selinput.dtx; ...
%    The class ltxdoc loads the configuration file ltxdoc.cfg
%    if available. Here you can specify further options, e.g.
%    use A4 as paper format:
%       \PassOptionsToClass{a4paper}{article}
%
%    Programm calls to get the documentation (example):
%       pdflatex selinput.dtx
%       makeindex -s gind.ist selinput.idx
%       pdflatex selinput.dtx
%       makeindex -s gind.ist selinput.idx
%       pdflatex selinput.dtx
%
% Installation:
%    TDS:tex/latex/oberdiek/selinput.sty
%    TDS:doc/latex/oberdiek/selinput.pdf
%    TDS:doc/latex/oberdiek/test/selinput-test1.tex
%    TDS:doc/latex/oberdiek/test/selinput-test2.tex
%    TDS:doc/latex/oberdiek/test/selinput-test3.tex
%    TDS:doc/latex/oberdiek/test/selinput-test4.tex
%    TDS:doc/latex/oberdiek/test/selinput-test5.tex
%    TDS:source/latex/oberdiek/selinput.dtx
%
%<*ignore>
\begingroup
  \catcode123=1 %
  \catcode125=2 %
  \def\x{LaTeX2e}%
\expandafter\endgroup
\ifcase 0\ifx\install y1\fi\expandafter
         \ifx\csname processbatchFile\endcsname\relax\else1\fi
         \ifx\fmtname\x\else 1\fi\relax
\else\csname fi\endcsname
%</ignore>
%<*install>
\input docstrip.tex
\Msg{************************************************************************}
\Msg{* Installation}
\Msg{* Package: selinput 2007/09/09 v1.2 Semi-automatic input encoding detection (HO)}
\Msg{************************************************************************}

\keepsilent
\askforoverwritefalse

\let\MetaPrefix\relax
\preamble

This is a generated file.

Project: selinput
Version: 2007/09/09 v1.2

Copyright (C) 2007 by
   Heiko Oberdiek <heiko.oberdiek at googlemail.com>

This work may be distributed and/or modified under the
conditions of the LaTeX Project Public License, either
version 1.3c of this license or (at your option) any later
version. This version of this license is in
   http://www.latex-project.org/lppl/lppl-1-3c.txt
and the latest version of this license is in
   http://www.latex-project.org/lppl.txt
and version 1.3 or later is part of all distributions of
LaTeX version 2005/12/01 or later.

This work has the LPPL maintenance status "maintained".

This Current Maintainer of this work is Heiko Oberdiek.

This work consists of the main source file selinput.dtx
and the derived files
   selinput.sty, selinput.pdf, selinput.ins, selinput.drv,
   selinput-test1.tex, selinput-test2.tex, selinput-test3.tex,
   selinput-test4.tex, selinput-test5.tex.

\endpreamble
\let\MetaPrefix\DoubleperCent

\generate{%
  \file{selinput.ins}{\from{selinput.dtx}{install}}%
  \file{selinput.drv}{\from{selinput.dtx}{driver}}%
  \usedir{tex/latex/oberdiek}%
  \file{selinput.sty}{\from{selinput.dtx}{package}}%
  \usedir{doc/latex/oberdiek/test}%
  \file{selinput-test1.tex}{\from{selinput.dtx}{test,test1}}%
  \file{selinput-test2.tex}{\from{selinput.dtx}{test,test2}}%
  \file{selinput-test3.tex}{\from{selinput.dtx}{test,test3}}%
  \file{selinput-test4.tex}{\from{selinput.dtx}{test,test4}}%
  \file{selinput-test5.tex}{\from{selinput.dtx}{test,test5}}%
  \nopreamble
  \nopostamble
  \usedir{source/latex/oberdiek/catalogue}%
  \file{selinput.xml}{\from{selinput.dtx}{catalogue}}%
}

\catcode32=13\relax% active space
\let =\space%
\Msg{************************************************************************}
\Msg{*}
\Msg{* To finish the installation you have to move the following}
\Msg{* file into a directory searched by TeX:}
\Msg{*}
\Msg{*     selinput.sty}
\Msg{*}
\Msg{* To produce the documentation run the file `selinput.drv'}
\Msg{* through LaTeX.}
\Msg{*}
\Msg{* Happy TeXing!}
\Msg{*}
\Msg{************************************************************************}

\endbatchfile
%</install>
%<*ignore>
\fi
%</ignore>
%<*driver>
\NeedsTeXFormat{LaTeX2e}
\ProvidesFile{selinput.drv}%
  [2007/09/09 v1.2 Semi-automatic input encoding detection (HO)]%
\documentclass{ltxdoc}
\usepackage[T1]{fontenc}
\usepackage{textcomp}
\usepackage{lmodern}
\usepackage{holtxdoc}[2011/11/22]
\usepackage{color}
\begin{document}
  \DocInput{selinput.dtx}%
\end{document}
%</driver>
% \fi
%
% \CheckSum{389}
%
% \CharacterTable
%  {Upper-case    \A\B\C\D\E\F\G\H\I\J\K\L\M\N\O\P\Q\R\S\T\U\V\W\X\Y\Z
%   Lower-case    \a\b\c\d\e\f\g\h\i\j\k\l\m\n\o\p\q\r\s\t\u\v\w\x\y\z
%   Digits        \0\1\2\3\4\5\6\7\8\9
%   Exclamation   \!     Double quote  \"     Hash (number) \#
%   Dollar        \$     Percent       \%     Ampersand     \&
%   Acute accent  \'     Left paren    \(     Right paren   \)
%   Asterisk      \*     Plus          \+     Comma         \,
%   Minus         \-     Point         \.     Solidus       \/
%   Colon         \:     Semicolon     \;     Less than     \<
%   Equals        \=     Greater than  \>     Question mark \?
%   Commercial at \@     Left bracket  \[     Backslash     \\
%   Right bracket \]     Circumflex    \^     Underscore    \_
%   Grave accent  \`     Left brace    \{     Vertical bar  \|
%   Right brace   \}     Tilde         \~}
%
% \GetFileInfo{selinput.drv}
%
% \title{The \xpackage{selinput} package}
% \date{2007/09/09 v1.2}
% \author{Heiko Oberdiek\\\xemail{heiko.oberdiek at googlemail.com}}
%
% \maketitle
%
% \begin{abstract}
% This package selects the input encoding by specifying between
% input characters and their glyph names.
% \end{abstract}
%
% \tableofcontents
%
% \newcommand*{\EM}{\textcolor{blue}}
% \newcommand*{\ExampleText}{^^A
%   Umlauts:\ \EM{\"A\"O\"U\"a\"o\"u\ss}^^A
% }
%
% \section{Documentation}
%
% \subsection{Introduction}
%
% \LaTeX\ supports the direct use of 8-bit characters by means
% of package \xpackage{inputenc}. However you must know
% and specify the encoding, e.g.:
% \begin{quote}
%   \ttfamily
%   |\documentclass{article}|\\
%   |\usepackage[|\EM{latin1}|]{inputenc}|\\
%   |% or \usepackage[|\EM{utf8}|]{inputenc}|\\
%   |% or \usepackage[|\EM{??}|]{inputenc}|\\
%   |\begin{document}|\\
%   |  |\ExampleText\\
%   |\end{document}|
% \end{quote}
%
% If the document is transferred in an environment that
% uses a different encoding, then there are programs that
% convert the input characters. Examples for conversion
% of file \xfile{test.tex}
% from encoding latin1 (ISO-8859-1) to UTF-8:
% \begin{quote}
%   \ttfamily
%   |recode ISO-8859-1..UTF-8 test.tex|\\
%   |recode latin1..utf8 test.tex|\\
%   |iconv --from-code ISO-8859-1|\\
%   \hphantom{iconv}| --to-code UTF-8|\\
%   \hphantom{iconv}| --output testnew.tex|\\
%   \hphantom{iconv}| test.tex|\\
%   |iconv -f latin1 -t utf8 -o testnew.tex test.tex|
% \end{quote}
% However, the encoding name for package \xpackage{inputenc}
% must be changed:
% \begin{quote}
%    |\usepackage[latin1]{inputenc}| $\rightarrow$
%    |\usepackage[utf8]{inputenc}|\kern-4pt\relax
% \end{quote}
% Of course, unless you are using some clever editor
% that knows package \xpackage{inputenc}, recodes
% the file and adjusts the option at the same time.
% But most editors can perhaps recode the file, but
% they let the option untouched.
%
% Therefore package \xpackage{selinput} chooses another way for
% specifying the input encoding. The encoding name is not needed
% at all. Some 8-bit characters are identified by their glyph
% name and the package chooses an appropriate encoding, example:
% \begin{quote}
%   \ttfamily
%   |\documentclass{article}|\\
%   |\usepackage{selinput}|\\
%   |\SelectInputMappings{|\\
%   |  adieresis={|\EM{\"a}|}|,\\
%   |  germandbls={|\EM{\ss}|}|,\\
%   |  Euro={|\EM{\texteuro}|}|,\\
%   |}|\\
%   |\begin{document}|\\
%   |  |\ExampleText\\
%   |\end{document}|
% \end{quote}
%
% \subsection{User interface}
%
% \begin{declcs}{SelectInputEncodingList} \M{encoding list}
% \end{declcs}
% \cs{SelectInputEncodingList} expects a comma separated list of
% encoding names. Example:
% \begin{quote}
%   |\SelectInputEncodingList{utf8,ansinew,mac-roman}|
% \end{quote}
% The encodings of package \xpackage{inputenx} are used as default.
%
% \begin{declcs}{SelectInputMappings} \M{mapping pairs}
% \end{declcs}
% A mapping pair consists of a glyph name and its input
% character:
% \begin{quote}
%   |\SelectInputMappings{|\\
%   |  adieresis={|\EM{\"a}|}|,\\
%   |  germandbls={|\EM{\ss}|}|,\\
%   |  Euro={|\EM{\texteuro}|}|,\\
%   |}|
% \end{quote}
% The supported glyph names can be found in file \xfile{ix-name.def}
% of project \xpackage{inputenx} \cite{inputenx}. The names are
% basically taken from Adobe's glyphlists \cite{adobe:glyphlist,adobe:aglfn}.
% As many pairs are needed as necessary to identify the encoding.
% Example with insufficient pairs:
% \begin{quote}
%   \ttfamily
%   |\SelectInputEncodingSet{latin1,latin9}|\\
%   |\SelectInputMappings{|\\
%   |  adieresis={|\EM{\"a}|}|,\\
%   |  germandbls={|\EM{\ss}|}|,\\
%   |}|\\
%   \ExampleText| and Euro: |\EM{\textcurrency} (wrong)
% \end{quote}
% The first encoding \xoption{latin1} passes the constraints given
% by the mapping pairs. However the Euro symbol is not part of
% the encoding. Thus a mapping pair with the Euro symbol
% solves the problem. In fact the symbol alone already succeeds in selecting
% between \xoption{latin1} and \xoption{latin9}:
% \begin{quote}
%   \ttfamily
%   |\SelectInputEncodingSet{latin1,latin9}|\\
%   |\SelectInputMappings{|\\
%   |  Euro={|\EM{\texteuro}|},|\\
%   |}|\\
%   \ExampleText| and Euro: |\EM{\texteuro}
% \end{quote}
%
% \subsection{Options}
%
% \begin{description}
% \item[\xoption{warning}:]
%   The selected encoding is written
%   by \cs{PackageInfo} into the \xfile{.log} file only.
%   Option \xoption{warning} changes it to \cs{PackageWarning}.
%   Then the selected encoding is shown on the terminal as well.
% \item[\xoption{ucs}:]
%   The encoding file \xfile{utf8x} of package \cs{ucs} requires
%   that the package itself is loaded before.
%   If the package is not loaded, then the option \xoption{ucs}
%   will load package \xpackage{ucs} if the detected encoding is
%   UTF-8 (limited to the preamble, packages cannot be loaded later).
% \item[\xoption{utf8=\dots}:]
%   The option allows to specify other encoding files
%   for UTF-8 than \LaTeX's \xfile{utf8.def}. For example,
%   |utf8=utf-8| will load \xfile{utf-8.def} instead.
% \end{description}
%
% \subsection{Encodings}
%
% Package \xpackage{stringenc} \cite{stringenc}
% is used for testing the encoding. Thus the encoding
% name must be known by this package. Then the found
% encoding is loaded by \cs{inputencoding} by package
% \xpackage{inputenc} or \cs{InputEncoding} if package
% \xpackage{inputenx} is loaded.
%
% The supported encodings are present in the encoding list,
% thus usually the encoding names do not matter.
% If the list is set by \cs{SelectInputEncodingList},
% then you can use the names that work for package
% \xpackage{inputenc} and are known by package \xpackage{stringenc},
% for example: \xoption{latin1}, \xoption{x-iso-8859-1}. Encoding
% file names of package \xpackage{inputenx} are prefixed with \xfile{x-}.
% The prefix can be dropped, if package \xpackage{inputenx} is loaded.
%
% \StopEventually{
% }
%
% \section{Implementation}
%
%    \begin{macrocode}
%<*package>
\NeedsTeXFormat{LaTeX2e}
\ProvidesPackage{selinput}
  [2007/09/09 v1.2 Semi-automatic input encoding detection (HO)]%
%    \end{macrocode}
%
%    \begin{macrocode}
\RequirePackage{inputenc}
\RequirePackage{kvsetkeys}[2006/10/19]
\RequirePackage{stringenc}[2007/06/16]
\RequirePackage{kvoptions}
%    \end{macrocode}
%    \begin{macro}{\SelectInputEncodingList}
%    \begin{macrocode}
\newcommand*{\SelectInputEncodingList}{%
  \let\SIE@EncodingList\@empty
  \kvsetkeys{SelInputEnc}%
}
%    \end{macrocode}
%    \end{macro}
%    \begin{macro}{\SelectInputMappings}
%    \begin{macrocode}
\newcommand*{\SelectInputMappings}[1]{%
  \SIE@LoadNameDefs
  \let\SIE@StringUnicode\@empty
  \let\SIE@StringDest\@empty
  \kvsetkeys{SelInputMap}{#1}%
  \ifx\\SIE@StringUnicode\SIE@StringDest\\%
    \PackageError{selinput}{%
      No mappings specified%
    }\@ehc
  \else
    \EdefUnescapeHex\SIE@StringUnicode\SIE@StringUnicode
    \let\SIE@Encoding\@empty
    \@for\SIE@EncodingTest:=\SIE@EncodingList\do{%
      \ifx\SIE@Encoding\@empty
        \StringEncodingConvertTest\SIE@temp\SIE@StringUnicode
                                  {utf16be}\SIE@EncodingTest{%
          \ifx\SIE@temp\SIE@StringDest
            \let\SIE@Encoding\SIE@EncodingTest
          \fi
        }{}%
      \fi
    }%
    \ifx\SIE@Encoding\@empty
      \StringEncodingConvertTest\SIE@temp\SIE@StringDest
                                {ascii}{utf16be}{%
        \def\SIE@Encoding{ascii}%
        \SIE@Info{selinput}{%
          Matching encoding not found, but input characters%
          \MessageBreak
          are 7-bit (possibly editor replacements).%
          \MessageBreak
          Hence using ascii encoding%
        }%
      }{}%
    \fi
    \ifx\SIE@Encoding\@empty
      \PackageError{selinput}{%
        Cannot find a matching encoding%
      }\@ehd
    \else
      \ifx\SIE@Encoding\SIE@EncodingUTFviii
        \SIE@LoadUnicodePackage
        \ifx\SIE@UseUTFviii\@empty
        \else
          \let\SIE@Encoding\SIE@UseUTFviii
        \fi
      \fi
      \begingroup\expandafter\expandafter\expandafter\endgroup
      \expandafter\ifx\csname InputEncoding\endcsname\relax
        \inputencoding\SIE@Encoding
      \else
        \InputEncoding\SIE@Encoding
      \fi
      \SIE@Info{selinput}{Encoding `\SIE@Encoding' selected}%
    \fi
  \fi
}
%    \end{macrocode}
%    \end{macro}
%    \begin{macro}{\SIE@LoadNameDefs}
%    \begin{macrocode}
\def\SIE@LoadNameDefs{%
  \begingroup
    \endlinechar=\m@ne
    \catcode92=0 % backslash
    \catcode123=1 % left curly brace/beginning of group
    \catcode125=2 % right curly brace/end of group
    \catcode37=14 % percent/comment character
    \@makeother\[%
    \@makeother\]%
    \@makeother\.%
    \@makeother\(%
    \@makeother\)%
    \@makeother\/%
    \@makeother\-%
    \let\InputenxName\SelectInputDefineMapping
    \InputIfFileExists{ix-name.def}{}{%
      \PackageError{selinput}{%
        Missing `ix-name.def' (part of package `inputenx')%
      }\@ehd
    }%
    \global\let\SIE@LoadNameDefs\relax
  \endgroup
}
%    \end{macrocode}
%    \end{macro}
%    \begin{macro}{\SelectInputDefineMapping}
%    \begin{macrocode}
\newcommand*{\SelectInputDefineMapping}[1]{%
  \expandafter\gdef\csname SIE@@#1\endcsname
}
%    \end{macrocode}
%    \end{macro}
%    \begin{macrocode}
\kv@set@family@handler{SelInputMap}{%
  \@onelevel@sanitize\kv@key
  \ifx\kv@value\relax
    \PackageError{selinput}{%
      Missing input character for `\kv@key'%
    }\@ehc
  \else
    \@onelevel@sanitize\kv@value
    \ifx\kv@value\@empty
      \PackageError{selinput}{%
        Input character got lost?\MessageBreak
        Missing input character for `\kv@key'%
      }\@ehc
    \else
      \@ifundefined{SIE@@\kv@key}{%
        \PackageWarning{selinput}{%
          Missing definition for `\kv@key'%
        }%
      }{%
        \edef\SIE@StringDest{%
          \SIE@StringDest
          \kv@value
        }%
        \edef\SIE@StringUnicode{%
          \SIE@StringUnicode
          \csname SIE@@\kv@key\endcsname
        }%
      }%
    \fi
  \fi
}
%    \end{macrocode}
%    \begin{macrocode}
\kv@set@family@handler{SelInputEnc}{%
  \@onelevel@sanitize\kv@key
  \ifx\kv@value\relax
    \ifx\SIE@EncodingList\@empty
      \let\SIE@EncodingList\kv@key
    \else
      \edef\SIE@EncodingList{\SIE@EncodingList,\kv@key}%
    \fi
  \else
    \@onelevel@sanitize\kv@value
    \PackageError{selinput}{%
      Illegal key value pair (\kv@key=\kv@value)\MessagBreak
      in encoding list%
    }\@ehc
  \fi
}
%    \end{macrocode}
%
%    \begin{macro}{\SIE@LoadUnicodePackage}
%    \begin{macrocode}
\def\SIE@LoadUnicodePackage{%
  \@ifpackageloaded\SIE@UnicodePackage{}{%
    \RequirePackage\SIE@UnicodePackage\relax
  }%
  \SIE@PatchUCS
  \global\let\SIE@LoadUnicodePackage\relax
}
\let\SIE@show\show
\def\SIE@PatchUCS{%
  \AtBeginDocument{%
    \expandafter\ifx\csname ver@ucsencs.def\endcsname\relax
    \else
      \let\show\SIE@show
    \fi
  }%
}
\SIE@PatchUCS
%    \end{macrocode}
%    \end{macro}
%    \begin{macrocode}
\AtBeginDocument{%
  \let\SIE@LoadUnicodePackage\relax
}
%    \end{macrocode}
%    \begin{macro}{\SIE@EncodingUTFviii}
%    \begin{macrocode}
\def\SIE@EncodingUTFviii{utf8}
\@onelevel@sanitize\SIE@EncodingUTFviii
%    \end{macrocode}
%    \end{macro}
%    \begin{macro}{\SIE@EncodingUTFviiix}
%    \begin{macrocode}
\def\SIE@EncodingUTFviiix{utf8x}
\@onelevel@sanitize\SIE@EncodingUTFviiix
%    \end{macrocode}
%    \end{macro}
%
%    \begin{macrocode}
\let\SIE@UnicodePackage\@empty
\let\SIE@UseUTFviii\@empty
\let\SIE@Info\PackageInfo
%    \end{macrocode}
%    \begin{macrocode}
\SetupKeyvalOptions{%
  family=SelInput,%
  prefix=SelInput@%
}
\define@key{SelInput}{utf8}{%
  \def\SIE@UseUTFviii{#1}%
  \@onelevel@sanitize\SIE@UseUTFviii
}
\DeclareBoolOption{ucs}
\DeclareVoidOption{warning}{%
  \let\SIE@Info\PackageWarning
}
\ProcessKeyvalOptions{SelInput}
\ifSelInput@ucs
  \def\SIE@UnicodePackage{ucs}%
  \ifx\SIE@UseUTFviii\@empty
    \let\SIE@UseUTFviii\SIE@EncodingUTFviiix
  \fi
\else
  \ifx\SIE@UseUTFviii\@empty
    \@ifpackageloaded{ucs}{%
      \let\SIE@UseUTFviii\SIE@EncodingUTFviiix
    }{%
      \let\SIE@UseUTFviii\SIE@EncodingUTFviii
    }%
  \fi
\fi
%    \end{macrocode}
%
%    \begin{macro}{\SIE@EncodingList}
%    \begin{macrocode}
\edef\SIE@EncodingList{%
  utf8,%
  x-iso-8859-1,%
  x-iso-8859-15,%
  x-cp1252,% ansinew
  x-mac-roman,%
  x-iso-8859-2,%
  x-iso-8859-3,%
  x-iso-8859-4,%
  x-iso-8859-5,%
  x-iso-8859-6,%
  x-iso-8859-7,%
  x-iso-8859-8,%
  x-iso-8859-9,%
  x-iso-8859-10,%
  x-iso-8859-11,%
  x-iso-8859-13,%
  x-iso-8859-14,%
  x-iso-8859-15,%
  x-mac-centeuro,%
  x-mac-cyrillic,%
  x-koi8-r,%
  x-cp1250,%
  x-cp1251,%
  x-cp1257,%
  x-cp437,%
  x-cp850,%
  x-cp852,%
  x-cp855,%
  x-cp858,%
  x-cp865,%
  x-cp866,%
  x-nextstep,%
  x-dec-mcs%
}%
\@onelevel@sanitize\SIE@EncodingList
%    \end{macrocode}
%    \end{macro}
%
%    \begin{macrocode}
%</package>
%    \end{macrocode}
%
% \section{Test}
%
%    \begin{macrocode}
%<*test>
\NeedsTeXFormat{LaTeX2e}
\documentclass{minimal}
\usepackage{textcomp}
\usepackage{qstest}
%    \end{macrocode}
%    \begin{macrocode}
%<*test1|test2|test3>
\makeatletter
\let\BeginDocumentText\@empty
\def\TestEncoding#1#2{%
  \SelectInputMappings{#2}%
  \Expect*{\SIE@Encoding}{#1}%
  \Expect*{\inputencodingname}{#1}%
  \g@addto@macro\BeginDocumentText{%
    \SelectInputMappings{#2}%
    \Expect*{\SIE@Encoding}{#1}%
    \textbf{\SIE@Encoding:} %
    \kvsetkeys{test}{#2}\par
  }%
}
\def\TestKey#1#2{%
  \define@key{test}{#1}{%
    \sbox0{##1}%
    \sbox2{#2}%
    \Expect*{wd:\the\wd0, ht:\the\ht0, dp:\the\dp0}%
           *{wd:\the\wd2, ht:\the\ht2, dp:\the\dp2}%
    [#1=##1] % hash-ok
  }%
}
\RequirePackage{keyval}
\TestKey{adieresis}{\"a}
\TestKey{germandbls}{\ss}
\TestKey{Euro}{\texteuro}
\makeatother
\usepackage[
  warning,%
%<test2>  utf8=utf-8,
%<test3>  ucs,
]{selinput}
%<test1|test3>\inputencoding{ascii}
%<test2>\inputencoding{utf-8}
%<test3>\usepackage{ucs}
\begin{qstest}{preamble}{}
  \TestEncoding{x-iso-8859-15}{%
    adieresis=^^e4,%
    germandbls=^^df,%
    Euro=^^a4,%
  }%
  \TestEncoding{x-cp1252}{%
    adieresis=^^e4,%
    germandbls=^^df,%
    Euro=^^80,%
  }%
%<test1>  \TestEncoding{utf8}{%
%<test2>  \TestEncoding{utf-8}{%
%<test3>  \TestEncoding{utf8x}{%
    adieresis=^^c3^^a4,%
    germandbls=^^c3^^9f,%
%<!test2>    Euro=^^e2^^82^^ac,
  }%
\end{qstest}
%<test3>\let\ifUnicodeOptiongraphics\iffalse
\begin{document}
\begin{qstest}{document}{}
%<test3>\makeatletter
  \BeginDocumentText
\end{qstest}
%</test1|test2|test3>
%    \end{macrocode}
%
%    \begin{macrocode}
%<*test4>
\usepackage[warning,ucs]{selinput}
\SelectInputMappings{%
    adieresis=^^c3^^a4,%
    germandbls=^^c3^^9f,%
    Euro=^^e2^^82^^ac,%
}
\begin{qstest}{encoding}{}
  \Expect*{\inputencodingname}{utf8x}%
\end{qstest}
\begin{document}
  adieresis=^^c3^^a4, %
  germandbls=^^c3^^9f, %
  Euro=^^e2^^82^^ac%
%</test4>
%    \end{macrocode}
%
%    \begin{macrocode}
%<*test5>
\usepackage[warning,ucs]{selinput}
\SelectInputMappings{%
    adieresis={\"a},%
    germandbls={{\ss}},%
    Euro=\texteuro{},%
}
\begin{qstest}{encoding}{}
  \Expect*{\inputencodingname}{ascii}%
\end{qstest}
\begin{document}
  adieresis={\"a}, %
  germandbls={{\ss}}, %
  Euro=\texteuro{}%
%</test5>
%    \end{macrocode}
%
%    \begin{macrocode}
\end{document}
%</test>
%    \end{macrocode}
%
% \section{Installation}
%
% \subsection{Download}
%
% \paragraph{Package.} This package is available on
% CTAN\footnote{\url{ftp://ftp.ctan.org/tex-archive/}}:
% \begin{description}
% \item[\CTAN{macros/latex/contrib/oberdiek/selinput.dtx}] The source file.
% \item[\CTAN{macros/latex/contrib/oberdiek/selinput.pdf}] Documentation.
% \end{description}
%
%
% \paragraph{Bundle.} All the packages of the bundle `oberdiek'
% are also available in a TDS compliant ZIP archive. There
% the packages are already unpacked and the documentation files
% are generated. The files and directories obey the TDS standard.
% \begin{description}
% \item[\CTAN{install/macros/latex/contrib/oberdiek.tds.zip}]
% \end{description}
% \emph{TDS} refers to the standard ``A Directory Structure
% for \TeX\ Files'' (\CTAN{tds/tds.pdf}). Directories
% with \xfile{texmf} in their name are usually organized this way.
%
% \subsection{Bundle installation}
%
% \paragraph{Unpacking.} Unpack the \xfile{oberdiek.tds.zip} in the
% TDS tree (also known as \xfile{texmf} tree) of your choice.
% Example (linux):
% \begin{quote}
%   |unzip oberdiek.tds.zip -d ~/texmf|
% \end{quote}
%
% \paragraph{Script installation.}
% Check the directory \xfile{TDS:scripts/oberdiek/} for
% scripts that need further installation steps.
% Package \xpackage{attachfile2} comes with the Perl script
% \xfile{pdfatfi.pl} that should be installed in such a way
% that it can be called as \texttt{pdfatfi}.
% Example (linux):
% \begin{quote}
%   |chmod +x scripts/oberdiek/pdfatfi.pl|\\
%   |cp scripts/oberdiek/pdfatfi.pl /usr/local/bin/|
% \end{quote}
%
% \subsection{Package installation}
%
% \paragraph{Unpacking.} The \xfile{.dtx} file is a self-extracting
% \docstrip\ archive. The files are extracted by running the
% \xfile{.dtx} through \plainTeX:
% \begin{quote}
%   \verb|tex selinput.dtx|
% \end{quote}
%
% \paragraph{TDS.} Now the different files must be moved into
% the different directories in your installation TDS tree
% (also known as \xfile{texmf} tree):
% \begin{quote}
% \def\t{^^A
% \begin{tabular}{@{}>{\ttfamily}l@{ $\rightarrow$ }>{\ttfamily}l@{}}
%   selinput.sty & tex/latex/oberdiek/selinput.sty\\
%   selinput.pdf & doc/latex/oberdiek/selinput.pdf\\
%   test/selinput-test1.tex & doc/latex/oberdiek/test/selinput-test1.tex\\
%   test/selinput-test2.tex & doc/latex/oberdiek/test/selinput-test2.tex\\
%   test/selinput-test3.tex & doc/latex/oberdiek/test/selinput-test3.tex\\
%   test/selinput-test4.tex & doc/latex/oberdiek/test/selinput-test4.tex\\
%   test/selinput-test5.tex & doc/latex/oberdiek/test/selinput-test5.tex\\
%   selinput.dtx & source/latex/oberdiek/selinput.dtx\\
% \end{tabular}^^A
% }^^A
% \sbox0{\t}^^A
% \ifdim\wd0>\linewidth
%   \begingroup
%     \advance\linewidth by\leftmargin
%     \advance\linewidth by\rightmargin
%   \edef\x{\endgroup
%     \def\noexpand\lw{\the\linewidth}^^A
%   }\x
%   \def\lwbox{^^A
%     \leavevmode
%     \hbox to \linewidth{^^A
%       \kern-\leftmargin\relax
%       \hss
%       \usebox0
%       \hss
%       \kern-\rightmargin\relax
%     }^^A
%   }^^A
%   \ifdim\wd0>\lw
%     \sbox0{\small\t}^^A
%     \ifdim\wd0>\linewidth
%       \ifdim\wd0>\lw
%         \sbox0{\footnotesize\t}^^A
%         \ifdim\wd0>\linewidth
%           \ifdim\wd0>\lw
%             \sbox0{\scriptsize\t}^^A
%             \ifdim\wd0>\linewidth
%               \ifdim\wd0>\lw
%                 \sbox0{\tiny\t}^^A
%                 \ifdim\wd0>\linewidth
%                   \lwbox
%                 \else
%                   \usebox0
%                 \fi
%               \else
%                 \lwbox
%               \fi
%             \else
%               \usebox0
%             \fi
%           \else
%             \lwbox
%           \fi
%         \else
%           \usebox0
%         \fi
%       \else
%         \lwbox
%       \fi
%     \else
%       \usebox0
%     \fi
%   \else
%     \lwbox
%   \fi
% \else
%   \usebox0
% \fi
% \end{quote}
% If you have a \xfile{docstrip.cfg} that configures and enables \docstrip's
% TDS installing feature, then some files can already be in the right
% place, see the documentation of \docstrip.
%
% \subsection{Refresh file name databases}
%
% If your \TeX~distribution
% (\teTeX, \mikTeX, \dots) relies on file name databases, you must refresh
% these. For example, \teTeX\ users run \verb|texhash| or
% \verb|mktexlsr|.
%
% \subsection{Some details for the interested}
%
% \paragraph{Attached source.}
%
% The PDF documentation on CTAN also includes the
% \xfile{.dtx} source file. It can be extracted by
% AcrobatReader 6 or higher. Another option is \textsf{pdftk},
% e.g. unpack the file into the current directory:
% \begin{quote}
%   \verb|pdftk selinput.pdf unpack_files output .|
% \end{quote}
%
% \paragraph{Unpacking with \LaTeX.}
% The \xfile{.dtx} chooses its action depending on the format:
% \begin{description}
% \item[\plainTeX:] Run \docstrip\ and extract the files.
% \item[\LaTeX:] Generate the documentation.
% \end{description}
% If you insist on using \LaTeX\ for \docstrip\ (really,
% \docstrip\ does not need \LaTeX), then inform the autodetect routine
% about your intention:
% \begin{quote}
%   \verb|latex \let\install=y% \iffalse meta-comment
%
% File: selinput.dtx
% Version: 2007/09/09 v1.2
% Info: Semi-automatic input encoding detection
%
% Copyright (C) 2007 by
%    Heiko Oberdiek <heiko.oberdiek at googlemail.com>
%
% This work may be distributed and/or modified under the
% conditions of the LaTeX Project Public License, either
% version 1.3c of this license or (at your option) any later
% version. This version of this license is in
%    http://www.latex-project.org/lppl/lppl-1-3c.txt
% and the latest version of this license is in
%    http://www.latex-project.org/lppl.txt
% and version 1.3 or later is part of all distributions of
% LaTeX version 2005/12/01 or later.
%
% This work has the LPPL maintenance status "maintained".
%
% This Current Maintainer of this work is Heiko Oberdiek.
%
% This work consists of the main source file selinput.dtx
% and the derived files
%    selinput.sty, selinput.pdf, selinput.ins, selinput.drv,
%    selinput-test1.tex, selinput-test2.tex, selinput-test3.tex,
%    selinput-test4.tex, selinput-test5.tex.
%
% Distribution:
%    CTAN:macros/latex/contrib/oberdiek/selinput.dtx
%    CTAN:macros/latex/contrib/oberdiek/selinput.pdf
%
% Unpacking:
%    (a) If selinput.ins is present:
%           tex selinput.ins
%    (b) Without selinput.ins:
%           tex selinput.dtx
%    (c) If you insist on using LaTeX
%           latex \let\install=y\input{selinput.dtx}
%        (quote the arguments according to the demands of your shell)
%
% Documentation:
%    (a) If selinput.drv is present:
%           latex selinput.drv
%    (b) Without selinput.drv:
%           latex selinput.dtx; ...
%    The class ltxdoc loads the configuration file ltxdoc.cfg
%    if available. Here you can specify further options, e.g.
%    use A4 as paper format:
%       \PassOptionsToClass{a4paper}{article}
%
%    Programm calls to get the documentation (example):
%       pdflatex selinput.dtx
%       makeindex -s gind.ist selinput.idx
%       pdflatex selinput.dtx
%       makeindex -s gind.ist selinput.idx
%       pdflatex selinput.dtx
%
% Installation:
%    TDS:tex/latex/oberdiek/selinput.sty
%    TDS:doc/latex/oberdiek/selinput.pdf
%    TDS:doc/latex/oberdiek/test/selinput-test1.tex
%    TDS:doc/latex/oberdiek/test/selinput-test2.tex
%    TDS:doc/latex/oberdiek/test/selinput-test3.tex
%    TDS:doc/latex/oberdiek/test/selinput-test4.tex
%    TDS:doc/latex/oberdiek/test/selinput-test5.tex
%    TDS:source/latex/oberdiek/selinput.dtx
%
%<*ignore>
\begingroup
  \catcode123=1 %
  \catcode125=2 %
  \def\x{LaTeX2e}%
\expandafter\endgroup
\ifcase 0\ifx\install y1\fi\expandafter
         \ifx\csname processbatchFile\endcsname\relax\else1\fi
         \ifx\fmtname\x\else 1\fi\relax
\else\csname fi\endcsname
%</ignore>
%<*install>
\input docstrip.tex
\Msg{************************************************************************}
\Msg{* Installation}
\Msg{* Package: selinput 2007/09/09 v1.2 Semi-automatic input encoding detection (HO)}
\Msg{************************************************************************}

\keepsilent
\askforoverwritefalse

\let\MetaPrefix\relax
\preamble

This is a generated file.

Project: selinput
Version: 2007/09/09 v1.2

Copyright (C) 2007 by
   Heiko Oberdiek <heiko.oberdiek at googlemail.com>

This work may be distributed and/or modified under the
conditions of the LaTeX Project Public License, either
version 1.3c of this license or (at your option) any later
version. This version of this license is in
   http://www.latex-project.org/lppl/lppl-1-3c.txt
and the latest version of this license is in
   http://www.latex-project.org/lppl.txt
and version 1.3 or later is part of all distributions of
LaTeX version 2005/12/01 or later.

This work has the LPPL maintenance status "maintained".

This Current Maintainer of this work is Heiko Oberdiek.

This work consists of the main source file selinput.dtx
and the derived files
   selinput.sty, selinput.pdf, selinput.ins, selinput.drv,
   selinput-test1.tex, selinput-test2.tex, selinput-test3.tex,
   selinput-test4.tex, selinput-test5.tex.

\endpreamble
\let\MetaPrefix\DoubleperCent

\generate{%
  \file{selinput.ins}{\from{selinput.dtx}{install}}%
  \file{selinput.drv}{\from{selinput.dtx}{driver}}%
  \usedir{tex/latex/oberdiek}%
  \file{selinput.sty}{\from{selinput.dtx}{package}}%
  \usedir{doc/latex/oberdiek/test}%
  \file{selinput-test1.tex}{\from{selinput.dtx}{test,test1}}%
  \file{selinput-test2.tex}{\from{selinput.dtx}{test,test2}}%
  \file{selinput-test3.tex}{\from{selinput.dtx}{test,test3}}%
  \file{selinput-test4.tex}{\from{selinput.dtx}{test,test4}}%
  \file{selinput-test5.tex}{\from{selinput.dtx}{test,test5}}%
  \nopreamble
  \nopostamble
  \usedir{source/latex/oberdiek/catalogue}%
  \file{selinput.xml}{\from{selinput.dtx}{catalogue}}%
}

\catcode32=13\relax% active space
\let =\space%
\Msg{************************************************************************}
\Msg{*}
\Msg{* To finish the installation you have to move the following}
\Msg{* file into a directory searched by TeX:}
\Msg{*}
\Msg{*     selinput.sty}
\Msg{*}
\Msg{* To produce the documentation run the file `selinput.drv'}
\Msg{* through LaTeX.}
\Msg{*}
\Msg{* Happy TeXing!}
\Msg{*}
\Msg{************************************************************************}

\endbatchfile
%</install>
%<*ignore>
\fi
%</ignore>
%<*driver>
\NeedsTeXFormat{LaTeX2e}
\ProvidesFile{selinput.drv}%
  [2007/09/09 v1.2 Semi-automatic input encoding detection (HO)]%
\documentclass{ltxdoc}
\usepackage[T1]{fontenc}
\usepackage{textcomp}
\usepackage{lmodern}
\usepackage{holtxdoc}[2011/11/22]
\usepackage{color}
\begin{document}
  \DocInput{selinput.dtx}%
\end{document}
%</driver>
% \fi
%
% \CheckSum{389}
%
% \CharacterTable
%  {Upper-case    \A\B\C\D\E\F\G\H\I\J\K\L\M\N\O\P\Q\R\S\T\U\V\W\X\Y\Z
%   Lower-case    \a\b\c\d\e\f\g\h\i\j\k\l\m\n\o\p\q\r\s\t\u\v\w\x\y\z
%   Digits        \0\1\2\3\4\5\6\7\8\9
%   Exclamation   \!     Double quote  \"     Hash (number) \#
%   Dollar        \$     Percent       \%     Ampersand     \&
%   Acute accent  \'     Left paren    \(     Right paren   \)
%   Asterisk      \*     Plus          \+     Comma         \,
%   Minus         \-     Point         \.     Solidus       \/
%   Colon         \:     Semicolon     \;     Less than     \<
%   Equals        \=     Greater than  \>     Question mark \?
%   Commercial at \@     Left bracket  \[     Backslash     \\
%   Right bracket \]     Circumflex    \^     Underscore    \_
%   Grave accent  \`     Left brace    \{     Vertical bar  \|
%   Right brace   \}     Tilde         \~}
%
% \GetFileInfo{selinput.drv}
%
% \title{The \xpackage{selinput} package}
% \date{2007/09/09 v1.2}
% \author{Heiko Oberdiek\\\xemail{heiko.oberdiek at googlemail.com}}
%
% \maketitle
%
% \begin{abstract}
% This package selects the input encoding by specifying between
% input characters and their glyph names.
% \end{abstract}
%
% \tableofcontents
%
% \newcommand*{\EM}{\textcolor{blue}}
% \newcommand*{\ExampleText}{^^A
%   Umlauts:\ \EM{\"A\"O\"U\"a\"o\"u\ss}^^A
% }
%
% \section{Documentation}
%
% \subsection{Introduction}
%
% \LaTeX\ supports the direct use of 8-bit characters by means
% of package \xpackage{inputenc}. However you must know
% and specify the encoding, e.g.:
% \begin{quote}
%   \ttfamily
%   |\documentclass{article}|\\
%   |\usepackage[|\EM{latin1}|]{inputenc}|\\
%   |% or \usepackage[|\EM{utf8}|]{inputenc}|\\
%   |% or \usepackage[|\EM{??}|]{inputenc}|\\
%   |\begin{document}|\\
%   |  |\ExampleText\\
%   |\end{document}|
% \end{quote}
%
% If the document is transferred in an environment that
% uses a different encoding, then there are programs that
% convert the input characters. Examples for conversion
% of file \xfile{test.tex}
% from encoding latin1 (ISO-8859-1) to UTF-8:
% \begin{quote}
%   \ttfamily
%   |recode ISO-8859-1..UTF-8 test.tex|\\
%   |recode latin1..utf8 test.tex|\\
%   |iconv --from-code ISO-8859-1|\\
%   \hphantom{iconv}| --to-code UTF-8|\\
%   \hphantom{iconv}| --output testnew.tex|\\
%   \hphantom{iconv}| test.tex|\\
%   |iconv -f latin1 -t utf8 -o testnew.tex test.tex|
% \end{quote}
% However, the encoding name for package \xpackage{inputenc}
% must be changed:
% \begin{quote}
%    |\usepackage[latin1]{inputenc}| $\rightarrow$
%    |\usepackage[utf8]{inputenc}|\kern-4pt\relax
% \end{quote}
% Of course, unless you are using some clever editor
% that knows package \xpackage{inputenc}, recodes
% the file and adjusts the option at the same time.
% But most editors can perhaps recode the file, but
% they let the option untouched.
%
% Therefore package \xpackage{selinput} chooses another way for
% specifying the input encoding. The encoding name is not needed
% at all. Some 8-bit characters are identified by their glyph
% name and the package chooses an appropriate encoding, example:
% \begin{quote}
%   \ttfamily
%   |\documentclass{article}|\\
%   |\usepackage{selinput}|\\
%   |\SelectInputMappings{|\\
%   |  adieresis={|\EM{\"a}|}|,\\
%   |  germandbls={|\EM{\ss}|}|,\\
%   |  Euro={|\EM{\texteuro}|}|,\\
%   |}|\\
%   |\begin{document}|\\
%   |  |\ExampleText\\
%   |\end{document}|
% \end{quote}
%
% \subsection{User interface}
%
% \begin{declcs}{SelectInputEncodingList} \M{encoding list}
% \end{declcs}
% \cs{SelectInputEncodingList} expects a comma separated list of
% encoding names. Example:
% \begin{quote}
%   |\SelectInputEncodingList{utf8,ansinew,mac-roman}|
% \end{quote}
% The encodings of package \xpackage{inputenx} are used as default.
%
% \begin{declcs}{SelectInputMappings} \M{mapping pairs}
% \end{declcs}
% A mapping pair consists of a glyph name and its input
% character:
% \begin{quote}
%   |\SelectInputMappings{|\\
%   |  adieresis={|\EM{\"a}|}|,\\
%   |  germandbls={|\EM{\ss}|}|,\\
%   |  Euro={|\EM{\texteuro}|}|,\\
%   |}|
% \end{quote}
% The supported glyph names can be found in file \xfile{ix-name.def}
% of project \xpackage{inputenx} \cite{inputenx}. The names are
% basically taken from Adobe's glyphlists \cite{adobe:glyphlist,adobe:aglfn}.
% As many pairs are needed as necessary to identify the encoding.
% Example with insufficient pairs:
% \begin{quote}
%   \ttfamily
%   |\SelectInputEncodingSet{latin1,latin9}|\\
%   |\SelectInputMappings{|\\
%   |  adieresis={|\EM{\"a}|}|,\\
%   |  germandbls={|\EM{\ss}|}|,\\
%   |}|\\
%   \ExampleText| and Euro: |\EM{\textcurrency} (wrong)
% \end{quote}
% The first encoding \xoption{latin1} passes the constraints given
% by the mapping pairs. However the Euro symbol is not part of
% the encoding. Thus a mapping pair with the Euro symbol
% solves the problem. In fact the symbol alone already succeeds in selecting
% between \xoption{latin1} and \xoption{latin9}:
% \begin{quote}
%   \ttfamily
%   |\SelectInputEncodingSet{latin1,latin9}|\\
%   |\SelectInputMappings{|\\
%   |  Euro={|\EM{\texteuro}|},|\\
%   |}|\\
%   \ExampleText| and Euro: |\EM{\texteuro}
% \end{quote}
%
% \subsection{Options}
%
% \begin{description}
% \item[\xoption{warning}:]
%   The selected encoding is written
%   by \cs{PackageInfo} into the \xfile{.log} file only.
%   Option \xoption{warning} changes it to \cs{PackageWarning}.
%   Then the selected encoding is shown on the terminal as well.
% \item[\xoption{ucs}:]
%   The encoding file \xfile{utf8x} of package \cs{ucs} requires
%   that the package itself is loaded before.
%   If the package is not loaded, then the option \xoption{ucs}
%   will load package \xpackage{ucs} if the detected encoding is
%   UTF-8 (limited to the preamble, packages cannot be loaded later).
% \item[\xoption{utf8=\dots}:]
%   The option allows to specify other encoding files
%   for UTF-8 than \LaTeX's \xfile{utf8.def}. For example,
%   |utf8=utf-8| will load \xfile{utf-8.def} instead.
% \end{description}
%
% \subsection{Encodings}
%
% Package \xpackage{stringenc} \cite{stringenc}
% is used for testing the encoding. Thus the encoding
% name must be known by this package. Then the found
% encoding is loaded by \cs{inputencoding} by package
% \xpackage{inputenc} or \cs{InputEncoding} if package
% \xpackage{inputenx} is loaded.
%
% The supported encodings are present in the encoding list,
% thus usually the encoding names do not matter.
% If the list is set by \cs{SelectInputEncodingList},
% then you can use the names that work for package
% \xpackage{inputenc} and are known by package \xpackage{stringenc},
% for example: \xoption{latin1}, \xoption{x-iso-8859-1}. Encoding
% file names of package \xpackage{inputenx} are prefixed with \xfile{x-}.
% The prefix can be dropped, if package \xpackage{inputenx} is loaded.
%
% \StopEventually{
% }
%
% \section{Implementation}
%
%    \begin{macrocode}
%<*package>
\NeedsTeXFormat{LaTeX2e}
\ProvidesPackage{selinput}
  [2007/09/09 v1.2 Semi-automatic input encoding detection (HO)]%
%    \end{macrocode}
%
%    \begin{macrocode}
\RequirePackage{inputenc}
\RequirePackage{kvsetkeys}[2006/10/19]
\RequirePackage{stringenc}[2007/06/16]
\RequirePackage{kvoptions}
%    \end{macrocode}
%    \begin{macro}{\SelectInputEncodingList}
%    \begin{macrocode}
\newcommand*{\SelectInputEncodingList}{%
  \let\SIE@EncodingList\@empty
  \kvsetkeys{SelInputEnc}%
}
%    \end{macrocode}
%    \end{macro}
%    \begin{macro}{\SelectInputMappings}
%    \begin{macrocode}
\newcommand*{\SelectInputMappings}[1]{%
  \SIE@LoadNameDefs
  \let\SIE@StringUnicode\@empty
  \let\SIE@StringDest\@empty
  \kvsetkeys{SelInputMap}{#1}%
  \ifx\\SIE@StringUnicode\SIE@StringDest\\%
    \PackageError{selinput}{%
      No mappings specified%
    }\@ehc
  \else
    \EdefUnescapeHex\SIE@StringUnicode\SIE@StringUnicode
    \let\SIE@Encoding\@empty
    \@for\SIE@EncodingTest:=\SIE@EncodingList\do{%
      \ifx\SIE@Encoding\@empty
        \StringEncodingConvertTest\SIE@temp\SIE@StringUnicode
                                  {utf16be}\SIE@EncodingTest{%
          \ifx\SIE@temp\SIE@StringDest
            \let\SIE@Encoding\SIE@EncodingTest
          \fi
        }{}%
      \fi
    }%
    \ifx\SIE@Encoding\@empty
      \StringEncodingConvertTest\SIE@temp\SIE@StringDest
                                {ascii}{utf16be}{%
        \def\SIE@Encoding{ascii}%
        \SIE@Info{selinput}{%
          Matching encoding not found, but input characters%
          \MessageBreak
          are 7-bit (possibly editor replacements).%
          \MessageBreak
          Hence using ascii encoding%
        }%
      }{}%
    \fi
    \ifx\SIE@Encoding\@empty
      \PackageError{selinput}{%
        Cannot find a matching encoding%
      }\@ehd
    \else
      \ifx\SIE@Encoding\SIE@EncodingUTFviii
        \SIE@LoadUnicodePackage
        \ifx\SIE@UseUTFviii\@empty
        \else
          \let\SIE@Encoding\SIE@UseUTFviii
        \fi
      \fi
      \begingroup\expandafter\expandafter\expandafter\endgroup
      \expandafter\ifx\csname InputEncoding\endcsname\relax
        \inputencoding\SIE@Encoding
      \else
        \InputEncoding\SIE@Encoding
      \fi
      \SIE@Info{selinput}{Encoding `\SIE@Encoding' selected}%
    \fi
  \fi
}
%    \end{macrocode}
%    \end{macro}
%    \begin{macro}{\SIE@LoadNameDefs}
%    \begin{macrocode}
\def\SIE@LoadNameDefs{%
  \begingroup
    \endlinechar=\m@ne
    \catcode92=0 % backslash
    \catcode123=1 % left curly brace/beginning of group
    \catcode125=2 % right curly brace/end of group
    \catcode37=14 % percent/comment character
    \@makeother\[%
    \@makeother\]%
    \@makeother\.%
    \@makeother\(%
    \@makeother\)%
    \@makeother\/%
    \@makeother\-%
    \let\InputenxName\SelectInputDefineMapping
    \InputIfFileExists{ix-name.def}{}{%
      \PackageError{selinput}{%
        Missing `ix-name.def' (part of package `inputenx')%
      }\@ehd
    }%
    \global\let\SIE@LoadNameDefs\relax
  \endgroup
}
%    \end{macrocode}
%    \end{macro}
%    \begin{macro}{\SelectInputDefineMapping}
%    \begin{macrocode}
\newcommand*{\SelectInputDefineMapping}[1]{%
  \expandafter\gdef\csname SIE@@#1\endcsname
}
%    \end{macrocode}
%    \end{macro}
%    \begin{macrocode}
\kv@set@family@handler{SelInputMap}{%
  \@onelevel@sanitize\kv@key
  \ifx\kv@value\relax
    \PackageError{selinput}{%
      Missing input character for `\kv@key'%
    }\@ehc
  \else
    \@onelevel@sanitize\kv@value
    \ifx\kv@value\@empty
      \PackageError{selinput}{%
        Input character got lost?\MessageBreak
        Missing input character for `\kv@key'%
      }\@ehc
    \else
      \@ifundefined{SIE@@\kv@key}{%
        \PackageWarning{selinput}{%
          Missing definition for `\kv@key'%
        }%
      }{%
        \edef\SIE@StringDest{%
          \SIE@StringDest
          \kv@value
        }%
        \edef\SIE@StringUnicode{%
          \SIE@StringUnicode
          \csname SIE@@\kv@key\endcsname
        }%
      }%
    \fi
  \fi
}
%    \end{macrocode}
%    \begin{macrocode}
\kv@set@family@handler{SelInputEnc}{%
  \@onelevel@sanitize\kv@key
  \ifx\kv@value\relax
    \ifx\SIE@EncodingList\@empty
      \let\SIE@EncodingList\kv@key
    \else
      \edef\SIE@EncodingList{\SIE@EncodingList,\kv@key}%
    \fi
  \else
    \@onelevel@sanitize\kv@value
    \PackageError{selinput}{%
      Illegal key value pair (\kv@key=\kv@value)\MessagBreak
      in encoding list%
    }\@ehc
  \fi
}
%    \end{macrocode}
%
%    \begin{macro}{\SIE@LoadUnicodePackage}
%    \begin{macrocode}
\def\SIE@LoadUnicodePackage{%
  \@ifpackageloaded\SIE@UnicodePackage{}{%
    \RequirePackage\SIE@UnicodePackage\relax
  }%
  \SIE@PatchUCS
  \global\let\SIE@LoadUnicodePackage\relax
}
\let\SIE@show\show
\def\SIE@PatchUCS{%
  \AtBeginDocument{%
    \expandafter\ifx\csname ver@ucsencs.def\endcsname\relax
    \else
      \let\show\SIE@show
    \fi
  }%
}
\SIE@PatchUCS
%    \end{macrocode}
%    \end{macro}
%    \begin{macrocode}
\AtBeginDocument{%
  \let\SIE@LoadUnicodePackage\relax
}
%    \end{macrocode}
%    \begin{macro}{\SIE@EncodingUTFviii}
%    \begin{macrocode}
\def\SIE@EncodingUTFviii{utf8}
\@onelevel@sanitize\SIE@EncodingUTFviii
%    \end{macrocode}
%    \end{macro}
%    \begin{macro}{\SIE@EncodingUTFviiix}
%    \begin{macrocode}
\def\SIE@EncodingUTFviiix{utf8x}
\@onelevel@sanitize\SIE@EncodingUTFviiix
%    \end{macrocode}
%    \end{macro}
%
%    \begin{macrocode}
\let\SIE@UnicodePackage\@empty
\let\SIE@UseUTFviii\@empty
\let\SIE@Info\PackageInfo
%    \end{macrocode}
%    \begin{macrocode}
\SetupKeyvalOptions{%
  family=SelInput,%
  prefix=SelInput@%
}
\define@key{SelInput}{utf8}{%
  \def\SIE@UseUTFviii{#1}%
  \@onelevel@sanitize\SIE@UseUTFviii
}
\DeclareBoolOption{ucs}
\DeclareVoidOption{warning}{%
  \let\SIE@Info\PackageWarning
}
\ProcessKeyvalOptions{SelInput}
\ifSelInput@ucs
  \def\SIE@UnicodePackage{ucs}%
  \ifx\SIE@UseUTFviii\@empty
    \let\SIE@UseUTFviii\SIE@EncodingUTFviiix
  \fi
\else
  \ifx\SIE@UseUTFviii\@empty
    \@ifpackageloaded{ucs}{%
      \let\SIE@UseUTFviii\SIE@EncodingUTFviiix
    }{%
      \let\SIE@UseUTFviii\SIE@EncodingUTFviii
    }%
  \fi
\fi
%    \end{macrocode}
%
%    \begin{macro}{\SIE@EncodingList}
%    \begin{macrocode}
\edef\SIE@EncodingList{%
  utf8,%
  x-iso-8859-1,%
  x-iso-8859-15,%
  x-cp1252,% ansinew
  x-mac-roman,%
  x-iso-8859-2,%
  x-iso-8859-3,%
  x-iso-8859-4,%
  x-iso-8859-5,%
  x-iso-8859-6,%
  x-iso-8859-7,%
  x-iso-8859-8,%
  x-iso-8859-9,%
  x-iso-8859-10,%
  x-iso-8859-11,%
  x-iso-8859-13,%
  x-iso-8859-14,%
  x-iso-8859-15,%
  x-mac-centeuro,%
  x-mac-cyrillic,%
  x-koi8-r,%
  x-cp1250,%
  x-cp1251,%
  x-cp1257,%
  x-cp437,%
  x-cp850,%
  x-cp852,%
  x-cp855,%
  x-cp858,%
  x-cp865,%
  x-cp866,%
  x-nextstep,%
  x-dec-mcs%
}%
\@onelevel@sanitize\SIE@EncodingList
%    \end{macrocode}
%    \end{macro}
%
%    \begin{macrocode}
%</package>
%    \end{macrocode}
%
% \section{Test}
%
%    \begin{macrocode}
%<*test>
\NeedsTeXFormat{LaTeX2e}
\documentclass{minimal}
\usepackage{textcomp}
\usepackage{qstest}
%    \end{macrocode}
%    \begin{macrocode}
%<*test1|test2|test3>
\makeatletter
\let\BeginDocumentText\@empty
\def\TestEncoding#1#2{%
  \SelectInputMappings{#2}%
  \Expect*{\SIE@Encoding}{#1}%
  \Expect*{\inputencodingname}{#1}%
  \g@addto@macro\BeginDocumentText{%
    \SelectInputMappings{#2}%
    \Expect*{\SIE@Encoding}{#1}%
    \textbf{\SIE@Encoding:} %
    \kvsetkeys{test}{#2}\par
  }%
}
\def\TestKey#1#2{%
  \define@key{test}{#1}{%
    \sbox0{##1}%
    \sbox2{#2}%
    \Expect*{wd:\the\wd0, ht:\the\ht0, dp:\the\dp0}%
           *{wd:\the\wd2, ht:\the\ht2, dp:\the\dp2}%
    [#1=##1] % hash-ok
  }%
}
\RequirePackage{keyval}
\TestKey{adieresis}{\"a}
\TestKey{germandbls}{\ss}
\TestKey{Euro}{\texteuro}
\makeatother
\usepackage[
  warning,%
%<test2>  utf8=utf-8,
%<test3>  ucs,
]{selinput}
%<test1|test3>\inputencoding{ascii}
%<test2>\inputencoding{utf-8}
%<test3>\usepackage{ucs}
\begin{qstest}{preamble}{}
  \TestEncoding{x-iso-8859-15}{%
    adieresis=^^e4,%
    germandbls=^^df,%
    Euro=^^a4,%
  }%
  \TestEncoding{x-cp1252}{%
    adieresis=^^e4,%
    germandbls=^^df,%
    Euro=^^80,%
  }%
%<test1>  \TestEncoding{utf8}{%
%<test2>  \TestEncoding{utf-8}{%
%<test3>  \TestEncoding{utf8x}{%
    adieresis=^^c3^^a4,%
    germandbls=^^c3^^9f,%
%<!test2>    Euro=^^e2^^82^^ac,
  }%
\end{qstest}
%<test3>\let\ifUnicodeOptiongraphics\iffalse
\begin{document}
\begin{qstest}{document}{}
%<test3>\makeatletter
  \BeginDocumentText
\end{qstest}
%</test1|test2|test3>
%    \end{macrocode}
%
%    \begin{macrocode}
%<*test4>
\usepackage[warning,ucs]{selinput}
\SelectInputMappings{%
    adieresis=^^c3^^a4,%
    germandbls=^^c3^^9f,%
    Euro=^^e2^^82^^ac,%
}
\begin{qstest}{encoding}{}
  \Expect*{\inputencodingname}{utf8x}%
\end{qstest}
\begin{document}
  adieresis=^^c3^^a4, %
  germandbls=^^c3^^9f, %
  Euro=^^e2^^82^^ac%
%</test4>
%    \end{macrocode}
%
%    \begin{macrocode}
%<*test5>
\usepackage[warning,ucs]{selinput}
\SelectInputMappings{%
    adieresis={\"a},%
    germandbls={{\ss}},%
    Euro=\texteuro{},%
}
\begin{qstest}{encoding}{}
  \Expect*{\inputencodingname}{ascii}%
\end{qstest}
\begin{document}
  adieresis={\"a}, %
  germandbls={{\ss}}, %
  Euro=\texteuro{}%
%</test5>
%    \end{macrocode}
%
%    \begin{macrocode}
\end{document}
%</test>
%    \end{macrocode}
%
% \section{Installation}
%
% \subsection{Download}
%
% \paragraph{Package.} This package is available on
% CTAN\footnote{\url{ftp://ftp.ctan.org/tex-archive/}}:
% \begin{description}
% \item[\CTAN{macros/latex/contrib/oberdiek/selinput.dtx}] The source file.
% \item[\CTAN{macros/latex/contrib/oberdiek/selinput.pdf}] Documentation.
% \end{description}
%
%
% \paragraph{Bundle.} All the packages of the bundle `oberdiek'
% are also available in a TDS compliant ZIP archive. There
% the packages are already unpacked and the documentation files
% are generated. The files and directories obey the TDS standard.
% \begin{description}
% \item[\CTAN{install/macros/latex/contrib/oberdiek.tds.zip}]
% \end{description}
% \emph{TDS} refers to the standard ``A Directory Structure
% for \TeX\ Files'' (\CTAN{tds/tds.pdf}). Directories
% with \xfile{texmf} in their name are usually organized this way.
%
% \subsection{Bundle installation}
%
% \paragraph{Unpacking.} Unpack the \xfile{oberdiek.tds.zip} in the
% TDS tree (also known as \xfile{texmf} tree) of your choice.
% Example (linux):
% \begin{quote}
%   |unzip oberdiek.tds.zip -d ~/texmf|
% \end{quote}
%
% \paragraph{Script installation.}
% Check the directory \xfile{TDS:scripts/oberdiek/} for
% scripts that need further installation steps.
% Package \xpackage{attachfile2} comes with the Perl script
% \xfile{pdfatfi.pl} that should be installed in such a way
% that it can be called as \texttt{pdfatfi}.
% Example (linux):
% \begin{quote}
%   |chmod +x scripts/oberdiek/pdfatfi.pl|\\
%   |cp scripts/oberdiek/pdfatfi.pl /usr/local/bin/|
% \end{quote}
%
% \subsection{Package installation}
%
% \paragraph{Unpacking.} The \xfile{.dtx} file is a self-extracting
% \docstrip\ archive. The files are extracted by running the
% \xfile{.dtx} through \plainTeX:
% \begin{quote}
%   \verb|tex selinput.dtx|
% \end{quote}
%
% \paragraph{TDS.} Now the different files must be moved into
% the different directories in your installation TDS tree
% (also known as \xfile{texmf} tree):
% \begin{quote}
% \def\t{^^A
% \begin{tabular}{@{}>{\ttfamily}l@{ $\rightarrow$ }>{\ttfamily}l@{}}
%   selinput.sty & tex/latex/oberdiek/selinput.sty\\
%   selinput.pdf & doc/latex/oberdiek/selinput.pdf\\
%   test/selinput-test1.tex & doc/latex/oberdiek/test/selinput-test1.tex\\
%   test/selinput-test2.tex & doc/latex/oberdiek/test/selinput-test2.tex\\
%   test/selinput-test3.tex & doc/latex/oberdiek/test/selinput-test3.tex\\
%   test/selinput-test4.tex & doc/latex/oberdiek/test/selinput-test4.tex\\
%   test/selinput-test5.tex & doc/latex/oberdiek/test/selinput-test5.tex\\
%   selinput.dtx & source/latex/oberdiek/selinput.dtx\\
% \end{tabular}^^A
% }^^A
% \sbox0{\t}^^A
% \ifdim\wd0>\linewidth
%   \begingroup
%     \advance\linewidth by\leftmargin
%     \advance\linewidth by\rightmargin
%   \edef\x{\endgroup
%     \def\noexpand\lw{\the\linewidth}^^A
%   }\x
%   \def\lwbox{^^A
%     \leavevmode
%     \hbox to \linewidth{^^A
%       \kern-\leftmargin\relax
%       \hss
%       \usebox0
%       \hss
%       \kern-\rightmargin\relax
%     }^^A
%   }^^A
%   \ifdim\wd0>\lw
%     \sbox0{\small\t}^^A
%     \ifdim\wd0>\linewidth
%       \ifdim\wd0>\lw
%         \sbox0{\footnotesize\t}^^A
%         \ifdim\wd0>\linewidth
%           \ifdim\wd0>\lw
%             \sbox0{\scriptsize\t}^^A
%             \ifdim\wd0>\linewidth
%               \ifdim\wd0>\lw
%                 \sbox0{\tiny\t}^^A
%                 \ifdim\wd0>\linewidth
%                   \lwbox
%                 \else
%                   \usebox0
%                 \fi
%               \else
%                 \lwbox
%               \fi
%             \else
%               \usebox0
%             \fi
%           \else
%             \lwbox
%           \fi
%         \else
%           \usebox0
%         \fi
%       \else
%         \lwbox
%       \fi
%     \else
%       \usebox0
%     \fi
%   \else
%     \lwbox
%   \fi
% \else
%   \usebox0
% \fi
% \end{quote}
% If you have a \xfile{docstrip.cfg} that configures and enables \docstrip's
% TDS installing feature, then some files can already be in the right
% place, see the documentation of \docstrip.
%
% \subsection{Refresh file name databases}
%
% If your \TeX~distribution
% (\teTeX, \mikTeX, \dots) relies on file name databases, you must refresh
% these. For example, \teTeX\ users run \verb|texhash| or
% \verb|mktexlsr|.
%
% \subsection{Some details for the interested}
%
% \paragraph{Attached source.}
%
% The PDF documentation on CTAN also includes the
% \xfile{.dtx} source file. It can be extracted by
% AcrobatReader 6 or higher. Another option is \textsf{pdftk},
% e.g. unpack the file into the current directory:
% \begin{quote}
%   \verb|pdftk selinput.pdf unpack_files output .|
% \end{quote}
%
% \paragraph{Unpacking with \LaTeX.}
% The \xfile{.dtx} chooses its action depending on the format:
% \begin{description}
% \item[\plainTeX:] Run \docstrip\ and extract the files.
% \item[\LaTeX:] Generate the documentation.
% \end{description}
% If you insist on using \LaTeX\ for \docstrip\ (really,
% \docstrip\ does not need \LaTeX), then inform the autodetect routine
% about your intention:
% \begin{quote}
%   \verb|latex \let\install=y\input{selinput.dtx}|
% \end{quote}
% Do not forget to quote the argument according to the demands
% of your shell.
%
% \paragraph{Generating the documentation.}
% You can use both the \xfile{.dtx} or the \xfile{.drv} to generate
% the documentation. The process can be configured by the
% configuration file \xfile{ltxdoc.cfg}. For instance, put this
% line into this file, if you want to have A4 as paper format:
% \begin{quote}
%   \verb|\PassOptionsToClass{a4paper}{article}|
% \end{quote}
% An example follows how to generate the
% documentation with pdf\LaTeX:
% \begin{quote}
%\begin{verbatim}
%pdflatex selinput.dtx
%makeindex -s gind.ist selinput.idx
%pdflatex selinput.dtx
%makeindex -s gind.ist selinput.idx
%pdflatex selinput.dtx
%\end{verbatim}
% \end{quote}
%
% \section{Catalogue}
%
% The following XML file can be used as source for the
% \href{http://mirror.ctan.org/help/Catalogue/catalogue.html}{\TeX\ Catalogue}.
% The elements \texttt{caption} and \texttt{description} are imported
% from the original XML file from the Catalogue.
% The name of the XML file in the Catalogue is \xfile{selinput.xml}.
%    \begin{macrocode}
%<*catalogue>
<?xml version='1.0' encoding='us-ascii'?>
<!DOCTYPE entry SYSTEM 'catalogue.dtd'>
<entry datestamp='$Date$' modifier='$Author$' id='selinput'>
  <name>selinput</name>
  <caption>Semi-automatic detection of input encoding.</caption>
  <authorref id='auth:oberdiek'/>
  <copyright owner='Heiko Oberdiek' year='2007'/>
  <license type='lppl1.3'/>
  <version number='1.2'/>
  <description>
    This package selects the input encoding by specifying pairs
    of input characters and their glyph names.
    <p/>
    The package is part of the <xref refid='oberdiek'>oberdiek</xref>
    bundle.
  </description>
  <documentation details='Package documentation'
      href='ctan:/macros/latex/contrib/oberdiek/selinput.pdf'/>
  <ctan file='true' path='/macros/latex/contrib/oberdiek/selinput.dtx'/>
  <miktex location='oberdiek'/>
  <texlive location='oberdiek'/>
  <install path='/macros/latex/contrib/oberdiek/oberdiek.tds.zip'/>
</entry>
%</catalogue>
%    \end{macrocode}
%
% \begin{thebibliography}{9}
% \bibitem{inputenx}
%   Heiko Oberdiek: \textit{The \xpackage{inputenx} package};
%   2007-04-11 v1.1;
%   \CTAN{macros/latex/contrib/oberdiek/inputenx.pdf}.
%
% \bibitem{adobe:glyphlist}
%   Adobe: \textit{Adobe Glyph List};
%   2002-09-20 v2.0;
%   \url{http://partners.adobe.com/public/developer/en/opentype/glyphlist.txt}.
%
% \bibitem{adobe:aglfn}
%   Adobe: \textit{Adobe Glyph List For New Fonts};
%   2005-11-18 v1.5;
%   \url{http://partners.adobe.com/public/developer/en/opentype/aglfn13.txt}.
%
% \bibitem{stringenc}
%   Heiko Oberdiek: \textit{The \xpackage{stringenc} package};
%   2007-06-16 v1.1;
%   \CTAN{macros/latex/contrib/oberdiek/stringenc.pdf}.
%
% \end{thebibliography}
%
% \begin{History}
%   \begin{Version}{2007/06/16 v1.0}
%   \item
%     First version.
%   \end{Version}
%   \begin{Version}{2007/06/20 v1.1}
%   \item
%     Requested date for package \xpackage{stringenc} fixed.
%   \end{Version}
%   \begin{Version}{2007/09/09 v1.2}
%   \item
%     Line end fixed.
%   \end{Version}
% \end{History}
%
% \PrintIndex
%
% \Finale
\endinput
|
% \end{quote}
% Do not forget to quote the argument according to the demands
% of your shell.
%
% \paragraph{Generating the documentation.}
% You can use both the \xfile{.dtx} or the \xfile{.drv} to generate
% the documentation. The process can be configured by the
% configuration file \xfile{ltxdoc.cfg}. For instance, put this
% line into this file, if you want to have A4 as paper format:
% \begin{quote}
%   \verb|\PassOptionsToClass{a4paper}{article}|
% \end{quote}
% An example follows how to generate the
% documentation with pdf\LaTeX:
% \begin{quote}
%\begin{verbatim}
%pdflatex selinput.dtx
%makeindex -s gind.ist selinput.idx
%pdflatex selinput.dtx
%makeindex -s gind.ist selinput.idx
%pdflatex selinput.dtx
%\end{verbatim}
% \end{quote}
%
% \section{Catalogue}
%
% The following XML file can be used as source for the
% \href{http://mirror.ctan.org/help/Catalogue/catalogue.html}{\TeX\ Catalogue}.
% The elements \texttt{caption} and \texttt{description} are imported
% from the original XML file from the Catalogue.
% The name of the XML file in the Catalogue is \xfile{selinput.xml}.
%    \begin{macrocode}
%<*catalogue>
<?xml version='1.0' encoding='us-ascii'?>
<!DOCTYPE entry SYSTEM 'catalogue.dtd'>
<entry datestamp='$Date$' modifier='$Author$' id='selinput'>
  <name>selinput</name>
  <caption>Semi-automatic detection of input encoding.</caption>
  <authorref id='auth:oberdiek'/>
  <copyright owner='Heiko Oberdiek' year='2007'/>
  <license type='lppl1.3'/>
  <version number='1.2'/>
  <description>
    This package selects the input encoding by specifying pairs
    of input characters and their glyph names.
    <p/>
    The package is part of the <xref refid='oberdiek'>oberdiek</xref>
    bundle.
  </description>
  <documentation details='Package documentation'
      href='ctan:/macros/latex/contrib/oberdiek/selinput.pdf'/>
  <ctan file='true' path='/macros/latex/contrib/oberdiek/selinput.dtx'/>
  <miktex location='oberdiek'/>
  <texlive location='oberdiek'/>
  <install path='/macros/latex/contrib/oberdiek/oberdiek.tds.zip'/>
</entry>
%</catalogue>
%    \end{macrocode}
%
% \begin{thebibliography}{9}
% \bibitem{inputenx}
%   Heiko Oberdiek: \textit{The \xpackage{inputenx} package};
%   2007-04-11 v1.1;
%   \CTAN{macros/latex/contrib/oberdiek/inputenx.pdf}.
%
% \bibitem{adobe:glyphlist}
%   Adobe: \textit{Adobe Glyph List};
%   2002-09-20 v2.0;
%   \url{http://partners.adobe.com/public/developer/en/opentype/glyphlist.txt}.
%
% \bibitem{adobe:aglfn}
%   Adobe: \textit{Adobe Glyph List For New Fonts};
%   2005-11-18 v1.5;
%   \url{http://partners.adobe.com/public/developer/en/opentype/aglfn13.txt}.
%
% \bibitem{stringenc}
%   Heiko Oberdiek: \textit{The \xpackage{stringenc} package};
%   2007-06-16 v1.1;
%   \CTAN{macros/latex/contrib/oberdiek/stringenc.pdf}.
%
% \end{thebibliography}
%
% \begin{History}
%   \begin{Version}{2007/06/16 v1.0}
%   \item
%     First version.
%   \end{Version}
%   \begin{Version}{2007/06/20 v1.1}
%   \item
%     Requested date for package \xpackage{stringenc} fixed.
%   \end{Version}
%   \begin{Version}{2007/09/09 v1.2}
%   \item
%     Line end fixed.
%   \end{Version}
% \end{History}
%
% \PrintIndex
%
% \Finale
\endinput
|
% \end{quote}
% Do not forget to quote the argument according to the demands
% of your shell.
%
% \paragraph{Generating the documentation.}
% You can use both the \xfile{.dtx} or the \xfile{.drv} to generate
% the documentation. The process can be configured by the
% configuration file \xfile{ltxdoc.cfg}. For instance, put this
% line into this file, if you want to have A4 as paper format:
% \begin{quote}
%   \verb|\PassOptionsToClass{a4paper}{article}|
% \end{quote}
% An example follows how to generate the
% documentation with pdf\LaTeX:
% \begin{quote}
%\begin{verbatim}
%pdflatex selinput.dtx
%makeindex -s gind.ist selinput.idx
%pdflatex selinput.dtx
%makeindex -s gind.ist selinput.idx
%pdflatex selinput.dtx
%\end{verbatim}
% \end{quote}
%
% \section{Catalogue}
%
% The following XML file can be used as source for the
% \href{http://mirror.ctan.org/help/Catalogue/catalogue.html}{\TeX\ Catalogue}.
% The elements \texttt{caption} and \texttt{description} are imported
% from the original XML file from the Catalogue.
% The name of the XML file in the Catalogue is \xfile{selinput.xml}.
%    \begin{macrocode}
%<*catalogue>
<?xml version='1.0' encoding='us-ascii'?>
<!DOCTYPE entry SYSTEM 'catalogue.dtd'>
<entry datestamp='$Date$' modifier='$Author$' id='selinput'>
  <name>selinput</name>
  <caption>Semi-automatic detection of input encoding.</caption>
  <authorref id='auth:oberdiek'/>
  <copyright owner='Heiko Oberdiek' year='2007'/>
  <license type='lppl1.3'/>
  <version number='1.2'/>
  <description>
    This package selects the input encoding by specifying pairs
    of input characters and their glyph names.
    <p/>
    The package is part of the <xref refid='oberdiek'>oberdiek</xref>
    bundle.
  </description>
  <documentation details='Package documentation'
      href='ctan:/macros/latex/contrib/oberdiek/selinput.pdf'/>
  <ctan file='true' path='/macros/latex/contrib/oberdiek/selinput.dtx'/>
  <miktex location='oberdiek'/>
  <texlive location='oberdiek'/>
  <install path='/macros/latex/contrib/oberdiek/oberdiek.tds.zip'/>
</entry>
%</catalogue>
%    \end{macrocode}
%
% \begin{thebibliography}{9}
% \bibitem{inputenx}
%   Heiko Oberdiek: \textit{The \xpackage{inputenx} package};
%   2007-04-11 v1.1;
%   \CTAN{macros/latex/contrib/oberdiek/inputenx.pdf}.
%
% \bibitem{adobe:glyphlist}
%   Adobe: \textit{Adobe Glyph List};
%   2002-09-20 v2.0;
%   \url{http://partners.adobe.com/public/developer/en/opentype/glyphlist.txt}.
%
% \bibitem{adobe:aglfn}
%   Adobe: \textit{Adobe Glyph List For New Fonts};
%   2005-11-18 v1.5;
%   \url{http://partners.adobe.com/public/developer/en/opentype/aglfn13.txt}.
%
% \bibitem{stringenc}
%   Heiko Oberdiek: \textit{The \xpackage{stringenc} package};
%   2007-06-16 v1.1;
%   \CTAN{macros/latex/contrib/oberdiek/stringenc.pdf}.
%
% \end{thebibliography}
%
% \begin{History}
%   \begin{Version}{2007/06/16 v1.0}
%   \item
%     First version.
%   \end{Version}
%   \begin{Version}{2007/06/20 v1.1}
%   \item
%     Requested date for package \xpackage{stringenc} fixed.
%   \end{Version}
%   \begin{Version}{2007/09/09 v1.2}
%   \item
%     Line end fixed.
%   \end{Version}
% \end{History}
%
% \PrintIndex
%
% \Finale
\endinput

%        (quote the arguments according to the demands of your shell)
%
% Documentation:
%    (a) If selinput.drv is present:
%           latex selinput.drv
%    (b) Without selinput.drv:
%           latex selinput.dtx; ...
%    The class ltxdoc loads the configuration file ltxdoc.cfg
%    if available. Here you can specify further options, e.g.
%    use A4 as paper format:
%       \PassOptionsToClass{a4paper}{article}
%
%    Programm calls to get the documentation (example):
%       pdflatex selinput.dtx
%       makeindex -s gind.ist selinput.idx
%       pdflatex selinput.dtx
%       makeindex -s gind.ist selinput.idx
%       pdflatex selinput.dtx
%
% Installation:
%    TDS:tex/latex/oberdiek/selinput.sty
%    TDS:doc/latex/oberdiek/selinput.pdf
%    TDS:doc/latex/oberdiek/test/selinput-test1.tex
%    TDS:doc/latex/oberdiek/test/selinput-test2.tex
%    TDS:doc/latex/oberdiek/test/selinput-test3.tex
%    TDS:doc/latex/oberdiek/test/selinput-test4.tex
%    TDS:doc/latex/oberdiek/test/selinput-test5.tex
%    TDS:source/latex/oberdiek/selinput.dtx
%
%<*ignore>
\begingroup
  \catcode123=1 %
  \catcode125=2 %
  \def\x{LaTeX2e}%
\expandafter\endgroup
\ifcase 0\ifx\install y1\fi\expandafter
         \ifx\csname processbatchFile\endcsname\relax\else1\fi
         \ifx\fmtname\x\else 1\fi\relax
\else\csname fi\endcsname
%</ignore>
%<*install>
\input docstrip.tex
\Msg{************************************************************************}
\Msg{* Installation}
\Msg{* Package: selinput 2007/09/09 v1.2 Semi-automatic input encoding detection (HO)}
\Msg{************************************************************************}

\keepsilent
\askforoverwritefalse

\let\MetaPrefix\relax
\preamble

This is a generated file.

Project: selinput
Version: 2007/09/09 v1.2

Copyright (C) 2007 by
   Heiko Oberdiek <heiko.oberdiek at googlemail.com>

This work may be distributed and/or modified under the
conditions of the LaTeX Project Public License, either
version 1.3c of this license or (at your option) any later
version. This version of this license is in
   http://www.latex-project.org/lppl/lppl-1-3c.txt
and the latest version of this license is in
   http://www.latex-project.org/lppl.txt
and version 1.3 or later is part of all distributions of
LaTeX version 2005/12/01 or later.

This work has the LPPL maintenance status "maintained".

This Current Maintainer of this work is Heiko Oberdiek.

This work consists of the main source file selinput.dtx
and the derived files
   selinput.sty, selinput.pdf, selinput.ins, selinput.drv,
   selinput-test1.tex, selinput-test2.tex, selinput-test3.tex,
   selinput-test4.tex, selinput-test5.tex.

\endpreamble
\let\MetaPrefix\DoubleperCent

\generate{%
  \file{selinput.ins}{\from{selinput.dtx}{install}}%
  \file{selinput.drv}{\from{selinput.dtx}{driver}}%
  \usedir{tex/latex/oberdiek}%
  \file{selinput.sty}{\from{selinput.dtx}{package}}%
  \usedir{doc/latex/oberdiek/test}%
  \file{selinput-test1.tex}{\from{selinput.dtx}{test,test1}}%
  \file{selinput-test2.tex}{\from{selinput.dtx}{test,test2}}%
  \file{selinput-test3.tex}{\from{selinput.dtx}{test,test3}}%
  \file{selinput-test4.tex}{\from{selinput.dtx}{test,test4}}%
  \file{selinput-test5.tex}{\from{selinput.dtx}{test,test5}}%
  \nopreamble
  \nopostamble
  \usedir{source/latex/oberdiek/catalogue}%
  \file{selinput.xml}{\from{selinput.dtx}{catalogue}}%
}

\catcode32=13\relax% active space
\let =\space%
\Msg{************************************************************************}
\Msg{*}
\Msg{* To finish the installation you have to move the following}
\Msg{* file into a directory searched by TeX:}
\Msg{*}
\Msg{*     selinput.sty}
\Msg{*}
\Msg{* To produce the documentation run the file `selinput.drv'}
\Msg{* through LaTeX.}
\Msg{*}
\Msg{* Happy TeXing!}
\Msg{*}
\Msg{************************************************************************}

\endbatchfile
%</install>
%<*ignore>
\fi
%</ignore>
%<*driver>
\NeedsTeXFormat{LaTeX2e}
\ProvidesFile{selinput.drv}%
  [2007/09/09 v1.2 Semi-automatic input encoding detection (HO)]%
\documentclass{ltxdoc}
\usepackage[T1]{fontenc}
\usepackage{textcomp}
\usepackage{lmodern}
\usepackage{holtxdoc}[2011/11/22]
\usepackage{color}
\begin{document}
  \DocInput{selinput.dtx}%
\end{document}
%</driver>
% \fi
%
% \CheckSum{389}
%
% \CharacterTable
%  {Upper-case    \A\B\C\D\E\F\G\H\I\J\K\L\M\N\O\P\Q\R\S\T\U\V\W\X\Y\Z
%   Lower-case    \a\b\c\d\e\f\g\h\i\j\k\l\m\n\o\p\q\r\s\t\u\v\w\x\y\z
%   Digits        \0\1\2\3\4\5\6\7\8\9
%   Exclamation   \!     Double quote  \"     Hash (number) \#
%   Dollar        \$     Percent       \%     Ampersand     \&
%   Acute accent  \'     Left paren    \(     Right paren   \)
%   Asterisk      \*     Plus          \+     Comma         \,
%   Minus         \-     Point         \.     Solidus       \/
%   Colon         \:     Semicolon     \;     Less than     \<
%   Equals        \=     Greater than  \>     Question mark \?
%   Commercial at \@     Left bracket  \[     Backslash     \\
%   Right bracket \]     Circumflex    \^     Underscore    \_
%   Grave accent  \`     Left brace    \{     Vertical bar  \|
%   Right brace   \}     Tilde         \~}
%
% \GetFileInfo{selinput.drv}
%
% \title{The \xpackage{selinput} package}
% \date{2007/09/09 v1.2}
% \author{Heiko Oberdiek\\\xemail{heiko.oberdiek at googlemail.com}}
%
% \maketitle
%
% \begin{abstract}
% This package selects the input encoding by specifying between
% input characters and their glyph names.
% \end{abstract}
%
% \tableofcontents
%
% \newcommand*{\EM}{\textcolor{blue}}
% \newcommand*{\ExampleText}{^^A
%   Umlauts:\ \EM{\"A\"O\"U\"a\"o\"u\ss}^^A
% }
%
% \section{Documentation}
%
% \subsection{Introduction}
%
% \LaTeX\ supports the direct use of 8-bit characters by means
% of package \xpackage{inputenc}. However you must know
% and specify the encoding, e.g.:
% \begin{quote}
%   \ttfamily
%   |\documentclass{article}|\\
%   |\usepackage[|\EM{latin1}|]{inputenc}|\\
%   |% or \usepackage[|\EM{utf8}|]{inputenc}|\\
%   |% or \usepackage[|\EM{??}|]{inputenc}|\\
%   |\begin{document}|\\
%   |  |\ExampleText\\
%   |\end{document}|
% \end{quote}
%
% If the document is transferred in an environment that
% uses a different encoding, then there are programs that
% convert the input characters. Examples for conversion
% of file \xfile{test.tex}
% from encoding latin1 (ISO-8859-1) to UTF-8:
% \begin{quote}
%   \ttfamily
%   |recode ISO-8859-1..UTF-8 test.tex|\\
%   |recode latin1..utf8 test.tex|\\
%   |iconv --from-code ISO-8859-1|\\
%   \hphantom{iconv}| --to-code UTF-8|\\
%   \hphantom{iconv}| --output testnew.tex|\\
%   \hphantom{iconv}| test.tex|\\
%   |iconv -f latin1 -t utf8 -o testnew.tex test.tex|
% \end{quote}
% However, the encoding name for package \xpackage{inputenc}
% must be changed:
% \begin{quote}
%    |\usepackage[latin1]{inputenc}| $\rightarrow$
%    |\usepackage[utf8]{inputenc}|\kern-4pt\relax
% \end{quote}
% Of course, unless you are using some clever editor
% that knows package \xpackage{inputenc}, recodes
% the file and adjusts the option at the same time.
% But most editors can perhaps recode the file, but
% they let the option untouched.
%
% Therefore package \xpackage{selinput} chooses another way for
% specifying the input encoding. The encoding name is not needed
% at all. Some 8-bit characters are identified by their glyph
% name and the package chooses an appropriate encoding, example:
% \begin{quote}
%   \ttfamily
%   |\documentclass{article}|\\
%   |\usepackage{selinput}|\\
%   |\SelectInputMappings{|\\
%   |  adieresis={|\EM{\"a}|}|,\\
%   |  germandbls={|\EM{\ss}|}|,\\
%   |  Euro={|\EM{\texteuro}|}|,\\
%   |}|\\
%   |\begin{document}|\\
%   |  |\ExampleText\\
%   |\end{document}|
% \end{quote}
%
% \subsection{User interface}
%
% \begin{declcs}{SelectInputEncodingList} \M{encoding list}
% \end{declcs}
% \cs{SelectInputEncodingList} expects a comma separated list of
% encoding names. Example:
% \begin{quote}
%   |\SelectInputEncodingList{utf8,ansinew,mac-roman}|
% \end{quote}
% The encodings of package \xpackage{inputenx} are used as default.
%
% \begin{declcs}{SelectInputMappings} \M{mapping pairs}
% \end{declcs}
% A mapping pair consists of a glyph name and its input
% character:
% \begin{quote}
%   |\SelectInputMappings{|\\
%   |  adieresis={|\EM{\"a}|}|,\\
%   |  germandbls={|\EM{\ss}|}|,\\
%   |  Euro={|\EM{\texteuro}|}|,\\
%   |}|
% \end{quote}
% The supported glyph names can be found in file \xfile{ix-name.def}
% of project \xpackage{inputenx} \cite{inputenx}. The names are
% basically taken from Adobe's glyphlists \cite{adobe:glyphlist,adobe:aglfn}.
% As many pairs are needed as necessary to identify the encoding.
% Example with insufficient pairs:
% \begin{quote}
%   \ttfamily
%   |\SelectInputEncodingSet{latin1,latin9}|\\
%   |\SelectInputMappings{|\\
%   |  adieresis={|\EM{\"a}|}|,\\
%   |  germandbls={|\EM{\ss}|}|,\\
%   |}|\\
%   \ExampleText| and Euro: |\EM{\textcurrency} (wrong)
% \end{quote}
% The first encoding \xoption{latin1} passes the constraints given
% by the mapping pairs. However the Euro symbol is not part of
% the encoding. Thus a mapping pair with the Euro symbol
% solves the problem. In fact the symbol alone already succeeds in selecting
% between \xoption{latin1} and \xoption{latin9}:
% \begin{quote}
%   \ttfamily
%   |\SelectInputEncodingSet{latin1,latin9}|\\
%   |\SelectInputMappings{|\\
%   |  Euro={|\EM{\texteuro}|},|\\
%   |}|\\
%   \ExampleText| and Euro: |\EM{\texteuro}
% \end{quote}
%
% \subsection{Options}
%
% \begin{description}
% \item[\xoption{warning}:]
%   The selected encoding is written
%   by \cs{PackageInfo} into the \xfile{.log} file only.
%   Option \xoption{warning} changes it to \cs{PackageWarning}.
%   Then the selected encoding is shown on the terminal as well.
% \item[\xoption{ucs}:]
%   The encoding file \xfile{utf8x} of package \cs{ucs} requires
%   that the package itself is loaded before.
%   If the package is not loaded, then the option \xoption{ucs}
%   will load package \xpackage{ucs} if the detected encoding is
%   UTF-8 (limited to the preamble, packages cannot be loaded later).
% \item[\xoption{utf8=\dots}:]
%   The option allows to specify other encoding files
%   for UTF-8 than \LaTeX's \xfile{utf8.def}. For example,
%   |utf8=utf-8| will load \xfile{utf-8.def} instead.
% \end{description}
%
% \subsection{Encodings}
%
% Package \xpackage{stringenc} \cite{stringenc}
% is used for testing the encoding. Thus the encoding
% name must be known by this package. Then the found
% encoding is loaded by \cs{inputencoding} by package
% \xpackage{inputenc} or \cs{InputEncoding} if package
% \xpackage{inputenx} is loaded.
%
% The supported encodings are present in the encoding list,
% thus usually the encoding names do not matter.
% If the list is set by \cs{SelectInputEncodingList},
% then you can use the names that work for package
% \xpackage{inputenc} and are known by package \xpackage{stringenc},
% for example: \xoption{latin1}, \xoption{x-iso-8859-1}. Encoding
% file names of package \xpackage{inputenx} are prefixed with \xfile{x-}.
% The prefix can be dropped, if package \xpackage{inputenx} is loaded.
%
% \StopEventually{
% }
%
% \section{Implementation}
%
%    \begin{macrocode}
%<*package>
\NeedsTeXFormat{LaTeX2e}
\ProvidesPackage{selinput}
  [2007/09/09 v1.2 Semi-automatic input encoding detection (HO)]%
%    \end{macrocode}
%
%    \begin{macrocode}
\RequirePackage{inputenc}
\RequirePackage{kvsetkeys}[2006/10/19]
\RequirePackage{stringenc}[2007/06/16]
\RequirePackage{kvoptions}
%    \end{macrocode}
%    \begin{macro}{\SelectInputEncodingList}
%    \begin{macrocode}
\newcommand*{\SelectInputEncodingList}{%
  \let\SIE@EncodingList\@empty
  \kvsetkeys{SelInputEnc}%
}
%    \end{macrocode}
%    \end{macro}
%    \begin{macro}{\SelectInputMappings}
%    \begin{macrocode}
\newcommand*{\SelectInputMappings}[1]{%
  \SIE@LoadNameDefs
  \let\SIE@StringUnicode\@empty
  \let\SIE@StringDest\@empty
  \kvsetkeys{SelInputMap}{#1}%
  \ifx\\SIE@StringUnicode\SIE@StringDest\\%
    \PackageError{selinput}{%
      No mappings specified%
    }\@ehc
  \else
    \EdefUnescapeHex\SIE@StringUnicode\SIE@StringUnicode
    \let\SIE@Encoding\@empty
    \@for\SIE@EncodingTest:=\SIE@EncodingList\do{%
      \ifx\SIE@Encoding\@empty
        \StringEncodingConvertTest\SIE@temp\SIE@StringUnicode
                                  {utf16be}\SIE@EncodingTest{%
          \ifx\SIE@temp\SIE@StringDest
            \let\SIE@Encoding\SIE@EncodingTest
          \fi
        }{}%
      \fi
    }%
    \ifx\SIE@Encoding\@empty
      \StringEncodingConvertTest\SIE@temp\SIE@StringDest
                                {ascii}{utf16be}{%
        \def\SIE@Encoding{ascii}%
        \SIE@Info{selinput}{%
          Matching encoding not found, but input characters%
          \MessageBreak
          are 7-bit (possibly editor replacements).%
          \MessageBreak
          Hence using ascii encoding%
        }%
      }{}%
    \fi
    \ifx\SIE@Encoding\@empty
      \PackageError{selinput}{%
        Cannot find a matching encoding%
      }\@ehd
    \else
      \ifx\SIE@Encoding\SIE@EncodingUTFviii
        \SIE@LoadUnicodePackage
        \ifx\SIE@UseUTFviii\@empty
        \else
          \let\SIE@Encoding\SIE@UseUTFviii
        \fi
      \fi
      \begingroup\expandafter\expandafter\expandafter\endgroup
      \expandafter\ifx\csname InputEncoding\endcsname\relax
        \inputencoding\SIE@Encoding
      \else
        \InputEncoding\SIE@Encoding
      \fi
      \SIE@Info{selinput}{Encoding `\SIE@Encoding' selected}%
    \fi
  \fi
}
%    \end{macrocode}
%    \end{macro}
%    \begin{macro}{\SIE@LoadNameDefs}
%    \begin{macrocode}
\def\SIE@LoadNameDefs{%
  \begingroup
    \endlinechar=\m@ne
    \catcode92=0 % backslash
    \catcode123=1 % left curly brace/beginning of group
    \catcode125=2 % right curly brace/end of group
    \catcode37=14 % percent/comment character
    \@makeother\[%
    \@makeother\]%
    \@makeother\.%
    \@makeother\(%
    \@makeother\)%
    \@makeother\/%
    \@makeother\-%
    \let\InputenxName\SelectInputDefineMapping
    \InputIfFileExists{ix-name.def}{}{%
      \PackageError{selinput}{%
        Missing `ix-name.def' (part of package `inputenx')%
      }\@ehd
    }%
    \global\let\SIE@LoadNameDefs\relax
  \endgroup
}
%    \end{macrocode}
%    \end{macro}
%    \begin{macro}{\SelectInputDefineMapping}
%    \begin{macrocode}
\newcommand*{\SelectInputDefineMapping}[1]{%
  \expandafter\gdef\csname SIE@@#1\endcsname
}
%    \end{macrocode}
%    \end{macro}
%    \begin{macrocode}
\kv@set@family@handler{SelInputMap}{%
  \@onelevel@sanitize\kv@key
  \ifx\kv@value\relax
    \PackageError{selinput}{%
      Missing input character for `\kv@key'%
    }\@ehc
  \else
    \@onelevel@sanitize\kv@value
    \ifx\kv@value\@empty
      \PackageError{selinput}{%
        Input character got lost?\MessageBreak
        Missing input character for `\kv@key'%
      }\@ehc
    \else
      \@ifundefined{SIE@@\kv@key}{%
        \PackageWarning{selinput}{%
          Missing definition for `\kv@key'%
        }%
      }{%
        \edef\SIE@StringDest{%
          \SIE@StringDest
          \kv@value
        }%
        \edef\SIE@StringUnicode{%
          \SIE@StringUnicode
          \csname SIE@@\kv@key\endcsname
        }%
      }%
    \fi
  \fi
}
%    \end{macrocode}
%    \begin{macrocode}
\kv@set@family@handler{SelInputEnc}{%
  \@onelevel@sanitize\kv@key
  \ifx\kv@value\relax
    \ifx\SIE@EncodingList\@empty
      \let\SIE@EncodingList\kv@key
    \else
      \edef\SIE@EncodingList{\SIE@EncodingList,\kv@key}%
    \fi
  \else
    \@onelevel@sanitize\kv@value
    \PackageError{selinput}{%
      Illegal key value pair (\kv@key=\kv@value)\MessagBreak
      in encoding list%
    }\@ehc
  \fi
}
%    \end{macrocode}
%
%    \begin{macro}{\SIE@LoadUnicodePackage}
%    \begin{macrocode}
\def\SIE@LoadUnicodePackage{%
  \@ifpackageloaded\SIE@UnicodePackage{}{%
    \RequirePackage\SIE@UnicodePackage\relax
  }%
  \SIE@PatchUCS
  \global\let\SIE@LoadUnicodePackage\relax
}
\let\SIE@show\show
\def\SIE@PatchUCS{%
  \AtBeginDocument{%
    \expandafter\ifx\csname ver@ucsencs.def\endcsname\relax
    \else
      \let\show\SIE@show
    \fi
  }%
}
\SIE@PatchUCS
%    \end{macrocode}
%    \end{macro}
%    \begin{macrocode}
\AtBeginDocument{%
  \let\SIE@LoadUnicodePackage\relax
}
%    \end{macrocode}
%    \begin{macro}{\SIE@EncodingUTFviii}
%    \begin{macrocode}
\def\SIE@EncodingUTFviii{utf8}
\@onelevel@sanitize\SIE@EncodingUTFviii
%    \end{macrocode}
%    \end{macro}
%    \begin{macro}{\SIE@EncodingUTFviiix}
%    \begin{macrocode}
\def\SIE@EncodingUTFviiix{utf8x}
\@onelevel@sanitize\SIE@EncodingUTFviiix
%    \end{macrocode}
%    \end{macro}
%
%    \begin{macrocode}
\let\SIE@UnicodePackage\@empty
\let\SIE@UseUTFviii\@empty
\let\SIE@Info\PackageInfo
%    \end{macrocode}
%    \begin{macrocode}
\SetupKeyvalOptions{%
  family=SelInput,%
  prefix=SelInput@%
}
\define@key{SelInput}{utf8}{%
  \def\SIE@UseUTFviii{#1}%
  \@onelevel@sanitize\SIE@UseUTFviii
}
\DeclareBoolOption{ucs}
\DeclareVoidOption{warning}{%
  \let\SIE@Info\PackageWarning
}
\ProcessKeyvalOptions{SelInput}
\ifSelInput@ucs
  \def\SIE@UnicodePackage{ucs}%
  \ifx\SIE@UseUTFviii\@empty
    \let\SIE@UseUTFviii\SIE@EncodingUTFviiix
  \fi
\else
  \ifx\SIE@UseUTFviii\@empty
    \@ifpackageloaded{ucs}{%
      \let\SIE@UseUTFviii\SIE@EncodingUTFviiix
    }{%
      \let\SIE@UseUTFviii\SIE@EncodingUTFviii
    }%
  \fi
\fi
%    \end{macrocode}
%
%    \begin{macro}{\SIE@EncodingList}
%    \begin{macrocode}
\edef\SIE@EncodingList{%
  utf8,%
  x-iso-8859-1,%
  x-iso-8859-15,%
  x-cp1252,% ansinew
  x-mac-roman,%
  x-iso-8859-2,%
  x-iso-8859-3,%
  x-iso-8859-4,%
  x-iso-8859-5,%
  x-iso-8859-6,%
  x-iso-8859-7,%
  x-iso-8859-8,%
  x-iso-8859-9,%
  x-iso-8859-10,%
  x-iso-8859-11,%
  x-iso-8859-13,%
  x-iso-8859-14,%
  x-iso-8859-15,%
  x-mac-centeuro,%
  x-mac-cyrillic,%
  x-koi8-r,%
  x-cp1250,%
  x-cp1251,%
  x-cp1257,%
  x-cp437,%
  x-cp850,%
  x-cp852,%
  x-cp855,%
  x-cp858,%
  x-cp865,%
  x-cp866,%
  x-nextstep,%
  x-dec-mcs%
}%
\@onelevel@sanitize\SIE@EncodingList
%    \end{macrocode}
%    \end{macro}
%
%    \begin{macrocode}
%</package>
%    \end{macrocode}
%
% \section{Test}
%
%    \begin{macrocode}
%<*test>
\NeedsTeXFormat{LaTeX2e}
\documentclass{minimal}
\usepackage{textcomp}
\usepackage{qstest}
%    \end{macrocode}
%    \begin{macrocode}
%<*test1|test2|test3>
\makeatletter
\let\BeginDocumentText\@empty
\def\TestEncoding#1#2{%
  \SelectInputMappings{#2}%
  \Expect*{\SIE@Encoding}{#1}%
  \Expect*{\inputencodingname}{#1}%
  \g@addto@macro\BeginDocumentText{%
    \SelectInputMappings{#2}%
    \Expect*{\SIE@Encoding}{#1}%
    \textbf{\SIE@Encoding:} %
    \kvsetkeys{test}{#2}\par
  }%
}
\def\TestKey#1#2{%
  \define@key{test}{#1}{%
    \sbox0{##1}%
    \sbox2{#2}%
    \Expect*{wd:\the\wd0, ht:\the\ht0, dp:\the\dp0}%
           *{wd:\the\wd2, ht:\the\ht2, dp:\the\dp2}%
    [#1=##1] % hash-ok
  }%
}
\RequirePackage{keyval}
\TestKey{adieresis}{\"a}
\TestKey{germandbls}{\ss}
\TestKey{Euro}{\texteuro}
\makeatother
\usepackage[
  warning,%
%<test2>  utf8=utf-8,
%<test3>  ucs,
]{selinput}
%<test1|test3>\inputencoding{ascii}
%<test2>\inputencoding{utf-8}
%<test3>\usepackage{ucs}
\begin{qstest}{preamble}{}
  \TestEncoding{x-iso-8859-15}{%
    adieresis=^^e4,%
    germandbls=^^df,%
    Euro=^^a4,%
  }%
  \TestEncoding{x-cp1252}{%
    adieresis=^^e4,%
    germandbls=^^df,%
    Euro=^^80,%
  }%
%<test1>  \TestEncoding{utf8}{%
%<test2>  \TestEncoding{utf-8}{%
%<test3>  \TestEncoding{utf8x}{%
    adieresis=^^c3^^a4,%
    germandbls=^^c3^^9f,%
%<!test2>    Euro=^^e2^^82^^ac,
  }%
\end{qstest}
%<test3>\let\ifUnicodeOptiongraphics\iffalse
\begin{document}
\begin{qstest}{document}{}
%<test3>\makeatletter
  \BeginDocumentText
\end{qstest}
%</test1|test2|test3>
%    \end{macrocode}
%
%    \begin{macrocode}
%<*test4>
\usepackage[warning,ucs]{selinput}
\SelectInputMappings{%
    adieresis=^^c3^^a4,%
    germandbls=^^c3^^9f,%
    Euro=^^e2^^82^^ac,%
}
\begin{qstest}{encoding}{}
  \Expect*{\inputencodingname}{utf8x}%
\end{qstest}
\begin{document}
  adieresis=^^c3^^a4, %
  germandbls=^^c3^^9f, %
  Euro=^^e2^^82^^ac%
%</test4>
%    \end{macrocode}
%
%    \begin{macrocode}
%<*test5>
\usepackage[warning,ucs]{selinput}
\SelectInputMappings{%
    adieresis={\"a},%
    germandbls={{\ss}},%
    Euro=\texteuro{},%
}
\begin{qstest}{encoding}{}
  \Expect*{\inputencodingname}{ascii}%
\end{qstest}
\begin{document}
  adieresis={\"a}, %
  germandbls={{\ss}}, %
  Euro=\texteuro{}%
%</test5>
%    \end{macrocode}
%
%    \begin{macrocode}
\end{document}
%</test>
%    \end{macrocode}
%
% \section{Installation}
%
% \subsection{Download}
%
% \paragraph{Package.} This package is available on
% CTAN\footnote{\url{ftp://ftp.ctan.org/tex-archive/}}:
% \begin{description}
% \item[\CTAN{macros/latex/contrib/oberdiek/selinput.dtx}] The source file.
% \item[\CTAN{macros/latex/contrib/oberdiek/selinput.pdf}] Documentation.
% \end{description}
%
%
% \paragraph{Bundle.} All the packages of the bundle `oberdiek'
% are also available in a TDS compliant ZIP archive. There
% the packages are already unpacked and the documentation files
% are generated. The files and directories obey the TDS standard.
% \begin{description}
% \item[\CTAN{install/macros/latex/contrib/oberdiek.tds.zip}]
% \end{description}
% \emph{TDS} refers to the standard ``A Directory Structure
% for \TeX\ Files'' (\CTAN{tds/tds.pdf}). Directories
% with \xfile{texmf} in their name are usually organized this way.
%
% \subsection{Bundle installation}
%
% \paragraph{Unpacking.} Unpack the \xfile{oberdiek.tds.zip} in the
% TDS tree (also known as \xfile{texmf} tree) of your choice.
% Example (linux):
% \begin{quote}
%   |unzip oberdiek.tds.zip -d ~/texmf|
% \end{quote}
%
% \paragraph{Script installation.}
% Check the directory \xfile{TDS:scripts/oberdiek/} for
% scripts that need further installation steps.
% Package \xpackage{attachfile2} comes with the Perl script
% \xfile{pdfatfi.pl} that should be installed in such a way
% that it can be called as \texttt{pdfatfi}.
% Example (linux):
% \begin{quote}
%   |chmod +x scripts/oberdiek/pdfatfi.pl|\\
%   |cp scripts/oberdiek/pdfatfi.pl /usr/local/bin/|
% \end{quote}
%
% \subsection{Package installation}
%
% \paragraph{Unpacking.} The \xfile{.dtx} file is a self-extracting
% \docstrip\ archive. The files are extracted by running the
% \xfile{.dtx} through \plainTeX:
% \begin{quote}
%   \verb|tex selinput.dtx|
% \end{quote}
%
% \paragraph{TDS.} Now the different files must be moved into
% the different directories in your installation TDS tree
% (also known as \xfile{texmf} tree):
% \begin{quote}
% \def\t{^^A
% \begin{tabular}{@{}>{\ttfamily}l@{ $\rightarrow$ }>{\ttfamily}l@{}}
%   selinput.sty & tex/latex/oberdiek/selinput.sty\\
%   selinput.pdf & doc/latex/oberdiek/selinput.pdf\\
%   test/selinput-test1.tex & doc/latex/oberdiek/test/selinput-test1.tex\\
%   test/selinput-test2.tex & doc/latex/oberdiek/test/selinput-test2.tex\\
%   test/selinput-test3.tex & doc/latex/oberdiek/test/selinput-test3.tex\\
%   test/selinput-test4.tex & doc/latex/oberdiek/test/selinput-test4.tex\\
%   test/selinput-test5.tex & doc/latex/oberdiek/test/selinput-test5.tex\\
%   selinput.dtx & source/latex/oberdiek/selinput.dtx\\
% \end{tabular}^^A
% }^^A
% \sbox0{\t}^^A
% \ifdim\wd0>\linewidth
%   \begingroup
%     \advance\linewidth by\leftmargin
%     \advance\linewidth by\rightmargin
%   \edef\x{\endgroup
%     \def\noexpand\lw{\the\linewidth}^^A
%   }\x
%   \def\lwbox{^^A
%     \leavevmode
%     \hbox to \linewidth{^^A
%       \kern-\leftmargin\relax
%       \hss
%       \usebox0
%       \hss
%       \kern-\rightmargin\relax
%     }^^A
%   }^^A
%   \ifdim\wd0>\lw
%     \sbox0{\small\t}^^A
%     \ifdim\wd0>\linewidth
%       \ifdim\wd0>\lw
%         \sbox0{\footnotesize\t}^^A
%         \ifdim\wd0>\linewidth
%           \ifdim\wd0>\lw
%             \sbox0{\scriptsize\t}^^A
%             \ifdim\wd0>\linewidth
%               \ifdim\wd0>\lw
%                 \sbox0{\tiny\t}^^A
%                 \ifdim\wd0>\linewidth
%                   \lwbox
%                 \else
%                   \usebox0
%                 \fi
%               \else
%                 \lwbox
%               \fi
%             \else
%               \usebox0
%             \fi
%           \else
%             \lwbox
%           \fi
%         \else
%           \usebox0
%         \fi
%       \else
%         \lwbox
%       \fi
%     \else
%       \usebox0
%     \fi
%   \else
%     \lwbox
%   \fi
% \else
%   \usebox0
% \fi
% \end{quote}
% If you have a \xfile{docstrip.cfg} that configures and enables \docstrip's
% TDS installing feature, then some files can already be in the right
% place, see the documentation of \docstrip.
%
% \subsection{Refresh file name databases}
%
% If your \TeX~distribution
% (\teTeX, \mikTeX, \dots) relies on file name databases, you must refresh
% these. For example, \teTeX\ users run \verb|texhash| or
% \verb|mktexlsr|.
%
% \subsection{Some details for the interested}
%
% \paragraph{Attached source.}
%
% The PDF documentation on CTAN also includes the
% \xfile{.dtx} source file. It can be extracted by
% AcrobatReader 6 or higher. Another option is \textsf{pdftk},
% e.g. unpack the file into the current directory:
% \begin{quote}
%   \verb|pdftk selinput.pdf unpack_files output .|
% \end{quote}
%
% \paragraph{Unpacking with \LaTeX.}
% The \xfile{.dtx} chooses its action depending on the format:
% \begin{description}
% \item[\plainTeX:] Run \docstrip\ and extract the files.
% \item[\LaTeX:] Generate the documentation.
% \end{description}
% If you insist on using \LaTeX\ for \docstrip\ (really,
% \docstrip\ does not need \LaTeX), then inform the autodetect routine
% about your intention:
% \begin{quote}
%   \verb|latex \let\install=y% \iffalse meta-comment
%
% File: selinput.dtx
% Version: 2007/09/09 v1.2
% Info: Semi-automatic input encoding detection
%
% Copyright (C) 2007 by
%    Heiko Oberdiek <heiko.oberdiek at googlemail.com>
%
% This work may be distributed and/or modified under the
% conditions of the LaTeX Project Public License, either
% version 1.3c of this license or (at your option) any later
% version. This version of this license is in
%    http://www.latex-project.org/lppl/lppl-1-3c.txt
% and the latest version of this license is in
%    http://www.latex-project.org/lppl.txt
% and version 1.3 or later is part of all distributions of
% LaTeX version 2005/12/01 or later.
%
% This work has the LPPL maintenance status "maintained".
%
% This Current Maintainer of this work is Heiko Oberdiek.
%
% This work consists of the main source file selinput.dtx
% and the derived files
%    selinput.sty, selinput.pdf, selinput.ins, selinput.drv,
%    selinput-test1.tex, selinput-test2.tex, selinput-test3.tex,
%    selinput-test4.tex, selinput-test5.tex.
%
% Distribution:
%    CTAN:macros/latex/contrib/oberdiek/selinput.dtx
%    CTAN:macros/latex/contrib/oberdiek/selinput.pdf
%
% Unpacking:
%    (a) If selinput.ins is present:
%           tex selinput.ins
%    (b) Without selinput.ins:
%           tex selinput.dtx
%    (c) If you insist on using LaTeX
%           latex \let\install=y% \iffalse meta-comment
%
% File: selinput.dtx
% Version: 2007/09/09 v1.2
% Info: Semi-automatic input encoding detection
%
% Copyright (C) 2007 by
%    Heiko Oberdiek <heiko.oberdiek at googlemail.com>
%
% This work may be distributed and/or modified under the
% conditions of the LaTeX Project Public License, either
% version 1.3c of this license or (at your option) any later
% version. This version of this license is in
%    http://www.latex-project.org/lppl/lppl-1-3c.txt
% and the latest version of this license is in
%    http://www.latex-project.org/lppl.txt
% and version 1.3 or later is part of all distributions of
% LaTeX version 2005/12/01 or later.
%
% This work has the LPPL maintenance status "maintained".
%
% This Current Maintainer of this work is Heiko Oberdiek.
%
% This work consists of the main source file selinput.dtx
% and the derived files
%    selinput.sty, selinput.pdf, selinput.ins, selinput.drv,
%    selinput-test1.tex, selinput-test2.tex, selinput-test3.tex,
%    selinput-test4.tex, selinput-test5.tex.
%
% Distribution:
%    CTAN:macros/latex/contrib/oberdiek/selinput.dtx
%    CTAN:macros/latex/contrib/oberdiek/selinput.pdf
%
% Unpacking:
%    (a) If selinput.ins is present:
%           tex selinput.ins
%    (b) Without selinput.ins:
%           tex selinput.dtx
%    (c) If you insist on using LaTeX
%           latex \let\install=y% \iffalse meta-comment
%
% File: selinput.dtx
% Version: 2007/09/09 v1.2
% Info: Semi-automatic input encoding detection
%
% Copyright (C) 2007 by
%    Heiko Oberdiek <heiko.oberdiek at googlemail.com>
%
% This work may be distributed and/or modified under the
% conditions of the LaTeX Project Public License, either
% version 1.3c of this license or (at your option) any later
% version. This version of this license is in
%    http://www.latex-project.org/lppl/lppl-1-3c.txt
% and the latest version of this license is in
%    http://www.latex-project.org/lppl.txt
% and version 1.3 or later is part of all distributions of
% LaTeX version 2005/12/01 or later.
%
% This work has the LPPL maintenance status "maintained".
%
% This Current Maintainer of this work is Heiko Oberdiek.
%
% This work consists of the main source file selinput.dtx
% and the derived files
%    selinput.sty, selinput.pdf, selinput.ins, selinput.drv,
%    selinput-test1.tex, selinput-test2.tex, selinput-test3.tex,
%    selinput-test4.tex, selinput-test5.tex.
%
% Distribution:
%    CTAN:macros/latex/contrib/oberdiek/selinput.dtx
%    CTAN:macros/latex/contrib/oberdiek/selinput.pdf
%
% Unpacking:
%    (a) If selinput.ins is present:
%           tex selinput.ins
%    (b) Without selinput.ins:
%           tex selinput.dtx
%    (c) If you insist on using LaTeX
%           latex \let\install=y\input{selinput.dtx}
%        (quote the arguments according to the demands of your shell)
%
% Documentation:
%    (a) If selinput.drv is present:
%           latex selinput.drv
%    (b) Without selinput.drv:
%           latex selinput.dtx; ...
%    The class ltxdoc loads the configuration file ltxdoc.cfg
%    if available. Here you can specify further options, e.g.
%    use A4 as paper format:
%       \PassOptionsToClass{a4paper}{article}
%
%    Programm calls to get the documentation (example):
%       pdflatex selinput.dtx
%       makeindex -s gind.ist selinput.idx
%       pdflatex selinput.dtx
%       makeindex -s gind.ist selinput.idx
%       pdflatex selinput.dtx
%
% Installation:
%    TDS:tex/latex/oberdiek/selinput.sty
%    TDS:doc/latex/oberdiek/selinput.pdf
%    TDS:doc/latex/oberdiek/test/selinput-test1.tex
%    TDS:doc/latex/oberdiek/test/selinput-test2.tex
%    TDS:doc/latex/oberdiek/test/selinput-test3.tex
%    TDS:doc/latex/oberdiek/test/selinput-test4.tex
%    TDS:doc/latex/oberdiek/test/selinput-test5.tex
%    TDS:source/latex/oberdiek/selinput.dtx
%
%<*ignore>
\begingroup
  \catcode123=1 %
  \catcode125=2 %
  \def\x{LaTeX2e}%
\expandafter\endgroup
\ifcase 0\ifx\install y1\fi\expandafter
         \ifx\csname processbatchFile\endcsname\relax\else1\fi
         \ifx\fmtname\x\else 1\fi\relax
\else\csname fi\endcsname
%</ignore>
%<*install>
\input docstrip.tex
\Msg{************************************************************************}
\Msg{* Installation}
\Msg{* Package: selinput 2007/09/09 v1.2 Semi-automatic input encoding detection (HO)}
\Msg{************************************************************************}

\keepsilent
\askforoverwritefalse

\let\MetaPrefix\relax
\preamble

This is a generated file.

Project: selinput
Version: 2007/09/09 v1.2

Copyright (C) 2007 by
   Heiko Oberdiek <heiko.oberdiek at googlemail.com>

This work may be distributed and/or modified under the
conditions of the LaTeX Project Public License, either
version 1.3c of this license or (at your option) any later
version. This version of this license is in
   http://www.latex-project.org/lppl/lppl-1-3c.txt
and the latest version of this license is in
   http://www.latex-project.org/lppl.txt
and version 1.3 or later is part of all distributions of
LaTeX version 2005/12/01 or later.

This work has the LPPL maintenance status "maintained".

This Current Maintainer of this work is Heiko Oberdiek.

This work consists of the main source file selinput.dtx
and the derived files
   selinput.sty, selinput.pdf, selinput.ins, selinput.drv,
   selinput-test1.tex, selinput-test2.tex, selinput-test3.tex,
   selinput-test4.tex, selinput-test5.tex.

\endpreamble
\let\MetaPrefix\DoubleperCent

\generate{%
  \file{selinput.ins}{\from{selinput.dtx}{install}}%
  \file{selinput.drv}{\from{selinput.dtx}{driver}}%
  \usedir{tex/latex/oberdiek}%
  \file{selinput.sty}{\from{selinput.dtx}{package}}%
  \usedir{doc/latex/oberdiek/test}%
  \file{selinput-test1.tex}{\from{selinput.dtx}{test,test1}}%
  \file{selinput-test2.tex}{\from{selinput.dtx}{test,test2}}%
  \file{selinput-test3.tex}{\from{selinput.dtx}{test,test3}}%
  \file{selinput-test4.tex}{\from{selinput.dtx}{test,test4}}%
  \file{selinput-test5.tex}{\from{selinput.dtx}{test,test5}}%
  \nopreamble
  \nopostamble
  \usedir{source/latex/oberdiek/catalogue}%
  \file{selinput.xml}{\from{selinput.dtx}{catalogue}}%
}

\catcode32=13\relax% active space
\let =\space%
\Msg{************************************************************************}
\Msg{*}
\Msg{* To finish the installation you have to move the following}
\Msg{* file into a directory searched by TeX:}
\Msg{*}
\Msg{*     selinput.sty}
\Msg{*}
\Msg{* To produce the documentation run the file `selinput.drv'}
\Msg{* through LaTeX.}
\Msg{*}
\Msg{* Happy TeXing!}
\Msg{*}
\Msg{************************************************************************}

\endbatchfile
%</install>
%<*ignore>
\fi
%</ignore>
%<*driver>
\NeedsTeXFormat{LaTeX2e}
\ProvidesFile{selinput.drv}%
  [2007/09/09 v1.2 Semi-automatic input encoding detection (HO)]%
\documentclass{ltxdoc}
\usepackage[T1]{fontenc}
\usepackage{textcomp}
\usepackage{lmodern}
\usepackage{holtxdoc}[2011/11/22]
\usepackage{color}
\begin{document}
  \DocInput{selinput.dtx}%
\end{document}
%</driver>
% \fi
%
% \CheckSum{389}
%
% \CharacterTable
%  {Upper-case    \A\B\C\D\E\F\G\H\I\J\K\L\M\N\O\P\Q\R\S\T\U\V\W\X\Y\Z
%   Lower-case    \a\b\c\d\e\f\g\h\i\j\k\l\m\n\o\p\q\r\s\t\u\v\w\x\y\z
%   Digits        \0\1\2\3\4\5\6\7\8\9
%   Exclamation   \!     Double quote  \"     Hash (number) \#
%   Dollar        \$     Percent       \%     Ampersand     \&
%   Acute accent  \'     Left paren    \(     Right paren   \)
%   Asterisk      \*     Plus          \+     Comma         \,
%   Minus         \-     Point         \.     Solidus       \/
%   Colon         \:     Semicolon     \;     Less than     \<
%   Equals        \=     Greater than  \>     Question mark \?
%   Commercial at \@     Left bracket  \[     Backslash     \\
%   Right bracket \]     Circumflex    \^     Underscore    \_
%   Grave accent  \`     Left brace    \{     Vertical bar  \|
%   Right brace   \}     Tilde         \~}
%
% \GetFileInfo{selinput.drv}
%
% \title{The \xpackage{selinput} package}
% \date{2007/09/09 v1.2}
% \author{Heiko Oberdiek\\\xemail{heiko.oberdiek at googlemail.com}}
%
% \maketitle
%
% \begin{abstract}
% This package selects the input encoding by specifying between
% input characters and their glyph names.
% \end{abstract}
%
% \tableofcontents
%
% \newcommand*{\EM}{\textcolor{blue}}
% \newcommand*{\ExampleText}{^^A
%   Umlauts:\ \EM{\"A\"O\"U\"a\"o\"u\ss}^^A
% }
%
% \section{Documentation}
%
% \subsection{Introduction}
%
% \LaTeX\ supports the direct use of 8-bit characters by means
% of package \xpackage{inputenc}. However you must know
% and specify the encoding, e.g.:
% \begin{quote}
%   \ttfamily
%   |\documentclass{article}|\\
%   |\usepackage[|\EM{latin1}|]{inputenc}|\\
%   |% or \usepackage[|\EM{utf8}|]{inputenc}|\\
%   |% or \usepackage[|\EM{??}|]{inputenc}|\\
%   |\begin{document}|\\
%   |  |\ExampleText\\
%   |\end{document}|
% \end{quote}
%
% If the document is transferred in an environment that
% uses a different encoding, then there are programs that
% convert the input characters. Examples for conversion
% of file \xfile{test.tex}
% from encoding latin1 (ISO-8859-1) to UTF-8:
% \begin{quote}
%   \ttfamily
%   |recode ISO-8859-1..UTF-8 test.tex|\\
%   |recode latin1..utf8 test.tex|\\
%   |iconv --from-code ISO-8859-1|\\
%   \hphantom{iconv}| --to-code UTF-8|\\
%   \hphantom{iconv}| --output testnew.tex|\\
%   \hphantom{iconv}| test.tex|\\
%   |iconv -f latin1 -t utf8 -o testnew.tex test.tex|
% \end{quote}
% However, the encoding name for package \xpackage{inputenc}
% must be changed:
% \begin{quote}
%    |\usepackage[latin1]{inputenc}| $\rightarrow$
%    |\usepackage[utf8]{inputenc}|\kern-4pt\relax
% \end{quote}
% Of course, unless you are using some clever editor
% that knows package \xpackage{inputenc}, recodes
% the file and adjusts the option at the same time.
% But most editors can perhaps recode the file, but
% they let the option untouched.
%
% Therefore package \xpackage{selinput} chooses another way for
% specifying the input encoding. The encoding name is not needed
% at all. Some 8-bit characters are identified by their glyph
% name and the package chooses an appropriate encoding, example:
% \begin{quote}
%   \ttfamily
%   |\documentclass{article}|\\
%   |\usepackage{selinput}|\\
%   |\SelectInputMappings{|\\
%   |  adieresis={|\EM{\"a}|}|,\\
%   |  germandbls={|\EM{\ss}|}|,\\
%   |  Euro={|\EM{\texteuro}|}|,\\
%   |}|\\
%   |\begin{document}|\\
%   |  |\ExampleText\\
%   |\end{document}|
% \end{quote}
%
% \subsection{User interface}
%
% \begin{declcs}{SelectInputEncodingList} \M{encoding list}
% \end{declcs}
% \cs{SelectInputEncodingList} expects a comma separated list of
% encoding names. Example:
% \begin{quote}
%   |\SelectInputEncodingList{utf8,ansinew,mac-roman}|
% \end{quote}
% The encodings of package \xpackage{inputenx} are used as default.
%
% \begin{declcs}{SelectInputMappings} \M{mapping pairs}
% \end{declcs}
% A mapping pair consists of a glyph name and its input
% character:
% \begin{quote}
%   |\SelectInputMappings{|\\
%   |  adieresis={|\EM{\"a}|}|,\\
%   |  germandbls={|\EM{\ss}|}|,\\
%   |  Euro={|\EM{\texteuro}|}|,\\
%   |}|
% \end{quote}
% The supported glyph names can be found in file \xfile{ix-name.def}
% of project \xpackage{inputenx} \cite{inputenx}. The names are
% basically taken from Adobe's glyphlists \cite{adobe:glyphlist,adobe:aglfn}.
% As many pairs are needed as necessary to identify the encoding.
% Example with insufficient pairs:
% \begin{quote}
%   \ttfamily
%   |\SelectInputEncodingSet{latin1,latin9}|\\
%   |\SelectInputMappings{|\\
%   |  adieresis={|\EM{\"a}|}|,\\
%   |  germandbls={|\EM{\ss}|}|,\\
%   |}|\\
%   \ExampleText| and Euro: |\EM{\textcurrency} (wrong)
% \end{quote}
% The first encoding \xoption{latin1} passes the constraints given
% by the mapping pairs. However the Euro symbol is not part of
% the encoding. Thus a mapping pair with the Euro symbol
% solves the problem. In fact the symbol alone already succeeds in selecting
% between \xoption{latin1} and \xoption{latin9}:
% \begin{quote}
%   \ttfamily
%   |\SelectInputEncodingSet{latin1,latin9}|\\
%   |\SelectInputMappings{|\\
%   |  Euro={|\EM{\texteuro}|},|\\
%   |}|\\
%   \ExampleText| and Euro: |\EM{\texteuro}
% \end{quote}
%
% \subsection{Options}
%
% \begin{description}
% \item[\xoption{warning}:]
%   The selected encoding is written
%   by \cs{PackageInfo} into the \xfile{.log} file only.
%   Option \xoption{warning} changes it to \cs{PackageWarning}.
%   Then the selected encoding is shown on the terminal as well.
% \item[\xoption{ucs}:]
%   The encoding file \xfile{utf8x} of package \cs{ucs} requires
%   that the package itself is loaded before.
%   If the package is not loaded, then the option \xoption{ucs}
%   will load package \xpackage{ucs} if the detected encoding is
%   UTF-8 (limited to the preamble, packages cannot be loaded later).
% \item[\xoption{utf8=\dots}:]
%   The option allows to specify other encoding files
%   for UTF-8 than \LaTeX's \xfile{utf8.def}. For example,
%   |utf8=utf-8| will load \xfile{utf-8.def} instead.
% \end{description}
%
% \subsection{Encodings}
%
% Package \xpackage{stringenc} \cite{stringenc}
% is used for testing the encoding. Thus the encoding
% name must be known by this package. Then the found
% encoding is loaded by \cs{inputencoding} by package
% \xpackage{inputenc} or \cs{InputEncoding} if package
% \xpackage{inputenx} is loaded.
%
% The supported encodings are present in the encoding list,
% thus usually the encoding names do not matter.
% If the list is set by \cs{SelectInputEncodingList},
% then you can use the names that work for package
% \xpackage{inputenc} and are known by package \xpackage{stringenc},
% for example: \xoption{latin1}, \xoption{x-iso-8859-1}. Encoding
% file names of package \xpackage{inputenx} are prefixed with \xfile{x-}.
% The prefix can be dropped, if package \xpackage{inputenx} is loaded.
%
% \StopEventually{
% }
%
% \section{Implementation}
%
%    \begin{macrocode}
%<*package>
\NeedsTeXFormat{LaTeX2e}
\ProvidesPackage{selinput}
  [2007/09/09 v1.2 Semi-automatic input encoding detection (HO)]%
%    \end{macrocode}
%
%    \begin{macrocode}
\RequirePackage{inputenc}
\RequirePackage{kvsetkeys}[2006/10/19]
\RequirePackage{stringenc}[2007/06/16]
\RequirePackage{kvoptions}
%    \end{macrocode}
%    \begin{macro}{\SelectInputEncodingList}
%    \begin{macrocode}
\newcommand*{\SelectInputEncodingList}{%
  \let\SIE@EncodingList\@empty
  \kvsetkeys{SelInputEnc}%
}
%    \end{macrocode}
%    \end{macro}
%    \begin{macro}{\SelectInputMappings}
%    \begin{macrocode}
\newcommand*{\SelectInputMappings}[1]{%
  \SIE@LoadNameDefs
  \let\SIE@StringUnicode\@empty
  \let\SIE@StringDest\@empty
  \kvsetkeys{SelInputMap}{#1}%
  \ifx\\SIE@StringUnicode\SIE@StringDest\\%
    \PackageError{selinput}{%
      No mappings specified%
    }\@ehc
  \else
    \EdefUnescapeHex\SIE@StringUnicode\SIE@StringUnicode
    \let\SIE@Encoding\@empty
    \@for\SIE@EncodingTest:=\SIE@EncodingList\do{%
      \ifx\SIE@Encoding\@empty
        \StringEncodingConvertTest\SIE@temp\SIE@StringUnicode
                                  {utf16be}\SIE@EncodingTest{%
          \ifx\SIE@temp\SIE@StringDest
            \let\SIE@Encoding\SIE@EncodingTest
          \fi
        }{}%
      \fi
    }%
    \ifx\SIE@Encoding\@empty
      \StringEncodingConvertTest\SIE@temp\SIE@StringDest
                                {ascii}{utf16be}{%
        \def\SIE@Encoding{ascii}%
        \SIE@Info{selinput}{%
          Matching encoding not found, but input characters%
          \MessageBreak
          are 7-bit (possibly editor replacements).%
          \MessageBreak
          Hence using ascii encoding%
        }%
      }{}%
    \fi
    \ifx\SIE@Encoding\@empty
      \PackageError{selinput}{%
        Cannot find a matching encoding%
      }\@ehd
    \else
      \ifx\SIE@Encoding\SIE@EncodingUTFviii
        \SIE@LoadUnicodePackage
        \ifx\SIE@UseUTFviii\@empty
        \else
          \let\SIE@Encoding\SIE@UseUTFviii
        \fi
      \fi
      \begingroup\expandafter\expandafter\expandafter\endgroup
      \expandafter\ifx\csname InputEncoding\endcsname\relax
        \inputencoding\SIE@Encoding
      \else
        \InputEncoding\SIE@Encoding
      \fi
      \SIE@Info{selinput}{Encoding `\SIE@Encoding' selected}%
    \fi
  \fi
}
%    \end{macrocode}
%    \end{macro}
%    \begin{macro}{\SIE@LoadNameDefs}
%    \begin{macrocode}
\def\SIE@LoadNameDefs{%
  \begingroup
    \endlinechar=\m@ne
    \catcode92=0 % backslash
    \catcode123=1 % left curly brace/beginning of group
    \catcode125=2 % right curly brace/end of group
    \catcode37=14 % percent/comment character
    \@makeother\[%
    \@makeother\]%
    \@makeother\.%
    \@makeother\(%
    \@makeother\)%
    \@makeother\/%
    \@makeother\-%
    \let\InputenxName\SelectInputDefineMapping
    \InputIfFileExists{ix-name.def}{}{%
      \PackageError{selinput}{%
        Missing `ix-name.def' (part of package `inputenx')%
      }\@ehd
    }%
    \global\let\SIE@LoadNameDefs\relax
  \endgroup
}
%    \end{macrocode}
%    \end{macro}
%    \begin{macro}{\SelectInputDefineMapping}
%    \begin{macrocode}
\newcommand*{\SelectInputDefineMapping}[1]{%
  \expandafter\gdef\csname SIE@@#1\endcsname
}
%    \end{macrocode}
%    \end{macro}
%    \begin{macrocode}
\kv@set@family@handler{SelInputMap}{%
  \@onelevel@sanitize\kv@key
  \ifx\kv@value\relax
    \PackageError{selinput}{%
      Missing input character for `\kv@key'%
    }\@ehc
  \else
    \@onelevel@sanitize\kv@value
    \ifx\kv@value\@empty
      \PackageError{selinput}{%
        Input character got lost?\MessageBreak
        Missing input character for `\kv@key'%
      }\@ehc
    \else
      \@ifundefined{SIE@@\kv@key}{%
        \PackageWarning{selinput}{%
          Missing definition for `\kv@key'%
        }%
      }{%
        \edef\SIE@StringDest{%
          \SIE@StringDest
          \kv@value
        }%
        \edef\SIE@StringUnicode{%
          \SIE@StringUnicode
          \csname SIE@@\kv@key\endcsname
        }%
      }%
    \fi
  \fi
}
%    \end{macrocode}
%    \begin{macrocode}
\kv@set@family@handler{SelInputEnc}{%
  \@onelevel@sanitize\kv@key
  \ifx\kv@value\relax
    \ifx\SIE@EncodingList\@empty
      \let\SIE@EncodingList\kv@key
    \else
      \edef\SIE@EncodingList{\SIE@EncodingList,\kv@key}%
    \fi
  \else
    \@onelevel@sanitize\kv@value
    \PackageError{selinput}{%
      Illegal key value pair (\kv@key=\kv@value)\MessagBreak
      in encoding list%
    }\@ehc
  \fi
}
%    \end{macrocode}
%
%    \begin{macro}{\SIE@LoadUnicodePackage}
%    \begin{macrocode}
\def\SIE@LoadUnicodePackage{%
  \@ifpackageloaded\SIE@UnicodePackage{}{%
    \RequirePackage\SIE@UnicodePackage\relax
  }%
  \SIE@PatchUCS
  \global\let\SIE@LoadUnicodePackage\relax
}
\let\SIE@show\show
\def\SIE@PatchUCS{%
  \AtBeginDocument{%
    \expandafter\ifx\csname ver@ucsencs.def\endcsname\relax
    \else
      \let\show\SIE@show
    \fi
  }%
}
\SIE@PatchUCS
%    \end{macrocode}
%    \end{macro}
%    \begin{macrocode}
\AtBeginDocument{%
  \let\SIE@LoadUnicodePackage\relax
}
%    \end{macrocode}
%    \begin{macro}{\SIE@EncodingUTFviii}
%    \begin{macrocode}
\def\SIE@EncodingUTFviii{utf8}
\@onelevel@sanitize\SIE@EncodingUTFviii
%    \end{macrocode}
%    \end{macro}
%    \begin{macro}{\SIE@EncodingUTFviiix}
%    \begin{macrocode}
\def\SIE@EncodingUTFviiix{utf8x}
\@onelevel@sanitize\SIE@EncodingUTFviiix
%    \end{macrocode}
%    \end{macro}
%
%    \begin{macrocode}
\let\SIE@UnicodePackage\@empty
\let\SIE@UseUTFviii\@empty
\let\SIE@Info\PackageInfo
%    \end{macrocode}
%    \begin{macrocode}
\SetupKeyvalOptions{%
  family=SelInput,%
  prefix=SelInput@%
}
\define@key{SelInput}{utf8}{%
  \def\SIE@UseUTFviii{#1}%
  \@onelevel@sanitize\SIE@UseUTFviii
}
\DeclareBoolOption{ucs}
\DeclareVoidOption{warning}{%
  \let\SIE@Info\PackageWarning
}
\ProcessKeyvalOptions{SelInput}
\ifSelInput@ucs
  \def\SIE@UnicodePackage{ucs}%
  \ifx\SIE@UseUTFviii\@empty
    \let\SIE@UseUTFviii\SIE@EncodingUTFviiix
  \fi
\else
  \ifx\SIE@UseUTFviii\@empty
    \@ifpackageloaded{ucs}{%
      \let\SIE@UseUTFviii\SIE@EncodingUTFviiix
    }{%
      \let\SIE@UseUTFviii\SIE@EncodingUTFviii
    }%
  \fi
\fi
%    \end{macrocode}
%
%    \begin{macro}{\SIE@EncodingList}
%    \begin{macrocode}
\edef\SIE@EncodingList{%
  utf8,%
  x-iso-8859-1,%
  x-iso-8859-15,%
  x-cp1252,% ansinew
  x-mac-roman,%
  x-iso-8859-2,%
  x-iso-8859-3,%
  x-iso-8859-4,%
  x-iso-8859-5,%
  x-iso-8859-6,%
  x-iso-8859-7,%
  x-iso-8859-8,%
  x-iso-8859-9,%
  x-iso-8859-10,%
  x-iso-8859-11,%
  x-iso-8859-13,%
  x-iso-8859-14,%
  x-iso-8859-15,%
  x-mac-centeuro,%
  x-mac-cyrillic,%
  x-koi8-r,%
  x-cp1250,%
  x-cp1251,%
  x-cp1257,%
  x-cp437,%
  x-cp850,%
  x-cp852,%
  x-cp855,%
  x-cp858,%
  x-cp865,%
  x-cp866,%
  x-nextstep,%
  x-dec-mcs%
}%
\@onelevel@sanitize\SIE@EncodingList
%    \end{macrocode}
%    \end{macro}
%
%    \begin{macrocode}
%</package>
%    \end{macrocode}
%
% \section{Test}
%
%    \begin{macrocode}
%<*test>
\NeedsTeXFormat{LaTeX2e}
\documentclass{minimal}
\usepackage{textcomp}
\usepackage{qstest}
%    \end{macrocode}
%    \begin{macrocode}
%<*test1|test2|test3>
\makeatletter
\let\BeginDocumentText\@empty
\def\TestEncoding#1#2{%
  \SelectInputMappings{#2}%
  \Expect*{\SIE@Encoding}{#1}%
  \Expect*{\inputencodingname}{#1}%
  \g@addto@macro\BeginDocumentText{%
    \SelectInputMappings{#2}%
    \Expect*{\SIE@Encoding}{#1}%
    \textbf{\SIE@Encoding:} %
    \kvsetkeys{test}{#2}\par
  }%
}
\def\TestKey#1#2{%
  \define@key{test}{#1}{%
    \sbox0{##1}%
    \sbox2{#2}%
    \Expect*{wd:\the\wd0, ht:\the\ht0, dp:\the\dp0}%
           *{wd:\the\wd2, ht:\the\ht2, dp:\the\dp2}%
    [#1=##1] % hash-ok
  }%
}
\RequirePackage{keyval}
\TestKey{adieresis}{\"a}
\TestKey{germandbls}{\ss}
\TestKey{Euro}{\texteuro}
\makeatother
\usepackage[
  warning,%
%<test2>  utf8=utf-8,
%<test3>  ucs,
]{selinput}
%<test1|test3>\inputencoding{ascii}
%<test2>\inputencoding{utf-8}
%<test3>\usepackage{ucs}
\begin{qstest}{preamble}{}
  \TestEncoding{x-iso-8859-15}{%
    adieresis=^^e4,%
    germandbls=^^df,%
    Euro=^^a4,%
  }%
  \TestEncoding{x-cp1252}{%
    adieresis=^^e4,%
    germandbls=^^df,%
    Euro=^^80,%
  }%
%<test1>  \TestEncoding{utf8}{%
%<test2>  \TestEncoding{utf-8}{%
%<test3>  \TestEncoding{utf8x}{%
    adieresis=^^c3^^a4,%
    germandbls=^^c3^^9f,%
%<!test2>    Euro=^^e2^^82^^ac,
  }%
\end{qstest}
%<test3>\let\ifUnicodeOptiongraphics\iffalse
\begin{document}
\begin{qstest}{document}{}
%<test3>\makeatletter
  \BeginDocumentText
\end{qstest}
%</test1|test2|test3>
%    \end{macrocode}
%
%    \begin{macrocode}
%<*test4>
\usepackage[warning,ucs]{selinput}
\SelectInputMappings{%
    adieresis=^^c3^^a4,%
    germandbls=^^c3^^9f,%
    Euro=^^e2^^82^^ac,%
}
\begin{qstest}{encoding}{}
  \Expect*{\inputencodingname}{utf8x}%
\end{qstest}
\begin{document}
  adieresis=^^c3^^a4, %
  germandbls=^^c3^^9f, %
  Euro=^^e2^^82^^ac%
%</test4>
%    \end{macrocode}
%
%    \begin{macrocode}
%<*test5>
\usepackage[warning,ucs]{selinput}
\SelectInputMappings{%
    adieresis={\"a},%
    germandbls={{\ss}},%
    Euro=\texteuro{},%
}
\begin{qstest}{encoding}{}
  \Expect*{\inputencodingname}{ascii}%
\end{qstest}
\begin{document}
  adieresis={\"a}, %
  germandbls={{\ss}}, %
  Euro=\texteuro{}%
%</test5>
%    \end{macrocode}
%
%    \begin{macrocode}
\end{document}
%</test>
%    \end{macrocode}
%
% \section{Installation}
%
% \subsection{Download}
%
% \paragraph{Package.} This package is available on
% CTAN\footnote{\url{ftp://ftp.ctan.org/tex-archive/}}:
% \begin{description}
% \item[\CTAN{macros/latex/contrib/oberdiek/selinput.dtx}] The source file.
% \item[\CTAN{macros/latex/contrib/oberdiek/selinput.pdf}] Documentation.
% \end{description}
%
%
% \paragraph{Bundle.} All the packages of the bundle `oberdiek'
% are also available in a TDS compliant ZIP archive. There
% the packages are already unpacked and the documentation files
% are generated. The files and directories obey the TDS standard.
% \begin{description}
% \item[\CTAN{install/macros/latex/contrib/oberdiek.tds.zip}]
% \end{description}
% \emph{TDS} refers to the standard ``A Directory Structure
% for \TeX\ Files'' (\CTAN{tds/tds.pdf}). Directories
% with \xfile{texmf} in their name are usually organized this way.
%
% \subsection{Bundle installation}
%
% \paragraph{Unpacking.} Unpack the \xfile{oberdiek.tds.zip} in the
% TDS tree (also known as \xfile{texmf} tree) of your choice.
% Example (linux):
% \begin{quote}
%   |unzip oberdiek.tds.zip -d ~/texmf|
% \end{quote}
%
% \paragraph{Script installation.}
% Check the directory \xfile{TDS:scripts/oberdiek/} for
% scripts that need further installation steps.
% Package \xpackage{attachfile2} comes with the Perl script
% \xfile{pdfatfi.pl} that should be installed in such a way
% that it can be called as \texttt{pdfatfi}.
% Example (linux):
% \begin{quote}
%   |chmod +x scripts/oberdiek/pdfatfi.pl|\\
%   |cp scripts/oberdiek/pdfatfi.pl /usr/local/bin/|
% \end{quote}
%
% \subsection{Package installation}
%
% \paragraph{Unpacking.} The \xfile{.dtx} file is a self-extracting
% \docstrip\ archive. The files are extracted by running the
% \xfile{.dtx} through \plainTeX:
% \begin{quote}
%   \verb|tex selinput.dtx|
% \end{quote}
%
% \paragraph{TDS.} Now the different files must be moved into
% the different directories in your installation TDS tree
% (also known as \xfile{texmf} tree):
% \begin{quote}
% \def\t{^^A
% \begin{tabular}{@{}>{\ttfamily}l@{ $\rightarrow$ }>{\ttfamily}l@{}}
%   selinput.sty & tex/latex/oberdiek/selinput.sty\\
%   selinput.pdf & doc/latex/oberdiek/selinput.pdf\\
%   test/selinput-test1.tex & doc/latex/oberdiek/test/selinput-test1.tex\\
%   test/selinput-test2.tex & doc/latex/oberdiek/test/selinput-test2.tex\\
%   test/selinput-test3.tex & doc/latex/oberdiek/test/selinput-test3.tex\\
%   test/selinput-test4.tex & doc/latex/oberdiek/test/selinput-test4.tex\\
%   test/selinput-test5.tex & doc/latex/oberdiek/test/selinput-test5.tex\\
%   selinput.dtx & source/latex/oberdiek/selinput.dtx\\
% \end{tabular}^^A
% }^^A
% \sbox0{\t}^^A
% \ifdim\wd0>\linewidth
%   \begingroup
%     \advance\linewidth by\leftmargin
%     \advance\linewidth by\rightmargin
%   \edef\x{\endgroup
%     \def\noexpand\lw{\the\linewidth}^^A
%   }\x
%   \def\lwbox{^^A
%     \leavevmode
%     \hbox to \linewidth{^^A
%       \kern-\leftmargin\relax
%       \hss
%       \usebox0
%       \hss
%       \kern-\rightmargin\relax
%     }^^A
%   }^^A
%   \ifdim\wd0>\lw
%     \sbox0{\small\t}^^A
%     \ifdim\wd0>\linewidth
%       \ifdim\wd0>\lw
%         \sbox0{\footnotesize\t}^^A
%         \ifdim\wd0>\linewidth
%           \ifdim\wd0>\lw
%             \sbox0{\scriptsize\t}^^A
%             \ifdim\wd0>\linewidth
%               \ifdim\wd0>\lw
%                 \sbox0{\tiny\t}^^A
%                 \ifdim\wd0>\linewidth
%                   \lwbox
%                 \else
%                   \usebox0
%                 \fi
%               \else
%                 \lwbox
%               \fi
%             \else
%               \usebox0
%             \fi
%           \else
%             \lwbox
%           \fi
%         \else
%           \usebox0
%         \fi
%       \else
%         \lwbox
%       \fi
%     \else
%       \usebox0
%     \fi
%   \else
%     \lwbox
%   \fi
% \else
%   \usebox0
% \fi
% \end{quote}
% If you have a \xfile{docstrip.cfg} that configures and enables \docstrip's
% TDS installing feature, then some files can already be in the right
% place, see the documentation of \docstrip.
%
% \subsection{Refresh file name databases}
%
% If your \TeX~distribution
% (\teTeX, \mikTeX, \dots) relies on file name databases, you must refresh
% these. For example, \teTeX\ users run \verb|texhash| or
% \verb|mktexlsr|.
%
% \subsection{Some details for the interested}
%
% \paragraph{Attached source.}
%
% The PDF documentation on CTAN also includes the
% \xfile{.dtx} source file. It can be extracted by
% AcrobatReader 6 or higher. Another option is \textsf{pdftk},
% e.g. unpack the file into the current directory:
% \begin{quote}
%   \verb|pdftk selinput.pdf unpack_files output .|
% \end{quote}
%
% \paragraph{Unpacking with \LaTeX.}
% The \xfile{.dtx} chooses its action depending on the format:
% \begin{description}
% \item[\plainTeX:] Run \docstrip\ and extract the files.
% \item[\LaTeX:] Generate the documentation.
% \end{description}
% If you insist on using \LaTeX\ for \docstrip\ (really,
% \docstrip\ does not need \LaTeX), then inform the autodetect routine
% about your intention:
% \begin{quote}
%   \verb|latex \let\install=y\input{selinput.dtx}|
% \end{quote}
% Do not forget to quote the argument according to the demands
% of your shell.
%
% \paragraph{Generating the documentation.}
% You can use both the \xfile{.dtx} or the \xfile{.drv} to generate
% the documentation. The process can be configured by the
% configuration file \xfile{ltxdoc.cfg}. For instance, put this
% line into this file, if you want to have A4 as paper format:
% \begin{quote}
%   \verb|\PassOptionsToClass{a4paper}{article}|
% \end{quote}
% An example follows how to generate the
% documentation with pdf\LaTeX:
% \begin{quote}
%\begin{verbatim}
%pdflatex selinput.dtx
%makeindex -s gind.ist selinput.idx
%pdflatex selinput.dtx
%makeindex -s gind.ist selinput.idx
%pdflatex selinput.dtx
%\end{verbatim}
% \end{quote}
%
% \section{Catalogue}
%
% The following XML file can be used as source for the
% \href{http://mirror.ctan.org/help/Catalogue/catalogue.html}{\TeX\ Catalogue}.
% The elements \texttt{caption} and \texttt{description} are imported
% from the original XML file from the Catalogue.
% The name of the XML file in the Catalogue is \xfile{selinput.xml}.
%    \begin{macrocode}
%<*catalogue>
<?xml version='1.0' encoding='us-ascii'?>
<!DOCTYPE entry SYSTEM 'catalogue.dtd'>
<entry datestamp='$Date$' modifier='$Author$' id='selinput'>
  <name>selinput</name>
  <caption>Semi-automatic detection of input encoding.</caption>
  <authorref id='auth:oberdiek'/>
  <copyright owner='Heiko Oberdiek' year='2007'/>
  <license type='lppl1.3'/>
  <version number='1.2'/>
  <description>
    This package selects the input encoding by specifying pairs
    of input characters and their glyph names.
    <p/>
    The package is part of the <xref refid='oberdiek'>oberdiek</xref>
    bundle.
  </description>
  <documentation details='Package documentation'
      href='ctan:/macros/latex/contrib/oberdiek/selinput.pdf'/>
  <ctan file='true' path='/macros/latex/contrib/oberdiek/selinput.dtx'/>
  <miktex location='oberdiek'/>
  <texlive location='oberdiek'/>
  <install path='/macros/latex/contrib/oberdiek/oberdiek.tds.zip'/>
</entry>
%</catalogue>
%    \end{macrocode}
%
% \begin{thebibliography}{9}
% \bibitem{inputenx}
%   Heiko Oberdiek: \textit{The \xpackage{inputenx} package};
%   2007-04-11 v1.1;
%   \CTAN{macros/latex/contrib/oberdiek/inputenx.pdf}.
%
% \bibitem{adobe:glyphlist}
%   Adobe: \textit{Adobe Glyph List};
%   2002-09-20 v2.0;
%   \url{http://partners.adobe.com/public/developer/en/opentype/glyphlist.txt}.
%
% \bibitem{adobe:aglfn}
%   Adobe: \textit{Adobe Glyph List For New Fonts};
%   2005-11-18 v1.5;
%   \url{http://partners.adobe.com/public/developer/en/opentype/aglfn13.txt}.
%
% \bibitem{stringenc}
%   Heiko Oberdiek: \textit{The \xpackage{stringenc} package};
%   2007-06-16 v1.1;
%   \CTAN{macros/latex/contrib/oberdiek/stringenc.pdf}.
%
% \end{thebibliography}
%
% \begin{History}
%   \begin{Version}{2007/06/16 v1.0}
%   \item
%     First version.
%   \end{Version}
%   \begin{Version}{2007/06/20 v1.1}
%   \item
%     Requested date for package \xpackage{stringenc} fixed.
%   \end{Version}
%   \begin{Version}{2007/09/09 v1.2}
%   \item
%     Line end fixed.
%   \end{Version}
% \end{History}
%
% \PrintIndex
%
% \Finale
\endinput

%        (quote the arguments according to the demands of your shell)
%
% Documentation:
%    (a) If selinput.drv is present:
%           latex selinput.drv
%    (b) Without selinput.drv:
%           latex selinput.dtx; ...
%    The class ltxdoc loads the configuration file ltxdoc.cfg
%    if available. Here you can specify further options, e.g.
%    use A4 as paper format:
%       \PassOptionsToClass{a4paper}{article}
%
%    Programm calls to get the documentation (example):
%       pdflatex selinput.dtx
%       makeindex -s gind.ist selinput.idx
%       pdflatex selinput.dtx
%       makeindex -s gind.ist selinput.idx
%       pdflatex selinput.dtx
%
% Installation:
%    TDS:tex/latex/oberdiek/selinput.sty
%    TDS:doc/latex/oberdiek/selinput.pdf
%    TDS:doc/latex/oberdiek/test/selinput-test1.tex
%    TDS:doc/latex/oberdiek/test/selinput-test2.tex
%    TDS:doc/latex/oberdiek/test/selinput-test3.tex
%    TDS:doc/latex/oberdiek/test/selinput-test4.tex
%    TDS:doc/latex/oberdiek/test/selinput-test5.tex
%    TDS:source/latex/oberdiek/selinput.dtx
%
%<*ignore>
\begingroup
  \catcode123=1 %
  \catcode125=2 %
  \def\x{LaTeX2e}%
\expandafter\endgroup
\ifcase 0\ifx\install y1\fi\expandafter
         \ifx\csname processbatchFile\endcsname\relax\else1\fi
         \ifx\fmtname\x\else 1\fi\relax
\else\csname fi\endcsname
%</ignore>
%<*install>
\input docstrip.tex
\Msg{************************************************************************}
\Msg{* Installation}
\Msg{* Package: selinput 2007/09/09 v1.2 Semi-automatic input encoding detection (HO)}
\Msg{************************************************************************}

\keepsilent
\askforoverwritefalse

\let\MetaPrefix\relax
\preamble

This is a generated file.

Project: selinput
Version: 2007/09/09 v1.2

Copyright (C) 2007 by
   Heiko Oberdiek <heiko.oberdiek at googlemail.com>

This work may be distributed and/or modified under the
conditions of the LaTeX Project Public License, either
version 1.3c of this license or (at your option) any later
version. This version of this license is in
   http://www.latex-project.org/lppl/lppl-1-3c.txt
and the latest version of this license is in
   http://www.latex-project.org/lppl.txt
and version 1.3 or later is part of all distributions of
LaTeX version 2005/12/01 or later.

This work has the LPPL maintenance status "maintained".

This Current Maintainer of this work is Heiko Oberdiek.

This work consists of the main source file selinput.dtx
and the derived files
   selinput.sty, selinput.pdf, selinput.ins, selinput.drv,
   selinput-test1.tex, selinput-test2.tex, selinput-test3.tex,
   selinput-test4.tex, selinput-test5.tex.

\endpreamble
\let\MetaPrefix\DoubleperCent

\generate{%
  \file{selinput.ins}{\from{selinput.dtx}{install}}%
  \file{selinput.drv}{\from{selinput.dtx}{driver}}%
  \usedir{tex/latex/oberdiek}%
  \file{selinput.sty}{\from{selinput.dtx}{package}}%
  \usedir{doc/latex/oberdiek/test}%
  \file{selinput-test1.tex}{\from{selinput.dtx}{test,test1}}%
  \file{selinput-test2.tex}{\from{selinput.dtx}{test,test2}}%
  \file{selinput-test3.tex}{\from{selinput.dtx}{test,test3}}%
  \file{selinput-test4.tex}{\from{selinput.dtx}{test,test4}}%
  \file{selinput-test5.tex}{\from{selinput.dtx}{test,test5}}%
  \nopreamble
  \nopostamble
  \usedir{source/latex/oberdiek/catalogue}%
  \file{selinput.xml}{\from{selinput.dtx}{catalogue}}%
}

\catcode32=13\relax% active space
\let =\space%
\Msg{************************************************************************}
\Msg{*}
\Msg{* To finish the installation you have to move the following}
\Msg{* file into a directory searched by TeX:}
\Msg{*}
\Msg{*     selinput.sty}
\Msg{*}
\Msg{* To produce the documentation run the file `selinput.drv'}
\Msg{* through LaTeX.}
\Msg{*}
\Msg{* Happy TeXing!}
\Msg{*}
\Msg{************************************************************************}

\endbatchfile
%</install>
%<*ignore>
\fi
%</ignore>
%<*driver>
\NeedsTeXFormat{LaTeX2e}
\ProvidesFile{selinput.drv}%
  [2007/09/09 v1.2 Semi-automatic input encoding detection (HO)]%
\documentclass{ltxdoc}
\usepackage[T1]{fontenc}
\usepackage{textcomp}
\usepackage{lmodern}
\usepackage{holtxdoc}[2011/11/22]
\usepackage{color}
\begin{document}
  \DocInput{selinput.dtx}%
\end{document}
%</driver>
% \fi
%
% \CheckSum{389}
%
% \CharacterTable
%  {Upper-case    \A\B\C\D\E\F\G\H\I\J\K\L\M\N\O\P\Q\R\S\T\U\V\W\X\Y\Z
%   Lower-case    \a\b\c\d\e\f\g\h\i\j\k\l\m\n\o\p\q\r\s\t\u\v\w\x\y\z
%   Digits        \0\1\2\3\4\5\6\7\8\9
%   Exclamation   \!     Double quote  \"     Hash (number) \#
%   Dollar        \$     Percent       \%     Ampersand     \&
%   Acute accent  \'     Left paren    \(     Right paren   \)
%   Asterisk      \*     Plus          \+     Comma         \,
%   Minus         \-     Point         \.     Solidus       \/
%   Colon         \:     Semicolon     \;     Less than     \<
%   Equals        \=     Greater than  \>     Question mark \?
%   Commercial at \@     Left bracket  \[     Backslash     \\
%   Right bracket \]     Circumflex    \^     Underscore    \_
%   Grave accent  \`     Left brace    \{     Vertical bar  \|
%   Right brace   \}     Tilde         \~}
%
% \GetFileInfo{selinput.drv}
%
% \title{The \xpackage{selinput} package}
% \date{2007/09/09 v1.2}
% \author{Heiko Oberdiek\\\xemail{heiko.oberdiek at googlemail.com}}
%
% \maketitle
%
% \begin{abstract}
% This package selects the input encoding by specifying between
% input characters and their glyph names.
% \end{abstract}
%
% \tableofcontents
%
% \newcommand*{\EM}{\textcolor{blue}}
% \newcommand*{\ExampleText}{^^A
%   Umlauts:\ \EM{\"A\"O\"U\"a\"o\"u\ss}^^A
% }
%
% \section{Documentation}
%
% \subsection{Introduction}
%
% \LaTeX\ supports the direct use of 8-bit characters by means
% of package \xpackage{inputenc}. However you must know
% and specify the encoding, e.g.:
% \begin{quote}
%   \ttfamily
%   |\documentclass{article}|\\
%   |\usepackage[|\EM{latin1}|]{inputenc}|\\
%   |% or \usepackage[|\EM{utf8}|]{inputenc}|\\
%   |% or \usepackage[|\EM{??}|]{inputenc}|\\
%   |\begin{document}|\\
%   |  |\ExampleText\\
%   |\end{document}|
% \end{quote}
%
% If the document is transferred in an environment that
% uses a different encoding, then there are programs that
% convert the input characters. Examples for conversion
% of file \xfile{test.tex}
% from encoding latin1 (ISO-8859-1) to UTF-8:
% \begin{quote}
%   \ttfamily
%   |recode ISO-8859-1..UTF-8 test.tex|\\
%   |recode latin1..utf8 test.tex|\\
%   |iconv --from-code ISO-8859-1|\\
%   \hphantom{iconv}| --to-code UTF-8|\\
%   \hphantom{iconv}| --output testnew.tex|\\
%   \hphantom{iconv}| test.tex|\\
%   |iconv -f latin1 -t utf8 -o testnew.tex test.tex|
% \end{quote}
% However, the encoding name for package \xpackage{inputenc}
% must be changed:
% \begin{quote}
%    |\usepackage[latin1]{inputenc}| $\rightarrow$
%    |\usepackage[utf8]{inputenc}|\kern-4pt\relax
% \end{quote}
% Of course, unless you are using some clever editor
% that knows package \xpackage{inputenc}, recodes
% the file and adjusts the option at the same time.
% But most editors can perhaps recode the file, but
% they let the option untouched.
%
% Therefore package \xpackage{selinput} chooses another way for
% specifying the input encoding. The encoding name is not needed
% at all. Some 8-bit characters are identified by their glyph
% name and the package chooses an appropriate encoding, example:
% \begin{quote}
%   \ttfamily
%   |\documentclass{article}|\\
%   |\usepackage{selinput}|\\
%   |\SelectInputMappings{|\\
%   |  adieresis={|\EM{\"a}|}|,\\
%   |  germandbls={|\EM{\ss}|}|,\\
%   |  Euro={|\EM{\texteuro}|}|,\\
%   |}|\\
%   |\begin{document}|\\
%   |  |\ExampleText\\
%   |\end{document}|
% \end{quote}
%
% \subsection{User interface}
%
% \begin{declcs}{SelectInputEncodingList} \M{encoding list}
% \end{declcs}
% \cs{SelectInputEncodingList} expects a comma separated list of
% encoding names. Example:
% \begin{quote}
%   |\SelectInputEncodingList{utf8,ansinew,mac-roman}|
% \end{quote}
% The encodings of package \xpackage{inputenx} are used as default.
%
% \begin{declcs}{SelectInputMappings} \M{mapping pairs}
% \end{declcs}
% A mapping pair consists of a glyph name and its input
% character:
% \begin{quote}
%   |\SelectInputMappings{|\\
%   |  adieresis={|\EM{\"a}|}|,\\
%   |  germandbls={|\EM{\ss}|}|,\\
%   |  Euro={|\EM{\texteuro}|}|,\\
%   |}|
% \end{quote}
% The supported glyph names can be found in file \xfile{ix-name.def}
% of project \xpackage{inputenx} \cite{inputenx}. The names are
% basically taken from Adobe's glyphlists \cite{adobe:glyphlist,adobe:aglfn}.
% As many pairs are needed as necessary to identify the encoding.
% Example with insufficient pairs:
% \begin{quote}
%   \ttfamily
%   |\SelectInputEncodingSet{latin1,latin9}|\\
%   |\SelectInputMappings{|\\
%   |  adieresis={|\EM{\"a}|}|,\\
%   |  germandbls={|\EM{\ss}|}|,\\
%   |}|\\
%   \ExampleText| and Euro: |\EM{\textcurrency} (wrong)
% \end{quote}
% The first encoding \xoption{latin1} passes the constraints given
% by the mapping pairs. However the Euro symbol is not part of
% the encoding. Thus a mapping pair with the Euro symbol
% solves the problem. In fact the symbol alone already succeeds in selecting
% between \xoption{latin1} and \xoption{latin9}:
% \begin{quote}
%   \ttfamily
%   |\SelectInputEncodingSet{latin1,latin9}|\\
%   |\SelectInputMappings{|\\
%   |  Euro={|\EM{\texteuro}|},|\\
%   |}|\\
%   \ExampleText| and Euro: |\EM{\texteuro}
% \end{quote}
%
% \subsection{Options}
%
% \begin{description}
% \item[\xoption{warning}:]
%   The selected encoding is written
%   by \cs{PackageInfo} into the \xfile{.log} file only.
%   Option \xoption{warning} changes it to \cs{PackageWarning}.
%   Then the selected encoding is shown on the terminal as well.
% \item[\xoption{ucs}:]
%   The encoding file \xfile{utf8x} of package \cs{ucs} requires
%   that the package itself is loaded before.
%   If the package is not loaded, then the option \xoption{ucs}
%   will load package \xpackage{ucs} if the detected encoding is
%   UTF-8 (limited to the preamble, packages cannot be loaded later).
% \item[\xoption{utf8=\dots}:]
%   The option allows to specify other encoding files
%   for UTF-8 than \LaTeX's \xfile{utf8.def}. For example,
%   |utf8=utf-8| will load \xfile{utf-8.def} instead.
% \end{description}
%
% \subsection{Encodings}
%
% Package \xpackage{stringenc} \cite{stringenc}
% is used for testing the encoding. Thus the encoding
% name must be known by this package. Then the found
% encoding is loaded by \cs{inputencoding} by package
% \xpackage{inputenc} or \cs{InputEncoding} if package
% \xpackage{inputenx} is loaded.
%
% The supported encodings are present in the encoding list,
% thus usually the encoding names do not matter.
% If the list is set by \cs{SelectInputEncodingList},
% then you can use the names that work for package
% \xpackage{inputenc} and are known by package \xpackage{stringenc},
% for example: \xoption{latin1}, \xoption{x-iso-8859-1}. Encoding
% file names of package \xpackage{inputenx} are prefixed with \xfile{x-}.
% The prefix can be dropped, if package \xpackage{inputenx} is loaded.
%
% \StopEventually{
% }
%
% \section{Implementation}
%
%    \begin{macrocode}
%<*package>
\NeedsTeXFormat{LaTeX2e}
\ProvidesPackage{selinput}
  [2007/09/09 v1.2 Semi-automatic input encoding detection (HO)]%
%    \end{macrocode}
%
%    \begin{macrocode}
\RequirePackage{inputenc}
\RequirePackage{kvsetkeys}[2006/10/19]
\RequirePackage{stringenc}[2007/06/16]
\RequirePackage{kvoptions}
%    \end{macrocode}
%    \begin{macro}{\SelectInputEncodingList}
%    \begin{macrocode}
\newcommand*{\SelectInputEncodingList}{%
  \let\SIE@EncodingList\@empty
  \kvsetkeys{SelInputEnc}%
}
%    \end{macrocode}
%    \end{macro}
%    \begin{macro}{\SelectInputMappings}
%    \begin{macrocode}
\newcommand*{\SelectInputMappings}[1]{%
  \SIE@LoadNameDefs
  \let\SIE@StringUnicode\@empty
  \let\SIE@StringDest\@empty
  \kvsetkeys{SelInputMap}{#1}%
  \ifx\\SIE@StringUnicode\SIE@StringDest\\%
    \PackageError{selinput}{%
      No mappings specified%
    }\@ehc
  \else
    \EdefUnescapeHex\SIE@StringUnicode\SIE@StringUnicode
    \let\SIE@Encoding\@empty
    \@for\SIE@EncodingTest:=\SIE@EncodingList\do{%
      \ifx\SIE@Encoding\@empty
        \StringEncodingConvertTest\SIE@temp\SIE@StringUnicode
                                  {utf16be}\SIE@EncodingTest{%
          \ifx\SIE@temp\SIE@StringDest
            \let\SIE@Encoding\SIE@EncodingTest
          \fi
        }{}%
      \fi
    }%
    \ifx\SIE@Encoding\@empty
      \StringEncodingConvertTest\SIE@temp\SIE@StringDest
                                {ascii}{utf16be}{%
        \def\SIE@Encoding{ascii}%
        \SIE@Info{selinput}{%
          Matching encoding not found, but input characters%
          \MessageBreak
          are 7-bit (possibly editor replacements).%
          \MessageBreak
          Hence using ascii encoding%
        }%
      }{}%
    \fi
    \ifx\SIE@Encoding\@empty
      \PackageError{selinput}{%
        Cannot find a matching encoding%
      }\@ehd
    \else
      \ifx\SIE@Encoding\SIE@EncodingUTFviii
        \SIE@LoadUnicodePackage
        \ifx\SIE@UseUTFviii\@empty
        \else
          \let\SIE@Encoding\SIE@UseUTFviii
        \fi
      \fi
      \begingroup\expandafter\expandafter\expandafter\endgroup
      \expandafter\ifx\csname InputEncoding\endcsname\relax
        \inputencoding\SIE@Encoding
      \else
        \InputEncoding\SIE@Encoding
      \fi
      \SIE@Info{selinput}{Encoding `\SIE@Encoding' selected}%
    \fi
  \fi
}
%    \end{macrocode}
%    \end{macro}
%    \begin{macro}{\SIE@LoadNameDefs}
%    \begin{macrocode}
\def\SIE@LoadNameDefs{%
  \begingroup
    \endlinechar=\m@ne
    \catcode92=0 % backslash
    \catcode123=1 % left curly brace/beginning of group
    \catcode125=2 % right curly brace/end of group
    \catcode37=14 % percent/comment character
    \@makeother\[%
    \@makeother\]%
    \@makeother\.%
    \@makeother\(%
    \@makeother\)%
    \@makeother\/%
    \@makeother\-%
    \let\InputenxName\SelectInputDefineMapping
    \InputIfFileExists{ix-name.def}{}{%
      \PackageError{selinput}{%
        Missing `ix-name.def' (part of package `inputenx')%
      }\@ehd
    }%
    \global\let\SIE@LoadNameDefs\relax
  \endgroup
}
%    \end{macrocode}
%    \end{macro}
%    \begin{macro}{\SelectInputDefineMapping}
%    \begin{macrocode}
\newcommand*{\SelectInputDefineMapping}[1]{%
  \expandafter\gdef\csname SIE@@#1\endcsname
}
%    \end{macrocode}
%    \end{macro}
%    \begin{macrocode}
\kv@set@family@handler{SelInputMap}{%
  \@onelevel@sanitize\kv@key
  \ifx\kv@value\relax
    \PackageError{selinput}{%
      Missing input character for `\kv@key'%
    }\@ehc
  \else
    \@onelevel@sanitize\kv@value
    \ifx\kv@value\@empty
      \PackageError{selinput}{%
        Input character got lost?\MessageBreak
        Missing input character for `\kv@key'%
      }\@ehc
    \else
      \@ifundefined{SIE@@\kv@key}{%
        \PackageWarning{selinput}{%
          Missing definition for `\kv@key'%
        }%
      }{%
        \edef\SIE@StringDest{%
          \SIE@StringDest
          \kv@value
        }%
        \edef\SIE@StringUnicode{%
          \SIE@StringUnicode
          \csname SIE@@\kv@key\endcsname
        }%
      }%
    \fi
  \fi
}
%    \end{macrocode}
%    \begin{macrocode}
\kv@set@family@handler{SelInputEnc}{%
  \@onelevel@sanitize\kv@key
  \ifx\kv@value\relax
    \ifx\SIE@EncodingList\@empty
      \let\SIE@EncodingList\kv@key
    \else
      \edef\SIE@EncodingList{\SIE@EncodingList,\kv@key}%
    \fi
  \else
    \@onelevel@sanitize\kv@value
    \PackageError{selinput}{%
      Illegal key value pair (\kv@key=\kv@value)\MessagBreak
      in encoding list%
    }\@ehc
  \fi
}
%    \end{macrocode}
%
%    \begin{macro}{\SIE@LoadUnicodePackage}
%    \begin{macrocode}
\def\SIE@LoadUnicodePackage{%
  \@ifpackageloaded\SIE@UnicodePackage{}{%
    \RequirePackage\SIE@UnicodePackage\relax
  }%
  \SIE@PatchUCS
  \global\let\SIE@LoadUnicodePackage\relax
}
\let\SIE@show\show
\def\SIE@PatchUCS{%
  \AtBeginDocument{%
    \expandafter\ifx\csname ver@ucsencs.def\endcsname\relax
    \else
      \let\show\SIE@show
    \fi
  }%
}
\SIE@PatchUCS
%    \end{macrocode}
%    \end{macro}
%    \begin{macrocode}
\AtBeginDocument{%
  \let\SIE@LoadUnicodePackage\relax
}
%    \end{macrocode}
%    \begin{macro}{\SIE@EncodingUTFviii}
%    \begin{macrocode}
\def\SIE@EncodingUTFviii{utf8}
\@onelevel@sanitize\SIE@EncodingUTFviii
%    \end{macrocode}
%    \end{macro}
%    \begin{macro}{\SIE@EncodingUTFviiix}
%    \begin{macrocode}
\def\SIE@EncodingUTFviiix{utf8x}
\@onelevel@sanitize\SIE@EncodingUTFviiix
%    \end{macrocode}
%    \end{macro}
%
%    \begin{macrocode}
\let\SIE@UnicodePackage\@empty
\let\SIE@UseUTFviii\@empty
\let\SIE@Info\PackageInfo
%    \end{macrocode}
%    \begin{macrocode}
\SetupKeyvalOptions{%
  family=SelInput,%
  prefix=SelInput@%
}
\define@key{SelInput}{utf8}{%
  \def\SIE@UseUTFviii{#1}%
  \@onelevel@sanitize\SIE@UseUTFviii
}
\DeclareBoolOption{ucs}
\DeclareVoidOption{warning}{%
  \let\SIE@Info\PackageWarning
}
\ProcessKeyvalOptions{SelInput}
\ifSelInput@ucs
  \def\SIE@UnicodePackage{ucs}%
  \ifx\SIE@UseUTFviii\@empty
    \let\SIE@UseUTFviii\SIE@EncodingUTFviiix
  \fi
\else
  \ifx\SIE@UseUTFviii\@empty
    \@ifpackageloaded{ucs}{%
      \let\SIE@UseUTFviii\SIE@EncodingUTFviiix
    }{%
      \let\SIE@UseUTFviii\SIE@EncodingUTFviii
    }%
  \fi
\fi
%    \end{macrocode}
%
%    \begin{macro}{\SIE@EncodingList}
%    \begin{macrocode}
\edef\SIE@EncodingList{%
  utf8,%
  x-iso-8859-1,%
  x-iso-8859-15,%
  x-cp1252,% ansinew
  x-mac-roman,%
  x-iso-8859-2,%
  x-iso-8859-3,%
  x-iso-8859-4,%
  x-iso-8859-5,%
  x-iso-8859-6,%
  x-iso-8859-7,%
  x-iso-8859-8,%
  x-iso-8859-9,%
  x-iso-8859-10,%
  x-iso-8859-11,%
  x-iso-8859-13,%
  x-iso-8859-14,%
  x-iso-8859-15,%
  x-mac-centeuro,%
  x-mac-cyrillic,%
  x-koi8-r,%
  x-cp1250,%
  x-cp1251,%
  x-cp1257,%
  x-cp437,%
  x-cp850,%
  x-cp852,%
  x-cp855,%
  x-cp858,%
  x-cp865,%
  x-cp866,%
  x-nextstep,%
  x-dec-mcs%
}%
\@onelevel@sanitize\SIE@EncodingList
%    \end{macrocode}
%    \end{macro}
%
%    \begin{macrocode}
%</package>
%    \end{macrocode}
%
% \section{Test}
%
%    \begin{macrocode}
%<*test>
\NeedsTeXFormat{LaTeX2e}
\documentclass{minimal}
\usepackage{textcomp}
\usepackage{qstest}
%    \end{macrocode}
%    \begin{macrocode}
%<*test1|test2|test3>
\makeatletter
\let\BeginDocumentText\@empty
\def\TestEncoding#1#2{%
  \SelectInputMappings{#2}%
  \Expect*{\SIE@Encoding}{#1}%
  \Expect*{\inputencodingname}{#1}%
  \g@addto@macro\BeginDocumentText{%
    \SelectInputMappings{#2}%
    \Expect*{\SIE@Encoding}{#1}%
    \textbf{\SIE@Encoding:} %
    \kvsetkeys{test}{#2}\par
  }%
}
\def\TestKey#1#2{%
  \define@key{test}{#1}{%
    \sbox0{##1}%
    \sbox2{#2}%
    \Expect*{wd:\the\wd0, ht:\the\ht0, dp:\the\dp0}%
           *{wd:\the\wd2, ht:\the\ht2, dp:\the\dp2}%
    [#1=##1] % hash-ok
  }%
}
\RequirePackage{keyval}
\TestKey{adieresis}{\"a}
\TestKey{germandbls}{\ss}
\TestKey{Euro}{\texteuro}
\makeatother
\usepackage[
  warning,%
%<test2>  utf8=utf-8,
%<test3>  ucs,
]{selinput}
%<test1|test3>\inputencoding{ascii}
%<test2>\inputencoding{utf-8}
%<test3>\usepackage{ucs}
\begin{qstest}{preamble}{}
  \TestEncoding{x-iso-8859-15}{%
    adieresis=^^e4,%
    germandbls=^^df,%
    Euro=^^a4,%
  }%
  \TestEncoding{x-cp1252}{%
    adieresis=^^e4,%
    germandbls=^^df,%
    Euro=^^80,%
  }%
%<test1>  \TestEncoding{utf8}{%
%<test2>  \TestEncoding{utf-8}{%
%<test3>  \TestEncoding{utf8x}{%
    adieresis=^^c3^^a4,%
    germandbls=^^c3^^9f,%
%<!test2>    Euro=^^e2^^82^^ac,
  }%
\end{qstest}
%<test3>\let\ifUnicodeOptiongraphics\iffalse
\begin{document}
\begin{qstest}{document}{}
%<test3>\makeatletter
  \BeginDocumentText
\end{qstest}
%</test1|test2|test3>
%    \end{macrocode}
%
%    \begin{macrocode}
%<*test4>
\usepackage[warning,ucs]{selinput}
\SelectInputMappings{%
    adieresis=^^c3^^a4,%
    germandbls=^^c3^^9f,%
    Euro=^^e2^^82^^ac,%
}
\begin{qstest}{encoding}{}
  \Expect*{\inputencodingname}{utf8x}%
\end{qstest}
\begin{document}
  adieresis=^^c3^^a4, %
  germandbls=^^c3^^9f, %
  Euro=^^e2^^82^^ac%
%</test4>
%    \end{macrocode}
%
%    \begin{macrocode}
%<*test5>
\usepackage[warning,ucs]{selinput}
\SelectInputMappings{%
    adieresis={\"a},%
    germandbls={{\ss}},%
    Euro=\texteuro{},%
}
\begin{qstest}{encoding}{}
  \Expect*{\inputencodingname}{ascii}%
\end{qstest}
\begin{document}
  adieresis={\"a}, %
  germandbls={{\ss}}, %
  Euro=\texteuro{}%
%</test5>
%    \end{macrocode}
%
%    \begin{macrocode}
\end{document}
%</test>
%    \end{macrocode}
%
% \section{Installation}
%
% \subsection{Download}
%
% \paragraph{Package.} This package is available on
% CTAN\footnote{\url{ftp://ftp.ctan.org/tex-archive/}}:
% \begin{description}
% \item[\CTAN{macros/latex/contrib/oberdiek/selinput.dtx}] The source file.
% \item[\CTAN{macros/latex/contrib/oberdiek/selinput.pdf}] Documentation.
% \end{description}
%
%
% \paragraph{Bundle.} All the packages of the bundle `oberdiek'
% are also available in a TDS compliant ZIP archive. There
% the packages are already unpacked and the documentation files
% are generated. The files and directories obey the TDS standard.
% \begin{description}
% \item[\CTAN{install/macros/latex/contrib/oberdiek.tds.zip}]
% \end{description}
% \emph{TDS} refers to the standard ``A Directory Structure
% for \TeX\ Files'' (\CTAN{tds/tds.pdf}). Directories
% with \xfile{texmf} in their name are usually organized this way.
%
% \subsection{Bundle installation}
%
% \paragraph{Unpacking.} Unpack the \xfile{oberdiek.tds.zip} in the
% TDS tree (also known as \xfile{texmf} tree) of your choice.
% Example (linux):
% \begin{quote}
%   |unzip oberdiek.tds.zip -d ~/texmf|
% \end{quote}
%
% \paragraph{Script installation.}
% Check the directory \xfile{TDS:scripts/oberdiek/} for
% scripts that need further installation steps.
% Package \xpackage{attachfile2} comes with the Perl script
% \xfile{pdfatfi.pl} that should be installed in such a way
% that it can be called as \texttt{pdfatfi}.
% Example (linux):
% \begin{quote}
%   |chmod +x scripts/oberdiek/pdfatfi.pl|\\
%   |cp scripts/oberdiek/pdfatfi.pl /usr/local/bin/|
% \end{quote}
%
% \subsection{Package installation}
%
% \paragraph{Unpacking.} The \xfile{.dtx} file is a self-extracting
% \docstrip\ archive. The files are extracted by running the
% \xfile{.dtx} through \plainTeX:
% \begin{quote}
%   \verb|tex selinput.dtx|
% \end{quote}
%
% \paragraph{TDS.} Now the different files must be moved into
% the different directories in your installation TDS tree
% (also known as \xfile{texmf} tree):
% \begin{quote}
% \def\t{^^A
% \begin{tabular}{@{}>{\ttfamily}l@{ $\rightarrow$ }>{\ttfamily}l@{}}
%   selinput.sty & tex/latex/oberdiek/selinput.sty\\
%   selinput.pdf & doc/latex/oberdiek/selinput.pdf\\
%   test/selinput-test1.tex & doc/latex/oberdiek/test/selinput-test1.tex\\
%   test/selinput-test2.tex & doc/latex/oberdiek/test/selinput-test2.tex\\
%   test/selinput-test3.tex & doc/latex/oberdiek/test/selinput-test3.tex\\
%   test/selinput-test4.tex & doc/latex/oberdiek/test/selinput-test4.tex\\
%   test/selinput-test5.tex & doc/latex/oberdiek/test/selinput-test5.tex\\
%   selinput.dtx & source/latex/oberdiek/selinput.dtx\\
% \end{tabular}^^A
% }^^A
% \sbox0{\t}^^A
% \ifdim\wd0>\linewidth
%   \begingroup
%     \advance\linewidth by\leftmargin
%     \advance\linewidth by\rightmargin
%   \edef\x{\endgroup
%     \def\noexpand\lw{\the\linewidth}^^A
%   }\x
%   \def\lwbox{^^A
%     \leavevmode
%     \hbox to \linewidth{^^A
%       \kern-\leftmargin\relax
%       \hss
%       \usebox0
%       \hss
%       \kern-\rightmargin\relax
%     }^^A
%   }^^A
%   \ifdim\wd0>\lw
%     \sbox0{\small\t}^^A
%     \ifdim\wd0>\linewidth
%       \ifdim\wd0>\lw
%         \sbox0{\footnotesize\t}^^A
%         \ifdim\wd0>\linewidth
%           \ifdim\wd0>\lw
%             \sbox0{\scriptsize\t}^^A
%             \ifdim\wd0>\linewidth
%               \ifdim\wd0>\lw
%                 \sbox0{\tiny\t}^^A
%                 \ifdim\wd0>\linewidth
%                   \lwbox
%                 \else
%                   \usebox0
%                 \fi
%               \else
%                 \lwbox
%               \fi
%             \else
%               \usebox0
%             \fi
%           \else
%             \lwbox
%           \fi
%         \else
%           \usebox0
%         \fi
%       \else
%         \lwbox
%       \fi
%     \else
%       \usebox0
%     \fi
%   \else
%     \lwbox
%   \fi
% \else
%   \usebox0
% \fi
% \end{quote}
% If you have a \xfile{docstrip.cfg} that configures and enables \docstrip's
% TDS installing feature, then some files can already be in the right
% place, see the documentation of \docstrip.
%
% \subsection{Refresh file name databases}
%
% If your \TeX~distribution
% (\teTeX, \mikTeX, \dots) relies on file name databases, you must refresh
% these. For example, \teTeX\ users run \verb|texhash| or
% \verb|mktexlsr|.
%
% \subsection{Some details for the interested}
%
% \paragraph{Attached source.}
%
% The PDF documentation on CTAN also includes the
% \xfile{.dtx} source file. It can be extracted by
% AcrobatReader 6 or higher. Another option is \textsf{pdftk},
% e.g. unpack the file into the current directory:
% \begin{quote}
%   \verb|pdftk selinput.pdf unpack_files output .|
% \end{quote}
%
% \paragraph{Unpacking with \LaTeX.}
% The \xfile{.dtx} chooses its action depending on the format:
% \begin{description}
% \item[\plainTeX:] Run \docstrip\ and extract the files.
% \item[\LaTeX:] Generate the documentation.
% \end{description}
% If you insist on using \LaTeX\ for \docstrip\ (really,
% \docstrip\ does not need \LaTeX), then inform the autodetect routine
% about your intention:
% \begin{quote}
%   \verb|latex \let\install=y% \iffalse meta-comment
%
% File: selinput.dtx
% Version: 2007/09/09 v1.2
% Info: Semi-automatic input encoding detection
%
% Copyright (C) 2007 by
%    Heiko Oberdiek <heiko.oberdiek at googlemail.com>
%
% This work may be distributed and/or modified under the
% conditions of the LaTeX Project Public License, either
% version 1.3c of this license or (at your option) any later
% version. This version of this license is in
%    http://www.latex-project.org/lppl/lppl-1-3c.txt
% and the latest version of this license is in
%    http://www.latex-project.org/lppl.txt
% and version 1.3 or later is part of all distributions of
% LaTeX version 2005/12/01 or later.
%
% This work has the LPPL maintenance status "maintained".
%
% This Current Maintainer of this work is Heiko Oberdiek.
%
% This work consists of the main source file selinput.dtx
% and the derived files
%    selinput.sty, selinput.pdf, selinput.ins, selinput.drv,
%    selinput-test1.tex, selinput-test2.tex, selinput-test3.tex,
%    selinput-test4.tex, selinput-test5.tex.
%
% Distribution:
%    CTAN:macros/latex/contrib/oberdiek/selinput.dtx
%    CTAN:macros/latex/contrib/oberdiek/selinput.pdf
%
% Unpacking:
%    (a) If selinput.ins is present:
%           tex selinput.ins
%    (b) Without selinput.ins:
%           tex selinput.dtx
%    (c) If you insist on using LaTeX
%           latex \let\install=y\input{selinput.dtx}
%        (quote the arguments according to the demands of your shell)
%
% Documentation:
%    (a) If selinput.drv is present:
%           latex selinput.drv
%    (b) Without selinput.drv:
%           latex selinput.dtx; ...
%    The class ltxdoc loads the configuration file ltxdoc.cfg
%    if available. Here you can specify further options, e.g.
%    use A4 as paper format:
%       \PassOptionsToClass{a4paper}{article}
%
%    Programm calls to get the documentation (example):
%       pdflatex selinput.dtx
%       makeindex -s gind.ist selinput.idx
%       pdflatex selinput.dtx
%       makeindex -s gind.ist selinput.idx
%       pdflatex selinput.dtx
%
% Installation:
%    TDS:tex/latex/oberdiek/selinput.sty
%    TDS:doc/latex/oberdiek/selinput.pdf
%    TDS:doc/latex/oberdiek/test/selinput-test1.tex
%    TDS:doc/latex/oberdiek/test/selinput-test2.tex
%    TDS:doc/latex/oberdiek/test/selinput-test3.tex
%    TDS:doc/latex/oberdiek/test/selinput-test4.tex
%    TDS:doc/latex/oberdiek/test/selinput-test5.tex
%    TDS:source/latex/oberdiek/selinput.dtx
%
%<*ignore>
\begingroup
  \catcode123=1 %
  \catcode125=2 %
  \def\x{LaTeX2e}%
\expandafter\endgroup
\ifcase 0\ifx\install y1\fi\expandafter
         \ifx\csname processbatchFile\endcsname\relax\else1\fi
         \ifx\fmtname\x\else 1\fi\relax
\else\csname fi\endcsname
%</ignore>
%<*install>
\input docstrip.tex
\Msg{************************************************************************}
\Msg{* Installation}
\Msg{* Package: selinput 2007/09/09 v1.2 Semi-automatic input encoding detection (HO)}
\Msg{************************************************************************}

\keepsilent
\askforoverwritefalse

\let\MetaPrefix\relax
\preamble

This is a generated file.

Project: selinput
Version: 2007/09/09 v1.2

Copyright (C) 2007 by
   Heiko Oberdiek <heiko.oberdiek at googlemail.com>

This work may be distributed and/or modified under the
conditions of the LaTeX Project Public License, either
version 1.3c of this license or (at your option) any later
version. This version of this license is in
   http://www.latex-project.org/lppl/lppl-1-3c.txt
and the latest version of this license is in
   http://www.latex-project.org/lppl.txt
and version 1.3 or later is part of all distributions of
LaTeX version 2005/12/01 or later.

This work has the LPPL maintenance status "maintained".

This Current Maintainer of this work is Heiko Oberdiek.

This work consists of the main source file selinput.dtx
and the derived files
   selinput.sty, selinput.pdf, selinput.ins, selinput.drv,
   selinput-test1.tex, selinput-test2.tex, selinput-test3.tex,
   selinput-test4.tex, selinput-test5.tex.

\endpreamble
\let\MetaPrefix\DoubleperCent

\generate{%
  \file{selinput.ins}{\from{selinput.dtx}{install}}%
  \file{selinput.drv}{\from{selinput.dtx}{driver}}%
  \usedir{tex/latex/oberdiek}%
  \file{selinput.sty}{\from{selinput.dtx}{package}}%
  \usedir{doc/latex/oberdiek/test}%
  \file{selinput-test1.tex}{\from{selinput.dtx}{test,test1}}%
  \file{selinput-test2.tex}{\from{selinput.dtx}{test,test2}}%
  \file{selinput-test3.tex}{\from{selinput.dtx}{test,test3}}%
  \file{selinput-test4.tex}{\from{selinput.dtx}{test,test4}}%
  \file{selinput-test5.tex}{\from{selinput.dtx}{test,test5}}%
  \nopreamble
  \nopostamble
  \usedir{source/latex/oberdiek/catalogue}%
  \file{selinput.xml}{\from{selinput.dtx}{catalogue}}%
}

\catcode32=13\relax% active space
\let =\space%
\Msg{************************************************************************}
\Msg{*}
\Msg{* To finish the installation you have to move the following}
\Msg{* file into a directory searched by TeX:}
\Msg{*}
\Msg{*     selinput.sty}
\Msg{*}
\Msg{* To produce the documentation run the file `selinput.drv'}
\Msg{* through LaTeX.}
\Msg{*}
\Msg{* Happy TeXing!}
\Msg{*}
\Msg{************************************************************************}

\endbatchfile
%</install>
%<*ignore>
\fi
%</ignore>
%<*driver>
\NeedsTeXFormat{LaTeX2e}
\ProvidesFile{selinput.drv}%
  [2007/09/09 v1.2 Semi-automatic input encoding detection (HO)]%
\documentclass{ltxdoc}
\usepackage[T1]{fontenc}
\usepackage{textcomp}
\usepackage{lmodern}
\usepackage{holtxdoc}[2011/11/22]
\usepackage{color}
\begin{document}
  \DocInput{selinput.dtx}%
\end{document}
%</driver>
% \fi
%
% \CheckSum{389}
%
% \CharacterTable
%  {Upper-case    \A\B\C\D\E\F\G\H\I\J\K\L\M\N\O\P\Q\R\S\T\U\V\W\X\Y\Z
%   Lower-case    \a\b\c\d\e\f\g\h\i\j\k\l\m\n\o\p\q\r\s\t\u\v\w\x\y\z
%   Digits        \0\1\2\3\4\5\6\7\8\9
%   Exclamation   \!     Double quote  \"     Hash (number) \#
%   Dollar        \$     Percent       \%     Ampersand     \&
%   Acute accent  \'     Left paren    \(     Right paren   \)
%   Asterisk      \*     Plus          \+     Comma         \,
%   Minus         \-     Point         \.     Solidus       \/
%   Colon         \:     Semicolon     \;     Less than     \<
%   Equals        \=     Greater than  \>     Question mark \?
%   Commercial at \@     Left bracket  \[     Backslash     \\
%   Right bracket \]     Circumflex    \^     Underscore    \_
%   Grave accent  \`     Left brace    \{     Vertical bar  \|
%   Right brace   \}     Tilde         \~}
%
% \GetFileInfo{selinput.drv}
%
% \title{The \xpackage{selinput} package}
% \date{2007/09/09 v1.2}
% \author{Heiko Oberdiek\\\xemail{heiko.oberdiek at googlemail.com}}
%
% \maketitle
%
% \begin{abstract}
% This package selects the input encoding by specifying between
% input characters and their glyph names.
% \end{abstract}
%
% \tableofcontents
%
% \newcommand*{\EM}{\textcolor{blue}}
% \newcommand*{\ExampleText}{^^A
%   Umlauts:\ \EM{\"A\"O\"U\"a\"o\"u\ss}^^A
% }
%
% \section{Documentation}
%
% \subsection{Introduction}
%
% \LaTeX\ supports the direct use of 8-bit characters by means
% of package \xpackage{inputenc}. However you must know
% and specify the encoding, e.g.:
% \begin{quote}
%   \ttfamily
%   |\documentclass{article}|\\
%   |\usepackage[|\EM{latin1}|]{inputenc}|\\
%   |% or \usepackage[|\EM{utf8}|]{inputenc}|\\
%   |% or \usepackage[|\EM{??}|]{inputenc}|\\
%   |\begin{document}|\\
%   |  |\ExampleText\\
%   |\end{document}|
% \end{quote}
%
% If the document is transferred in an environment that
% uses a different encoding, then there are programs that
% convert the input characters. Examples for conversion
% of file \xfile{test.tex}
% from encoding latin1 (ISO-8859-1) to UTF-8:
% \begin{quote}
%   \ttfamily
%   |recode ISO-8859-1..UTF-8 test.tex|\\
%   |recode latin1..utf8 test.tex|\\
%   |iconv --from-code ISO-8859-1|\\
%   \hphantom{iconv}| --to-code UTF-8|\\
%   \hphantom{iconv}| --output testnew.tex|\\
%   \hphantom{iconv}| test.tex|\\
%   |iconv -f latin1 -t utf8 -o testnew.tex test.tex|
% \end{quote}
% However, the encoding name for package \xpackage{inputenc}
% must be changed:
% \begin{quote}
%    |\usepackage[latin1]{inputenc}| $\rightarrow$
%    |\usepackage[utf8]{inputenc}|\kern-4pt\relax
% \end{quote}
% Of course, unless you are using some clever editor
% that knows package \xpackage{inputenc}, recodes
% the file and adjusts the option at the same time.
% But most editors can perhaps recode the file, but
% they let the option untouched.
%
% Therefore package \xpackage{selinput} chooses another way for
% specifying the input encoding. The encoding name is not needed
% at all. Some 8-bit characters are identified by their glyph
% name and the package chooses an appropriate encoding, example:
% \begin{quote}
%   \ttfamily
%   |\documentclass{article}|\\
%   |\usepackage{selinput}|\\
%   |\SelectInputMappings{|\\
%   |  adieresis={|\EM{\"a}|}|,\\
%   |  germandbls={|\EM{\ss}|}|,\\
%   |  Euro={|\EM{\texteuro}|}|,\\
%   |}|\\
%   |\begin{document}|\\
%   |  |\ExampleText\\
%   |\end{document}|
% \end{quote}
%
% \subsection{User interface}
%
% \begin{declcs}{SelectInputEncodingList} \M{encoding list}
% \end{declcs}
% \cs{SelectInputEncodingList} expects a comma separated list of
% encoding names. Example:
% \begin{quote}
%   |\SelectInputEncodingList{utf8,ansinew,mac-roman}|
% \end{quote}
% The encodings of package \xpackage{inputenx} are used as default.
%
% \begin{declcs}{SelectInputMappings} \M{mapping pairs}
% \end{declcs}
% A mapping pair consists of a glyph name and its input
% character:
% \begin{quote}
%   |\SelectInputMappings{|\\
%   |  adieresis={|\EM{\"a}|}|,\\
%   |  germandbls={|\EM{\ss}|}|,\\
%   |  Euro={|\EM{\texteuro}|}|,\\
%   |}|
% \end{quote}
% The supported glyph names can be found in file \xfile{ix-name.def}
% of project \xpackage{inputenx} \cite{inputenx}. The names are
% basically taken from Adobe's glyphlists \cite{adobe:glyphlist,adobe:aglfn}.
% As many pairs are needed as necessary to identify the encoding.
% Example with insufficient pairs:
% \begin{quote}
%   \ttfamily
%   |\SelectInputEncodingSet{latin1,latin9}|\\
%   |\SelectInputMappings{|\\
%   |  adieresis={|\EM{\"a}|}|,\\
%   |  germandbls={|\EM{\ss}|}|,\\
%   |}|\\
%   \ExampleText| and Euro: |\EM{\textcurrency} (wrong)
% \end{quote}
% The first encoding \xoption{latin1} passes the constraints given
% by the mapping pairs. However the Euro symbol is not part of
% the encoding. Thus a mapping pair with the Euro symbol
% solves the problem. In fact the symbol alone already succeeds in selecting
% between \xoption{latin1} and \xoption{latin9}:
% \begin{quote}
%   \ttfamily
%   |\SelectInputEncodingSet{latin1,latin9}|\\
%   |\SelectInputMappings{|\\
%   |  Euro={|\EM{\texteuro}|},|\\
%   |}|\\
%   \ExampleText| and Euro: |\EM{\texteuro}
% \end{quote}
%
% \subsection{Options}
%
% \begin{description}
% \item[\xoption{warning}:]
%   The selected encoding is written
%   by \cs{PackageInfo} into the \xfile{.log} file only.
%   Option \xoption{warning} changes it to \cs{PackageWarning}.
%   Then the selected encoding is shown on the terminal as well.
% \item[\xoption{ucs}:]
%   The encoding file \xfile{utf8x} of package \cs{ucs} requires
%   that the package itself is loaded before.
%   If the package is not loaded, then the option \xoption{ucs}
%   will load package \xpackage{ucs} if the detected encoding is
%   UTF-8 (limited to the preamble, packages cannot be loaded later).
% \item[\xoption{utf8=\dots}:]
%   The option allows to specify other encoding files
%   for UTF-8 than \LaTeX's \xfile{utf8.def}. For example,
%   |utf8=utf-8| will load \xfile{utf-8.def} instead.
% \end{description}
%
% \subsection{Encodings}
%
% Package \xpackage{stringenc} \cite{stringenc}
% is used for testing the encoding. Thus the encoding
% name must be known by this package. Then the found
% encoding is loaded by \cs{inputencoding} by package
% \xpackage{inputenc} or \cs{InputEncoding} if package
% \xpackage{inputenx} is loaded.
%
% The supported encodings are present in the encoding list,
% thus usually the encoding names do not matter.
% If the list is set by \cs{SelectInputEncodingList},
% then you can use the names that work for package
% \xpackage{inputenc} and are known by package \xpackage{stringenc},
% for example: \xoption{latin1}, \xoption{x-iso-8859-1}. Encoding
% file names of package \xpackage{inputenx} are prefixed with \xfile{x-}.
% The prefix can be dropped, if package \xpackage{inputenx} is loaded.
%
% \StopEventually{
% }
%
% \section{Implementation}
%
%    \begin{macrocode}
%<*package>
\NeedsTeXFormat{LaTeX2e}
\ProvidesPackage{selinput}
  [2007/09/09 v1.2 Semi-automatic input encoding detection (HO)]%
%    \end{macrocode}
%
%    \begin{macrocode}
\RequirePackage{inputenc}
\RequirePackage{kvsetkeys}[2006/10/19]
\RequirePackage{stringenc}[2007/06/16]
\RequirePackage{kvoptions}
%    \end{macrocode}
%    \begin{macro}{\SelectInputEncodingList}
%    \begin{macrocode}
\newcommand*{\SelectInputEncodingList}{%
  \let\SIE@EncodingList\@empty
  \kvsetkeys{SelInputEnc}%
}
%    \end{macrocode}
%    \end{macro}
%    \begin{macro}{\SelectInputMappings}
%    \begin{macrocode}
\newcommand*{\SelectInputMappings}[1]{%
  \SIE@LoadNameDefs
  \let\SIE@StringUnicode\@empty
  \let\SIE@StringDest\@empty
  \kvsetkeys{SelInputMap}{#1}%
  \ifx\\SIE@StringUnicode\SIE@StringDest\\%
    \PackageError{selinput}{%
      No mappings specified%
    }\@ehc
  \else
    \EdefUnescapeHex\SIE@StringUnicode\SIE@StringUnicode
    \let\SIE@Encoding\@empty
    \@for\SIE@EncodingTest:=\SIE@EncodingList\do{%
      \ifx\SIE@Encoding\@empty
        \StringEncodingConvertTest\SIE@temp\SIE@StringUnicode
                                  {utf16be}\SIE@EncodingTest{%
          \ifx\SIE@temp\SIE@StringDest
            \let\SIE@Encoding\SIE@EncodingTest
          \fi
        }{}%
      \fi
    }%
    \ifx\SIE@Encoding\@empty
      \StringEncodingConvertTest\SIE@temp\SIE@StringDest
                                {ascii}{utf16be}{%
        \def\SIE@Encoding{ascii}%
        \SIE@Info{selinput}{%
          Matching encoding not found, but input characters%
          \MessageBreak
          are 7-bit (possibly editor replacements).%
          \MessageBreak
          Hence using ascii encoding%
        }%
      }{}%
    \fi
    \ifx\SIE@Encoding\@empty
      \PackageError{selinput}{%
        Cannot find a matching encoding%
      }\@ehd
    \else
      \ifx\SIE@Encoding\SIE@EncodingUTFviii
        \SIE@LoadUnicodePackage
        \ifx\SIE@UseUTFviii\@empty
        \else
          \let\SIE@Encoding\SIE@UseUTFviii
        \fi
      \fi
      \begingroup\expandafter\expandafter\expandafter\endgroup
      \expandafter\ifx\csname InputEncoding\endcsname\relax
        \inputencoding\SIE@Encoding
      \else
        \InputEncoding\SIE@Encoding
      \fi
      \SIE@Info{selinput}{Encoding `\SIE@Encoding' selected}%
    \fi
  \fi
}
%    \end{macrocode}
%    \end{macro}
%    \begin{macro}{\SIE@LoadNameDefs}
%    \begin{macrocode}
\def\SIE@LoadNameDefs{%
  \begingroup
    \endlinechar=\m@ne
    \catcode92=0 % backslash
    \catcode123=1 % left curly brace/beginning of group
    \catcode125=2 % right curly brace/end of group
    \catcode37=14 % percent/comment character
    \@makeother\[%
    \@makeother\]%
    \@makeother\.%
    \@makeother\(%
    \@makeother\)%
    \@makeother\/%
    \@makeother\-%
    \let\InputenxName\SelectInputDefineMapping
    \InputIfFileExists{ix-name.def}{}{%
      \PackageError{selinput}{%
        Missing `ix-name.def' (part of package `inputenx')%
      }\@ehd
    }%
    \global\let\SIE@LoadNameDefs\relax
  \endgroup
}
%    \end{macrocode}
%    \end{macro}
%    \begin{macro}{\SelectInputDefineMapping}
%    \begin{macrocode}
\newcommand*{\SelectInputDefineMapping}[1]{%
  \expandafter\gdef\csname SIE@@#1\endcsname
}
%    \end{macrocode}
%    \end{macro}
%    \begin{macrocode}
\kv@set@family@handler{SelInputMap}{%
  \@onelevel@sanitize\kv@key
  \ifx\kv@value\relax
    \PackageError{selinput}{%
      Missing input character for `\kv@key'%
    }\@ehc
  \else
    \@onelevel@sanitize\kv@value
    \ifx\kv@value\@empty
      \PackageError{selinput}{%
        Input character got lost?\MessageBreak
        Missing input character for `\kv@key'%
      }\@ehc
    \else
      \@ifundefined{SIE@@\kv@key}{%
        \PackageWarning{selinput}{%
          Missing definition for `\kv@key'%
        }%
      }{%
        \edef\SIE@StringDest{%
          \SIE@StringDest
          \kv@value
        }%
        \edef\SIE@StringUnicode{%
          \SIE@StringUnicode
          \csname SIE@@\kv@key\endcsname
        }%
      }%
    \fi
  \fi
}
%    \end{macrocode}
%    \begin{macrocode}
\kv@set@family@handler{SelInputEnc}{%
  \@onelevel@sanitize\kv@key
  \ifx\kv@value\relax
    \ifx\SIE@EncodingList\@empty
      \let\SIE@EncodingList\kv@key
    \else
      \edef\SIE@EncodingList{\SIE@EncodingList,\kv@key}%
    \fi
  \else
    \@onelevel@sanitize\kv@value
    \PackageError{selinput}{%
      Illegal key value pair (\kv@key=\kv@value)\MessagBreak
      in encoding list%
    }\@ehc
  \fi
}
%    \end{macrocode}
%
%    \begin{macro}{\SIE@LoadUnicodePackage}
%    \begin{macrocode}
\def\SIE@LoadUnicodePackage{%
  \@ifpackageloaded\SIE@UnicodePackage{}{%
    \RequirePackage\SIE@UnicodePackage\relax
  }%
  \SIE@PatchUCS
  \global\let\SIE@LoadUnicodePackage\relax
}
\let\SIE@show\show
\def\SIE@PatchUCS{%
  \AtBeginDocument{%
    \expandafter\ifx\csname ver@ucsencs.def\endcsname\relax
    \else
      \let\show\SIE@show
    \fi
  }%
}
\SIE@PatchUCS
%    \end{macrocode}
%    \end{macro}
%    \begin{macrocode}
\AtBeginDocument{%
  \let\SIE@LoadUnicodePackage\relax
}
%    \end{macrocode}
%    \begin{macro}{\SIE@EncodingUTFviii}
%    \begin{macrocode}
\def\SIE@EncodingUTFviii{utf8}
\@onelevel@sanitize\SIE@EncodingUTFviii
%    \end{macrocode}
%    \end{macro}
%    \begin{macro}{\SIE@EncodingUTFviiix}
%    \begin{macrocode}
\def\SIE@EncodingUTFviiix{utf8x}
\@onelevel@sanitize\SIE@EncodingUTFviiix
%    \end{macrocode}
%    \end{macro}
%
%    \begin{macrocode}
\let\SIE@UnicodePackage\@empty
\let\SIE@UseUTFviii\@empty
\let\SIE@Info\PackageInfo
%    \end{macrocode}
%    \begin{macrocode}
\SetupKeyvalOptions{%
  family=SelInput,%
  prefix=SelInput@%
}
\define@key{SelInput}{utf8}{%
  \def\SIE@UseUTFviii{#1}%
  \@onelevel@sanitize\SIE@UseUTFviii
}
\DeclareBoolOption{ucs}
\DeclareVoidOption{warning}{%
  \let\SIE@Info\PackageWarning
}
\ProcessKeyvalOptions{SelInput}
\ifSelInput@ucs
  \def\SIE@UnicodePackage{ucs}%
  \ifx\SIE@UseUTFviii\@empty
    \let\SIE@UseUTFviii\SIE@EncodingUTFviiix
  \fi
\else
  \ifx\SIE@UseUTFviii\@empty
    \@ifpackageloaded{ucs}{%
      \let\SIE@UseUTFviii\SIE@EncodingUTFviiix
    }{%
      \let\SIE@UseUTFviii\SIE@EncodingUTFviii
    }%
  \fi
\fi
%    \end{macrocode}
%
%    \begin{macro}{\SIE@EncodingList}
%    \begin{macrocode}
\edef\SIE@EncodingList{%
  utf8,%
  x-iso-8859-1,%
  x-iso-8859-15,%
  x-cp1252,% ansinew
  x-mac-roman,%
  x-iso-8859-2,%
  x-iso-8859-3,%
  x-iso-8859-4,%
  x-iso-8859-5,%
  x-iso-8859-6,%
  x-iso-8859-7,%
  x-iso-8859-8,%
  x-iso-8859-9,%
  x-iso-8859-10,%
  x-iso-8859-11,%
  x-iso-8859-13,%
  x-iso-8859-14,%
  x-iso-8859-15,%
  x-mac-centeuro,%
  x-mac-cyrillic,%
  x-koi8-r,%
  x-cp1250,%
  x-cp1251,%
  x-cp1257,%
  x-cp437,%
  x-cp850,%
  x-cp852,%
  x-cp855,%
  x-cp858,%
  x-cp865,%
  x-cp866,%
  x-nextstep,%
  x-dec-mcs%
}%
\@onelevel@sanitize\SIE@EncodingList
%    \end{macrocode}
%    \end{macro}
%
%    \begin{macrocode}
%</package>
%    \end{macrocode}
%
% \section{Test}
%
%    \begin{macrocode}
%<*test>
\NeedsTeXFormat{LaTeX2e}
\documentclass{minimal}
\usepackage{textcomp}
\usepackage{qstest}
%    \end{macrocode}
%    \begin{macrocode}
%<*test1|test2|test3>
\makeatletter
\let\BeginDocumentText\@empty
\def\TestEncoding#1#2{%
  \SelectInputMappings{#2}%
  \Expect*{\SIE@Encoding}{#1}%
  \Expect*{\inputencodingname}{#1}%
  \g@addto@macro\BeginDocumentText{%
    \SelectInputMappings{#2}%
    \Expect*{\SIE@Encoding}{#1}%
    \textbf{\SIE@Encoding:} %
    \kvsetkeys{test}{#2}\par
  }%
}
\def\TestKey#1#2{%
  \define@key{test}{#1}{%
    \sbox0{##1}%
    \sbox2{#2}%
    \Expect*{wd:\the\wd0, ht:\the\ht0, dp:\the\dp0}%
           *{wd:\the\wd2, ht:\the\ht2, dp:\the\dp2}%
    [#1=##1] % hash-ok
  }%
}
\RequirePackage{keyval}
\TestKey{adieresis}{\"a}
\TestKey{germandbls}{\ss}
\TestKey{Euro}{\texteuro}
\makeatother
\usepackage[
  warning,%
%<test2>  utf8=utf-8,
%<test3>  ucs,
]{selinput}
%<test1|test3>\inputencoding{ascii}
%<test2>\inputencoding{utf-8}
%<test3>\usepackage{ucs}
\begin{qstest}{preamble}{}
  \TestEncoding{x-iso-8859-15}{%
    adieresis=^^e4,%
    germandbls=^^df,%
    Euro=^^a4,%
  }%
  \TestEncoding{x-cp1252}{%
    adieresis=^^e4,%
    germandbls=^^df,%
    Euro=^^80,%
  }%
%<test1>  \TestEncoding{utf8}{%
%<test2>  \TestEncoding{utf-8}{%
%<test3>  \TestEncoding{utf8x}{%
    adieresis=^^c3^^a4,%
    germandbls=^^c3^^9f,%
%<!test2>    Euro=^^e2^^82^^ac,
  }%
\end{qstest}
%<test3>\let\ifUnicodeOptiongraphics\iffalse
\begin{document}
\begin{qstest}{document}{}
%<test3>\makeatletter
  \BeginDocumentText
\end{qstest}
%</test1|test2|test3>
%    \end{macrocode}
%
%    \begin{macrocode}
%<*test4>
\usepackage[warning,ucs]{selinput}
\SelectInputMappings{%
    adieresis=^^c3^^a4,%
    germandbls=^^c3^^9f,%
    Euro=^^e2^^82^^ac,%
}
\begin{qstest}{encoding}{}
  \Expect*{\inputencodingname}{utf8x}%
\end{qstest}
\begin{document}
  adieresis=^^c3^^a4, %
  germandbls=^^c3^^9f, %
  Euro=^^e2^^82^^ac%
%</test4>
%    \end{macrocode}
%
%    \begin{macrocode}
%<*test5>
\usepackage[warning,ucs]{selinput}
\SelectInputMappings{%
    adieresis={\"a},%
    germandbls={{\ss}},%
    Euro=\texteuro{},%
}
\begin{qstest}{encoding}{}
  \Expect*{\inputencodingname}{ascii}%
\end{qstest}
\begin{document}
  adieresis={\"a}, %
  germandbls={{\ss}}, %
  Euro=\texteuro{}%
%</test5>
%    \end{macrocode}
%
%    \begin{macrocode}
\end{document}
%</test>
%    \end{macrocode}
%
% \section{Installation}
%
% \subsection{Download}
%
% \paragraph{Package.} This package is available on
% CTAN\footnote{\url{ftp://ftp.ctan.org/tex-archive/}}:
% \begin{description}
% \item[\CTAN{macros/latex/contrib/oberdiek/selinput.dtx}] The source file.
% \item[\CTAN{macros/latex/contrib/oberdiek/selinput.pdf}] Documentation.
% \end{description}
%
%
% \paragraph{Bundle.} All the packages of the bundle `oberdiek'
% are also available in a TDS compliant ZIP archive. There
% the packages are already unpacked and the documentation files
% are generated. The files and directories obey the TDS standard.
% \begin{description}
% \item[\CTAN{install/macros/latex/contrib/oberdiek.tds.zip}]
% \end{description}
% \emph{TDS} refers to the standard ``A Directory Structure
% for \TeX\ Files'' (\CTAN{tds/tds.pdf}). Directories
% with \xfile{texmf} in their name are usually organized this way.
%
% \subsection{Bundle installation}
%
% \paragraph{Unpacking.} Unpack the \xfile{oberdiek.tds.zip} in the
% TDS tree (also known as \xfile{texmf} tree) of your choice.
% Example (linux):
% \begin{quote}
%   |unzip oberdiek.tds.zip -d ~/texmf|
% \end{quote}
%
% \paragraph{Script installation.}
% Check the directory \xfile{TDS:scripts/oberdiek/} for
% scripts that need further installation steps.
% Package \xpackage{attachfile2} comes with the Perl script
% \xfile{pdfatfi.pl} that should be installed in such a way
% that it can be called as \texttt{pdfatfi}.
% Example (linux):
% \begin{quote}
%   |chmod +x scripts/oberdiek/pdfatfi.pl|\\
%   |cp scripts/oberdiek/pdfatfi.pl /usr/local/bin/|
% \end{quote}
%
% \subsection{Package installation}
%
% \paragraph{Unpacking.} The \xfile{.dtx} file is a self-extracting
% \docstrip\ archive. The files are extracted by running the
% \xfile{.dtx} through \plainTeX:
% \begin{quote}
%   \verb|tex selinput.dtx|
% \end{quote}
%
% \paragraph{TDS.} Now the different files must be moved into
% the different directories in your installation TDS tree
% (also known as \xfile{texmf} tree):
% \begin{quote}
% \def\t{^^A
% \begin{tabular}{@{}>{\ttfamily}l@{ $\rightarrow$ }>{\ttfamily}l@{}}
%   selinput.sty & tex/latex/oberdiek/selinput.sty\\
%   selinput.pdf & doc/latex/oberdiek/selinput.pdf\\
%   test/selinput-test1.tex & doc/latex/oberdiek/test/selinput-test1.tex\\
%   test/selinput-test2.tex & doc/latex/oberdiek/test/selinput-test2.tex\\
%   test/selinput-test3.tex & doc/latex/oberdiek/test/selinput-test3.tex\\
%   test/selinput-test4.tex & doc/latex/oberdiek/test/selinput-test4.tex\\
%   test/selinput-test5.tex & doc/latex/oberdiek/test/selinput-test5.tex\\
%   selinput.dtx & source/latex/oberdiek/selinput.dtx\\
% \end{tabular}^^A
% }^^A
% \sbox0{\t}^^A
% \ifdim\wd0>\linewidth
%   \begingroup
%     \advance\linewidth by\leftmargin
%     \advance\linewidth by\rightmargin
%   \edef\x{\endgroup
%     \def\noexpand\lw{\the\linewidth}^^A
%   }\x
%   \def\lwbox{^^A
%     \leavevmode
%     \hbox to \linewidth{^^A
%       \kern-\leftmargin\relax
%       \hss
%       \usebox0
%       \hss
%       \kern-\rightmargin\relax
%     }^^A
%   }^^A
%   \ifdim\wd0>\lw
%     \sbox0{\small\t}^^A
%     \ifdim\wd0>\linewidth
%       \ifdim\wd0>\lw
%         \sbox0{\footnotesize\t}^^A
%         \ifdim\wd0>\linewidth
%           \ifdim\wd0>\lw
%             \sbox0{\scriptsize\t}^^A
%             \ifdim\wd0>\linewidth
%               \ifdim\wd0>\lw
%                 \sbox0{\tiny\t}^^A
%                 \ifdim\wd0>\linewidth
%                   \lwbox
%                 \else
%                   \usebox0
%                 \fi
%               \else
%                 \lwbox
%               \fi
%             \else
%               \usebox0
%             \fi
%           \else
%             \lwbox
%           \fi
%         \else
%           \usebox0
%         \fi
%       \else
%         \lwbox
%       \fi
%     \else
%       \usebox0
%     \fi
%   \else
%     \lwbox
%   \fi
% \else
%   \usebox0
% \fi
% \end{quote}
% If you have a \xfile{docstrip.cfg} that configures and enables \docstrip's
% TDS installing feature, then some files can already be in the right
% place, see the documentation of \docstrip.
%
% \subsection{Refresh file name databases}
%
% If your \TeX~distribution
% (\teTeX, \mikTeX, \dots) relies on file name databases, you must refresh
% these. For example, \teTeX\ users run \verb|texhash| or
% \verb|mktexlsr|.
%
% \subsection{Some details for the interested}
%
% \paragraph{Attached source.}
%
% The PDF documentation on CTAN also includes the
% \xfile{.dtx} source file. It can be extracted by
% AcrobatReader 6 or higher. Another option is \textsf{pdftk},
% e.g. unpack the file into the current directory:
% \begin{quote}
%   \verb|pdftk selinput.pdf unpack_files output .|
% \end{quote}
%
% \paragraph{Unpacking with \LaTeX.}
% The \xfile{.dtx} chooses its action depending on the format:
% \begin{description}
% \item[\plainTeX:] Run \docstrip\ and extract the files.
% \item[\LaTeX:] Generate the documentation.
% \end{description}
% If you insist on using \LaTeX\ for \docstrip\ (really,
% \docstrip\ does not need \LaTeX), then inform the autodetect routine
% about your intention:
% \begin{quote}
%   \verb|latex \let\install=y\input{selinput.dtx}|
% \end{quote}
% Do not forget to quote the argument according to the demands
% of your shell.
%
% \paragraph{Generating the documentation.}
% You can use both the \xfile{.dtx} or the \xfile{.drv} to generate
% the documentation. The process can be configured by the
% configuration file \xfile{ltxdoc.cfg}. For instance, put this
% line into this file, if you want to have A4 as paper format:
% \begin{quote}
%   \verb|\PassOptionsToClass{a4paper}{article}|
% \end{quote}
% An example follows how to generate the
% documentation with pdf\LaTeX:
% \begin{quote}
%\begin{verbatim}
%pdflatex selinput.dtx
%makeindex -s gind.ist selinput.idx
%pdflatex selinput.dtx
%makeindex -s gind.ist selinput.idx
%pdflatex selinput.dtx
%\end{verbatim}
% \end{quote}
%
% \section{Catalogue}
%
% The following XML file can be used as source for the
% \href{http://mirror.ctan.org/help/Catalogue/catalogue.html}{\TeX\ Catalogue}.
% The elements \texttt{caption} and \texttt{description} are imported
% from the original XML file from the Catalogue.
% The name of the XML file in the Catalogue is \xfile{selinput.xml}.
%    \begin{macrocode}
%<*catalogue>
<?xml version='1.0' encoding='us-ascii'?>
<!DOCTYPE entry SYSTEM 'catalogue.dtd'>
<entry datestamp='$Date$' modifier='$Author$' id='selinput'>
  <name>selinput</name>
  <caption>Semi-automatic detection of input encoding.</caption>
  <authorref id='auth:oberdiek'/>
  <copyright owner='Heiko Oberdiek' year='2007'/>
  <license type='lppl1.3'/>
  <version number='1.2'/>
  <description>
    This package selects the input encoding by specifying pairs
    of input characters and their glyph names.
    <p/>
    The package is part of the <xref refid='oberdiek'>oberdiek</xref>
    bundle.
  </description>
  <documentation details='Package documentation'
      href='ctan:/macros/latex/contrib/oberdiek/selinput.pdf'/>
  <ctan file='true' path='/macros/latex/contrib/oberdiek/selinput.dtx'/>
  <miktex location='oberdiek'/>
  <texlive location='oberdiek'/>
  <install path='/macros/latex/contrib/oberdiek/oberdiek.tds.zip'/>
</entry>
%</catalogue>
%    \end{macrocode}
%
% \begin{thebibliography}{9}
% \bibitem{inputenx}
%   Heiko Oberdiek: \textit{The \xpackage{inputenx} package};
%   2007-04-11 v1.1;
%   \CTAN{macros/latex/contrib/oberdiek/inputenx.pdf}.
%
% \bibitem{adobe:glyphlist}
%   Adobe: \textit{Adobe Glyph List};
%   2002-09-20 v2.0;
%   \url{http://partners.adobe.com/public/developer/en/opentype/glyphlist.txt}.
%
% \bibitem{adobe:aglfn}
%   Adobe: \textit{Adobe Glyph List For New Fonts};
%   2005-11-18 v1.5;
%   \url{http://partners.adobe.com/public/developer/en/opentype/aglfn13.txt}.
%
% \bibitem{stringenc}
%   Heiko Oberdiek: \textit{The \xpackage{stringenc} package};
%   2007-06-16 v1.1;
%   \CTAN{macros/latex/contrib/oberdiek/stringenc.pdf}.
%
% \end{thebibliography}
%
% \begin{History}
%   \begin{Version}{2007/06/16 v1.0}
%   \item
%     First version.
%   \end{Version}
%   \begin{Version}{2007/06/20 v1.1}
%   \item
%     Requested date for package \xpackage{stringenc} fixed.
%   \end{Version}
%   \begin{Version}{2007/09/09 v1.2}
%   \item
%     Line end fixed.
%   \end{Version}
% \end{History}
%
% \PrintIndex
%
% \Finale
\endinput
|
% \end{quote}
% Do not forget to quote the argument according to the demands
% of your shell.
%
% \paragraph{Generating the documentation.}
% You can use both the \xfile{.dtx} or the \xfile{.drv} to generate
% the documentation. The process can be configured by the
% configuration file \xfile{ltxdoc.cfg}. For instance, put this
% line into this file, if you want to have A4 as paper format:
% \begin{quote}
%   \verb|\PassOptionsToClass{a4paper}{article}|
% \end{quote}
% An example follows how to generate the
% documentation with pdf\LaTeX:
% \begin{quote}
%\begin{verbatim}
%pdflatex selinput.dtx
%makeindex -s gind.ist selinput.idx
%pdflatex selinput.dtx
%makeindex -s gind.ist selinput.idx
%pdflatex selinput.dtx
%\end{verbatim}
% \end{quote}
%
% \section{Catalogue}
%
% The following XML file can be used as source for the
% \href{http://mirror.ctan.org/help/Catalogue/catalogue.html}{\TeX\ Catalogue}.
% The elements \texttt{caption} and \texttt{description} are imported
% from the original XML file from the Catalogue.
% The name of the XML file in the Catalogue is \xfile{selinput.xml}.
%    \begin{macrocode}
%<*catalogue>
<?xml version='1.0' encoding='us-ascii'?>
<!DOCTYPE entry SYSTEM 'catalogue.dtd'>
<entry datestamp='$Date$' modifier='$Author$' id='selinput'>
  <name>selinput</name>
  <caption>Semi-automatic detection of input encoding.</caption>
  <authorref id='auth:oberdiek'/>
  <copyright owner='Heiko Oberdiek' year='2007'/>
  <license type='lppl1.3'/>
  <version number='1.2'/>
  <description>
    This package selects the input encoding by specifying pairs
    of input characters and their glyph names.
    <p/>
    The package is part of the <xref refid='oberdiek'>oberdiek</xref>
    bundle.
  </description>
  <documentation details='Package documentation'
      href='ctan:/macros/latex/contrib/oberdiek/selinput.pdf'/>
  <ctan file='true' path='/macros/latex/contrib/oberdiek/selinput.dtx'/>
  <miktex location='oberdiek'/>
  <texlive location='oberdiek'/>
  <install path='/macros/latex/contrib/oberdiek/oberdiek.tds.zip'/>
</entry>
%</catalogue>
%    \end{macrocode}
%
% \begin{thebibliography}{9}
% \bibitem{inputenx}
%   Heiko Oberdiek: \textit{The \xpackage{inputenx} package};
%   2007-04-11 v1.1;
%   \CTAN{macros/latex/contrib/oberdiek/inputenx.pdf}.
%
% \bibitem{adobe:glyphlist}
%   Adobe: \textit{Adobe Glyph List};
%   2002-09-20 v2.0;
%   \url{http://partners.adobe.com/public/developer/en/opentype/glyphlist.txt}.
%
% \bibitem{adobe:aglfn}
%   Adobe: \textit{Adobe Glyph List For New Fonts};
%   2005-11-18 v1.5;
%   \url{http://partners.adobe.com/public/developer/en/opentype/aglfn13.txt}.
%
% \bibitem{stringenc}
%   Heiko Oberdiek: \textit{The \xpackage{stringenc} package};
%   2007-06-16 v1.1;
%   \CTAN{macros/latex/contrib/oberdiek/stringenc.pdf}.
%
% \end{thebibliography}
%
% \begin{History}
%   \begin{Version}{2007/06/16 v1.0}
%   \item
%     First version.
%   \end{Version}
%   \begin{Version}{2007/06/20 v1.1}
%   \item
%     Requested date for package \xpackage{stringenc} fixed.
%   \end{Version}
%   \begin{Version}{2007/09/09 v1.2}
%   \item
%     Line end fixed.
%   \end{Version}
% \end{History}
%
% \PrintIndex
%
% \Finale
\endinput

%        (quote the arguments according to the demands of your shell)
%
% Documentation:
%    (a) If selinput.drv is present:
%           latex selinput.drv
%    (b) Without selinput.drv:
%           latex selinput.dtx; ...
%    The class ltxdoc loads the configuration file ltxdoc.cfg
%    if available. Here you can specify further options, e.g.
%    use A4 as paper format:
%       \PassOptionsToClass{a4paper}{article}
%
%    Programm calls to get the documentation (example):
%       pdflatex selinput.dtx
%       makeindex -s gind.ist selinput.idx
%       pdflatex selinput.dtx
%       makeindex -s gind.ist selinput.idx
%       pdflatex selinput.dtx
%
% Installation:
%    TDS:tex/latex/oberdiek/selinput.sty
%    TDS:doc/latex/oberdiek/selinput.pdf
%    TDS:doc/latex/oberdiek/test/selinput-test1.tex
%    TDS:doc/latex/oberdiek/test/selinput-test2.tex
%    TDS:doc/latex/oberdiek/test/selinput-test3.tex
%    TDS:doc/latex/oberdiek/test/selinput-test4.tex
%    TDS:doc/latex/oberdiek/test/selinput-test5.tex
%    TDS:source/latex/oberdiek/selinput.dtx
%
%<*ignore>
\begingroup
  \catcode123=1 %
  \catcode125=2 %
  \def\x{LaTeX2e}%
\expandafter\endgroup
\ifcase 0\ifx\install y1\fi\expandafter
         \ifx\csname processbatchFile\endcsname\relax\else1\fi
         \ifx\fmtname\x\else 1\fi\relax
\else\csname fi\endcsname
%</ignore>
%<*install>
\input docstrip.tex
\Msg{************************************************************************}
\Msg{* Installation}
\Msg{* Package: selinput 2007/09/09 v1.2 Semi-automatic input encoding detection (HO)}
\Msg{************************************************************************}

\keepsilent
\askforoverwritefalse

\let\MetaPrefix\relax
\preamble

This is a generated file.

Project: selinput
Version: 2007/09/09 v1.2

Copyright (C) 2007 by
   Heiko Oberdiek <heiko.oberdiek at googlemail.com>

This work may be distributed and/or modified under the
conditions of the LaTeX Project Public License, either
version 1.3c of this license or (at your option) any later
version. This version of this license is in
   http://www.latex-project.org/lppl/lppl-1-3c.txt
and the latest version of this license is in
   http://www.latex-project.org/lppl.txt
and version 1.3 or later is part of all distributions of
LaTeX version 2005/12/01 or later.

This work has the LPPL maintenance status "maintained".

This Current Maintainer of this work is Heiko Oberdiek.

This work consists of the main source file selinput.dtx
and the derived files
   selinput.sty, selinput.pdf, selinput.ins, selinput.drv,
   selinput-test1.tex, selinput-test2.tex, selinput-test3.tex,
   selinput-test4.tex, selinput-test5.tex.

\endpreamble
\let\MetaPrefix\DoubleperCent

\generate{%
  \file{selinput.ins}{\from{selinput.dtx}{install}}%
  \file{selinput.drv}{\from{selinput.dtx}{driver}}%
  \usedir{tex/latex/oberdiek}%
  \file{selinput.sty}{\from{selinput.dtx}{package}}%
  \usedir{doc/latex/oberdiek/test}%
  \file{selinput-test1.tex}{\from{selinput.dtx}{test,test1}}%
  \file{selinput-test2.tex}{\from{selinput.dtx}{test,test2}}%
  \file{selinput-test3.tex}{\from{selinput.dtx}{test,test3}}%
  \file{selinput-test4.tex}{\from{selinput.dtx}{test,test4}}%
  \file{selinput-test5.tex}{\from{selinput.dtx}{test,test5}}%
  \nopreamble
  \nopostamble
  \usedir{source/latex/oberdiek/catalogue}%
  \file{selinput.xml}{\from{selinput.dtx}{catalogue}}%
}

\catcode32=13\relax% active space
\let =\space%
\Msg{************************************************************************}
\Msg{*}
\Msg{* To finish the installation you have to move the following}
\Msg{* file into a directory searched by TeX:}
\Msg{*}
\Msg{*     selinput.sty}
\Msg{*}
\Msg{* To produce the documentation run the file `selinput.drv'}
\Msg{* through LaTeX.}
\Msg{*}
\Msg{* Happy TeXing!}
\Msg{*}
\Msg{************************************************************************}

\endbatchfile
%</install>
%<*ignore>
\fi
%</ignore>
%<*driver>
\NeedsTeXFormat{LaTeX2e}
\ProvidesFile{selinput.drv}%
  [2007/09/09 v1.2 Semi-automatic input encoding detection (HO)]%
\documentclass{ltxdoc}
\usepackage[T1]{fontenc}
\usepackage{textcomp}
\usepackage{lmodern}
\usepackage{holtxdoc}[2011/11/22]
\usepackage{color}
\begin{document}
  \DocInput{selinput.dtx}%
\end{document}
%</driver>
% \fi
%
% \CheckSum{389}
%
% \CharacterTable
%  {Upper-case    \A\B\C\D\E\F\G\H\I\J\K\L\M\N\O\P\Q\R\S\T\U\V\W\X\Y\Z
%   Lower-case    \a\b\c\d\e\f\g\h\i\j\k\l\m\n\o\p\q\r\s\t\u\v\w\x\y\z
%   Digits        \0\1\2\3\4\5\6\7\8\9
%   Exclamation   \!     Double quote  \"     Hash (number) \#
%   Dollar        \$     Percent       \%     Ampersand     \&
%   Acute accent  \'     Left paren    \(     Right paren   \)
%   Asterisk      \*     Plus          \+     Comma         \,
%   Minus         \-     Point         \.     Solidus       \/
%   Colon         \:     Semicolon     \;     Less than     \<
%   Equals        \=     Greater than  \>     Question mark \?
%   Commercial at \@     Left bracket  \[     Backslash     \\
%   Right bracket \]     Circumflex    \^     Underscore    \_
%   Grave accent  \`     Left brace    \{     Vertical bar  \|
%   Right brace   \}     Tilde         \~}
%
% \GetFileInfo{selinput.drv}
%
% \title{The \xpackage{selinput} package}
% \date{2007/09/09 v1.2}
% \author{Heiko Oberdiek\\\xemail{heiko.oberdiek at googlemail.com}}
%
% \maketitle
%
% \begin{abstract}
% This package selects the input encoding by specifying between
% input characters and their glyph names.
% \end{abstract}
%
% \tableofcontents
%
% \newcommand*{\EM}{\textcolor{blue}}
% \newcommand*{\ExampleText}{^^A
%   Umlauts:\ \EM{\"A\"O\"U\"a\"o\"u\ss}^^A
% }
%
% \section{Documentation}
%
% \subsection{Introduction}
%
% \LaTeX\ supports the direct use of 8-bit characters by means
% of package \xpackage{inputenc}. However you must know
% and specify the encoding, e.g.:
% \begin{quote}
%   \ttfamily
%   |\documentclass{article}|\\
%   |\usepackage[|\EM{latin1}|]{inputenc}|\\
%   |% or \usepackage[|\EM{utf8}|]{inputenc}|\\
%   |% or \usepackage[|\EM{??}|]{inputenc}|\\
%   |\begin{document}|\\
%   |  |\ExampleText\\
%   |\end{document}|
% \end{quote}
%
% If the document is transferred in an environment that
% uses a different encoding, then there are programs that
% convert the input characters. Examples for conversion
% of file \xfile{test.tex}
% from encoding latin1 (ISO-8859-1) to UTF-8:
% \begin{quote}
%   \ttfamily
%   |recode ISO-8859-1..UTF-8 test.tex|\\
%   |recode latin1..utf8 test.tex|\\
%   |iconv --from-code ISO-8859-1|\\
%   \hphantom{iconv}| --to-code UTF-8|\\
%   \hphantom{iconv}| --output testnew.tex|\\
%   \hphantom{iconv}| test.tex|\\
%   |iconv -f latin1 -t utf8 -o testnew.tex test.tex|
% \end{quote}
% However, the encoding name for package \xpackage{inputenc}
% must be changed:
% \begin{quote}
%    |\usepackage[latin1]{inputenc}| $\rightarrow$
%    |\usepackage[utf8]{inputenc}|\kern-4pt\relax
% \end{quote}
% Of course, unless you are using some clever editor
% that knows package \xpackage{inputenc}, recodes
% the file and adjusts the option at the same time.
% But most editors can perhaps recode the file, but
% they let the option untouched.
%
% Therefore package \xpackage{selinput} chooses another way for
% specifying the input encoding. The encoding name is not needed
% at all. Some 8-bit characters are identified by their glyph
% name and the package chooses an appropriate encoding, example:
% \begin{quote}
%   \ttfamily
%   |\documentclass{article}|\\
%   |\usepackage{selinput}|\\
%   |\SelectInputMappings{|\\
%   |  adieresis={|\EM{\"a}|}|,\\
%   |  germandbls={|\EM{\ss}|}|,\\
%   |  Euro={|\EM{\texteuro}|}|,\\
%   |}|\\
%   |\begin{document}|\\
%   |  |\ExampleText\\
%   |\end{document}|
% \end{quote}
%
% \subsection{User interface}
%
% \begin{declcs}{SelectInputEncodingList} \M{encoding list}
% \end{declcs}
% \cs{SelectInputEncodingList} expects a comma separated list of
% encoding names. Example:
% \begin{quote}
%   |\SelectInputEncodingList{utf8,ansinew,mac-roman}|
% \end{quote}
% The encodings of package \xpackage{inputenx} are used as default.
%
% \begin{declcs}{SelectInputMappings} \M{mapping pairs}
% \end{declcs}
% A mapping pair consists of a glyph name and its input
% character:
% \begin{quote}
%   |\SelectInputMappings{|\\
%   |  adieresis={|\EM{\"a}|}|,\\
%   |  germandbls={|\EM{\ss}|}|,\\
%   |  Euro={|\EM{\texteuro}|}|,\\
%   |}|
% \end{quote}
% The supported glyph names can be found in file \xfile{ix-name.def}
% of project \xpackage{inputenx} \cite{inputenx}. The names are
% basically taken from Adobe's glyphlists \cite{adobe:glyphlist,adobe:aglfn}.
% As many pairs are needed as necessary to identify the encoding.
% Example with insufficient pairs:
% \begin{quote}
%   \ttfamily
%   |\SelectInputEncodingSet{latin1,latin9}|\\
%   |\SelectInputMappings{|\\
%   |  adieresis={|\EM{\"a}|}|,\\
%   |  germandbls={|\EM{\ss}|}|,\\
%   |}|\\
%   \ExampleText| and Euro: |\EM{\textcurrency} (wrong)
% \end{quote}
% The first encoding \xoption{latin1} passes the constraints given
% by the mapping pairs. However the Euro symbol is not part of
% the encoding. Thus a mapping pair with the Euro symbol
% solves the problem. In fact the symbol alone already succeeds in selecting
% between \xoption{latin1} and \xoption{latin9}:
% \begin{quote}
%   \ttfamily
%   |\SelectInputEncodingSet{latin1,latin9}|\\
%   |\SelectInputMappings{|\\
%   |  Euro={|\EM{\texteuro}|},|\\
%   |}|\\
%   \ExampleText| and Euro: |\EM{\texteuro}
% \end{quote}
%
% \subsection{Options}
%
% \begin{description}
% \item[\xoption{warning}:]
%   The selected encoding is written
%   by \cs{PackageInfo} into the \xfile{.log} file only.
%   Option \xoption{warning} changes it to \cs{PackageWarning}.
%   Then the selected encoding is shown on the terminal as well.
% \item[\xoption{ucs}:]
%   The encoding file \xfile{utf8x} of package \cs{ucs} requires
%   that the package itself is loaded before.
%   If the package is not loaded, then the option \xoption{ucs}
%   will load package \xpackage{ucs} if the detected encoding is
%   UTF-8 (limited to the preamble, packages cannot be loaded later).
% \item[\xoption{utf8=\dots}:]
%   The option allows to specify other encoding files
%   for UTF-8 than \LaTeX's \xfile{utf8.def}. For example,
%   |utf8=utf-8| will load \xfile{utf-8.def} instead.
% \end{description}
%
% \subsection{Encodings}
%
% Package \xpackage{stringenc} \cite{stringenc}
% is used for testing the encoding. Thus the encoding
% name must be known by this package. Then the found
% encoding is loaded by \cs{inputencoding} by package
% \xpackage{inputenc} or \cs{InputEncoding} if package
% \xpackage{inputenx} is loaded.
%
% The supported encodings are present in the encoding list,
% thus usually the encoding names do not matter.
% If the list is set by \cs{SelectInputEncodingList},
% then you can use the names that work for package
% \xpackage{inputenc} and are known by package \xpackage{stringenc},
% for example: \xoption{latin1}, \xoption{x-iso-8859-1}. Encoding
% file names of package \xpackage{inputenx} are prefixed with \xfile{x-}.
% The prefix can be dropped, if package \xpackage{inputenx} is loaded.
%
% \StopEventually{
% }
%
% \section{Implementation}
%
%    \begin{macrocode}
%<*package>
\NeedsTeXFormat{LaTeX2e}
\ProvidesPackage{selinput}
  [2007/09/09 v1.2 Semi-automatic input encoding detection (HO)]%
%    \end{macrocode}
%
%    \begin{macrocode}
\RequirePackage{inputenc}
\RequirePackage{kvsetkeys}[2006/10/19]
\RequirePackage{stringenc}[2007/06/16]
\RequirePackage{kvoptions}
%    \end{macrocode}
%    \begin{macro}{\SelectInputEncodingList}
%    \begin{macrocode}
\newcommand*{\SelectInputEncodingList}{%
  \let\SIE@EncodingList\@empty
  \kvsetkeys{SelInputEnc}%
}
%    \end{macrocode}
%    \end{macro}
%    \begin{macro}{\SelectInputMappings}
%    \begin{macrocode}
\newcommand*{\SelectInputMappings}[1]{%
  \SIE@LoadNameDefs
  \let\SIE@StringUnicode\@empty
  \let\SIE@StringDest\@empty
  \kvsetkeys{SelInputMap}{#1}%
  \ifx\\SIE@StringUnicode\SIE@StringDest\\%
    \PackageError{selinput}{%
      No mappings specified%
    }\@ehc
  \else
    \EdefUnescapeHex\SIE@StringUnicode\SIE@StringUnicode
    \let\SIE@Encoding\@empty
    \@for\SIE@EncodingTest:=\SIE@EncodingList\do{%
      \ifx\SIE@Encoding\@empty
        \StringEncodingConvertTest\SIE@temp\SIE@StringUnicode
                                  {utf16be}\SIE@EncodingTest{%
          \ifx\SIE@temp\SIE@StringDest
            \let\SIE@Encoding\SIE@EncodingTest
          \fi
        }{}%
      \fi
    }%
    \ifx\SIE@Encoding\@empty
      \StringEncodingConvertTest\SIE@temp\SIE@StringDest
                                {ascii}{utf16be}{%
        \def\SIE@Encoding{ascii}%
        \SIE@Info{selinput}{%
          Matching encoding not found, but input characters%
          \MessageBreak
          are 7-bit (possibly editor replacements).%
          \MessageBreak
          Hence using ascii encoding%
        }%
      }{}%
    \fi
    \ifx\SIE@Encoding\@empty
      \PackageError{selinput}{%
        Cannot find a matching encoding%
      }\@ehd
    \else
      \ifx\SIE@Encoding\SIE@EncodingUTFviii
        \SIE@LoadUnicodePackage
        \ifx\SIE@UseUTFviii\@empty
        \else
          \let\SIE@Encoding\SIE@UseUTFviii
        \fi
      \fi
      \begingroup\expandafter\expandafter\expandafter\endgroup
      \expandafter\ifx\csname InputEncoding\endcsname\relax
        \inputencoding\SIE@Encoding
      \else
        \InputEncoding\SIE@Encoding
      \fi
      \SIE@Info{selinput}{Encoding `\SIE@Encoding' selected}%
    \fi
  \fi
}
%    \end{macrocode}
%    \end{macro}
%    \begin{macro}{\SIE@LoadNameDefs}
%    \begin{macrocode}
\def\SIE@LoadNameDefs{%
  \begingroup
    \endlinechar=\m@ne
    \catcode92=0 % backslash
    \catcode123=1 % left curly brace/beginning of group
    \catcode125=2 % right curly brace/end of group
    \catcode37=14 % percent/comment character
    \@makeother\[%
    \@makeother\]%
    \@makeother\.%
    \@makeother\(%
    \@makeother\)%
    \@makeother\/%
    \@makeother\-%
    \let\InputenxName\SelectInputDefineMapping
    \InputIfFileExists{ix-name.def}{}{%
      \PackageError{selinput}{%
        Missing `ix-name.def' (part of package `inputenx')%
      }\@ehd
    }%
    \global\let\SIE@LoadNameDefs\relax
  \endgroup
}
%    \end{macrocode}
%    \end{macro}
%    \begin{macro}{\SelectInputDefineMapping}
%    \begin{macrocode}
\newcommand*{\SelectInputDefineMapping}[1]{%
  \expandafter\gdef\csname SIE@@#1\endcsname
}
%    \end{macrocode}
%    \end{macro}
%    \begin{macrocode}
\kv@set@family@handler{SelInputMap}{%
  \@onelevel@sanitize\kv@key
  \ifx\kv@value\relax
    \PackageError{selinput}{%
      Missing input character for `\kv@key'%
    }\@ehc
  \else
    \@onelevel@sanitize\kv@value
    \ifx\kv@value\@empty
      \PackageError{selinput}{%
        Input character got lost?\MessageBreak
        Missing input character for `\kv@key'%
      }\@ehc
    \else
      \@ifundefined{SIE@@\kv@key}{%
        \PackageWarning{selinput}{%
          Missing definition for `\kv@key'%
        }%
      }{%
        \edef\SIE@StringDest{%
          \SIE@StringDest
          \kv@value
        }%
        \edef\SIE@StringUnicode{%
          \SIE@StringUnicode
          \csname SIE@@\kv@key\endcsname
        }%
      }%
    \fi
  \fi
}
%    \end{macrocode}
%    \begin{macrocode}
\kv@set@family@handler{SelInputEnc}{%
  \@onelevel@sanitize\kv@key
  \ifx\kv@value\relax
    \ifx\SIE@EncodingList\@empty
      \let\SIE@EncodingList\kv@key
    \else
      \edef\SIE@EncodingList{\SIE@EncodingList,\kv@key}%
    \fi
  \else
    \@onelevel@sanitize\kv@value
    \PackageError{selinput}{%
      Illegal key value pair (\kv@key=\kv@value)\MessagBreak
      in encoding list%
    }\@ehc
  \fi
}
%    \end{macrocode}
%
%    \begin{macro}{\SIE@LoadUnicodePackage}
%    \begin{macrocode}
\def\SIE@LoadUnicodePackage{%
  \@ifpackageloaded\SIE@UnicodePackage{}{%
    \RequirePackage\SIE@UnicodePackage\relax
  }%
  \SIE@PatchUCS
  \global\let\SIE@LoadUnicodePackage\relax
}
\let\SIE@show\show
\def\SIE@PatchUCS{%
  \AtBeginDocument{%
    \expandafter\ifx\csname ver@ucsencs.def\endcsname\relax
    \else
      \let\show\SIE@show
    \fi
  }%
}
\SIE@PatchUCS
%    \end{macrocode}
%    \end{macro}
%    \begin{macrocode}
\AtBeginDocument{%
  \let\SIE@LoadUnicodePackage\relax
}
%    \end{macrocode}
%    \begin{macro}{\SIE@EncodingUTFviii}
%    \begin{macrocode}
\def\SIE@EncodingUTFviii{utf8}
\@onelevel@sanitize\SIE@EncodingUTFviii
%    \end{macrocode}
%    \end{macro}
%    \begin{macro}{\SIE@EncodingUTFviiix}
%    \begin{macrocode}
\def\SIE@EncodingUTFviiix{utf8x}
\@onelevel@sanitize\SIE@EncodingUTFviiix
%    \end{macrocode}
%    \end{macro}
%
%    \begin{macrocode}
\let\SIE@UnicodePackage\@empty
\let\SIE@UseUTFviii\@empty
\let\SIE@Info\PackageInfo
%    \end{macrocode}
%    \begin{macrocode}
\SetupKeyvalOptions{%
  family=SelInput,%
  prefix=SelInput@%
}
\define@key{SelInput}{utf8}{%
  \def\SIE@UseUTFviii{#1}%
  \@onelevel@sanitize\SIE@UseUTFviii
}
\DeclareBoolOption{ucs}
\DeclareVoidOption{warning}{%
  \let\SIE@Info\PackageWarning
}
\ProcessKeyvalOptions{SelInput}
\ifSelInput@ucs
  \def\SIE@UnicodePackage{ucs}%
  \ifx\SIE@UseUTFviii\@empty
    \let\SIE@UseUTFviii\SIE@EncodingUTFviiix
  \fi
\else
  \ifx\SIE@UseUTFviii\@empty
    \@ifpackageloaded{ucs}{%
      \let\SIE@UseUTFviii\SIE@EncodingUTFviiix
    }{%
      \let\SIE@UseUTFviii\SIE@EncodingUTFviii
    }%
  \fi
\fi
%    \end{macrocode}
%
%    \begin{macro}{\SIE@EncodingList}
%    \begin{macrocode}
\edef\SIE@EncodingList{%
  utf8,%
  x-iso-8859-1,%
  x-iso-8859-15,%
  x-cp1252,% ansinew
  x-mac-roman,%
  x-iso-8859-2,%
  x-iso-8859-3,%
  x-iso-8859-4,%
  x-iso-8859-5,%
  x-iso-8859-6,%
  x-iso-8859-7,%
  x-iso-8859-8,%
  x-iso-8859-9,%
  x-iso-8859-10,%
  x-iso-8859-11,%
  x-iso-8859-13,%
  x-iso-8859-14,%
  x-iso-8859-15,%
  x-mac-centeuro,%
  x-mac-cyrillic,%
  x-koi8-r,%
  x-cp1250,%
  x-cp1251,%
  x-cp1257,%
  x-cp437,%
  x-cp850,%
  x-cp852,%
  x-cp855,%
  x-cp858,%
  x-cp865,%
  x-cp866,%
  x-nextstep,%
  x-dec-mcs%
}%
\@onelevel@sanitize\SIE@EncodingList
%    \end{macrocode}
%    \end{macro}
%
%    \begin{macrocode}
%</package>
%    \end{macrocode}
%
% \section{Test}
%
%    \begin{macrocode}
%<*test>
\NeedsTeXFormat{LaTeX2e}
\documentclass{minimal}
\usepackage{textcomp}
\usepackage{qstest}
%    \end{macrocode}
%    \begin{macrocode}
%<*test1|test2|test3>
\makeatletter
\let\BeginDocumentText\@empty
\def\TestEncoding#1#2{%
  \SelectInputMappings{#2}%
  \Expect*{\SIE@Encoding}{#1}%
  \Expect*{\inputencodingname}{#1}%
  \g@addto@macro\BeginDocumentText{%
    \SelectInputMappings{#2}%
    \Expect*{\SIE@Encoding}{#1}%
    \textbf{\SIE@Encoding:} %
    \kvsetkeys{test}{#2}\par
  }%
}
\def\TestKey#1#2{%
  \define@key{test}{#1}{%
    \sbox0{##1}%
    \sbox2{#2}%
    \Expect*{wd:\the\wd0, ht:\the\ht0, dp:\the\dp0}%
           *{wd:\the\wd2, ht:\the\ht2, dp:\the\dp2}%
    [#1=##1] % hash-ok
  }%
}
\RequirePackage{keyval}
\TestKey{adieresis}{\"a}
\TestKey{germandbls}{\ss}
\TestKey{Euro}{\texteuro}
\makeatother
\usepackage[
  warning,%
%<test2>  utf8=utf-8,
%<test3>  ucs,
]{selinput}
%<test1|test3>\inputencoding{ascii}
%<test2>\inputencoding{utf-8}
%<test3>\usepackage{ucs}
\begin{qstest}{preamble}{}
  \TestEncoding{x-iso-8859-15}{%
    adieresis=^^e4,%
    germandbls=^^df,%
    Euro=^^a4,%
  }%
  \TestEncoding{x-cp1252}{%
    adieresis=^^e4,%
    germandbls=^^df,%
    Euro=^^80,%
  }%
%<test1>  \TestEncoding{utf8}{%
%<test2>  \TestEncoding{utf-8}{%
%<test3>  \TestEncoding{utf8x}{%
    adieresis=^^c3^^a4,%
    germandbls=^^c3^^9f,%
%<!test2>    Euro=^^e2^^82^^ac,
  }%
\end{qstest}
%<test3>\let\ifUnicodeOptiongraphics\iffalse
\begin{document}
\begin{qstest}{document}{}
%<test3>\makeatletter
  \BeginDocumentText
\end{qstest}
%</test1|test2|test3>
%    \end{macrocode}
%
%    \begin{macrocode}
%<*test4>
\usepackage[warning,ucs]{selinput}
\SelectInputMappings{%
    adieresis=^^c3^^a4,%
    germandbls=^^c3^^9f,%
    Euro=^^e2^^82^^ac,%
}
\begin{qstest}{encoding}{}
  \Expect*{\inputencodingname}{utf8x}%
\end{qstest}
\begin{document}
  adieresis=^^c3^^a4, %
  germandbls=^^c3^^9f, %
  Euro=^^e2^^82^^ac%
%</test4>
%    \end{macrocode}
%
%    \begin{macrocode}
%<*test5>
\usepackage[warning,ucs]{selinput}
\SelectInputMappings{%
    adieresis={\"a},%
    germandbls={{\ss}},%
    Euro=\texteuro{},%
}
\begin{qstest}{encoding}{}
  \Expect*{\inputencodingname}{ascii}%
\end{qstest}
\begin{document}
  adieresis={\"a}, %
  germandbls={{\ss}}, %
  Euro=\texteuro{}%
%</test5>
%    \end{macrocode}
%
%    \begin{macrocode}
\end{document}
%</test>
%    \end{macrocode}
%
% \section{Installation}
%
% \subsection{Download}
%
% \paragraph{Package.} This package is available on
% CTAN\footnote{\url{ftp://ftp.ctan.org/tex-archive/}}:
% \begin{description}
% \item[\CTAN{macros/latex/contrib/oberdiek/selinput.dtx}] The source file.
% \item[\CTAN{macros/latex/contrib/oberdiek/selinput.pdf}] Documentation.
% \end{description}
%
%
% \paragraph{Bundle.} All the packages of the bundle `oberdiek'
% are also available in a TDS compliant ZIP archive. There
% the packages are already unpacked and the documentation files
% are generated. The files and directories obey the TDS standard.
% \begin{description}
% \item[\CTAN{install/macros/latex/contrib/oberdiek.tds.zip}]
% \end{description}
% \emph{TDS} refers to the standard ``A Directory Structure
% for \TeX\ Files'' (\CTAN{tds/tds.pdf}). Directories
% with \xfile{texmf} in their name are usually organized this way.
%
% \subsection{Bundle installation}
%
% \paragraph{Unpacking.} Unpack the \xfile{oberdiek.tds.zip} in the
% TDS tree (also known as \xfile{texmf} tree) of your choice.
% Example (linux):
% \begin{quote}
%   |unzip oberdiek.tds.zip -d ~/texmf|
% \end{quote}
%
% \paragraph{Script installation.}
% Check the directory \xfile{TDS:scripts/oberdiek/} for
% scripts that need further installation steps.
% Package \xpackage{attachfile2} comes with the Perl script
% \xfile{pdfatfi.pl} that should be installed in such a way
% that it can be called as \texttt{pdfatfi}.
% Example (linux):
% \begin{quote}
%   |chmod +x scripts/oberdiek/pdfatfi.pl|\\
%   |cp scripts/oberdiek/pdfatfi.pl /usr/local/bin/|
% \end{quote}
%
% \subsection{Package installation}
%
% \paragraph{Unpacking.} The \xfile{.dtx} file is a self-extracting
% \docstrip\ archive. The files are extracted by running the
% \xfile{.dtx} through \plainTeX:
% \begin{quote}
%   \verb|tex selinput.dtx|
% \end{quote}
%
% \paragraph{TDS.} Now the different files must be moved into
% the different directories in your installation TDS tree
% (also known as \xfile{texmf} tree):
% \begin{quote}
% \def\t{^^A
% \begin{tabular}{@{}>{\ttfamily}l@{ $\rightarrow$ }>{\ttfamily}l@{}}
%   selinput.sty & tex/latex/oberdiek/selinput.sty\\
%   selinput.pdf & doc/latex/oberdiek/selinput.pdf\\
%   test/selinput-test1.tex & doc/latex/oberdiek/test/selinput-test1.tex\\
%   test/selinput-test2.tex & doc/latex/oberdiek/test/selinput-test2.tex\\
%   test/selinput-test3.tex & doc/latex/oberdiek/test/selinput-test3.tex\\
%   test/selinput-test4.tex & doc/latex/oberdiek/test/selinput-test4.tex\\
%   test/selinput-test5.tex & doc/latex/oberdiek/test/selinput-test5.tex\\
%   selinput.dtx & source/latex/oberdiek/selinput.dtx\\
% \end{tabular}^^A
% }^^A
% \sbox0{\t}^^A
% \ifdim\wd0>\linewidth
%   \begingroup
%     \advance\linewidth by\leftmargin
%     \advance\linewidth by\rightmargin
%   \edef\x{\endgroup
%     \def\noexpand\lw{\the\linewidth}^^A
%   }\x
%   \def\lwbox{^^A
%     \leavevmode
%     \hbox to \linewidth{^^A
%       \kern-\leftmargin\relax
%       \hss
%       \usebox0
%       \hss
%       \kern-\rightmargin\relax
%     }^^A
%   }^^A
%   \ifdim\wd0>\lw
%     \sbox0{\small\t}^^A
%     \ifdim\wd0>\linewidth
%       \ifdim\wd0>\lw
%         \sbox0{\footnotesize\t}^^A
%         \ifdim\wd0>\linewidth
%           \ifdim\wd0>\lw
%             \sbox0{\scriptsize\t}^^A
%             \ifdim\wd0>\linewidth
%               \ifdim\wd0>\lw
%                 \sbox0{\tiny\t}^^A
%                 \ifdim\wd0>\linewidth
%                   \lwbox
%                 \else
%                   \usebox0
%                 \fi
%               \else
%                 \lwbox
%               \fi
%             \else
%               \usebox0
%             \fi
%           \else
%             \lwbox
%           \fi
%         \else
%           \usebox0
%         \fi
%       \else
%         \lwbox
%       \fi
%     \else
%       \usebox0
%     \fi
%   \else
%     \lwbox
%   \fi
% \else
%   \usebox0
% \fi
% \end{quote}
% If you have a \xfile{docstrip.cfg} that configures and enables \docstrip's
% TDS installing feature, then some files can already be in the right
% place, see the documentation of \docstrip.
%
% \subsection{Refresh file name databases}
%
% If your \TeX~distribution
% (\teTeX, \mikTeX, \dots) relies on file name databases, you must refresh
% these. For example, \teTeX\ users run \verb|texhash| or
% \verb|mktexlsr|.
%
% \subsection{Some details for the interested}
%
% \paragraph{Attached source.}
%
% The PDF documentation on CTAN also includes the
% \xfile{.dtx} source file. It can be extracted by
% AcrobatReader 6 or higher. Another option is \textsf{pdftk},
% e.g. unpack the file into the current directory:
% \begin{quote}
%   \verb|pdftk selinput.pdf unpack_files output .|
% \end{quote}
%
% \paragraph{Unpacking with \LaTeX.}
% The \xfile{.dtx} chooses its action depending on the format:
% \begin{description}
% \item[\plainTeX:] Run \docstrip\ and extract the files.
% \item[\LaTeX:] Generate the documentation.
% \end{description}
% If you insist on using \LaTeX\ for \docstrip\ (really,
% \docstrip\ does not need \LaTeX), then inform the autodetect routine
% about your intention:
% \begin{quote}
%   \verb|latex \let\install=y% \iffalse meta-comment
%
% File: selinput.dtx
% Version: 2007/09/09 v1.2
% Info: Semi-automatic input encoding detection
%
% Copyright (C) 2007 by
%    Heiko Oberdiek <heiko.oberdiek at googlemail.com>
%
% This work may be distributed and/or modified under the
% conditions of the LaTeX Project Public License, either
% version 1.3c of this license or (at your option) any later
% version. This version of this license is in
%    http://www.latex-project.org/lppl/lppl-1-3c.txt
% and the latest version of this license is in
%    http://www.latex-project.org/lppl.txt
% and version 1.3 or later is part of all distributions of
% LaTeX version 2005/12/01 or later.
%
% This work has the LPPL maintenance status "maintained".
%
% This Current Maintainer of this work is Heiko Oberdiek.
%
% This work consists of the main source file selinput.dtx
% and the derived files
%    selinput.sty, selinput.pdf, selinput.ins, selinput.drv,
%    selinput-test1.tex, selinput-test2.tex, selinput-test3.tex,
%    selinput-test4.tex, selinput-test5.tex.
%
% Distribution:
%    CTAN:macros/latex/contrib/oberdiek/selinput.dtx
%    CTAN:macros/latex/contrib/oberdiek/selinput.pdf
%
% Unpacking:
%    (a) If selinput.ins is present:
%           tex selinput.ins
%    (b) Without selinput.ins:
%           tex selinput.dtx
%    (c) If you insist on using LaTeX
%           latex \let\install=y% \iffalse meta-comment
%
% File: selinput.dtx
% Version: 2007/09/09 v1.2
% Info: Semi-automatic input encoding detection
%
% Copyright (C) 2007 by
%    Heiko Oberdiek <heiko.oberdiek at googlemail.com>
%
% This work may be distributed and/or modified under the
% conditions of the LaTeX Project Public License, either
% version 1.3c of this license or (at your option) any later
% version. This version of this license is in
%    http://www.latex-project.org/lppl/lppl-1-3c.txt
% and the latest version of this license is in
%    http://www.latex-project.org/lppl.txt
% and version 1.3 or later is part of all distributions of
% LaTeX version 2005/12/01 or later.
%
% This work has the LPPL maintenance status "maintained".
%
% This Current Maintainer of this work is Heiko Oberdiek.
%
% This work consists of the main source file selinput.dtx
% and the derived files
%    selinput.sty, selinput.pdf, selinput.ins, selinput.drv,
%    selinput-test1.tex, selinput-test2.tex, selinput-test3.tex,
%    selinput-test4.tex, selinput-test5.tex.
%
% Distribution:
%    CTAN:macros/latex/contrib/oberdiek/selinput.dtx
%    CTAN:macros/latex/contrib/oberdiek/selinput.pdf
%
% Unpacking:
%    (a) If selinput.ins is present:
%           tex selinput.ins
%    (b) Without selinput.ins:
%           tex selinput.dtx
%    (c) If you insist on using LaTeX
%           latex \let\install=y\input{selinput.dtx}
%        (quote the arguments according to the demands of your shell)
%
% Documentation:
%    (a) If selinput.drv is present:
%           latex selinput.drv
%    (b) Without selinput.drv:
%           latex selinput.dtx; ...
%    The class ltxdoc loads the configuration file ltxdoc.cfg
%    if available. Here you can specify further options, e.g.
%    use A4 as paper format:
%       \PassOptionsToClass{a4paper}{article}
%
%    Programm calls to get the documentation (example):
%       pdflatex selinput.dtx
%       makeindex -s gind.ist selinput.idx
%       pdflatex selinput.dtx
%       makeindex -s gind.ist selinput.idx
%       pdflatex selinput.dtx
%
% Installation:
%    TDS:tex/latex/oberdiek/selinput.sty
%    TDS:doc/latex/oberdiek/selinput.pdf
%    TDS:doc/latex/oberdiek/test/selinput-test1.tex
%    TDS:doc/latex/oberdiek/test/selinput-test2.tex
%    TDS:doc/latex/oberdiek/test/selinput-test3.tex
%    TDS:doc/latex/oberdiek/test/selinput-test4.tex
%    TDS:doc/latex/oberdiek/test/selinput-test5.tex
%    TDS:source/latex/oberdiek/selinput.dtx
%
%<*ignore>
\begingroup
  \catcode123=1 %
  \catcode125=2 %
  \def\x{LaTeX2e}%
\expandafter\endgroup
\ifcase 0\ifx\install y1\fi\expandafter
         \ifx\csname processbatchFile\endcsname\relax\else1\fi
         \ifx\fmtname\x\else 1\fi\relax
\else\csname fi\endcsname
%</ignore>
%<*install>
\input docstrip.tex
\Msg{************************************************************************}
\Msg{* Installation}
\Msg{* Package: selinput 2007/09/09 v1.2 Semi-automatic input encoding detection (HO)}
\Msg{************************************************************************}

\keepsilent
\askforoverwritefalse

\let\MetaPrefix\relax
\preamble

This is a generated file.

Project: selinput
Version: 2007/09/09 v1.2

Copyright (C) 2007 by
   Heiko Oberdiek <heiko.oberdiek at googlemail.com>

This work may be distributed and/or modified under the
conditions of the LaTeX Project Public License, either
version 1.3c of this license or (at your option) any later
version. This version of this license is in
   http://www.latex-project.org/lppl/lppl-1-3c.txt
and the latest version of this license is in
   http://www.latex-project.org/lppl.txt
and version 1.3 or later is part of all distributions of
LaTeX version 2005/12/01 or later.

This work has the LPPL maintenance status "maintained".

This Current Maintainer of this work is Heiko Oberdiek.

This work consists of the main source file selinput.dtx
and the derived files
   selinput.sty, selinput.pdf, selinput.ins, selinput.drv,
   selinput-test1.tex, selinput-test2.tex, selinput-test3.tex,
   selinput-test4.tex, selinput-test5.tex.

\endpreamble
\let\MetaPrefix\DoubleperCent

\generate{%
  \file{selinput.ins}{\from{selinput.dtx}{install}}%
  \file{selinput.drv}{\from{selinput.dtx}{driver}}%
  \usedir{tex/latex/oberdiek}%
  \file{selinput.sty}{\from{selinput.dtx}{package}}%
  \usedir{doc/latex/oberdiek/test}%
  \file{selinput-test1.tex}{\from{selinput.dtx}{test,test1}}%
  \file{selinput-test2.tex}{\from{selinput.dtx}{test,test2}}%
  \file{selinput-test3.tex}{\from{selinput.dtx}{test,test3}}%
  \file{selinput-test4.tex}{\from{selinput.dtx}{test,test4}}%
  \file{selinput-test5.tex}{\from{selinput.dtx}{test,test5}}%
  \nopreamble
  \nopostamble
  \usedir{source/latex/oberdiek/catalogue}%
  \file{selinput.xml}{\from{selinput.dtx}{catalogue}}%
}

\catcode32=13\relax% active space
\let =\space%
\Msg{************************************************************************}
\Msg{*}
\Msg{* To finish the installation you have to move the following}
\Msg{* file into a directory searched by TeX:}
\Msg{*}
\Msg{*     selinput.sty}
\Msg{*}
\Msg{* To produce the documentation run the file `selinput.drv'}
\Msg{* through LaTeX.}
\Msg{*}
\Msg{* Happy TeXing!}
\Msg{*}
\Msg{************************************************************************}

\endbatchfile
%</install>
%<*ignore>
\fi
%</ignore>
%<*driver>
\NeedsTeXFormat{LaTeX2e}
\ProvidesFile{selinput.drv}%
  [2007/09/09 v1.2 Semi-automatic input encoding detection (HO)]%
\documentclass{ltxdoc}
\usepackage[T1]{fontenc}
\usepackage{textcomp}
\usepackage{lmodern}
\usepackage{holtxdoc}[2011/11/22]
\usepackage{color}
\begin{document}
  \DocInput{selinput.dtx}%
\end{document}
%</driver>
% \fi
%
% \CheckSum{389}
%
% \CharacterTable
%  {Upper-case    \A\B\C\D\E\F\G\H\I\J\K\L\M\N\O\P\Q\R\S\T\U\V\W\X\Y\Z
%   Lower-case    \a\b\c\d\e\f\g\h\i\j\k\l\m\n\o\p\q\r\s\t\u\v\w\x\y\z
%   Digits        \0\1\2\3\4\5\6\7\8\9
%   Exclamation   \!     Double quote  \"     Hash (number) \#
%   Dollar        \$     Percent       \%     Ampersand     \&
%   Acute accent  \'     Left paren    \(     Right paren   \)
%   Asterisk      \*     Plus          \+     Comma         \,
%   Minus         \-     Point         \.     Solidus       \/
%   Colon         \:     Semicolon     \;     Less than     \<
%   Equals        \=     Greater than  \>     Question mark \?
%   Commercial at \@     Left bracket  \[     Backslash     \\
%   Right bracket \]     Circumflex    \^     Underscore    \_
%   Grave accent  \`     Left brace    \{     Vertical bar  \|
%   Right brace   \}     Tilde         \~}
%
% \GetFileInfo{selinput.drv}
%
% \title{The \xpackage{selinput} package}
% \date{2007/09/09 v1.2}
% \author{Heiko Oberdiek\\\xemail{heiko.oberdiek at googlemail.com}}
%
% \maketitle
%
% \begin{abstract}
% This package selects the input encoding by specifying between
% input characters and their glyph names.
% \end{abstract}
%
% \tableofcontents
%
% \newcommand*{\EM}{\textcolor{blue}}
% \newcommand*{\ExampleText}{^^A
%   Umlauts:\ \EM{\"A\"O\"U\"a\"o\"u\ss}^^A
% }
%
% \section{Documentation}
%
% \subsection{Introduction}
%
% \LaTeX\ supports the direct use of 8-bit characters by means
% of package \xpackage{inputenc}. However you must know
% and specify the encoding, e.g.:
% \begin{quote}
%   \ttfamily
%   |\documentclass{article}|\\
%   |\usepackage[|\EM{latin1}|]{inputenc}|\\
%   |% or \usepackage[|\EM{utf8}|]{inputenc}|\\
%   |% or \usepackage[|\EM{??}|]{inputenc}|\\
%   |\begin{document}|\\
%   |  |\ExampleText\\
%   |\end{document}|
% \end{quote}
%
% If the document is transferred in an environment that
% uses a different encoding, then there are programs that
% convert the input characters. Examples for conversion
% of file \xfile{test.tex}
% from encoding latin1 (ISO-8859-1) to UTF-8:
% \begin{quote}
%   \ttfamily
%   |recode ISO-8859-1..UTF-8 test.tex|\\
%   |recode latin1..utf8 test.tex|\\
%   |iconv --from-code ISO-8859-1|\\
%   \hphantom{iconv}| --to-code UTF-8|\\
%   \hphantom{iconv}| --output testnew.tex|\\
%   \hphantom{iconv}| test.tex|\\
%   |iconv -f latin1 -t utf8 -o testnew.tex test.tex|
% \end{quote}
% However, the encoding name for package \xpackage{inputenc}
% must be changed:
% \begin{quote}
%    |\usepackage[latin1]{inputenc}| $\rightarrow$
%    |\usepackage[utf8]{inputenc}|\kern-4pt\relax
% \end{quote}
% Of course, unless you are using some clever editor
% that knows package \xpackage{inputenc}, recodes
% the file and adjusts the option at the same time.
% But most editors can perhaps recode the file, but
% they let the option untouched.
%
% Therefore package \xpackage{selinput} chooses another way for
% specifying the input encoding. The encoding name is not needed
% at all. Some 8-bit characters are identified by their glyph
% name and the package chooses an appropriate encoding, example:
% \begin{quote}
%   \ttfamily
%   |\documentclass{article}|\\
%   |\usepackage{selinput}|\\
%   |\SelectInputMappings{|\\
%   |  adieresis={|\EM{\"a}|}|,\\
%   |  germandbls={|\EM{\ss}|}|,\\
%   |  Euro={|\EM{\texteuro}|}|,\\
%   |}|\\
%   |\begin{document}|\\
%   |  |\ExampleText\\
%   |\end{document}|
% \end{quote}
%
% \subsection{User interface}
%
% \begin{declcs}{SelectInputEncodingList} \M{encoding list}
% \end{declcs}
% \cs{SelectInputEncodingList} expects a comma separated list of
% encoding names. Example:
% \begin{quote}
%   |\SelectInputEncodingList{utf8,ansinew,mac-roman}|
% \end{quote}
% The encodings of package \xpackage{inputenx} are used as default.
%
% \begin{declcs}{SelectInputMappings} \M{mapping pairs}
% \end{declcs}
% A mapping pair consists of a glyph name and its input
% character:
% \begin{quote}
%   |\SelectInputMappings{|\\
%   |  adieresis={|\EM{\"a}|}|,\\
%   |  germandbls={|\EM{\ss}|}|,\\
%   |  Euro={|\EM{\texteuro}|}|,\\
%   |}|
% \end{quote}
% The supported glyph names can be found in file \xfile{ix-name.def}
% of project \xpackage{inputenx} \cite{inputenx}. The names are
% basically taken from Adobe's glyphlists \cite{adobe:glyphlist,adobe:aglfn}.
% As many pairs are needed as necessary to identify the encoding.
% Example with insufficient pairs:
% \begin{quote}
%   \ttfamily
%   |\SelectInputEncodingSet{latin1,latin9}|\\
%   |\SelectInputMappings{|\\
%   |  adieresis={|\EM{\"a}|}|,\\
%   |  germandbls={|\EM{\ss}|}|,\\
%   |}|\\
%   \ExampleText| and Euro: |\EM{\textcurrency} (wrong)
% \end{quote}
% The first encoding \xoption{latin1} passes the constraints given
% by the mapping pairs. However the Euro symbol is not part of
% the encoding. Thus a mapping pair with the Euro symbol
% solves the problem. In fact the symbol alone already succeeds in selecting
% between \xoption{latin1} and \xoption{latin9}:
% \begin{quote}
%   \ttfamily
%   |\SelectInputEncodingSet{latin1,latin9}|\\
%   |\SelectInputMappings{|\\
%   |  Euro={|\EM{\texteuro}|},|\\
%   |}|\\
%   \ExampleText| and Euro: |\EM{\texteuro}
% \end{quote}
%
% \subsection{Options}
%
% \begin{description}
% \item[\xoption{warning}:]
%   The selected encoding is written
%   by \cs{PackageInfo} into the \xfile{.log} file only.
%   Option \xoption{warning} changes it to \cs{PackageWarning}.
%   Then the selected encoding is shown on the terminal as well.
% \item[\xoption{ucs}:]
%   The encoding file \xfile{utf8x} of package \cs{ucs} requires
%   that the package itself is loaded before.
%   If the package is not loaded, then the option \xoption{ucs}
%   will load package \xpackage{ucs} if the detected encoding is
%   UTF-8 (limited to the preamble, packages cannot be loaded later).
% \item[\xoption{utf8=\dots}:]
%   The option allows to specify other encoding files
%   for UTF-8 than \LaTeX's \xfile{utf8.def}. For example,
%   |utf8=utf-8| will load \xfile{utf-8.def} instead.
% \end{description}
%
% \subsection{Encodings}
%
% Package \xpackage{stringenc} \cite{stringenc}
% is used for testing the encoding. Thus the encoding
% name must be known by this package. Then the found
% encoding is loaded by \cs{inputencoding} by package
% \xpackage{inputenc} or \cs{InputEncoding} if package
% \xpackage{inputenx} is loaded.
%
% The supported encodings are present in the encoding list,
% thus usually the encoding names do not matter.
% If the list is set by \cs{SelectInputEncodingList},
% then you can use the names that work for package
% \xpackage{inputenc} and are known by package \xpackage{stringenc},
% for example: \xoption{latin1}, \xoption{x-iso-8859-1}. Encoding
% file names of package \xpackage{inputenx} are prefixed with \xfile{x-}.
% The prefix can be dropped, if package \xpackage{inputenx} is loaded.
%
% \StopEventually{
% }
%
% \section{Implementation}
%
%    \begin{macrocode}
%<*package>
\NeedsTeXFormat{LaTeX2e}
\ProvidesPackage{selinput}
  [2007/09/09 v1.2 Semi-automatic input encoding detection (HO)]%
%    \end{macrocode}
%
%    \begin{macrocode}
\RequirePackage{inputenc}
\RequirePackage{kvsetkeys}[2006/10/19]
\RequirePackage{stringenc}[2007/06/16]
\RequirePackage{kvoptions}
%    \end{macrocode}
%    \begin{macro}{\SelectInputEncodingList}
%    \begin{macrocode}
\newcommand*{\SelectInputEncodingList}{%
  \let\SIE@EncodingList\@empty
  \kvsetkeys{SelInputEnc}%
}
%    \end{macrocode}
%    \end{macro}
%    \begin{macro}{\SelectInputMappings}
%    \begin{macrocode}
\newcommand*{\SelectInputMappings}[1]{%
  \SIE@LoadNameDefs
  \let\SIE@StringUnicode\@empty
  \let\SIE@StringDest\@empty
  \kvsetkeys{SelInputMap}{#1}%
  \ifx\\SIE@StringUnicode\SIE@StringDest\\%
    \PackageError{selinput}{%
      No mappings specified%
    }\@ehc
  \else
    \EdefUnescapeHex\SIE@StringUnicode\SIE@StringUnicode
    \let\SIE@Encoding\@empty
    \@for\SIE@EncodingTest:=\SIE@EncodingList\do{%
      \ifx\SIE@Encoding\@empty
        \StringEncodingConvertTest\SIE@temp\SIE@StringUnicode
                                  {utf16be}\SIE@EncodingTest{%
          \ifx\SIE@temp\SIE@StringDest
            \let\SIE@Encoding\SIE@EncodingTest
          \fi
        }{}%
      \fi
    }%
    \ifx\SIE@Encoding\@empty
      \StringEncodingConvertTest\SIE@temp\SIE@StringDest
                                {ascii}{utf16be}{%
        \def\SIE@Encoding{ascii}%
        \SIE@Info{selinput}{%
          Matching encoding not found, but input characters%
          \MessageBreak
          are 7-bit (possibly editor replacements).%
          \MessageBreak
          Hence using ascii encoding%
        }%
      }{}%
    \fi
    \ifx\SIE@Encoding\@empty
      \PackageError{selinput}{%
        Cannot find a matching encoding%
      }\@ehd
    \else
      \ifx\SIE@Encoding\SIE@EncodingUTFviii
        \SIE@LoadUnicodePackage
        \ifx\SIE@UseUTFviii\@empty
        \else
          \let\SIE@Encoding\SIE@UseUTFviii
        \fi
      \fi
      \begingroup\expandafter\expandafter\expandafter\endgroup
      \expandafter\ifx\csname InputEncoding\endcsname\relax
        \inputencoding\SIE@Encoding
      \else
        \InputEncoding\SIE@Encoding
      \fi
      \SIE@Info{selinput}{Encoding `\SIE@Encoding' selected}%
    \fi
  \fi
}
%    \end{macrocode}
%    \end{macro}
%    \begin{macro}{\SIE@LoadNameDefs}
%    \begin{macrocode}
\def\SIE@LoadNameDefs{%
  \begingroup
    \endlinechar=\m@ne
    \catcode92=0 % backslash
    \catcode123=1 % left curly brace/beginning of group
    \catcode125=2 % right curly brace/end of group
    \catcode37=14 % percent/comment character
    \@makeother\[%
    \@makeother\]%
    \@makeother\.%
    \@makeother\(%
    \@makeother\)%
    \@makeother\/%
    \@makeother\-%
    \let\InputenxName\SelectInputDefineMapping
    \InputIfFileExists{ix-name.def}{}{%
      \PackageError{selinput}{%
        Missing `ix-name.def' (part of package `inputenx')%
      }\@ehd
    }%
    \global\let\SIE@LoadNameDefs\relax
  \endgroup
}
%    \end{macrocode}
%    \end{macro}
%    \begin{macro}{\SelectInputDefineMapping}
%    \begin{macrocode}
\newcommand*{\SelectInputDefineMapping}[1]{%
  \expandafter\gdef\csname SIE@@#1\endcsname
}
%    \end{macrocode}
%    \end{macro}
%    \begin{macrocode}
\kv@set@family@handler{SelInputMap}{%
  \@onelevel@sanitize\kv@key
  \ifx\kv@value\relax
    \PackageError{selinput}{%
      Missing input character for `\kv@key'%
    }\@ehc
  \else
    \@onelevel@sanitize\kv@value
    \ifx\kv@value\@empty
      \PackageError{selinput}{%
        Input character got lost?\MessageBreak
        Missing input character for `\kv@key'%
      }\@ehc
    \else
      \@ifundefined{SIE@@\kv@key}{%
        \PackageWarning{selinput}{%
          Missing definition for `\kv@key'%
        }%
      }{%
        \edef\SIE@StringDest{%
          \SIE@StringDest
          \kv@value
        }%
        \edef\SIE@StringUnicode{%
          \SIE@StringUnicode
          \csname SIE@@\kv@key\endcsname
        }%
      }%
    \fi
  \fi
}
%    \end{macrocode}
%    \begin{macrocode}
\kv@set@family@handler{SelInputEnc}{%
  \@onelevel@sanitize\kv@key
  \ifx\kv@value\relax
    \ifx\SIE@EncodingList\@empty
      \let\SIE@EncodingList\kv@key
    \else
      \edef\SIE@EncodingList{\SIE@EncodingList,\kv@key}%
    \fi
  \else
    \@onelevel@sanitize\kv@value
    \PackageError{selinput}{%
      Illegal key value pair (\kv@key=\kv@value)\MessagBreak
      in encoding list%
    }\@ehc
  \fi
}
%    \end{macrocode}
%
%    \begin{macro}{\SIE@LoadUnicodePackage}
%    \begin{macrocode}
\def\SIE@LoadUnicodePackage{%
  \@ifpackageloaded\SIE@UnicodePackage{}{%
    \RequirePackage\SIE@UnicodePackage\relax
  }%
  \SIE@PatchUCS
  \global\let\SIE@LoadUnicodePackage\relax
}
\let\SIE@show\show
\def\SIE@PatchUCS{%
  \AtBeginDocument{%
    \expandafter\ifx\csname ver@ucsencs.def\endcsname\relax
    \else
      \let\show\SIE@show
    \fi
  }%
}
\SIE@PatchUCS
%    \end{macrocode}
%    \end{macro}
%    \begin{macrocode}
\AtBeginDocument{%
  \let\SIE@LoadUnicodePackage\relax
}
%    \end{macrocode}
%    \begin{macro}{\SIE@EncodingUTFviii}
%    \begin{macrocode}
\def\SIE@EncodingUTFviii{utf8}
\@onelevel@sanitize\SIE@EncodingUTFviii
%    \end{macrocode}
%    \end{macro}
%    \begin{macro}{\SIE@EncodingUTFviiix}
%    \begin{macrocode}
\def\SIE@EncodingUTFviiix{utf8x}
\@onelevel@sanitize\SIE@EncodingUTFviiix
%    \end{macrocode}
%    \end{macro}
%
%    \begin{macrocode}
\let\SIE@UnicodePackage\@empty
\let\SIE@UseUTFviii\@empty
\let\SIE@Info\PackageInfo
%    \end{macrocode}
%    \begin{macrocode}
\SetupKeyvalOptions{%
  family=SelInput,%
  prefix=SelInput@%
}
\define@key{SelInput}{utf8}{%
  \def\SIE@UseUTFviii{#1}%
  \@onelevel@sanitize\SIE@UseUTFviii
}
\DeclareBoolOption{ucs}
\DeclareVoidOption{warning}{%
  \let\SIE@Info\PackageWarning
}
\ProcessKeyvalOptions{SelInput}
\ifSelInput@ucs
  \def\SIE@UnicodePackage{ucs}%
  \ifx\SIE@UseUTFviii\@empty
    \let\SIE@UseUTFviii\SIE@EncodingUTFviiix
  \fi
\else
  \ifx\SIE@UseUTFviii\@empty
    \@ifpackageloaded{ucs}{%
      \let\SIE@UseUTFviii\SIE@EncodingUTFviiix
    }{%
      \let\SIE@UseUTFviii\SIE@EncodingUTFviii
    }%
  \fi
\fi
%    \end{macrocode}
%
%    \begin{macro}{\SIE@EncodingList}
%    \begin{macrocode}
\edef\SIE@EncodingList{%
  utf8,%
  x-iso-8859-1,%
  x-iso-8859-15,%
  x-cp1252,% ansinew
  x-mac-roman,%
  x-iso-8859-2,%
  x-iso-8859-3,%
  x-iso-8859-4,%
  x-iso-8859-5,%
  x-iso-8859-6,%
  x-iso-8859-7,%
  x-iso-8859-8,%
  x-iso-8859-9,%
  x-iso-8859-10,%
  x-iso-8859-11,%
  x-iso-8859-13,%
  x-iso-8859-14,%
  x-iso-8859-15,%
  x-mac-centeuro,%
  x-mac-cyrillic,%
  x-koi8-r,%
  x-cp1250,%
  x-cp1251,%
  x-cp1257,%
  x-cp437,%
  x-cp850,%
  x-cp852,%
  x-cp855,%
  x-cp858,%
  x-cp865,%
  x-cp866,%
  x-nextstep,%
  x-dec-mcs%
}%
\@onelevel@sanitize\SIE@EncodingList
%    \end{macrocode}
%    \end{macro}
%
%    \begin{macrocode}
%</package>
%    \end{macrocode}
%
% \section{Test}
%
%    \begin{macrocode}
%<*test>
\NeedsTeXFormat{LaTeX2e}
\documentclass{minimal}
\usepackage{textcomp}
\usepackage{qstest}
%    \end{macrocode}
%    \begin{macrocode}
%<*test1|test2|test3>
\makeatletter
\let\BeginDocumentText\@empty
\def\TestEncoding#1#2{%
  \SelectInputMappings{#2}%
  \Expect*{\SIE@Encoding}{#1}%
  \Expect*{\inputencodingname}{#1}%
  \g@addto@macro\BeginDocumentText{%
    \SelectInputMappings{#2}%
    \Expect*{\SIE@Encoding}{#1}%
    \textbf{\SIE@Encoding:} %
    \kvsetkeys{test}{#2}\par
  }%
}
\def\TestKey#1#2{%
  \define@key{test}{#1}{%
    \sbox0{##1}%
    \sbox2{#2}%
    \Expect*{wd:\the\wd0, ht:\the\ht0, dp:\the\dp0}%
           *{wd:\the\wd2, ht:\the\ht2, dp:\the\dp2}%
    [#1=##1] % hash-ok
  }%
}
\RequirePackage{keyval}
\TestKey{adieresis}{\"a}
\TestKey{germandbls}{\ss}
\TestKey{Euro}{\texteuro}
\makeatother
\usepackage[
  warning,%
%<test2>  utf8=utf-8,
%<test3>  ucs,
]{selinput}
%<test1|test3>\inputencoding{ascii}
%<test2>\inputencoding{utf-8}
%<test3>\usepackage{ucs}
\begin{qstest}{preamble}{}
  \TestEncoding{x-iso-8859-15}{%
    adieresis=^^e4,%
    germandbls=^^df,%
    Euro=^^a4,%
  }%
  \TestEncoding{x-cp1252}{%
    adieresis=^^e4,%
    germandbls=^^df,%
    Euro=^^80,%
  }%
%<test1>  \TestEncoding{utf8}{%
%<test2>  \TestEncoding{utf-8}{%
%<test3>  \TestEncoding{utf8x}{%
    adieresis=^^c3^^a4,%
    germandbls=^^c3^^9f,%
%<!test2>    Euro=^^e2^^82^^ac,
  }%
\end{qstest}
%<test3>\let\ifUnicodeOptiongraphics\iffalse
\begin{document}
\begin{qstest}{document}{}
%<test3>\makeatletter
  \BeginDocumentText
\end{qstest}
%</test1|test2|test3>
%    \end{macrocode}
%
%    \begin{macrocode}
%<*test4>
\usepackage[warning,ucs]{selinput}
\SelectInputMappings{%
    adieresis=^^c3^^a4,%
    germandbls=^^c3^^9f,%
    Euro=^^e2^^82^^ac,%
}
\begin{qstest}{encoding}{}
  \Expect*{\inputencodingname}{utf8x}%
\end{qstest}
\begin{document}
  adieresis=^^c3^^a4, %
  germandbls=^^c3^^9f, %
  Euro=^^e2^^82^^ac%
%</test4>
%    \end{macrocode}
%
%    \begin{macrocode}
%<*test5>
\usepackage[warning,ucs]{selinput}
\SelectInputMappings{%
    adieresis={\"a},%
    germandbls={{\ss}},%
    Euro=\texteuro{},%
}
\begin{qstest}{encoding}{}
  \Expect*{\inputencodingname}{ascii}%
\end{qstest}
\begin{document}
  adieresis={\"a}, %
  germandbls={{\ss}}, %
  Euro=\texteuro{}%
%</test5>
%    \end{macrocode}
%
%    \begin{macrocode}
\end{document}
%</test>
%    \end{macrocode}
%
% \section{Installation}
%
% \subsection{Download}
%
% \paragraph{Package.} This package is available on
% CTAN\footnote{\url{ftp://ftp.ctan.org/tex-archive/}}:
% \begin{description}
% \item[\CTAN{macros/latex/contrib/oberdiek/selinput.dtx}] The source file.
% \item[\CTAN{macros/latex/contrib/oberdiek/selinput.pdf}] Documentation.
% \end{description}
%
%
% \paragraph{Bundle.} All the packages of the bundle `oberdiek'
% are also available in a TDS compliant ZIP archive. There
% the packages are already unpacked and the documentation files
% are generated. The files and directories obey the TDS standard.
% \begin{description}
% \item[\CTAN{install/macros/latex/contrib/oberdiek.tds.zip}]
% \end{description}
% \emph{TDS} refers to the standard ``A Directory Structure
% for \TeX\ Files'' (\CTAN{tds/tds.pdf}). Directories
% with \xfile{texmf} in their name are usually organized this way.
%
% \subsection{Bundle installation}
%
% \paragraph{Unpacking.} Unpack the \xfile{oberdiek.tds.zip} in the
% TDS tree (also known as \xfile{texmf} tree) of your choice.
% Example (linux):
% \begin{quote}
%   |unzip oberdiek.tds.zip -d ~/texmf|
% \end{quote}
%
% \paragraph{Script installation.}
% Check the directory \xfile{TDS:scripts/oberdiek/} for
% scripts that need further installation steps.
% Package \xpackage{attachfile2} comes with the Perl script
% \xfile{pdfatfi.pl} that should be installed in such a way
% that it can be called as \texttt{pdfatfi}.
% Example (linux):
% \begin{quote}
%   |chmod +x scripts/oberdiek/pdfatfi.pl|\\
%   |cp scripts/oberdiek/pdfatfi.pl /usr/local/bin/|
% \end{quote}
%
% \subsection{Package installation}
%
% \paragraph{Unpacking.} The \xfile{.dtx} file is a self-extracting
% \docstrip\ archive. The files are extracted by running the
% \xfile{.dtx} through \plainTeX:
% \begin{quote}
%   \verb|tex selinput.dtx|
% \end{quote}
%
% \paragraph{TDS.} Now the different files must be moved into
% the different directories in your installation TDS tree
% (also known as \xfile{texmf} tree):
% \begin{quote}
% \def\t{^^A
% \begin{tabular}{@{}>{\ttfamily}l@{ $\rightarrow$ }>{\ttfamily}l@{}}
%   selinput.sty & tex/latex/oberdiek/selinput.sty\\
%   selinput.pdf & doc/latex/oberdiek/selinput.pdf\\
%   test/selinput-test1.tex & doc/latex/oberdiek/test/selinput-test1.tex\\
%   test/selinput-test2.tex & doc/latex/oberdiek/test/selinput-test2.tex\\
%   test/selinput-test3.tex & doc/latex/oberdiek/test/selinput-test3.tex\\
%   test/selinput-test4.tex & doc/latex/oberdiek/test/selinput-test4.tex\\
%   test/selinput-test5.tex & doc/latex/oberdiek/test/selinput-test5.tex\\
%   selinput.dtx & source/latex/oberdiek/selinput.dtx\\
% \end{tabular}^^A
% }^^A
% \sbox0{\t}^^A
% \ifdim\wd0>\linewidth
%   \begingroup
%     \advance\linewidth by\leftmargin
%     \advance\linewidth by\rightmargin
%   \edef\x{\endgroup
%     \def\noexpand\lw{\the\linewidth}^^A
%   }\x
%   \def\lwbox{^^A
%     \leavevmode
%     \hbox to \linewidth{^^A
%       \kern-\leftmargin\relax
%       \hss
%       \usebox0
%       \hss
%       \kern-\rightmargin\relax
%     }^^A
%   }^^A
%   \ifdim\wd0>\lw
%     \sbox0{\small\t}^^A
%     \ifdim\wd0>\linewidth
%       \ifdim\wd0>\lw
%         \sbox0{\footnotesize\t}^^A
%         \ifdim\wd0>\linewidth
%           \ifdim\wd0>\lw
%             \sbox0{\scriptsize\t}^^A
%             \ifdim\wd0>\linewidth
%               \ifdim\wd0>\lw
%                 \sbox0{\tiny\t}^^A
%                 \ifdim\wd0>\linewidth
%                   \lwbox
%                 \else
%                   \usebox0
%                 \fi
%               \else
%                 \lwbox
%               \fi
%             \else
%               \usebox0
%             \fi
%           \else
%             \lwbox
%           \fi
%         \else
%           \usebox0
%         \fi
%       \else
%         \lwbox
%       \fi
%     \else
%       \usebox0
%     \fi
%   \else
%     \lwbox
%   \fi
% \else
%   \usebox0
% \fi
% \end{quote}
% If you have a \xfile{docstrip.cfg} that configures and enables \docstrip's
% TDS installing feature, then some files can already be in the right
% place, see the documentation of \docstrip.
%
% \subsection{Refresh file name databases}
%
% If your \TeX~distribution
% (\teTeX, \mikTeX, \dots) relies on file name databases, you must refresh
% these. For example, \teTeX\ users run \verb|texhash| or
% \verb|mktexlsr|.
%
% \subsection{Some details for the interested}
%
% \paragraph{Attached source.}
%
% The PDF documentation on CTAN also includes the
% \xfile{.dtx} source file. It can be extracted by
% AcrobatReader 6 or higher. Another option is \textsf{pdftk},
% e.g. unpack the file into the current directory:
% \begin{quote}
%   \verb|pdftk selinput.pdf unpack_files output .|
% \end{quote}
%
% \paragraph{Unpacking with \LaTeX.}
% The \xfile{.dtx} chooses its action depending on the format:
% \begin{description}
% \item[\plainTeX:] Run \docstrip\ and extract the files.
% \item[\LaTeX:] Generate the documentation.
% \end{description}
% If you insist on using \LaTeX\ for \docstrip\ (really,
% \docstrip\ does not need \LaTeX), then inform the autodetect routine
% about your intention:
% \begin{quote}
%   \verb|latex \let\install=y\input{selinput.dtx}|
% \end{quote}
% Do not forget to quote the argument according to the demands
% of your shell.
%
% \paragraph{Generating the documentation.}
% You can use both the \xfile{.dtx} or the \xfile{.drv} to generate
% the documentation. The process can be configured by the
% configuration file \xfile{ltxdoc.cfg}. For instance, put this
% line into this file, if you want to have A4 as paper format:
% \begin{quote}
%   \verb|\PassOptionsToClass{a4paper}{article}|
% \end{quote}
% An example follows how to generate the
% documentation with pdf\LaTeX:
% \begin{quote}
%\begin{verbatim}
%pdflatex selinput.dtx
%makeindex -s gind.ist selinput.idx
%pdflatex selinput.dtx
%makeindex -s gind.ist selinput.idx
%pdflatex selinput.dtx
%\end{verbatim}
% \end{quote}
%
% \section{Catalogue}
%
% The following XML file can be used as source for the
% \href{http://mirror.ctan.org/help/Catalogue/catalogue.html}{\TeX\ Catalogue}.
% The elements \texttt{caption} and \texttt{description} are imported
% from the original XML file from the Catalogue.
% The name of the XML file in the Catalogue is \xfile{selinput.xml}.
%    \begin{macrocode}
%<*catalogue>
<?xml version='1.0' encoding='us-ascii'?>
<!DOCTYPE entry SYSTEM 'catalogue.dtd'>
<entry datestamp='$Date$' modifier='$Author$' id='selinput'>
  <name>selinput</name>
  <caption>Semi-automatic detection of input encoding.</caption>
  <authorref id='auth:oberdiek'/>
  <copyright owner='Heiko Oberdiek' year='2007'/>
  <license type='lppl1.3'/>
  <version number='1.2'/>
  <description>
    This package selects the input encoding by specifying pairs
    of input characters and their glyph names.
    <p/>
    The package is part of the <xref refid='oberdiek'>oberdiek</xref>
    bundle.
  </description>
  <documentation details='Package documentation'
      href='ctan:/macros/latex/contrib/oberdiek/selinput.pdf'/>
  <ctan file='true' path='/macros/latex/contrib/oberdiek/selinput.dtx'/>
  <miktex location='oberdiek'/>
  <texlive location='oberdiek'/>
  <install path='/macros/latex/contrib/oberdiek/oberdiek.tds.zip'/>
</entry>
%</catalogue>
%    \end{macrocode}
%
% \begin{thebibliography}{9}
% \bibitem{inputenx}
%   Heiko Oberdiek: \textit{The \xpackage{inputenx} package};
%   2007-04-11 v1.1;
%   \CTAN{macros/latex/contrib/oberdiek/inputenx.pdf}.
%
% \bibitem{adobe:glyphlist}
%   Adobe: \textit{Adobe Glyph List};
%   2002-09-20 v2.0;
%   \url{http://partners.adobe.com/public/developer/en/opentype/glyphlist.txt}.
%
% \bibitem{adobe:aglfn}
%   Adobe: \textit{Adobe Glyph List For New Fonts};
%   2005-11-18 v1.5;
%   \url{http://partners.adobe.com/public/developer/en/opentype/aglfn13.txt}.
%
% \bibitem{stringenc}
%   Heiko Oberdiek: \textit{The \xpackage{stringenc} package};
%   2007-06-16 v1.1;
%   \CTAN{macros/latex/contrib/oberdiek/stringenc.pdf}.
%
% \end{thebibliography}
%
% \begin{History}
%   \begin{Version}{2007/06/16 v1.0}
%   \item
%     First version.
%   \end{Version}
%   \begin{Version}{2007/06/20 v1.1}
%   \item
%     Requested date for package \xpackage{stringenc} fixed.
%   \end{Version}
%   \begin{Version}{2007/09/09 v1.2}
%   \item
%     Line end fixed.
%   \end{Version}
% \end{History}
%
% \PrintIndex
%
% \Finale
\endinput

%        (quote the arguments according to the demands of your shell)
%
% Documentation:
%    (a) If selinput.drv is present:
%           latex selinput.drv
%    (b) Without selinput.drv:
%           latex selinput.dtx; ...
%    The class ltxdoc loads the configuration file ltxdoc.cfg
%    if available. Here you can specify further options, e.g.
%    use A4 as paper format:
%       \PassOptionsToClass{a4paper}{article}
%
%    Programm calls to get the documentation (example):
%       pdflatex selinput.dtx
%       makeindex -s gind.ist selinput.idx
%       pdflatex selinput.dtx
%       makeindex -s gind.ist selinput.idx
%       pdflatex selinput.dtx
%
% Installation:
%    TDS:tex/latex/oberdiek/selinput.sty
%    TDS:doc/latex/oberdiek/selinput.pdf
%    TDS:doc/latex/oberdiek/test/selinput-test1.tex
%    TDS:doc/latex/oberdiek/test/selinput-test2.tex
%    TDS:doc/latex/oberdiek/test/selinput-test3.tex
%    TDS:doc/latex/oberdiek/test/selinput-test4.tex
%    TDS:doc/latex/oberdiek/test/selinput-test5.tex
%    TDS:source/latex/oberdiek/selinput.dtx
%
%<*ignore>
\begingroup
  \catcode123=1 %
  \catcode125=2 %
  \def\x{LaTeX2e}%
\expandafter\endgroup
\ifcase 0\ifx\install y1\fi\expandafter
         \ifx\csname processbatchFile\endcsname\relax\else1\fi
         \ifx\fmtname\x\else 1\fi\relax
\else\csname fi\endcsname
%</ignore>
%<*install>
\input docstrip.tex
\Msg{************************************************************************}
\Msg{* Installation}
\Msg{* Package: selinput 2007/09/09 v1.2 Semi-automatic input encoding detection (HO)}
\Msg{************************************************************************}

\keepsilent
\askforoverwritefalse

\let\MetaPrefix\relax
\preamble

This is a generated file.

Project: selinput
Version: 2007/09/09 v1.2

Copyright (C) 2007 by
   Heiko Oberdiek <heiko.oberdiek at googlemail.com>

This work may be distributed and/or modified under the
conditions of the LaTeX Project Public License, either
version 1.3c of this license or (at your option) any later
version. This version of this license is in
   http://www.latex-project.org/lppl/lppl-1-3c.txt
and the latest version of this license is in
   http://www.latex-project.org/lppl.txt
and version 1.3 or later is part of all distributions of
LaTeX version 2005/12/01 or later.

This work has the LPPL maintenance status "maintained".

This Current Maintainer of this work is Heiko Oberdiek.

This work consists of the main source file selinput.dtx
and the derived files
   selinput.sty, selinput.pdf, selinput.ins, selinput.drv,
   selinput-test1.tex, selinput-test2.tex, selinput-test3.tex,
   selinput-test4.tex, selinput-test5.tex.

\endpreamble
\let\MetaPrefix\DoubleperCent

\generate{%
  \file{selinput.ins}{\from{selinput.dtx}{install}}%
  \file{selinput.drv}{\from{selinput.dtx}{driver}}%
  \usedir{tex/latex/oberdiek}%
  \file{selinput.sty}{\from{selinput.dtx}{package}}%
  \usedir{doc/latex/oberdiek/test}%
  \file{selinput-test1.tex}{\from{selinput.dtx}{test,test1}}%
  \file{selinput-test2.tex}{\from{selinput.dtx}{test,test2}}%
  \file{selinput-test3.tex}{\from{selinput.dtx}{test,test3}}%
  \file{selinput-test4.tex}{\from{selinput.dtx}{test,test4}}%
  \file{selinput-test5.tex}{\from{selinput.dtx}{test,test5}}%
  \nopreamble
  \nopostamble
  \usedir{source/latex/oberdiek/catalogue}%
  \file{selinput.xml}{\from{selinput.dtx}{catalogue}}%
}

\catcode32=13\relax% active space
\let =\space%
\Msg{************************************************************************}
\Msg{*}
\Msg{* To finish the installation you have to move the following}
\Msg{* file into a directory searched by TeX:}
\Msg{*}
\Msg{*     selinput.sty}
\Msg{*}
\Msg{* To produce the documentation run the file `selinput.drv'}
\Msg{* through LaTeX.}
\Msg{*}
\Msg{* Happy TeXing!}
\Msg{*}
\Msg{************************************************************************}

\endbatchfile
%</install>
%<*ignore>
\fi
%</ignore>
%<*driver>
\NeedsTeXFormat{LaTeX2e}
\ProvidesFile{selinput.drv}%
  [2007/09/09 v1.2 Semi-automatic input encoding detection (HO)]%
\documentclass{ltxdoc}
\usepackage[T1]{fontenc}
\usepackage{textcomp}
\usepackage{lmodern}
\usepackage{holtxdoc}[2011/11/22]
\usepackage{color}
\begin{document}
  \DocInput{selinput.dtx}%
\end{document}
%</driver>
% \fi
%
% \CheckSum{389}
%
% \CharacterTable
%  {Upper-case    \A\B\C\D\E\F\G\H\I\J\K\L\M\N\O\P\Q\R\S\T\U\V\W\X\Y\Z
%   Lower-case    \a\b\c\d\e\f\g\h\i\j\k\l\m\n\o\p\q\r\s\t\u\v\w\x\y\z
%   Digits        \0\1\2\3\4\5\6\7\8\9
%   Exclamation   \!     Double quote  \"     Hash (number) \#
%   Dollar        \$     Percent       \%     Ampersand     \&
%   Acute accent  \'     Left paren    \(     Right paren   \)
%   Asterisk      \*     Plus          \+     Comma         \,
%   Minus         \-     Point         \.     Solidus       \/
%   Colon         \:     Semicolon     \;     Less than     \<
%   Equals        \=     Greater than  \>     Question mark \?
%   Commercial at \@     Left bracket  \[     Backslash     \\
%   Right bracket \]     Circumflex    \^     Underscore    \_
%   Grave accent  \`     Left brace    \{     Vertical bar  \|
%   Right brace   \}     Tilde         \~}
%
% \GetFileInfo{selinput.drv}
%
% \title{The \xpackage{selinput} package}
% \date{2007/09/09 v1.2}
% \author{Heiko Oberdiek\\\xemail{heiko.oberdiek at googlemail.com}}
%
% \maketitle
%
% \begin{abstract}
% This package selects the input encoding by specifying between
% input characters and their glyph names.
% \end{abstract}
%
% \tableofcontents
%
% \newcommand*{\EM}{\textcolor{blue}}
% \newcommand*{\ExampleText}{^^A
%   Umlauts:\ \EM{\"A\"O\"U\"a\"o\"u\ss}^^A
% }
%
% \section{Documentation}
%
% \subsection{Introduction}
%
% \LaTeX\ supports the direct use of 8-bit characters by means
% of package \xpackage{inputenc}. However you must know
% and specify the encoding, e.g.:
% \begin{quote}
%   \ttfamily
%   |\documentclass{article}|\\
%   |\usepackage[|\EM{latin1}|]{inputenc}|\\
%   |% or \usepackage[|\EM{utf8}|]{inputenc}|\\
%   |% or \usepackage[|\EM{??}|]{inputenc}|\\
%   |\begin{document}|\\
%   |  |\ExampleText\\
%   |\end{document}|
% \end{quote}
%
% If the document is transferred in an environment that
% uses a different encoding, then there are programs that
% convert the input characters. Examples for conversion
% of file \xfile{test.tex}
% from encoding latin1 (ISO-8859-1) to UTF-8:
% \begin{quote}
%   \ttfamily
%   |recode ISO-8859-1..UTF-8 test.tex|\\
%   |recode latin1..utf8 test.tex|\\
%   |iconv --from-code ISO-8859-1|\\
%   \hphantom{iconv}| --to-code UTF-8|\\
%   \hphantom{iconv}| --output testnew.tex|\\
%   \hphantom{iconv}| test.tex|\\
%   |iconv -f latin1 -t utf8 -o testnew.tex test.tex|
% \end{quote}
% However, the encoding name for package \xpackage{inputenc}
% must be changed:
% \begin{quote}
%    |\usepackage[latin1]{inputenc}| $\rightarrow$
%    |\usepackage[utf8]{inputenc}|\kern-4pt\relax
% \end{quote}
% Of course, unless you are using some clever editor
% that knows package \xpackage{inputenc}, recodes
% the file and adjusts the option at the same time.
% But most editors can perhaps recode the file, but
% they let the option untouched.
%
% Therefore package \xpackage{selinput} chooses another way for
% specifying the input encoding. The encoding name is not needed
% at all. Some 8-bit characters are identified by their glyph
% name and the package chooses an appropriate encoding, example:
% \begin{quote}
%   \ttfamily
%   |\documentclass{article}|\\
%   |\usepackage{selinput}|\\
%   |\SelectInputMappings{|\\
%   |  adieresis={|\EM{\"a}|}|,\\
%   |  germandbls={|\EM{\ss}|}|,\\
%   |  Euro={|\EM{\texteuro}|}|,\\
%   |}|\\
%   |\begin{document}|\\
%   |  |\ExampleText\\
%   |\end{document}|
% \end{quote}
%
% \subsection{User interface}
%
% \begin{declcs}{SelectInputEncodingList} \M{encoding list}
% \end{declcs}
% \cs{SelectInputEncodingList} expects a comma separated list of
% encoding names. Example:
% \begin{quote}
%   |\SelectInputEncodingList{utf8,ansinew,mac-roman}|
% \end{quote}
% The encodings of package \xpackage{inputenx} are used as default.
%
% \begin{declcs}{SelectInputMappings} \M{mapping pairs}
% \end{declcs}
% A mapping pair consists of a glyph name and its input
% character:
% \begin{quote}
%   |\SelectInputMappings{|\\
%   |  adieresis={|\EM{\"a}|}|,\\
%   |  germandbls={|\EM{\ss}|}|,\\
%   |  Euro={|\EM{\texteuro}|}|,\\
%   |}|
% \end{quote}
% The supported glyph names can be found in file \xfile{ix-name.def}
% of project \xpackage{inputenx} \cite{inputenx}. The names are
% basically taken from Adobe's glyphlists \cite{adobe:glyphlist,adobe:aglfn}.
% As many pairs are needed as necessary to identify the encoding.
% Example with insufficient pairs:
% \begin{quote}
%   \ttfamily
%   |\SelectInputEncodingSet{latin1,latin9}|\\
%   |\SelectInputMappings{|\\
%   |  adieresis={|\EM{\"a}|}|,\\
%   |  germandbls={|\EM{\ss}|}|,\\
%   |}|\\
%   \ExampleText| and Euro: |\EM{\textcurrency} (wrong)
% \end{quote}
% The first encoding \xoption{latin1} passes the constraints given
% by the mapping pairs. However the Euro symbol is not part of
% the encoding. Thus a mapping pair with the Euro symbol
% solves the problem. In fact the symbol alone already succeeds in selecting
% between \xoption{latin1} and \xoption{latin9}:
% \begin{quote}
%   \ttfamily
%   |\SelectInputEncodingSet{latin1,latin9}|\\
%   |\SelectInputMappings{|\\
%   |  Euro={|\EM{\texteuro}|},|\\
%   |}|\\
%   \ExampleText| and Euro: |\EM{\texteuro}
% \end{quote}
%
% \subsection{Options}
%
% \begin{description}
% \item[\xoption{warning}:]
%   The selected encoding is written
%   by \cs{PackageInfo} into the \xfile{.log} file only.
%   Option \xoption{warning} changes it to \cs{PackageWarning}.
%   Then the selected encoding is shown on the terminal as well.
% \item[\xoption{ucs}:]
%   The encoding file \xfile{utf8x} of package \cs{ucs} requires
%   that the package itself is loaded before.
%   If the package is not loaded, then the option \xoption{ucs}
%   will load package \xpackage{ucs} if the detected encoding is
%   UTF-8 (limited to the preamble, packages cannot be loaded later).
% \item[\xoption{utf8=\dots}:]
%   The option allows to specify other encoding files
%   for UTF-8 than \LaTeX's \xfile{utf8.def}. For example,
%   |utf8=utf-8| will load \xfile{utf-8.def} instead.
% \end{description}
%
% \subsection{Encodings}
%
% Package \xpackage{stringenc} \cite{stringenc}
% is used for testing the encoding. Thus the encoding
% name must be known by this package. Then the found
% encoding is loaded by \cs{inputencoding} by package
% \xpackage{inputenc} or \cs{InputEncoding} if package
% \xpackage{inputenx} is loaded.
%
% The supported encodings are present in the encoding list,
% thus usually the encoding names do not matter.
% If the list is set by \cs{SelectInputEncodingList},
% then you can use the names that work for package
% \xpackage{inputenc} and are known by package \xpackage{stringenc},
% for example: \xoption{latin1}, \xoption{x-iso-8859-1}. Encoding
% file names of package \xpackage{inputenx} are prefixed with \xfile{x-}.
% The prefix can be dropped, if package \xpackage{inputenx} is loaded.
%
% \StopEventually{
% }
%
% \section{Implementation}
%
%    \begin{macrocode}
%<*package>
\NeedsTeXFormat{LaTeX2e}
\ProvidesPackage{selinput}
  [2007/09/09 v1.2 Semi-automatic input encoding detection (HO)]%
%    \end{macrocode}
%
%    \begin{macrocode}
\RequirePackage{inputenc}
\RequirePackage{kvsetkeys}[2006/10/19]
\RequirePackage{stringenc}[2007/06/16]
\RequirePackage{kvoptions}
%    \end{macrocode}
%    \begin{macro}{\SelectInputEncodingList}
%    \begin{macrocode}
\newcommand*{\SelectInputEncodingList}{%
  \let\SIE@EncodingList\@empty
  \kvsetkeys{SelInputEnc}%
}
%    \end{macrocode}
%    \end{macro}
%    \begin{macro}{\SelectInputMappings}
%    \begin{macrocode}
\newcommand*{\SelectInputMappings}[1]{%
  \SIE@LoadNameDefs
  \let\SIE@StringUnicode\@empty
  \let\SIE@StringDest\@empty
  \kvsetkeys{SelInputMap}{#1}%
  \ifx\\SIE@StringUnicode\SIE@StringDest\\%
    \PackageError{selinput}{%
      No mappings specified%
    }\@ehc
  \else
    \EdefUnescapeHex\SIE@StringUnicode\SIE@StringUnicode
    \let\SIE@Encoding\@empty
    \@for\SIE@EncodingTest:=\SIE@EncodingList\do{%
      \ifx\SIE@Encoding\@empty
        \StringEncodingConvertTest\SIE@temp\SIE@StringUnicode
                                  {utf16be}\SIE@EncodingTest{%
          \ifx\SIE@temp\SIE@StringDest
            \let\SIE@Encoding\SIE@EncodingTest
          \fi
        }{}%
      \fi
    }%
    \ifx\SIE@Encoding\@empty
      \StringEncodingConvertTest\SIE@temp\SIE@StringDest
                                {ascii}{utf16be}{%
        \def\SIE@Encoding{ascii}%
        \SIE@Info{selinput}{%
          Matching encoding not found, but input characters%
          \MessageBreak
          are 7-bit (possibly editor replacements).%
          \MessageBreak
          Hence using ascii encoding%
        }%
      }{}%
    \fi
    \ifx\SIE@Encoding\@empty
      \PackageError{selinput}{%
        Cannot find a matching encoding%
      }\@ehd
    \else
      \ifx\SIE@Encoding\SIE@EncodingUTFviii
        \SIE@LoadUnicodePackage
        \ifx\SIE@UseUTFviii\@empty
        \else
          \let\SIE@Encoding\SIE@UseUTFviii
        \fi
      \fi
      \begingroup\expandafter\expandafter\expandafter\endgroup
      \expandafter\ifx\csname InputEncoding\endcsname\relax
        \inputencoding\SIE@Encoding
      \else
        \InputEncoding\SIE@Encoding
      \fi
      \SIE@Info{selinput}{Encoding `\SIE@Encoding' selected}%
    \fi
  \fi
}
%    \end{macrocode}
%    \end{macro}
%    \begin{macro}{\SIE@LoadNameDefs}
%    \begin{macrocode}
\def\SIE@LoadNameDefs{%
  \begingroup
    \endlinechar=\m@ne
    \catcode92=0 % backslash
    \catcode123=1 % left curly brace/beginning of group
    \catcode125=2 % right curly brace/end of group
    \catcode37=14 % percent/comment character
    \@makeother\[%
    \@makeother\]%
    \@makeother\.%
    \@makeother\(%
    \@makeother\)%
    \@makeother\/%
    \@makeother\-%
    \let\InputenxName\SelectInputDefineMapping
    \InputIfFileExists{ix-name.def}{}{%
      \PackageError{selinput}{%
        Missing `ix-name.def' (part of package `inputenx')%
      }\@ehd
    }%
    \global\let\SIE@LoadNameDefs\relax
  \endgroup
}
%    \end{macrocode}
%    \end{macro}
%    \begin{macro}{\SelectInputDefineMapping}
%    \begin{macrocode}
\newcommand*{\SelectInputDefineMapping}[1]{%
  \expandafter\gdef\csname SIE@@#1\endcsname
}
%    \end{macrocode}
%    \end{macro}
%    \begin{macrocode}
\kv@set@family@handler{SelInputMap}{%
  \@onelevel@sanitize\kv@key
  \ifx\kv@value\relax
    \PackageError{selinput}{%
      Missing input character for `\kv@key'%
    }\@ehc
  \else
    \@onelevel@sanitize\kv@value
    \ifx\kv@value\@empty
      \PackageError{selinput}{%
        Input character got lost?\MessageBreak
        Missing input character for `\kv@key'%
      }\@ehc
    \else
      \@ifundefined{SIE@@\kv@key}{%
        \PackageWarning{selinput}{%
          Missing definition for `\kv@key'%
        }%
      }{%
        \edef\SIE@StringDest{%
          \SIE@StringDest
          \kv@value
        }%
        \edef\SIE@StringUnicode{%
          \SIE@StringUnicode
          \csname SIE@@\kv@key\endcsname
        }%
      }%
    \fi
  \fi
}
%    \end{macrocode}
%    \begin{macrocode}
\kv@set@family@handler{SelInputEnc}{%
  \@onelevel@sanitize\kv@key
  \ifx\kv@value\relax
    \ifx\SIE@EncodingList\@empty
      \let\SIE@EncodingList\kv@key
    \else
      \edef\SIE@EncodingList{\SIE@EncodingList,\kv@key}%
    \fi
  \else
    \@onelevel@sanitize\kv@value
    \PackageError{selinput}{%
      Illegal key value pair (\kv@key=\kv@value)\MessagBreak
      in encoding list%
    }\@ehc
  \fi
}
%    \end{macrocode}
%
%    \begin{macro}{\SIE@LoadUnicodePackage}
%    \begin{macrocode}
\def\SIE@LoadUnicodePackage{%
  \@ifpackageloaded\SIE@UnicodePackage{}{%
    \RequirePackage\SIE@UnicodePackage\relax
  }%
  \SIE@PatchUCS
  \global\let\SIE@LoadUnicodePackage\relax
}
\let\SIE@show\show
\def\SIE@PatchUCS{%
  \AtBeginDocument{%
    \expandafter\ifx\csname ver@ucsencs.def\endcsname\relax
    \else
      \let\show\SIE@show
    \fi
  }%
}
\SIE@PatchUCS
%    \end{macrocode}
%    \end{macro}
%    \begin{macrocode}
\AtBeginDocument{%
  \let\SIE@LoadUnicodePackage\relax
}
%    \end{macrocode}
%    \begin{macro}{\SIE@EncodingUTFviii}
%    \begin{macrocode}
\def\SIE@EncodingUTFviii{utf8}
\@onelevel@sanitize\SIE@EncodingUTFviii
%    \end{macrocode}
%    \end{macro}
%    \begin{macro}{\SIE@EncodingUTFviiix}
%    \begin{macrocode}
\def\SIE@EncodingUTFviiix{utf8x}
\@onelevel@sanitize\SIE@EncodingUTFviiix
%    \end{macrocode}
%    \end{macro}
%
%    \begin{macrocode}
\let\SIE@UnicodePackage\@empty
\let\SIE@UseUTFviii\@empty
\let\SIE@Info\PackageInfo
%    \end{macrocode}
%    \begin{macrocode}
\SetupKeyvalOptions{%
  family=SelInput,%
  prefix=SelInput@%
}
\define@key{SelInput}{utf8}{%
  \def\SIE@UseUTFviii{#1}%
  \@onelevel@sanitize\SIE@UseUTFviii
}
\DeclareBoolOption{ucs}
\DeclareVoidOption{warning}{%
  \let\SIE@Info\PackageWarning
}
\ProcessKeyvalOptions{SelInput}
\ifSelInput@ucs
  \def\SIE@UnicodePackage{ucs}%
  \ifx\SIE@UseUTFviii\@empty
    \let\SIE@UseUTFviii\SIE@EncodingUTFviiix
  \fi
\else
  \ifx\SIE@UseUTFviii\@empty
    \@ifpackageloaded{ucs}{%
      \let\SIE@UseUTFviii\SIE@EncodingUTFviiix
    }{%
      \let\SIE@UseUTFviii\SIE@EncodingUTFviii
    }%
  \fi
\fi
%    \end{macrocode}
%
%    \begin{macro}{\SIE@EncodingList}
%    \begin{macrocode}
\edef\SIE@EncodingList{%
  utf8,%
  x-iso-8859-1,%
  x-iso-8859-15,%
  x-cp1252,% ansinew
  x-mac-roman,%
  x-iso-8859-2,%
  x-iso-8859-3,%
  x-iso-8859-4,%
  x-iso-8859-5,%
  x-iso-8859-6,%
  x-iso-8859-7,%
  x-iso-8859-8,%
  x-iso-8859-9,%
  x-iso-8859-10,%
  x-iso-8859-11,%
  x-iso-8859-13,%
  x-iso-8859-14,%
  x-iso-8859-15,%
  x-mac-centeuro,%
  x-mac-cyrillic,%
  x-koi8-r,%
  x-cp1250,%
  x-cp1251,%
  x-cp1257,%
  x-cp437,%
  x-cp850,%
  x-cp852,%
  x-cp855,%
  x-cp858,%
  x-cp865,%
  x-cp866,%
  x-nextstep,%
  x-dec-mcs%
}%
\@onelevel@sanitize\SIE@EncodingList
%    \end{macrocode}
%    \end{macro}
%
%    \begin{macrocode}
%</package>
%    \end{macrocode}
%
% \section{Test}
%
%    \begin{macrocode}
%<*test>
\NeedsTeXFormat{LaTeX2e}
\documentclass{minimal}
\usepackage{textcomp}
\usepackage{qstest}
%    \end{macrocode}
%    \begin{macrocode}
%<*test1|test2|test3>
\makeatletter
\let\BeginDocumentText\@empty
\def\TestEncoding#1#2{%
  \SelectInputMappings{#2}%
  \Expect*{\SIE@Encoding}{#1}%
  \Expect*{\inputencodingname}{#1}%
  \g@addto@macro\BeginDocumentText{%
    \SelectInputMappings{#2}%
    \Expect*{\SIE@Encoding}{#1}%
    \textbf{\SIE@Encoding:} %
    \kvsetkeys{test}{#2}\par
  }%
}
\def\TestKey#1#2{%
  \define@key{test}{#1}{%
    \sbox0{##1}%
    \sbox2{#2}%
    \Expect*{wd:\the\wd0, ht:\the\ht0, dp:\the\dp0}%
           *{wd:\the\wd2, ht:\the\ht2, dp:\the\dp2}%
    [#1=##1] % hash-ok
  }%
}
\RequirePackage{keyval}
\TestKey{adieresis}{\"a}
\TestKey{germandbls}{\ss}
\TestKey{Euro}{\texteuro}
\makeatother
\usepackage[
  warning,%
%<test2>  utf8=utf-8,
%<test3>  ucs,
]{selinput}
%<test1|test3>\inputencoding{ascii}
%<test2>\inputencoding{utf-8}
%<test3>\usepackage{ucs}
\begin{qstest}{preamble}{}
  \TestEncoding{x-iso-8859-15}{%
    adieresis=^^e4,%
    germandbls=^^df,%
    Euro=^^a4,%
  }%
  \TestEncoding{x-cp1252}{%
    adieresis=^^e4,%
    germandbls=^^df,%
    Euro=^^80,%
  }%
%<test1>  \TestEncoding{utf8}{%
%<test2>  \TestEncoding{utf-8}{%
%<test3>  \TestEncoding{utf8x}{%
    adieresis=^^c3^^a4,%
    germandbls=^^c3^^9f,%
%<!test2>    Euro=^^e2^^82^^ac,
  }%
\end{qstest}
%<test3>\let\ifUnicodeOptiongraphics\iffalse
\begin{document}
\begin{qstest}{document}{}
%<test3>\makeatletter
  \BeginDocumentText
\end{qstest}
%</test1|test2|test3>
%    \end{macrocode}
%
%    \begin{macrocode}
%<*test4>
\usepackage[warning,ucs]{selinput}
\SelectInputMappings{%
    adieresis=^^c3^^a4,%
    germandbls=^^c3^^9f,%
    Euro=^^e2^^82^^ac,%
}
\begin{qstest}{encoding}{}
  \Expect*{\inputencodingname}{utf8x}%
\end{qstest}
\begin{document}
  adieresis=^^c3^^a4, %
  germandbls=^^c3^^9f, %
  Euro=^^e2^^82^^ac%
%</test4>
%    \end{macrocode}
%
%    \begin{macrocode}
%<*test5>
\usepackage[warning,ucs]{selinput}
\SelectInputMappings{%
    adieresis={\"a},%
    germandbls={{\ss}},%
    Euro=\texteuro{},%
}
\begin{qstest}{encoding}{}
  \Expect*{\inputencodingname}{ascii}%
\end{qstest}
\begin{document}
  adieresis={\"a}, %
  germandbls={{\ss}}, %
  Euro=\texteuro{}%
%</test5>
%    \end{macrocode}
%
%    \begin{macrocode}
\end{document}
%</test>
%    \end{macrocode}
%
% \section{Installation}
%
% \subsection{Download}
%
% \paragraph{Package.} This package is available on
% CTAN\footnote{\url{ftp://ftp.ctan.org/tex-archive/}}:
% \begin{description}
% \item[\CTAN{macros/latex/contrib/oberdiek/selinput.dtx}] The source file.
% \item[\CTAN{macros/latex/contrib/oberdiek/selinput.pdf}] Documentation.
% \end{description}
%
%
% \paragraph{Bundle.} All the packages of the bundle `oberdiek'
% are also available in a TDS compliant ZIP archive. There
% the packages are already unpacked and the documentation files
% are generated. The files and directories obey the TDS standard.
% \begin{description}
% \item[\CTAN{install/macros/latex/contrib/oberdiek.tds.zip}]
% \end{description}
% \emph{TDS} refers to the standard ``A Directory Structure
% for \TeX\ Files'' (\CTAN{tds/tds.pdf}). Directories
% with \xfile{texmf} in their name are usually organized this way.
%
% \subsection{Bundle installation}
%
% \paragraph{Unpacking.} Unpack the \xfile{oberdiek.tds.zip} in the
% TDS tree (also known as \xfile{texmf} tree) of your choice.
% Example (linux):
% \begin{quote}
%   |unzip oberdiek.tds.zip -d ~/texmf|
% \end{quote}
%
% \paragraph{Script installation.}
% Check the directory \xfile{TDS:scripts/oberdiek/} for
% scripts that need further installation steps.
% Package \xpackage{attachfile2} comes with the Perl script
% \xfile{pdfatfi.pl} that should be installed in such a way
% that it can be called as \texttt{pdfatfi}.
% Example (linux):
% \begin{quote}
%   |chmod +x scripts/oberdiek/pdfatfi.pl|\\
%   |cp scripts/oberdiek/pdfatfi.pl /usr/local/bin/|
% \end{quote}
%
% \subsection{Package installation}
%
% \paragraph{Unpacking.} The \xfile{.dtx} file is a self-extracting
% \docstrip\ archive. The files are extracted by running the
% \xfile{.dtx} through \plainTeX:
% \begin{quote}
%   \verb|tex selinput.dtx|
% \end{quote}
%
% \paragraph{TDS.} Now the different files must be moved into
% the different directories in your installation TDS tree
% (also known as \xfile{texmf} tree):
% \begin{quote}
% \def\t{^^A
% \begin{tabular}{@{}>{\ttfamily}l@{ $\rightarrow$ }>{\ttfamily}l@{}}
%   selinput.sty & tex/latex/oberdiek/selinput.sty\\
%   selinput.pdf & doc/latex/oberdiek/selinput.pdf\\
%   test/selinput-test1.tex & doc/latex/oberdiek/test/selinput-test1.tex\\
%   test/selinput-test2.tex & doc/latex/oberdiek/test/selinput-test2.tex\\
%   test/selinput-test3.tex & doc/latex/oberdiek/test/selinput-test3.tex\\
%   test/selinput-test4.tex & doc/latex/oberdiek/test/selinput-test4.tex\\
%   test/selinput-test5.tex & doc/latex/oberdiek/test/selinput-test5.tex\\
%   selinput.dtx & source/latex/oberdiek/selinput.dtx\\
% \end{tabular}^^A
% }^^A
% \sbox0{\t}^^A
% \ifdim\wd0>\linewidth
%   \begingroup
%     \advance\linewidth by\leftmargin
%     \advance\linewidth by\rightmargin
%   \edef\x{\endgroup
%     \def\noexpand\lw{\the\linewidth}^^A
%   }\x
%   \def\lwbox{^^A
%     \leavevmode
%     \hbox to \linewidth{^^A
%       \kern-\leftmargin\relax
%       \hss
%       \usebox0
%       \hss
%       \kern-\rightmargin\relax
%     }^^A
%   }^^A
%   \ifdim\wd0>\lw
%     \sbox0{\small\t}^^A
%     \ifdim\wd0>\linewidth
%       \ifdim\wd0>\lw
%         \sbox0{\footnotesize\t}^^A
%         \ifdim\wd0>\linewidth
%           \ifdim\wd0>\lw
%             \sbox0{\scriptsize\t}^^A
%             \ifdim\wd0>\linewidth
%               \ifdim\wd0>\lw
%                 \sbox0{\tiny\t}^^A
%                 \ifdim\wd0>\linewidth
%                   \lwbox
%                 \else
%                   \usebox0
%                 \fi
%               \else
%                 \lwbox
%               \fi
%             \else
%               \usebox0
%             \fi
%           \else
%             \lwbox
%           \fi
%         \else
%           \usebox0
%         \fi
%       \else
%         \lwbox
%       \fi
%     \else
%       \usebox0
%     \fi
%   \else
%     \lwbox
%   \fi
% \else
%   \usebox0
% \fi
% \end{quote}
% If you have a \xfile{docstrip.cfg} that configures and enables \docstrip's
% TDS installing feature, then some files can already be in the right
% place, see the documentation of \docstrip.
%
% \subsection{Refresh file name databases}
%
% If your \TeX~distribution
% (\teTeX, \mikTeX, \dots) relies on file name databases, you must refresh
% these. For example, \teTeX\ users run \verb|texhash| or
% \verb|mktexlsr|.
%
% \subsection{Some details for the interested}
%
% \paragraph{Attached source.}
%
% The PDF documentation on CTAN also includes the
% \xfile{.dtx} source file. It can be extracted by
% AcrobatReader 6 or higher. Another option is \textsf{pdftk},
% e.g. unpack the file into the current directory:
% \begin{quote}
%   \verb|pdftk selinput.pdf unpack_files output .|
% \end{quote}
%
% \paragraph{Unpacking with \LaTeX.}
% The \xfile{.dtx} chooses its action depending on the format:
% \begin{description}
% \item[\plainTeX:] Run \docstrip\ and extract the files.
% \item[\LaTeX:] Generate the documentation.
% \end{description}
% If you insist on using \LaTeX\ for \docstrip\ (really,
% \docstrip\ does not need \LaTeX), then inform the autodetect routine
% about your intention:
% \begin{quote}
%   \verb|latex \let\install=y% \iffalse meta-comment
%
% File: selinput.dtx
% Version: 2007/09/09 v1.2
% Info: Semi-automatic input encoding detection
%
% Copyright (C) 2007 by
%    Heiko Oberdiek <heiko.oberdiek at googlemail.com>
%
% This work may be distributed and/or modified under the
% conditions of the LaTeX Project Public License, either
% version 1.3c of this license or (at your option) any later
% version. This version of this license is in
%    http://www.latex-project.org/lppl/lppl-1-3c.txt
% and the latest version of this license is in
%    http://www.latex-project.org/lppl.txt
% and version 1.3 or later is part of all distributions of
% LaTeX version 2005/12/01 or later.
%
% This work has the LPPL maintenance status "maintained".
%
% This Current Maintainer of this work is Heiko Oberdiek.
%
% This work consists of the main source file selinput.dtx
% and the derived files
%    selinput.sty, selinput.pdf, selinput.ins, selinput.drv,
%    selinput-test1.tex, selinput-test2.tex, selinput-test3.tex,
%    selinput-test4.tex, selinput-test5.tex.
%
% Distribution:
%    CTAN:macros/latex/contrib/oberdiek/selinput.dtx
%    CTAN:macros/latex/contrib/oberdiek/selinput.pdf
%
% Unpacking:
%    (a) If selinput.ins is present:
%           tex selinput.ins
%    (b) Without selinput.ins:
%           tex selinput.dtx
%    (c) If you insist on using LaTeX
%           latex \let\install=y\input{selinput.dtx}
%        (quote the arguments according to the demands of your shell)
%
% Documentation:
%    (a) If selinput.drv is present:
%           latex selinput.drv
%    (b) Without selinput.drv:
%           latex selinput.dtx; ...
%    The class ltxdoc loads the configuration file ltxdoc.cfg
%    if available. Here you can specify further options, e.g.
%    use A4 as paper format:
%       \PassOptionsToClass{a4paper}{article}
%
%    Programm calls to get the documentation (example):
%       pdflatex selinput.dtx
%       makeindex -s gind.ist selinput.idx
%       pdflatex selinput.dtx
%       makeindex -s gind.ist selinput.idx
%       pdflatex selinput.dtx
%
% Installation:
%    TDS:tex/latex/oberdiek/selinput.sty
%    TDS:doc/latex/oberdiek/selinput.pdf
%    TDS:doc/latex/oberdiek/test/selinput-test1.tex
%    TDS:doc/latex/oberdiek/test/selinput-test2.tex
%    TDS:doc/latex/oberdiek/test/selinput-test3.tex
%    TDS:doc/latex/oberdiek/test/selinput-test4.tex
%    TDS:doc/latex/oberdiek/test/selinput-test5.tex
%    TDS:source/latex/oberdiek/selinput.dtx
%
%<*ignore>
\begingroup
  \catcode123=1 %
  \catcode125=2 %
  \def\x{LaTeX2e}%
\expandafter\endgroup
\ifcase 0\ifx\install y1\fi\expandafter
         \ifx\csname processbatchFile\endcsname\relax\else1\fi
         \ifx\fmtname\x\else 1\fi\relax
\else\csname fi\endcsname
%</ignore>
%<*install>
\input docstrip.tex
\Msg{************************************************************************}
\Msg{* Installation}
\Msg{* Package: selinput 2007/09/09 v1.2 Semi-automatic input encoding detection (HO)}
\Msg{************************************************************************}

\keepsilent
\askforoverwritefalse

\let\MetaPrefix\relax
\preamble

This is a generated file.

Project: selinput
Version: 2007/09/09 v1.2

Copyright (C) 2007 by
   Heiko Oberdiek <heiko.oberdiek at googlemail.com>

This work may be distributed and/or modified under the
conditions of the LaTeX Project Public License, either
version 1.3c of this license or (at your option) any later
version. This version of this license is in
   http://www.latex-project.org/lppl/lppl-1-3c.txt
and the latest version of this license is in
   http://www.latex-project.org/lppl.txt
and version 1.3 or later is part of all distributions of
LaTeX version 2005/12/01 or later.

This work has the LPPL maintenance status "maintained".

This Current Maintainer of this work is Heiko Oberdiek.

This work consists of the main source file selinput.dtx
and the derived files
   selinput.sty, selinput.pdf, selinput.ins, selinput.drv,
   selinput-test1.tex, selinput-test2.tex, selinput-test3.tex,
   selinput-test4.tex, selinput-test5.tex.

\endpreamble
\let\MetaPrefix\DoubleperCent

\generate{%
  \file{selinput.ins}{\from{selinput.dtx}{install}}%
  \file{selinput.drv}{\from{selinput.dtx}{driver}}%
  \usedir{tex/latex/oberdiek}%
  \file{selinput.sty}{\from{selinput.dtx}{package}}%
  \usedir{doc/latex/oberdiek/test}%
  \file{selinput-test1.tex}{\from{selinput.dtx}{test,test1}}%
  \file{selinput-test2.tex}{\from{selinput.dtx}{test,test2}}%
  \file{selinput-test3.tex}{\from{selinput.dtx}{test,test3}}%
  \file{selinput-test4.tex}{\from{selinput.dtx}{test,test4}}%
  \file{selinput-test5.tex}{\from{selinput.dtx}{test,test5}}%
  \nopreamble
  \nopostamble
  \usedir{source/latex/oberdiek/catalogue}%
  \file{selinput.xml}{\from{selinput.dtx}{catalogue}}%
}

\catcode32=13\relax% active space
\let =\space%
\Msg{************************************************************************}
\Msg{*}
\Msg{* To finish the installation you have to move the following}
\Msg{* file into a directory searched by TeX:}
\Msg{*}
\Msg{*     selinput.sty}
\Msg{*}
\Msg{* To produce the documentation run the file `selinput.drv'}
\Msg{* through LaTeX.}
\Msg{*}
\Msg{* Happy TeXing!}
\Msg{*}
\Msg{************************************************************************}

\endbatchfile
%</install>
%<*ignore>
\fi
%</ignore>
%<*driver>
\NeedsTeXFormat{LaTeX2e}
\ProvidesFile{selinput.drv}%
  [2007/09/09 v1.2 Semi-automatic input encoding detection (HO)]%
\documentclass{ltxdoc}
\usepackage[T1]{fontenc}
\usepackage{textcomp}
\usepackage{lmodern}
\usepackage{holtxdoc}[2011/11/22]
\usepackage{color}
\begin{document}
  \DocInput{selinput.dtx}%
\end{document}
%</driver>
% \fi
%
% \CheckSum{389}
%
% \CharacterTable
%  {Upper-case    \A\B\C\D\E\F\G\H\I\J\K\L\M\N\O\P\Q\R\S\T\U\V\W\X\Y\Z
%   Lower-case    \a\b\c\d\e\f\g\h\i\j\k\l\m\n\o\p\q\r\s\t\u\v\w\x\y\z
%   Digits        \0\1\2\3\4\5\6\7\8\9
%   Exclamation   \!     Double quote  \"     Hash (number) \#
%   Dollar        \$     Percent       \%     Ampersand     \&
%   Acute accent  \'     Left paren    \(     Right paren   \)
%   Asterisk      \*     Plus          \+     Comma         \,
%   Minus         \-     Point         \.     Solidus       \/
%   Colon         \:     Semicolon     \;     Less than     \<
%   Equals        \=     Greater than  \>     Question mark \?
%   Commercial at \@     Left bracket  \[     Backslash     \\
%   Right bracket \]     Circumflex    \^     Underscore    \_
%   Grave accent  \`     Left brace    \{     Vertical bar  \|
%   Right brace   \}     Tilde         \~}
%
% \GetFileInfo{selinput.drv}
%
% \title{The \xpackage{selinput} package}
% \date{2007/09/09 v1.2}
% \author{Heiko Oberdiek\\\xemail{heiko.oberdiek at googlemail.com}}
%
% \maketitle
%
% \begin{abstract}
% This package selects the input encoding by specifying between
% input characters and their glyph names.
% \end{abstract}
%
% \tableofcontents
%
% \newcommand*{\EM}{\textcolor{blue}}
% \newcommand*{\ExampleText}{^^A
%   Umlauts:\ \EM{\"A\"O\"U\"a\"o\"u\ss}^^A
% }
%
% \section{Documentation}
%
% \subsection{Introduction}
%
% \LaTeX\ supports the direct use of 8-bit characters by means
% of package \xpackage{inputenc}. However you must know
% and specify the encoding, e.g.:
% \begin{quote}
%   \ttfamily
%   |\documentclass{article}|\\
%   |\usepackage[|\EM{latin1}|]{inputenc}|\\
%   |% or \usepackage[|\EM{utf8}|]{inputenc}|\\
%   |% or \usepackage[|\EM{??}|]{inputenc}|\\
%   |\begin{document}|\\
%   |  |\ExampleText\\
%   |\end{document}|
% \end{quote}
%
% If the document is transferred in an environment that
% uses a different encoding, then there are programs that
% convert the input characters. Examples for conversion
% of file \xfile{test.tex}
% from encoding latin1 (ISO-8859-1) to UTF-8:
% \begin{quote}
%   \ttfamily
%   |recode ISO-8859-1..UTF-8 test.tex|\\
%   |recode latin1..utf8 test.tex|\\
%   |iconv --from-code ISO-8859-1|\\
%   \hphantom{iconv}| --to-code UTF-8|\\
%   \hphantom{iconv}| --output testnew.tex|\\
%   \hphantom{iconv}| test.tex|\\
%   |iconv -f latin1 -t utf8 -o testnew.tex test.tex|
% \end{quote}
% However, the encoding name for package \xpackage{inputenc}
% must be changed:
% \begin{quote}
%    |\usepackage[latin1]{inputenc}| $\rightarrow$
%    |\usepackage[utf8]{inputenc}|\kern-4pt\relax
% \end{quote}
% Of course, unless you are using some clever editor
% that knows package \xpackage{inputenc}, recodes
% the file and adjusts the option at the same time.
% But most editors can perhaps recode the file, but
% they let the option untouched.
%
% Therefore package \xpackage{selinput} chooses another way for
% specifying the input encoding. The encoding name is not needed
% at all. Some 8-bit characters are identified by their glyph
% name and the package chooses an appropriate encoding, example:
% \begin{quote}
%   \ttfamily
%   |\documentclass{article}|\\
%   |\usepackage{selinput}|\\
%   |\SelectInputMappings{|\\
%   |  adieresis={|\EM{\"a}|}|,\\
%   |  germandbls={|\EM{\ss}|}|,\\
%   |  Euro={|\EM{\texteuro}|}|,\\
%   |}|\\
%   |\begin{document}|\\
%   |  |\ExampleText\\
%   |\end{document}|
% \end{quote}
%
% \subsection{User interface}
%
% \begin{declcs}{SelectInputEncodingList} \M{encoding list}
% \end{declcs}
% \cs{SelectInputEncodingList} expects a comma separated list of
% encoding names. Example:
% \begin{quote}
%   |\SelectInputEncodingList{utf8,ansinew,mac-roman}|
% \end{quote}
% The encodings of package \xpackage{inputenx} are used as default.
%
% \begin{declcs}{SelectInputMappings} \M{mapping pairs}
% \end{declcs}
% A mapping pair consists of a glyph name and its input
% character:
% \begin{quote}
%   |\SelectInputMappings{|\\
%   |  adieresis={|\EM{\"a}|}|,\\
%   |  germandbls={|\EM{\ss}|}|,\\
%   |  Euro={|\EM{\texteuro}|}|,\\
%   |}|
% \end{quote}
% The supported glyph names can be found in file \xfile{ix-name.def}
% of project \xpackage{inputenx} \cite{inputenx}. The names are
% basically taken from Adobe's glyphlists \cite{adobe:glyphlist,adobe:aglfn}.
% As many pairs are needed as necessary to identify the encoding.
% Example with insufficient pairs:
% \begin{quote}
%   \ttfamily
%   |\SelectInputEncodingSet{latin1,latin9}|\\
%   |\SelectInputMappings{|\\
%   |  adieresis={|\EM{\"a}|}|,\\
%   |  germandbls={|\EM{\ss}|}|,\\
%   |}|\\
%   \ExampleText| and Euro: |\EM{\textcurrency} (wrong)
% \end{quote}
% The first encoding \xoption{latin1} passes the constraints given
% by the mapping pairs. However the Euro symbol is not part of
% the encoding. Thus a mapping pair with the Euro symbol
% solves the problem. In fact the symbol alone already succeeds in selecting
% between \xoption{latin1} and \xoption{latin9}:
% \begin{quote}
%   \ttfamily
%   |\SelectInputEncodingSet{latin1,latin9}|\\
%   |\SelectInputMappings{|\\
%   |  Euro={|\EM{\texteuro}|},|\\
%   |}|\\
%   \ExampleText| and Euro: |\EM{\texteuro}
% \end{quote}
%
% \subsection{Options}
%
% \begin{description}
% \item[\xoption{warning}:]
%   The selected encoding is written
%   by \cs{PackageInfo} into the \xfile{.log} file only.
%   Option \xoption{warning} changes it to \cs{PackageWarning}.
%   Then the selected encoding is shown on the terminal as well.
% \item[\xoption{ucs}:]
%   The encoding file \xfile{utf8x} of package \cs{ucs} requires
%   that the package itself is loaded before.
%   If the package is not loaded, then the option \xoption{ucs}
%   will load package \xpackage{ucs} if the detected encoding is
%   UTF-8 (limited to the preamble, packages cannot be loaded later).
% \item[\xoption{utf8=\dots}:]
%   The option allows to specify other encoding files
%   for UTF-8 than \LaTeX's \xfile{utf8.def}. For example,
%   |utf8=utf-8| will load \xfile{utf-8.def} instead.
% \end{description}
%
% \subsection{Encodings}
%
% Package \xpackage{stringenc} \cite{stringenc}
% is used for testing the encoding. Thus the encoding
% name must be known by this package. Then the found
% encoding is loaded by \cs{inputencoding} by package
% \xpackage{inputenc} or \cs{InputEncoding} if package
% \xpackage{inputenx} is loaded.
%
% The supported encodings are present in the encoding list,
% thus usually the encoding names do not matter.
% If the list is set by \cs{SelectInputEncodingList},
% then you can use the names that work for package
% \xpackage{inputenc} and are known by package \xpackage{stringenc},
% for example: \xoption{latin1}, \xoption{x-iso-8859-1}. Encoding
% file names of package \xpackage{inputenx} are prefixed with \xfile{x-}.
% The prefix can be dropped, if package \xpackage{inputenx} is loaded.
%
% \StopEventually{
% }
%
% \section{Implementation}
%
%    \begin{macrocode}
%<*package>
\NeedsTeXFormat{LaTeX2e}
\ProvidesPackage{selinput}
  [2007/09/09 v1.2 Semi-automatic input encoding detection (HO)]%
%    \end{macrocode}
%
%    \begin{macrocode}
\RequirePackage{inputenc}
\RequirePackage{kvsetkeys}[2006/10/19]
\RequirePackage{stringenc}[2007/06/16]
\RequirePackage{kvoptions}
%    \end{macrocode}
%    \begin{macro}{\SelectInputEncodingList}
%    \begin{macrocode}
\newcommand*{\SelectInputEncodingList}{%
  \let\SIE@EncodingList\@empty
  \kvsetkeys{SelInputEnc}%
}
%    \end{macrocode}
%    \end{macro}
%    \begin{macro}{\SelectInputMappings}
%    \begin{macrocode}
\newcommand*{\SelectInputMappings}[1]{%
  \SIE@LoadNameDefs
  \let\SIE@StringUnicode\@empty
  \let\SIE@StringDest\@empty
  \kvsetkeys{SelInputMap}{#1}%
  \ifx\\SIE@StringUnicode\SIE@StringDest\\%
    \PackageError{selinput}{%
      No mappings specified%
    }\@ehc
  \else
    \EdefUnescapeHex\SIE@StringUnicode\SIE@StringUnicode
    \let\SIE@Encoding\@empty
    \@for\SIE@EncodingTest:=\SIE@EncodingList\do{%
      \ifx\SIE@Encoding\@empty
        \StringEncodingConvertTest\SIE@temp\SIE@StringUnicode
                                  {utf16be}\SIE@EncodingTest{%
          \ifx\SIE@temp\SIE@StringDest
            \let\SIE@Encoding\SIE@EncodingTest
          \fi
        }{}%
      \fi
    }%
    \ifx\SIE@Encoding\@empty
      \StringEncodingConvertTest\SIE@temp\SIE@StringDest
                                {ascii}{utf16be}{%
        \def\SIE@Encoding{ascii}%
        \SIE@Info{selinput}{%
          Matching encoding not found, but input characters%
          \MessageBreak
          are 7-bit (possibly editor replacements).%
          \MessageBreak
          Hence using ascii encoding%
        }%
      }{}%
    \fi
    \ifx\SIE@Encoding\@empty
      \PackageError{selinput}{%
        Cannot find a matching encoding%
      }\@ehd
    \else
      \ifx\SIE@Encoding\SIE@EncodingUTFviii
        \SIE@LoadUnicodePackage
        \ifx\SIE@UseUTFviii\@empty
        \else
          \let\SIE@Encoding\SIE@UseUTFviii
        \fi
      \fi
      \begingroup\expandafter\expandafter\expandafter\endgroup
      \expandafter\ifx\csname InputEncoding\endcsname\relax
        \inputencoding\SIE@Encoding
      \else
        \InputEncoding\SIE@Encoding
      \fi
      \SIE@Info{selinput}{Encoding `\SIE@Encoding' selected}%
    \fi
  \fi
}
%    \end{macrocode}
%    \end{macro}
%    \begin{macro}{\SIE@LoadNameDefs}
%    \begin{macrocode}
\def\SIE@LoadNameDefs{%
  \begingroup
    \endlinechar=\m@ne
    \catcode92=0 % backslash
    \catcode123=1 % left curly brace/beginning of group
    \catcode125=2 % right curly brace/end of group
    \catcode37=14 % percent/comment character
    \@makeother\[%
    \@makeother\]%
    \@makeother\.%
    \@makeother\(%
    \@makeother\)%
    \@makeother\/%
    \@makeother\-%
    \let\InputenxName\SelectInputDefineMapping
    \InputIfFileExists{ix-name.def}{}{%
      \PackageError{selinput}{%
        Missing `ix-name.def' (part of package `inputenx')%
      }\@ehd
    }%
    \global\let\SIE@LoadNameDefs\relax
  \endgroup
}
%    \end{macrocode}
%    \end{macro}
%    \begin{macro}{\SelectInputDefineMapping}
%    \begin{macrocode}
\newcommand*{\SelectInputDefineMapping}[1]{%
  \expandafter\gdef\csname SIE@@#1\endcsname
}
%    \end{macrocode}
%    \end{macro}
%    \begin{macrocode}
\kv@set@family@handler{SelInputMap}{%
  \@onelevel@sanitize\kv@key
  \ifx\kv@value\relax
    \PackageError{selinput}{%
      Missing input character for `\kv@key'%
    }\@ehc
  \else
    \@onelevel@sanitize\kv@value
    \ifx\kv@value\@empty
      \PackageError{selinput}{%
        Input character got lost?\MessageBreak
        Missing input character for `\kv@key'%
      }\@ehc
    \else
      \@ifundefined{SIE@@\kv@key}{%
        \PackageWarning{selinput}{%
          Missing definition for `\kv@key'%
        }%
      }{%
        \edef\SIE@StringDest{%
          \SIE@StringDest
          \kv@value
        }%
        \edef\SIE@StringUnicode{%
          \SIE@StringUnicode
          \csname SIE@@\kv@key\endcsname
        }%
      }%
    \fi
  \fi
}
%    \end{macrocode}
%    \begin{macrocode}
\kv@set@family@handler{SelInputEnc}{%
  \@onelevel@sanitize\kv@key
  \ifx\kv@value\relax
    \ifx\SIE@EncodingList\@empty
      \let\SIE@EncodingList\kv@key
    \else
      \edef\SIE@EncodingList{\SIE@EncodingList,\kv@key}%
    \fi
  \else
    \@onelevel@sanitize\kv@value
    \PackageError{selinput}{%
      Illegal key value pair (\kv@key=\kv@value)\MessagBreak
      in encoding list%
    }\@ehc
  \fi
}
%    \end{macrocode}
%
%    \begin{macro}{\SIE@LoadUnicodePackage}
%    \begin{macrocode}
\def\SIE@LoadUnicodePackage{%
  \@ifpackageloaded\SIE@UnicodePackage{}{%
    \RequirePackage\SIE@UnicodePackage\relax
  }%
  \SIE@PatchUCS
  \global\let\SIE@LoadUnicodePackage\relax
}
\let\SIE@show\show
\def\SIE@PatchUCS{%
  \AtBeginDocument{%
    \expandafter\ifx\csname ver@ucsencs.def\endcsname\relax
    \else
      \let\show\SIE@show
    \fi
  }%
}
\SIE@PatchUCS
%    \end{macrocode}
%    \end{macro}
%    \begin{macrocode}
\AtBeginDocument{%
  \let\SIE@LoadUnicodePackage\relax
}
%    \end{macrocode}
%    \begin{macro}{\SIE@EncodingUTFviii}
%    \begin{macrocode}
\def\SIE@EncodingUTFviii{utf8}
\@onelevel@sanitize\SIE@EncodingUTFviii
%    \end{macrocode}
%    \end{macro}
%    \begin{macro}{\SIE@EncodingUTFviiix}
%    \begin{macrocode}
\def\SIE@EncodingUTFviiix{utf8x}
\@onelevel@sanitize\SIE@EncodingUTFviiix
%    \end{macrocode}
%    \end{macro}
%
%    \begin{macrocode}
\let\SIE@UnicodePackage\@empty
\let\SIE@UseUTFviii\@empty
\let\SIE@Info\PackageInfo
%    \end{macrocode}
%    \begin{macrocode}
\SetupKeyvalOptions{%
  family=SelInput,%
  prefix=SelInput@%
}
\define@key{SelInput}{utf8}{%
  \def\SIE@UseUTFviii{#1}%
  \@onelevel@sanitize\SIE@UseUTFviii
}
\DeclareBoolOption{ucs}
\DeclareVoidOption{warning}{%
  \let\SIE@Info\PackageWarning
}
\ProcessKeyvalOptions{SelInput}
\ifSelInput@ucs
  \def\SIE@UnicodePackage{ucs}%
  \ifx\SIE@UseUTFviii\@empty
    \let\SIE@UseUTFviii\SIE@EncodingUTFviiix
  \fi
\else
  \ifx\SIE@UseUTFviii\@empty
    \@ifpackageloaded{ucs}{%
      \let\SIE@UseUTFviii\SIE@EncodingUTFviiix
    }{%
      \let\SIE@UseUTFviii\SIE@EncodingUTFviii
    }%
  \fi
\fi
%    \end{macrocode}
%
%    \begin{macro}{\SIE@EncodingList}
%    \begin{macrocode}
\edef\SIE@EncodingList{%
  utf8,%
  x-iso-8859-1,%
  x-iso-8859-15,%
  x-cp1252,% ansinew
  x-mac-roman,%
  x-iso-8859-2,%
  x-iso-8859-3,%
  x-iso-8859-4,%
  x-iso-8859-5,%
  x-iso-8859-6,%
  x-iso-8859-7,%
  x-iso-8859-8,%
  x-iso-8859-9,%
  x-iso-8859-10,%
  x-iso-8859-11,%
  x-iso-8859-13,%
  x-iso-8859-14,%
  x-iso-8859-15,%
  x-mac-centeuro,%
  x-mac-cyrillic,%
  x-koi8-r,%
  x-cp1250,%
  x-cp1251,%
  x-cp1257,%
  x-cp437,%
  x-cp850,%
  x-cp852,%
  x-cp855,%
  x-cp858,%
  x-cp865,%
  x-cp866,%
  x-nextstep,%
  x-dec-mcs%
}%
\@onelevel@sanitize\SIE@EncodingList
%    \end{macrocode}
%    \end{macro}
%
%    \begin{macrocode}
%</package>
%    \end{macrocode}
%
% \section{Test}
%
%    \begin{macrocode}
%<*test>
\NeedsTeXFormat{LaTeX2e}
\documentclass{minimal}
\usepackage{textcomp}
\usepackage{qstest}
%    \end{macrocode}
%    \begin{macrocode}
%<*test1|test2|test3>
\makeatletter
\let\BeginDocumentText\@empty
\def\TestEncoding#1#2{%
  \SelectInputMappings{#2}%
  \Expect*{\SIE@Encoding}{#1}%
  \Expect*{\inputencodingname}{#1}%
  \g@addto@macro\BeginDocumentText{%
    \SelectInputMappings{#2}%
    \Expect*{\SIE@Encoding}{#1}%
    \textbf{\SIE@Encoding:} %
    \kvsetkeys{test}{#2}\par
  }%
}
\def\TestKey#1#2{%
  \define@key{test}{#1}{%
    \sbox0{##1}%
    \sbox2{#2}%
    \Expect*{wd:\the\wd0, ht:\the\ht0, dp:\the\dp0}%
           *{wd:\the\wd2, ht:\the\ht2, dp:\the\dp2}%
    [#1=##1] % hash-ok
  }%
}
\RequirePackage{keyval}
\TestKey{adieresis}{\"a}
\TestKey{germandbls}{\ss}
\TestKey{Euro}{\texteuro}
\makeatother
\usepackage[
  warning,%
%<test2>  utf8=utf-8,
%<test3>  ucs,
]{selinput}
%<test1|test3>\inputencoding{ascii}
%<test2>\inputencoding{utf-8}
%<test3>\usepackage{ucs}
\begin{qstest}{preamble}{}
  \TestEncoding{x-iso-8859-15}{%
    adieresis=^^e4,%
    germandbls=^^df,%
    Euro=^^a4,%
  }%
  \TestEncoding{x-cp1252}{%
    adieresis=^^e4,%
    germandbls=^^df,%
    Euro=^^80,%
  }%
%<test1>  \TestEncoding{utf8}{%
%<test2>  \TestEncoding{utf-8}{%
%<test3>  \TestEncoding{utf8x}{%
    adieresis=^^c3^^a4,%
    germandbls=^^c3^^9f,%
%<!test2>    Euro=^^e2^^82^^ac,
  }%
\end{qstest}
%<test3>\let\ifUnicodeOptiongraphics\iffalse
\begin{document}
\begin{qstest}{document}{}
%<test3>\makeatletter
  \BeginDocumentText
\end{qstest}
%</test1|test2|test3>
%    \end{macrocode}
%
%    \begin{macrocode}
%<*test4>
\usepackage[warning,ucs]{selinput}
\SelectInputMappings{%
    adieresis=^^c3^^a4,%
    germandbls=^^c3^^9f,%
    Euro=^^e2^^82^^ac,%
}
\begin{qstest}{encoding}{}
  \Expect*{\inputencodingname}{utf8x}%
\end{qstest}
\begin{document}
  adieresis=^^c3^^a4, %
  germandbls=^^c3^^9f, %
  Euro=^^e2^^82^^ac%
%</test4>
%    \end{macrocode}
%
%    \begin{macrocode}
%<*test5>
\usepackage[warning,ucs]{selinput}
\SelectInputMappings{%
    adieresis={\"a},%
    germandbls={{\ss}},%
    Euro=\texteuro{},%
}
\begin{qstest}{encoding}{}
  \Expect*{\inputencodingname}{ascii}%
\end{qstest}
\begin{document}
  adieresis={\"a}, %
  germandbls={{\ss}}, %
  Euro=\texteuro{}%
%</test5>
%    \end{macrocode}
%
%    \begin{macrocode}
\end{document}
%</test>
%    \end{macrocode}
%
% \section{Installation}
%
% \subsection{Download}
%
% \paragraph{Package.} This package is available on
% CTAN\footnote{\url{ftp://ftp.ctan.org/tex-archive/}}:
% \begin{description}
% \item[\CTAN{macros/latex/contrib/oberdiek/selinput.dtx}] The source file.
% \item[\CTAN{macros/latex/contrib/oberdiek/selinput.pdf}] Documentation.
% \end{description}
%
%
% \paragraph{Bundle.} All the packages of the bundle `oberdiek'
% are also available in a TDS compliant ZIP archive. There
% the packages are already unpacked and the documentation files
% are generated. The files and directories obey the TDS standard.
% \begin{description}
% \item[\CTAN{install/macros/latex/contrib/oberdiek.tds.zip}]
% \end{description}
% \emph{TDS} refers to the standard ``A Directory Structure
% for \TeX\ Files'' (\CTAN{tds/tds.pdf}). Directories
% with \xfile{texmf} in their name are usually organized this way.
%
% \subsection{Bundle installation}
%
% \paragraph{Unpacking.} Unpack the \xfile{oberdiek.tds.zip} in the
% TDS tree (also known as \xfile{texmf} tree) of your choice.
% Example (linux):
% \begin{quote}
%   |unzip oberdiek.tds.zip -d ~/texmf|
% \end{quote}
%
% \paragraph{Script installation.}
% Check the directory \xfile{TDS:scripts/oberdiek/} for
% scripts that need further installation steps.
% Package \xpackage{attachfile2} comes with the Perl script
% \xfile{pdfatfi.pl} that should be installed in such a way
% that it can be called as \texttt{pdfatfi}.
% Example (linux):
% \begin{quote}
%   |chmod +x scripts/oberdiek/pdfatfi.pl|\\
%   |cp scripts/oberdiek/pdfatfi.pl /usr/local/bin/|
% \end{quote}
%
% \subsection{Package installation}
%
% \paragraph{Unpacking.} The \xfile{.dtx} file is a self-extracting
% \docstrip\ archive. The files are extracted by running the
% \xfile{.dtx} through \plainTeX:
% \begin{quote}
%   \verb|tex selinput.dtx|
% \end{quote}
%
% \paragraph{TDS.} Now the different files must be moved into
% the different directories in your installation TDS tree
% (also known as \xfile{texmf} tree):
% \begin{quote}
% \def\t{^^A
% \begin{tabular}{@{}>{\ttfamily}l@{ $\rightarrow$ }>{\ttfamily}l@{}}
%   selinput.sty & tex/latex/oberdiek/selinput.sty\\
%   selinput.pdf & doc/latex/oberdiek/selinput.pdf\\
%   test/selinput-test1.tex & doc/latex/oberdiek/test/selinput-test1.tex\\
%   test/selinput-test2.tex & doc/latex/oberdiek/test/selinput-test2.tex\\
%   test/selinput-test3.tex & doc/latex/oberdiek/test/selinput-test3.tex\\
%   test/selinput-test4.tex & doc/latex/oberdiek/test/selinput-test4.tex\\
%   test/selinput-test5.tex & doc/latex/oberdiek/test/selinput-test5.tex\\
%   selinput.dtx & source/latex/oberdiek/selinput.dtx\\
% \end{tabular}^^A
% }^^A
% \sbox0{\t}^^A
% \ifdim\wd0>\linewidth
%   \begingroup
%     \advance\linewidth by\leftmargin
%     \advance\linewidth by\rightmargin
%   \edef\x{\endgroup
%     \def\noexpand\lw{\the\linewidth}^^A
%   }\x
%   \def\lwbox{^^A
%     \leavevmode
%     \hbox to \linewidth{^^A
%       \kern-\leftmargin\relax
%       \hss
%       \usebox0
%       \hss
%       \kern-\rightmargin\relax
%     }^^A
%   }^^A
%   \ifdim\wd0>\lw
%     \sbox0{\small\t}^^A
%     \ifdim\wd0>\linewidth
%       \ifdim\wd0>\lw
%         \sbox0{\footnotesize\t}^^A
%         \ifdim\wd0>\linewidth
%           \ifdim\wd0>\lw
%             \sbox0{\scriptsize\t}^^A
%             \ifdim\wd0>\linewidth
%               \ifdim\wd0>\lw
%                 \sbox0{\tiny\t}^^A
%                 \ifdim\wd0>\linewidth
%                   \lwbox
%                 \else
%                   \usebox0
%                 \fi
%               \else
%                 \lwbox
%               \fi
%             \else
%               \usebox0
%             \fi
%           \else
%             \lwbox
%           \fi
%         \else
%           \usebox0
%         \fi
%       \else
%         \lwbox
%       \fi
%     \else
%       \usebox0
%     \fi
%   \else
%     \lwbox
%   \fi
% \else
%   \usebox0
% \fi
% \end{quote}
% If you have a \xfile{docstrip.cfg} that configures and enables \docstrip's
% TDS installing feature, then some files can already be in the right
% place, see the documentation of \docstrip.
%
% \subsection{Refresh file name databases}
%
% If your \TeX~distribution
% (\teTeX, \mikTeX, \dots) relies on file name databases, you must refresh
% these. For example, \teTeX\ users run \verb|texhash| or
% \verb|mktexlsr|.
%
% \subsection{Some details for the interested}
%
% \paragraph{Attached source.}
%
% The PDF documentation on CTAN also includes the
% \xfile{.dtx} source file. It can be extracted by
% AcrobatReader 6 or higher. Another option is \textsf{pdftk},
% e.g. unpack the file into the current directory:
% \begin{quote}
%   \verb|pdftk selinput.pdf unpack_files output .|
% \end{quote}
%
% \paragraph{Unpacking with \LaTeX.}
% The \xfile{.dtx} chooses its action depending on the format:
% \begin{description}
% \item[\plainTeX:] Run \docstrip\ and extract the files.
% \item[\LaTeX:] Generate the documentation.
% \end{description}
% If you insist on using \LaTeX\ for \docstrip\ (really,
% \docstrip\ does not need \LaTeX), then inform the autodetect routine
% about your intention:
% \begin{quote}
%   \verb|latex \let\install=y\input{selinput.dtx}|
% \end{quote}
% Do not forget to quote the argument according to the demands
% of your shell.
%
% \paragraph{Generating the documentation.}
% You can use both the \xfile{.dtx} or the \xfile{.drv} to generate
% the documentation. The process can be configured by the
% configuration file \xfile{ltxdoc.cfg}. For instance, put this
% line into this file, if you want to have A4 as paper format:
% \begin{quote}
%   \verb|\PassOptionsToClass{a4paper}{article}|
% \end{quote}
% An example follows how to generate the
% documentation with pdf\LaTeX:
% \begin{quote}
%\begin{verbatim}
%pdflatex selinput.dtx
%makeindex -s gind.ist selinput.idx
%pdflatex selinput.dtx
%makeindex -s gind.ist selinput.idx
%pdflatex selinput.dtx
%\end{verbatim}
% \end{quote}
%
% \section{Catalogue}
%
% The following XML file can be used as source for the
% \href{http://mirror.ctan.org/help/Catalogue/catalogue.html}{\TeX\ Catalogue}.
% The elements \texttt{caption} and \texttt{description} are imported
% from the original XML file from the Catalogue.
% The name of the XML file in the Catalogue is \xfile{selinput.xml}.
%    \begin{macrocode}
%<*catalogue>
<?xml version='1.0' encoding='us-ascii'?>
<!DOCTYPE entry SYSTEM 'catalogue.dtd'>
<entry datestamp='$Date$' modifier='$Author$' id='selinput'>
  <name>selinput</name>
  <caption>Semi-automatic detection of input encoding.</caption>
  <authorref id='auth:oberdiek'/>
  <copyright owner='Heiko Oberdiek' year='2007'/>
  <license type='lppl1.3'/>
  <version number='1.2'/>
  <description>
    This package selects the input encoding by specifying pairs
    of input characters and their glyph names.
    <p/>
    The package is part of the <xref refid='oberdiek'>oberdiek</xref>
    bundle.
  </description>
  <documentation details='Package documentation'
      href='ctan:/macros/latex/contrib/oberdiek/selinput.pdf'/>
  <ctan file='true' path='/macros/latex/contrib/oberdiek/selinput.dtx'/>
  <miktex location='oberdiek'/>
  <texlive location='oberdiek'/>
  <install path='/macros/latex/contrib/oberdiek/oberdiek.tds.zip'/>
</entry>
%</catalogue>
%    \end{macrocode}
%
% \begin{thebibliography}{9}
% \bibitem{inputenx}
%   Heiko Oberdiek: \textit{The \xpackage{inputenx} package};
%   2007-04-11 v1.1;
%   \CTAN{macros/latex/contrib/oberdiek/inputenx.pdf}.
%
% \bibitem{adobe:glyphlist}
%   Adobe: \textit{Adobe Glyph List};
%   2002-09-20 v2.0;
%   \url{http://partners.adobe.com/public/developer/en/opentype/glyphlist.txt}.
%
% \bibitem{adobe:aglfn}
%   Adobe: \textit{Adobe Glyph List For New Fonts};
%   2005-11-18 v1.5;
%   \url{http://partners.adobe.com/public/developer/en/opentype/aglfn13.txt}.
%
% \bibitem{stringenc}
%   Heiko Oberdiek: \textit{The \xpackage{stringenc} package};
%   2007-06-16 v1.1;
%   \CTAN{macros/latex/contrib/oberdiek/stringenc.pdf}.
%
% \end{thebibliography}
%
% \begin{History}
%   \begin{Version}{2007/06/16 v1.0}
%   \item
%     First version.
%   \end{Version}
%   \begin{Version}{2007/06/20 v1.1}
%   \item
%     Requested date for package \xpackage{stringenc} fixed.
%   \end{Version}
%   \begin{Version}{2007/09/09 v1.2}
%   \item
%     Line end fixed.
%   \end{Version}
% \end{History}
%
% \PrintIndex
%
% \Finale
\endinput
|
% \end{quote}
% Do not forget to quote the argument according to the demands
% of your shell.
%
% \paragraph{Generating the documentation.}
% You can use both the \xfile{.dtx} or the \xfile{.drv} to generate
% the documentation. The process can be configured by the
% configuration file \xfile{ltxdoc.cfg}. For instance, put this
% line into this file, if you want to have A4 as paper format:
% \begin{quote}
%   \verb|\PassOptionsToClass{a4paper}{article}|
% \end{quote}
% An example follows how to generate the
% documentation with pdf\LaTeX:
% \begin{quote}
%\begin{verbatim}
%pdflatex selinput.dtx
%makeindex -s gind.ist selinput.idx
%pdflatex selinput.dtx
%makeindex -s gind.ist selinput.idx
%pdflatex selinput.dtx
%\end{verbatim}
% \end{quote}
%
% \section{Catalogue}
%
% The following XML file can be used as source for the
% \href{http://mirror.ctan.org/help/Catalogue/catalogue.html}{\TeX\ Catalogue}.
% The elements \texttt{caption} and \texttt{description} are imported
% from the original XML file from the Catalogue.
% The name of the XML file in the Catalogue is \xfile{selinput.xml}.
%    \begin{macrocode}
%<*catalogue>
<?xml version='1.0' encoding='us-ascii'?>
<!DOCTYPE entry SYSTEM 'catalogue.dtd'>
<entry datestamp='$Date$' modifier='$Author$' id='selinput'>
  <name>selinput</name>
  <caption>Semi-automatic detection of input encoding.</caption>
  <authorref id='auth:oberdiek'/>
  <copyright owner='Heiko Oberdiek' year='2007'/>
  <license type='lppl1.3'/>
  <version number='1.2'/>
  <description>
    This package selects the input encoding by specifying pairs
    of input characters and their glyph names.
    <p/>
    The package is part of the <xref refid='oberdiek'>oberdiek</xref>
    bundle.
  </description>
  <documentation details='Package documentation'
      href='ctan:/macros/latex/contrib/oberdiek/selinput.pdf'/>
  <ctan file='true' path='/macros/latex/contrib/oberdiek/selinput.dtx'/>
  <miktex location='oberdiek'/>
  <texlive location='oberdiek'/>
  <install path='/macros/latex/contrib/oberdiek/oberdiek.tds.zip'/>
</entry>
%</catalogue>
%    \end{macrocode}
%
% \begin{thebibliography}{9}
% \bibitem{inputenx}
%   Heiko Oberdiek: \textit{The \xpackage{inputenx} package};
%   2007-04-11 v1.1;
%   \CTAN{macros/latex/contrib/oberdiek/inputenx.pdf}.
%
% \bibitem{adobe:glyphlist}
%   Adobe: \textit{Adobe Glyph List};
%   2002-09-20 v2.0;
%   \url{http://partners.adobe.com/public/developer/en/opentype/glyphlist.txt}.
%
% \bibitem{adobe:aglfn}
%   Adobe: \textit{Adobe Glyph List For New Fonts};
%   2005-11-18 v1.5;
%   \url{http://partners.adobe.com/public/developer/en/opentype/aglfn13.txt}.
%
% \bibitem{stringenc}
%   Heiko Oberdiek: \textit{The \xpackage{stringenc} package};
%   2007-06-16 v1.1;
%   \CTAN{macros/latex/contrib/oberdiek/stringenc.pdf}.
%
% \end{thebibliography}
%
% \begin{History}
%   \begin{Version}{2007/06/16 v1.0}
%   \item
%     First version.
%   \end{Version}
%   \begin{Version}{2007/06/20 v1.1}
%   \item
%     Requested date for package \xpackage{stringenc} fixed.
%   \end{Version}
%   \begin{Version}{2007/09/09 v1.2}
%   \item
%     Line end fixed.
%   \end{Version}
% \end{History}
%
% \PrintIndex
%
% \Finale
\endinput
|
% \end{quote}
% Do not forget to quote the argument according to the demands
% of your shell.
%
% \paragraph{Generating the documentation.}
% You can use both the \xfile{.dtx} or the \xfile{.drv} to generate
% the documentation. The process can be configured by the
% configuration file \xfile{ltxdoc.cfg}. For instance, put this
% line into this file, if you want to have A4 as paper format:
% \begin{quote}
%   \verb|\PassOptionsToClass{a4paper}{article}|
% \end{quote}
% An example follows how to generate the
% documentation with pdf\LaTeX:
% \begin{quote}
%\begin{verbatim}
%pdflatex selinput.dtx
%makeindex -s gind.ist selinput.idx
%pdflatex selinput.dtx
%makeindex -s gind.ist selinput.idx
%pdflatex selinput.dtx
%\end{verbatim}
% \end{quote}
%
% \section{Catalogue}
%
% The following XML file can be used as source for the
% \href{http://mirror.ctan.org/help/Catalogue/catalogue.html}{\TeX\ Catalogue}.
% The elements \texttt{caption} and \texttt{description} are imported
% from the original XML file from the Catalogue.
% The name of the XML file in the Catalogue is \xfile{selinput.xml}.
%    \begin{macrocode}
%<*catalogue>
<?xml version='1.0' encoding='us-ascii'?>
<!DOCTYPE entry SYSTEM 'catalogue.dtd'>
<entry datestamp='$Date$' modifier='$Author$' id='selinput'>
  <name>selinput</name>
  <caption>Semi-automatic detection of input encoding.</caption>
  <authorref id='auth:oberdiek'/>
  <copyright owner='Heiko Oberdiek' year='2007'/>
  <license type='lppl1.3'/>
  <version number='1.2'/>
  <description>
    This package selects the input encoding by specifying pairs
    of input characters and their glyph names.
    <p/>
    The package is part of the <xref refid='oberdiek'>oberdiek</xref>
    bundle.
  </description>
  <documentation details='Package documentation'
      href='ctan:/macros/latex/contrib/oberdiek/selinput.pdf'/>
  <ctan file='true' path='/macros/latex/contrib/oberdiek/selinput.dtx'/>
  <miktex location='oberdiek'/>
  <texlive location='oberdiek'/>
  <install path='/macros/latex/contrib/oberdiek/oberdiek.tds.zip'/>
</entry>
%</catalogue>
%    \end{macrocode}
%
% \begin{thebibliography}{9}
% \bibitem{inputenx}
%   Heiko Oberdiek: \textit{The \xpackage{inputenx} package};
%   2007-04-11 v1.1;
%   \CTAN{macros/latex/contrib/oberdiek/inputenx.pdf}.
%
% \bibitem{adobe:glyphlist}
%   Adobe: \textit{Adobe Glyph List};
%   2002-09-20 v2.0;
%   \url{http://partners.adobe.com/public/developer/en/opentype/glyphlist.txt}.
%
% \bibitem{adobe:aglfn}
%   Adobe: \textit{Adobe Glyph List For New Fonts};
%   2005-11-18 v1.5;
%   \url{http://partners.adobe.com/public/developer/en/opentype/aglfn13.txt}.
%
% \bibitem{stringenc}
%   Heiko Oberdiek: \textit{The \xpackage{stringenc} package};
%   2007-06-16 v1.1;
%   \CTAN{macros/latex/contrib/oberdiek/stringenc.pdf}.
%
% \end{thebibliography}
%
% \begin{History}
%   \begin{Version}{2007/06/16 v1.0}
%   \item
%     First version.
%   \end{Version}
%   \begin{Version}{2007/06/20 v1.1}
%   \item
%     Requested date for package \xpackage{stringenc} fixed.
%   \end{Version}
%   \begin{Version}{2007/09/09 v1.2}
%   \item
%     Line end fixed.
%   \end{Version}
% \end{History}
%
% \PrintIndex
%
% \Finale
\endinput
|
% \end{quote}
% Do not forget to quote the argument according to the demands
% of your shell.
%
% \paragraph{Generating the documentation.}
% You can use both the \xfile{.dtx} or the \xfile{.drv} to generate
% the documentation. The process can be configured by the
% configuration file \xfile{ltxdoc.cfg}. For instance, put this
% line into this file, if you want to have A4 as paper format:
% \begin{quote}
%   \verb|\PassOptionsToClass{a4paper}{article}|
% \end{quote}
% An example follows how to generate the
% documentation with pdf\LaTeX:
% \begin{quote}
%\begin{verbatim}
%pdflatex selinput.dtx
%makeindex -s gind.ist selinput.idx
%pdflatex selinput.dtx
%makeindex -s gind.ist selinput.idx
%pdflatex selinput.dtx
%\end{verbatim}
% \end{quote}
%
% \section{Catalogue}
%
% The following XML file can be used as source for the
% \href{http://mirror.ctan.org/help/Catalogue/catalogue.html}{\TeX\ Catalogue}.
% The elements \texttt{caption} and \texttt{description} are imported
% from the original XML file from the Catalogue.
% The name of the XML file in the Catalogue is \xfile{selinput.xml}.
%    \begin{macrocode}
%<*catalogue>
<?xml version='1.0' encoding='us-ascii'?>
<!DOCTYPE entry SYSTEM 'catalogue.dtd'>
<entry datestamp='$Date$' modifier='$Author$' id='selinput'>
  <name>selinput</name>
  <caption>Semi-automatic detection of input encoding.</caption>
  <authorref id='auth:oberdiek'/>
  <copyright owner='Heiko Oberdiek' year='2007'/>
  <license type='lppl1.3'/>
  <version number='1.2'/>
  <description>
    This package selects the input encoding by specifying pairs
    of input characters and their glyph names.
    <p/>
    The package is part of the <xref refid='oberdiek'>oberdiek</xref>
    bundle.
  </description>
  <documentation details='Package documentation'
      href='ctan:/macros/latex/contrib/oberdiek/selinput.pdf'/>
  <ctan file='true' path='/macros/latex/contrib/oberdiek/selinput.dtx'/>
  <miktex location='oberdiek'/>
  <texlive location='oberdiek'/>
  <install path='/macros/latex/contrib/oberdiek/oberdiek.tds.zip'/>
</entry>
%</catalogue>
%    \end{macrocode}
%
% \begin{thebibliography}{9}
% \bibitem{inputenx}
%   Heiko Oberdiek: \textit{The \xpackage{inputenx} package};
%   2007-04-11 v1.1;
%   \CTAN{macros/latex/contrib/oberdiek/inputenx.pdf}.
%
% \bibitem{adobe:glyphlist}
%   Adobe: \textit{Adobe Glyph List};
%   2002-09-20 v2.0;
%   \url{http://partners.adobe.com/public/developer/en/opentype/glyphlist.txt}.
%
% \bibitem{adobe:aglfn}
%   Adobe: \textit{Adobe Glyph List For New Fonts};
%   2005-11-18 v1.5;
%   \url{http://partners.adobe.com/public/developer/en/opentype/aglfn13.txt}.
%
% \bibitem{stringenc}
%   Heiko Oberdiek: \textit{The \xpackage{stringenc} package};
%   2007-06-16 v1.1;
%   \CTAN{macros/latex/contrib/oberdiek/stringenc.pdf}.
%
% \end{thebibliography}
%
% \begin{History}
%   \begin{Version}{2007/06/16 v1.0}
%   \item
%     First version.
%   \end{Version}
%   \begin{Version}{2007/06/20 v1.1}
%   \item
%     Requested date for package \xpackage{stringenc} fixed.
%   \end{Version}
%   \begin{Version}{2007/09/09 v1.2}
%   \item
%     Line end fixed.
%   \end{Version}
% \end{History}
%
% \PrintIndex
%
% \Finale
\endinput
